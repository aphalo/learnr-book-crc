\documentclass[paper=a4,10pt,div=17,headsepline,BCOR=12mm,twoside,open=right]{scrbook}\usepackage{knitr}

\usepackage[utf8]{inputenc}

\title{Notes on using R}
\author{Pedro J. Aphalo}
\date{Git: tag\gitVtagn , committed with hash \gitAbbrevHash\ on \gitAuthorIsoDate}

% using Lucida bright using now free package from PC TeX
\usepackage[lucidasmallscale,nofontinfo,seriftt=true,math-style=iso]{lucimatx}
% replace option altbullet
\renewcommand{\labelitemi}{%
{\UseTextSymbol{OMS}\textbullet}}
% needed for Lucida
\linespread{1.04}

\usepackage[footinfo, missing={`none'}]{gitinfo} % remember to setup Git hooks

\usepackage{hologo}

\usepackage[british]{babel}
\usepackage{csquotes}

\usepackage{graphicx}
\DeclareGraphicsExtensions{.jpg,.pdf}

\usepackage{microtype}
\usepackage[style=authoryear-comp,firstinits,sortcites,maxcitenames=2,%
    mincitenames=1,maxbibnames=10,minbibnames=10,backref,uniquename=mininit,%
    uniquelist=minyear,sortfirstinits=true]{biblatex}%,refsection=chapter

\usepackage[hyperindex,bookmarks,pdfview=FitB,%backref,
            pdftitle={Notes on using R},%
            pdfkeywords={R, statistics, data analysis, plotting},%
            pdfsubject={R},%
            pdfauthor={Pedro J. Aphalo}%
            ]{hyperref}

\usepackage{framed}
\renewenvironment{shaded}{%
  \def\FrameCommand{\fboxsep=\FrameSep \colorbox{shadecolor}}%
  \MakeFramed{\advance\hsize-\width \FrameRestore\FrameRestore}}%
 {\endMakeFramed}
\definecolor{shadecolor}{gray}{0.80}

\usepackage{abbrev}
\usepackage{usingr}
\usepackage{breakurl}
%\usepackage{imakeidx}

\addbibresource{rbooks.bib}
\addbibresource{references.bib}

%\makeindex
\IfFileExists{upquote.sty}{\usepackage{upquote}}{}
\begin{document}

% knitr setup










% \thispagestyle{empty}
% \titleLL
% \clearpage

\maketitle

%\frontmatter
%\begin{titlingpage}
%  \maketitle
%\titleLL
%\end{titlingpage}

\tableofcontents

%\listoftables

%\listoffigures

%\mainmatter

\chapter*{Preface}

This series of Notes cover different aspects of the use of R. They are meant to be use as a complement to a course or book, as explanations
are short and terse. We do not discuss here statistics, just R as a tool and language for data manipulation and display.










% !Rnw root = appendix.main.Rnw



\chapter[Plots with ggpplot, ggrepel and ggpmisc]{Plots with \ggplot, \ggrepel and \ggpmisc}\label{chap:R:plotting}

\section{Packages used in this chapter}

For executing the examples listed in this chapter you need first to load the following packages from the library:

\begin{knitrout}\footnotesize
\definecolor{shadecolor}{rgb}{0.969, 0.969, 0.969}\color{fgcolor}\begin{kframe}
\begin{alltt}
\hlkwd{library}\hlstd{(ggplot2)}
\hlkwd{library}\hlstd{(ggrepel)}
\hlkwd{library}\hlstd{(ggpmisc)}
\end{alltt}
\end{kframe}
\end{knitrout}

We set a font larger size than the default
\begin{knitrout}\footnotesize
\definecolor{shadecolor}{rgb}{0.969, 0.969, 0.969}\color{fgcolor}\begin{kframe}
\begin{alltt}
\hlkwd{theme_set}\hlstd{(}\hlkwd{theme_grey}\hlstd{(}\hlnum{16}\hlstd{))}
\end{alltt}
\end{kframe}
\end{knitrout}




\section[ggpmisc]{\ggpmisc}

\subsection{New stats}

Package \ggpmisc provides new stats: \code{stat\_peaks()}, \code{stat\_valleys()}, and \code{stat\_poly\_eq()}. Peaks and valleys are local (or global) maxima and minima. These stats return the $x$ and $y$ values at the peaks or valleys plus suitable labels, and default aesthetics that make easy their use with several different geoms, including \code{geom\_point}, \code{geom\_text}, \code{geom\_label}, \code{geom\_vline}, \code{geom\_hline} and \code{geom\_rug}, and also with geoms defined by package \ggrepel. Some examples follow.

\subsection{Peaks and valleys}
\begin{knitrout}\footnotesize
\definecolor{shadecolor}{rgb}{0.969, 0.969, 0.969}\color{fgcolor}\begin{kframe}
\begin{alltt}
\hlstd{lynx.df} \hlkwb{<-} \hlkwd{data.frame}\hlstd{(}\hlkwc{year} \hlstd{=} \hlkwd{as.numeric}\hlstd{(}\hlkwd{time}\hlstd{(lynx)),} \hlkwc{lynx} \hlstd{=} \hlkwd{as.matrix}\hlstd{(lynx))}
\end{alltt}
\end{kframe}
\end{knitrout}
 
\begin{knitrout}\footnotesize
\definecolor{shadecolor}{rgb}{0.969, 0.969, 0.969}\color{fgcolor}\begin{kframe}
\begin{alltt}
\hlkwd{ggplot}\hlstd{(lynx.df,} \hlkwd{aes}\hlstd{(year, lynx))} \hlopt{+} \hlkwd{geom_line}\hlstd{()} \hlopt{+}
  \hlkwd{stat_peaks}\hlstd{(}\hlkwc{colour} \hlstd{=} \hlstr{"red"}\hlstd{)} \hlopt{+}
  \hlkwd{stat_peaks}\hlstd{(}\hlkwc{geom} \hlstd{=} \hlstr{"text"}\hlstd{,} \hlkwc{colour} \hlstd{=} \hlstr{"red"}\hlstd{,}
             \hlkwc{vjust} \hlstd{=} \hlopt{-}\hlnum{0.5}\hlstd{,} \hlkwc{x.label.fmt} \hlstd{=} \hlstr{"%4.0f"}\hlstd{)} \hlopt{+}
  \hlkwd{stat_valleys}\hlstd{(}\hlkwc{colour} \hlstd{=} \hlstr{"blue"}\hlstd{)} \hlopt{+}
  \hlkwd{stat_valleys}\hlstd{(}\hlkwc{geom} \hlstd{=} \hlstr{"text"}\hlstd{,} \hlkwc{colour} \hlstd{=} \hlstr{"blue"}\hlstd{,}
               \hlkwc{vjust} \hlstd{=} \hlnum{1.5}\hlstd{,} \hlkwc{x.label.fmt} \hlstd{=} \hlstr{"%4.0f"}\hlstd{)} \hlopt{+}
  \hlkwd{ylim}\hlstd{(}\hlopt{-}\hlnum{100}\hlstd{,} \hlnum{7300}\hlstd{)}
\end{alltt}
\end{kframe}

{\centering \includegraphics[width=.63\textwidth]{figure/pos-unnamed-chunk-11-1} 

}



\end{knitrout}

\begin{knitrout}\footnotesize
\definecolor{shadecolor}{rgb}{0.969, 0.969, 0.969}\color{fgcolor}\begin{kframe}
\begin{alltt}
\hlkwd{ggplot}\hlstd{(lynx.df,} \hlkwd{aes}\hlstd{(year, lynx))} \hlopt{+} \hlkwd{geom_line}\hlstd{()} \hlopt{+}
  \hlkwd{stat_peaks}\hlstd{(}\hlkwc{colour} \hlstd{=} \hlstr{"red"}\hlstd{)} \hlopt{+}
  \hlkwd{stat_peaks}\hlstd{(}\hlkwc{geom} \hlstd{=} \hlstr{"rug"}\hlstd{,} \hlkwc{colour} \hlstd{=} \hlstr{"red"}\hlstd{)} \hlopt{+}
  \hlkwd{stat_peaks}\hlstd{(}\hlkwc{geom} \hlstd{=} \hlstr{"text"}\hlstd{,} \hlkwc{colour} \hlstd{=} \hlstr{"red"}\hlstd{,}
             \hlkwc{vjust} \hlstd{=} \hlopt{-}\hlnum{0.5}\hlstd{,} \hlkwc{x.label.fmt} \hlstd{=} \hlstr{"%4.0f"}\hlstd{)} \hlopt{+}
  \hlkwd{ylim}\hlstd{(}\hlnum{NA}\hlstd{,} \hlnum{7300}\hlstd{)}
\end{alltt}
\end{kframe}

{\centering \includegraphics[width=.63\textwidth]{figure/pos-unnamed-chunk-12-1} 

}



\end{knitrout}

\begin{knitrout}\footnotesize
\definecolor{shadecolor}{rgb}{0.969, 0.969, 0.969}\color{fgcolor}\begin{kframe}
\begin{alltt}
\hlkwd{ggplot}\hlstd{(lynx.df,} \hlkwd{aes}\hlstd{(year, lynx))} \hlopt{+} \hlkwd{geom_line}\hlstd{()} \hlopt{+}
  \hlkwd{stat_peaks}\hlstd{(}\hlkwc{colour} \hlstd{=} \hlstr{"red"}\hlstd{)} \hlopt{+}
  \hlkwd{stat_peaks}\hlstd{(}\hlkwc{geom} \hlstd{=} \hlstr{"rug"}\hlstd{,} \hlkwc{colour} \hlstd{=} \hlstr{"red"}\hlstd{)} \hlopt{+}
  \hlkwd{stat_valleys}\hlstd{(}\hlkwc{colour} \hlstd{=} \hlstr{"blue"}\hlstd{)} \hlopt{+}
  \hlkwd{stat_valleys}\hlstd{(}\hlkwc{geom} \hlstd{=} \hlstr{"rug"}\hlstd{,} \hlkwc{colour} \hlstd{=} \hlstr{"blue"}\hlstd{)}
\end{alltt}
\end{kframe}

{\centering \includegraphics[width=.63\textwidth]{figure/pos-unnamed-chunk-13-1} 

}



\end{knitrout}

\begin{knitrout}\footnotesize
\definecolor{shadecolor}{rgb}{0.969, 0.969, 0.969}\color{fgcolor}\begin{kframe}
\begin{alltt}
\hlkwd{ggplot}\hlstd{(lynx.df,} \hlkwd{aes}\hlstd{(year, lynx))} \hlopt{+} \hlkwd{geom_line}\hlstd{()} \hlopt{+}
  \hlkwd{stat_peaks}\hlstd{(}\hlkwc{colour} \hlstd{=} \hlstr{"red"}\hlstd{)} \hlopt{+}
  \hlkwd{stat_peaks}\hlstd{(}\hlkwc{geom} \hlstd{=} \hlstr{"rug"}\hlstd{,} \hlkwc{colour} \hlstd{=} \hlstr{"red"}\hlstd{)} \hlopt{+}
  \hlkwd{stat_peaks}\hlstd{(}\hlkwc{geom} \hlstd{=} \hlstr{"text"}\hlstd{,} \hlkwc{colour} \hlstd{=} \hlstr{"red"}\hlstd{,}
             \hlkwc{hjust} \hlstd{=} \hlopt{-}\hlnum{0.1}\hlstd{,} \hlkwc{label.fmt} \hlstd{=} \hlstr{"%4.0f"}\hlstd{,}
             \hlkwc{angle} \hlstd{=} \hlnum{90}\hlstd{,} \hlkwc{size} \hlstd{=} \hlkwd{rel}\hlstd{(}\hlnum{2}\hlstd{),}
             \hlkwd{aes}\hlstd{(}\hlkwc{label} \hlstd{=} \hlkwd{paste}\hlstd{(..y.label..,}
                               \hlstr{"skins in year"}\hlstd{, ..x.label..)))} \hlopt{+}
  \hlkwd{stat_valleys}\hlstd{(}\hlkwc{colour} \hlstd{=} \hlstr{"blue"}\hlstd{)} \hlopt{+}
  \hlkwd{stat_valleys}\hlstd{(}\hlkwc{geom} \hlstd{=} \hlstr{"rug"}\hlstd{,} \hlkwc{colour} \hlstd{=} \hlstr{"blue"}\hlstd{)} \hlopt{+}
  \hlkwd{stat_valleys}\hlstd{(}\hlkwc{geom} \hlstd{=} \hlstr{"text"}\hlstd{,} \hlkwc{colour} \hlstd{=} \hlstr{"blue"}\hlstd{,}
             \hlkwc{hjust} \hlstd{=} \hlopt{-}\hlnum{0.1}\hlstd{,} \hlkwc{vjust} \hlstd{=} \hlnum{1}\hlstd{,} \hlkwc{label.fmt} \hlstd{=} \hlstr{"%4.0f"}\hlstd{,}
             \hlkwc{angle} \hlstd{=} \hlnum{90}\hlstd{,} \hlkwc{size} \hlstd{=} \hlkwd{rel}\hlstd{(}\hlnum{2}\hlstd{),}
             \hlkwd{aes}\hlstd{(}\hlkwc{label} \hlstd{=} \hlkwd{paste}\hlstd{(..y.label..,}
                               \hlstr{"skins in year"}\hlstd{, ..x.label..)))} \hlopt{+}
  \hlkwd{ylim}\hlstd{(}\hlnum{NA}\hlstd{,} \hlnum{10000}\hlstd{)}
\end{alltt}
\end{kframe}

{\centering \includegraphics[width=.63\textwidth]{figure/pos-unnamed-chunk-14-1} 

}



\end{knitrout}

Of course, if one finds use for it, the peaks and/or valleys can be plotted on their own.

\begin{knitrout}\footnotesize
\definecolor{shadecolor}{rgb}{0.969, 0.969, 0.969}\color{fgcolor}\begin{kframe}
\begin{alltt}
\hlkwd{ggplot}\hlstd{(lynx.df,} \hlkwd{aes}\hlstd{(year, lynx))} \hlopt{+}
  \hlkwd{stat_peaks}\hlstd{(}\hlkwc{geom} \hlstd{=} \hlstr{"line"}\hlstd{)} \hlopt{+} \hlkwd{stat_valleys}\hlstd{(}\hlkwc{geom} \hlstd{=} \hlstr{"line"}\hlstd{)}
\end{alltt}
\end{kframe}

{\centering \includegraphics[width=.63\textwidth]{figure/pos-unnamed-chunk-15-1} 

}



\end{knitrout}
\subsection{Learning and/or debugging}

A very simple stat named \code{stat\_debug()} can save the work of adding print statements to the code of stats to get information about what data is being passed to the \code{compute\_group()} function. Because the code of this function is stored in a \code{ggproto} object, at the moment it is impossible to directly set breakpoints in it. This stat may also help users diagnose problems with the mapping of aesthetics in their code or just get a better idea of how the internals of \ggplot work.

\begin{knitrout}\footnotesize
\definecolor{shadecolor}{rgb}{0.969, 0.969, 0.969}\color{fgcolor}\begin{kframe}
\begin{alltt}
\hlkwd{ggplot}\hlstd{(lynx.df,} \hlkwd{aes}\hlstd{(year, lynx))} \hlopt{+} \hlkwd{geom_line}\hlstd{()} \hlopt{+}
  \hlkwd{stat_debug}\hlstd{(}\hlkwc{alpha} \hlstd{=} \hlnum{0.8}\hlstd{)}
\end{alltt}
\begin{verbatim}
##        x    y nrow ncol           colnames group
## 1 1877.5 3515  114    4 x, y, PANEL, group    -1
##   PANEL
## 1     1
\end{verbatim}
\end{kframe}

{\centering \includegraphics[width=.63\textwidth]{figure/pos-unnamed-chunk-16-1} 

}



\end{knitrout}

\begin{knitrout}\footnotesize
\definecolor{shadecolor}{rgb}{0.969, 0.969, 0.969}\color{fgcolor}\begin{kframe}
\begin{alltt}
\hlstd{lynx.df}\hlopt{$}\hlstd{century} \hlkwb{<-} \hlkwd{ifelse}\hlstd{(lynx.df}\hlopt{$}\hlstd{year} \hlopt{>=} \hlnum{1900}\hlstd{,} \hlstr{"XX"}\hlstd{,} \hlstr{"XIX"}\hlstd{)}
\hlkwd{ggplot}\hlstd{(lynx.df,} \hlkwd{aes}\hlstd{(year, lynx,} \hlkwc{color} \hlstd{= century))} \hlopt{+}
  \hlkwd{geom_line}\hlstd{()} \hlopt{+}
  \hlkwd{stat_debug}\hlstd{(}\hlkwc{alpha} \hlstd{=} \hlnum{0.8}\hlstd{,} \hlkwc{size} \hlstd{=} \hlkwd{rel}\hlstd{(}\hlnum{2.5}\hlstd{))}
\end{alltt}
\begin{verbatim}
##      x    y nrow ncol                   colnames
## 1 1860 3380   79    5 x, y, colour, PANEL, group
##   group PANEL
## 1     1     1
##      x      y nrow ncol
## 1 1917 3535.5   35    5
##                     colnames group PANEL
## 1 x, y, colour, PANEL, group     2     1
\end{verbatim}
\end{kframe}

{\centering \includegraphics[width=.63\textwidth]{figure/pos-unnamed-chunk-17-1} 

}



\end{knitrout}

\section[ggrepel]{\ggrepel}

\subsection{New geoms}

Package \ggrepel provides two new geoms: \code{geom\_text\_repel} and \code{geom\_label\_repel}. They are used similarly to \code{geom\_text} and \code{geom\_label} but the text or labels ``repel'' each other so that they rarely overlap unless the plot is very crowded. 


\begin{knitrout}\footnotesize
\definecolor{shadecolor}{rgb}{0.969, 0.969, 0.969}\color{fgcolor}\begin{kframe}
\begin{alltt}
\hlkwd{try}\hlstd{(}\hlkwd{detach}\hlstd{(package}\hlopt{:}\hlstd{ggpmisc))}
\hlkwd{try}\hlstd{(}\hlkwd{detach}\hlstd{(package}\hlopt{:}\hlstd{ggrepel))}
\hlkwd{try}\hlstd{(}\hlkwd{detach}\hlstd{(package}\hlopt{:}\hlstd{ggplot2))}
\end{alltt}
\end{kframe}
\end{knitrout}





\chapter{Further reading about R}\label{chap:R:readings}

\section{Introductory texts}

\nocite{Dalgaard2008,Zuur2009,Teetor2011}

\section{Texts on specific aspects}

\nocite{Chang2013,Fox2002,Fox2010,Faraway2004,Faraway2006,Everitt2011}

\section{Advanced texts}

\nocite{Xie2013,wickham2015,wickham2014advanced,Pinheiro2000,Murrell2011,Matloff2011,Ihaka1996}

\printbibliography

\end{document}

\appendix

\chapter{Build information}

\begin{knitrout}\footnotesize
\definecolor{shadecolor}{rgb}{0.969, 0.969, 0.969}\color{fgcolor}\begin{kframe}
\begin{alltt}
\hlkwd{Sys.info}\hlstd{()}
\end{alltt}
\begin{verbatim}
##                      sysname 
##                    "Windows" 
##                      release 
##                      "7 x64" 
##                      version 
## "build 7601, Service Pack 1" 
##                     nodename 
##                      "MUSTI" 
##                      machine 
##                     "x86-64" 
##                        login 
##                     "aphalo" 
##                         user 
##                     "aphalo" 
##               effective_user 
##                     "aphalo"
\end{verbatim}
\end{kframe}
\end{knitrout}



\begin{knitrout}\footnotesize
\definecolor{shadecolor}{rgb}{0.969, 0.969, 0.969}\color{fgcolor}\begin{kframe}
\begin{alltt}
\hlkwd{sessionInfo}\hlstd{()}
\end{alltt}
\begin{verbatim}
## R version 3.2.3 (2015-12-10)
## Platform: x86_64-w64-mingw32/x64 (64-bit)
## Running under: Windows 7 x64 (build 7601) Service Pack 1
## 
## locale:
## [1] LC_COLLATE=English_United Kingdom.1252 
## [2] LC_CTYPE=English_United Kingdom.1252   
## [3] LC_MONETARY=English_United Kingdom.1252
## [4] LC_NUMERIC=C                           
## [5] LC_TIME=English_United Kingdom.1252    
## 
## attached base packages:
## [1] tools     stats     graphics  grDevices
## [5] utils     datasets  base     
## 
## other attached packages:
## [1] stringr_1.0.0 knitr_1.12   
## 
## loaded via a namespace (and not attached):
##  [1] ggrepel_0.4.2    Rcpp_0.12.3     
##  [3] splus2R_1.2-0    digest_0.6.9    
##  [5] assertthat_0.1   dplyr_0.4.3     
##  [7] R6_2.1.1         grid_3.2.3      
##  [9] plyr_1.8.3       DBI_0.3.1       
## [11] ggpmisc_0.2.0    gtable_0.1.2    
## [13] formatR_1.2.1    magrittr_1.5    
## [15] evaluate_0.8     scales_0.3.0    
## [17] highr_0.5.1      ggplot2_2.0.0   
## [19] stringi_1.0-1    lazyeval_0.1.10 
## [21] labeling_0.3     munsell_0.4.2   
## [23] parallel_3.2.3   colorspace_1.2-6
## [25] methods_3.2.3
\end{verbatim}
\end{kframe}
\end{knitrout}

%\backmatter

\end{document}
