\chapter*{Preface}

\begin{VF}
``Suppose that you want to teach the `cat' concept to a very young child. Do you explain that a cat is a relatively small, primarily carnivorous mammal with retractible claws, a distinctive sonic output, etc.? I'll bet not. You probably show the kid a lot of different cats, saying `kitty' each time, until it gets the idea. To put it more generally, generalizations are best made by abstraction from experience.''

\VA{R. P. Boas}{\emph{Can we make mathematics intelligible?}, 1981}\nocite{Boas1981}
\end{VF}

%\dictum[R. P. Boas (1981) Can we make mathematics intelligible?, \emph{American Mathematical Monthly} \textbf{88:} 727-731.]{"Suppose that you want to teach the `cat' concept to a very young child. Do you explain that a cat is a relatively small, primarily carnivorous mammal with retractible claws, a distinctive sonic output, etc.? I'll bet not. You probably show the kid a lot of different cats, saying `kitty' each time, until it gets the idea. To put it more generally, generalizations are best made by abstraction from experience."}


% Such pauses are not a miss use of our time. To learn a natural language we need to interact with other speakers, we need feedback. In the case of R, we can get feedback both from the outcomes from our utterances to the computer, and from other \Rlang users.}
\noindent
This book covers different aspects of the use of the \Rlang language. Chapters \ref{chap:R:introduction} to \ref{chap:R:functions} describe the \Rlang language itself. Later chapters describe extensions to the \Rlang language available through contributed \emph{packages}, the \emph{grammar of data} and the \emph{grammar of graphics}. In this book, explanations are concise but contain pointers to additional sources of information, so as to encourage the development of a routine of independent exploration. This is not an arbitrary decision, this is the normal \emph{modus operandi} of most of us who use \Rlang regularly for a variety of different problems. Some have called approaches like the one used here ``learning the hard way,'' but I would call it ``learning to be independent.''

I do not discuss statistics or data analysis methods in this book; I describe \Rlang as a language for data manipulation and display. The idea is for you to learn the \Rlang language in a way comparable to how children learn a language: they work out what the rules are, simply by listening to people speak and trying to utter what they want to tell their parents. Of course, small children receive some guidance, but they are not taught a prescriptive set of rules like when learning a second language at school. Instead of listening, you will read code, and instead of speaking, you will try to execute \Rlang  code statements on a computer---i.e., you will try your hand at using \Rlang to tell a computer what you want it to compute. I do provide explanations and guidance, but the idea of this book is for you to use the numerous examples to find out by yourself the overall patterns and coding philosophy behind the \Rlang language. Instead of parents being the sound board for your first utterances in \Rlang, the computer will play this role. You will \emph{play} by modifying the examples, see how the computer responds: does \Rlang understand you or not? Using a language actively is the most efficient way of learning it. By using it, I mean actually reading, writing, and running scripts or programs (copying and pasting, or typing ready-made examples from books or the internet, does not qualify as using a language).

I have been using \Rlang since around 1998 or 1999, but I am still constantly learning new things about \Rlang itself and \Rlang packages. With time, it has replaced in my work as a researcher and teacher several other pieces of software: \pgrmnameNI{SPSS}, \pgrmnameNI{Systat}, \pgrmnameNI{Origin}, \pgrmnameNI{MS-Excel}, and it has become a central piece of the tool set I use for producing lecture slides, notes, books and even web pages. This is to say that it is the most useful piece of software and programming language I have ever learned to use. Of course, in time it will be replaced by something better, but at the moment it is a key language to learn for anybody with a need to analyze and display data.

What is a language? A language is a system of communication. \Rlang as a language allows us to communicate with other members of the \Rlang community, and with computers. As all languages in active use, \Rlang evolves. New ``words'' and new ``constructs'' are incorporated into the language, and some earlier frequently used ones are relegated to the fringes of the corpus. I describe current usage and ``modisms'' of the \Rlang language in a way accessible to a readership unfamiliar with computer science but with some background in data analysis as used in biology, engineering, or humanities.

When teaching, I tend to lean toward challenging students rather than telling an over-simplified story. There are two reasons for this. First, I prefer as a student, and I learn best myself if the going is not too easy. Second, if I would hide the tricky bits of the \Rlang language, it would make reader's life much more difficult later on. You will not remember all the details; nobody could. However, you most likely will remember or develop a sense of in which situations you need to be careful or should check the details. So, I will expose you not only to the usual cases, but also to several exceptions and counterintuitive features of the language, which I have highlighted with icons. Reading this book will be about exploring a new world; this book aims to be a travel guide, but neither a traveler's account, nor a cookbook of \Rlang recipes.

Keep in mind that it is impossible to remember everything about \Rlang! The \Rlang language, in a broad sense, is vast because its capabilities can be expanded with independently developed packages. Learning to use \Rlang consists of learning the basics plus developing the skill of finding your way in \Rlang and its documentation.  In 2017, the number of packages available in the Comprehensive \Rlang Archive Network (CRAN) broke the 10,000 barrier. CRAN is the most important, but not only, public repository for \Rlang packages. How good a command of the \Rlang language and packages a user needs depends on the type of activities to be carried out. This book attempts to train you in the use of the \Rlang language itself, and of popular \Rlang language extensions for data manipulation and graphical display. Given the availability of numerous books on statistical analysis with \Rlang, in the present book I will cover only the bare minimum of this subject. The same is true for package development in \Rlang. This book seats somewhere between, aiming at teaching programming in-the-small: the use of \Rlang to automate the drudgery of data manipulation, including the different steps spanning from data input and exploration to the production of publication-quality illustrations.

As with all ``rich'' languages, there are many different ways of doing things in \Rlang, and there is in almost all cases no one-size-fits-all solution to a problem. There is always a compromise involved, usually between time spent by the user and processing time required in the computer. Many of the packages that are most popular nowadays did not exist when I started using \Rlang, and many of these packages make new approaches available. One could write many different \Rlang books with a given aim using substantially different ways of achieving the same results. In this book, I limit myself to packages that are currently popular and/or that I consider elegantly designed. I have in particular tried to limit myself to packages with similar design philosophies, especially in relation to their interfaces. What is elegant design, and in particular what is a friendly user interface, depends strongly on each user's preferences and previous experience. Consequently, the contents of the book are strongly biased by my own preferences. I have tried to write examples in ways that execute fast without compromising readability. I encourage readers to take this book as a starting point for exploring the very many packages, styles, and approaches which I have not described.

I will appreciate suggestions for further examples, and notification of errors and unclear sections. Many of the examples here have been collected from diverse sources over many years and, because of this, not all sources are acknowledged. If you recognize any example as yours or someone else's, please let me know so that I can add a proper acknowledgement. I warmly thank the students that over the years have asked the questions and posed the problems that have helped me write this text and correct the mistakes and voids of previous versions. I have also received help on online forums and in person from numerous people, learned from archived e-mail list messages, blog posts, books, articles, tutorials, webinars, and by struggling to solve some new problems on my own. In many ways this text owes much more to people who are not authors than to myself. However, as I am the one who has written this version and decided what to include and exclude, as author, I take full responsibility for any errors and inaccuracies.

Why have I chosen the title ``\emph{Learn R \ldots\ as You Learned Your Mother Tongue}''? On one hand, because this book is based on exploration and practice. On the other hand, because you will be exposed to current usage and not spared the quirks of the language. When we use our mother tongue in everyday life, we do not think about grammar rules or sentence structure, except for the trickier or unfamiliar situations. My aim is for this book to help you grow to use \Rlang in this same way, to become fluent in \Rlang. The book is structured around the elements of languages with chapter titles that highlight the parallels between natural languages like English and the \Rlang language.

\emph{I encourage you to approach \Rlang like a child approaches his or her mother tongue when first learning to speak: do not struggle, just play, and fool around with \Rlang! If the going gets difficult and frustrating, take a break! If you get a new insight, take a break to enjoy the victory!
}%\end{framed}

\section*{Acknowledgements}
First I should thank Jaakko Heinonen for introducing me to the then new R. Along the way many well known and not so famous experts have answered my questions in usenet and more recently in Stackoverflow. As time went by, answering other people's questions, both in the internet and in person, became the driving force for me to delve into the depths of the R language. Of course, I still get stuck from time to time and ask for help. I wish to warmly thank all the people I have interacted with in relation to R, including members of my own research group, students participating in the courses I have taught, colleagues I have collaborated with, authors of the books I have read and people I have only met online or at conferences. All of them have made it possible for me to write this book. This has been a time consuming endeavour which has kept me too many hours away from my family, so I specially thank Tarja, Rosa and Tomás for their understanding. I am indebted to Tarja Lehto, Titta Kotilainen and anonymous reviewers for many very helpful comments on different versions of the book manuscript, Rob Calver for his encouragement and patience during the whole duration of this book writing project.

\section*{Icons used to mark different content}

Text boxes are used throughout the book to highlight content that plays specific roles in the learning process or that require special attention from the reader. Each box contains one of five different icons that indicate the type of its contents as described below.\\[1.5ex]

\noindent
\playicon\ Signals \emph{playground} boxes which contain open-ended exercises---ideas and pieces of \Rlang code to play with at the \Rlang console.\\[1.5ex]

\noindent
\advplayicon\ Signals \emph{advanced playground} which will require more time to play with before grasping concepts than regular \emph{playground} boxes.\\[1.5ex]

\noindent
\ilAttention\ Signals important bits of information that must be remembered when using \Rlang---i.e., explain some unusual feature of the language.\\[1.5ex]

\noindent
\ilAdvanced\ Signals in-depth explanations of specific points that may require you to spend time thinking, which en general can be skipped on first reading, but to which you should return at a later, peaceful, time preferably with a cup of coffee or tea.\\[1.5ex]

\noindent
\infoicon\ Signals text boxes providing general information not directly related to the \Rlang language.\\[1.5ex]
%\end{framed}
%\newpage

%\newpage
%\begin{infobox}
%\noindent
%\textbf{Status as of 2016-11-23.} I have updated the manuscript to track package updates since the previous version uploaded six months ago, and added several examples of the new functionality added to packages \ggpmisc, \ggrepel, and \ggplot. I have written new sections on packages \viridis, \pkgname{gganimate}, \pkgname{ggstance}, \pkgname{ggbiplot}, \pkgname{ggforce}, \pkgname{ggtern} and \pkgname{ggalt}. Some of these sections are to be expanded, and additional sections are planned for other recently released packages.
%
%With respect to the chapter \textit{Storing and manipulating data with R} I have put it on hold, except for the introduction, until I can see a soon to be published book covering the same subject. Hadley Wickham has named the set of tools developed by him and his collaborators as \textit{tidyverse} to be described in the book titled \textit{R for Data Science} by Grolemund and Wickham (O'Reilly).
%
%An important update to \ggplot was released last week, and it includes changes to the behavior of some existing functions, specially faceting has become extensible through other packages. Several of the new facilities are described in the updated text and code included in this book and this pdf has been generated with up-to-date version of \ggplot and packages as available today from CRAN, except for \pkgname{ggtern} which was downloaded from Bitbucket minutes ago.
%
%The present update adds about 100 pages to the previous versions. I expect to upload a new update to this manuscript in one or two months time.
%
%\textbf{Status as of 2017-01-17.} Added ``playground'' exercises to the chapter describing \ggplot, and converted some of the examples earlier part of the main text into these playground items. Added icons to help readers quickly distinguish playground sections (\textcolor{blue}{\noticestd{"0055}}), information sections (\textcolor{blue}{\modpicts{"003D}}), warnings about things one needs to be specially aware of (\colorbox{yellow}{\typicons{"E136}}) and boxes with more advanced content that may require longer time/more effort to grasp (\typicons{"E04E}). Added to the sections \code{scales} and examples in the \ggplot chapter details about the use of colors in \Rlang and \ggplot2. Removed some redundant examples, and updated the section on \code{plotmath}. Added terms to the alphabetical index. Increased line-spacing to avoid uneven spacing with inline code bits.
%
%\textbf{Status as of 2017-02-09.} Wrote section on ggplot2 themes, and on using system- and Google fonts in ggpplots with the help of package \pkgname{showtext}. Expanded section on \ggplot's \code{annotation}, and revised some sections in the ``R scripts and Programming'' chapter. Started writing the data chapter. Wrote draft on writing and reading text files. Several other smaller edits to text and a few new examples.
%
%\textbf{Status as of 2017-02-14.} Wrote sections on reading and writing MS-Excel files, files from statistical programs such as SPSS, SyStat, etc., and NetCDF files. Also wrote sections on using URLs to directly read data, and on reading HTML and XML files directly, as well on using JSON to retrieve measured/logged data from IoT (internet of things) and similar intelligent physical sensors, micro-controller boards and sensor hubs with network access.
%
%\textbf{Status as of 2017-03-25.} Revised and expanded the chapter on plotting maps, adding a section on the manipulation and plotting of image data. Revised and expanded the chapter on extensions to \pkgname{ggplot2}, so that there are no longer empty sections. Wrote short chapter ``If and when \Rlang needs help.'' Revised and expanded the ``Introduction'' chapter. Added index entries, and additional citations to literature.
%
%\textbf{Status as of 2017-04-04.} Revised and expanded the chapter on using \Rpgrm as a calculator. Revised and expanded the ``Scripts'' chapter. Minor edits to ``Functions'' chapter. Continued writing chapter on data, writing a section on \Rlang native apply functions and added preliminary text for a pipes and tees section. Write intro to `tidyverse' and grammar of data manipulation. Added index entries, and a few additional citations to the literature. Spell checking.
%
%\textbf{Status as of 2017-04-08.} Completed writing first draft of chapter on data, writing all the previously missing sections on the ``grammar of data manipulation.'' Wrote two extended examples in the same chapter. Add table listing several extensions to \pkgname{ggplot2} not described in the book.
%
%\textbf{Status as of 2017-04-13.} Revised all chapters correcting some spelling mistakes, adding some explanatory text and indexing all functions and operators used. Thoroughly revised the Introduction chapter and the Preface. Expanded section on bar plots (now bar and column plots). Revised section on tile plots. Expanded section on factors in chapter 2, adding examples of reordering of factor labels, and making clearer the difference between the labels of the levels and the levels themselves.
%
%\textbf{Status as of 2017-04-29.} Tested with R 3.4.0. Package \pkgname{gganimate} needs to be installed from GitHub as the updated version is not yet in CRAN. Function \code{gg\_animate()} has been renamed \code{gganimate().}
%
%\textbf{Status as of 2017-05-14.} Submitted package \pkgname{learnrbook} to CRAN. Revised code in the book
%to use this new package. Small fixes after more testing. Added examples of plotting and labeling based on fits with \code{method = "nls"}, including use of the new \code{ggpmisc::stat\_fit\_tidy()}.
%
%\textbf{Status as of 2017-06-11.} Added sections on R-code bench marking and profiling for performance optimization. Added also an example of explicit compilation of a function defined in the R language. Added section on functions \code{assign()}, \code{get()} and \code{mget()}.
%
%\textbf{Status as of 2017-08-12.} Various edits to all chapters. Expanded section on \pkgname{ggpmisc} to include the new functionality added in version 0.2.15.9002: \code{geom\_table} and \code{stat\_fit\_tb}. Added section on package \pkgname{ggbeeswarm}. Added sections on packages \pkgname{magick} and on using \pgrmname{ImageJ} from \Rpgrm. Improved indexing and cross references.
%
%\textbf{Status as of 2017-10-25.} Edited the chapter on using R as a calculator, adding examples on insertion and deletion of members of lists and vectors, and also of use of \code{gl()} and \code{reorder()}. Edited sections on scale limits and added new section on coordinate limits to explain more thoroughly their differences and uses in chapter on plotting with \pkgname{ggplot2}. Added a section on package \pkgname{ggsignif} to the chapter on extensions to \pkgname{ggplot2}. Expanded section on \pkgname{ggpmisc} in the same chapter describing new functionality added in version 0.2.16.
%\pkgname{ggplo2} $>=$ 2.2.1.9000 is required by the current development version of \pkgname{ggpmisc}.
%
%\textbf{Status as of 2017-10-30.}  Add section on using pipes with \code{ggplot()} and layers.
%\end{infobox} 