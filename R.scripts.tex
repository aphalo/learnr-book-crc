% !Rnw root = appendix.main.Rnw

\begin{knitrout}
\definecolor{shadecolor}{rgb}{0.969, 0.969, 0.969}\color{fgcolor}\begin{kframe}


{\ttfamily\noindent\color{warningcolor}{\#\# Warning in file(con, "{}r"{}): cannot open file 'r4p.main.Rnw': No such file or directory}}

{\ttfamily\noindent\bfseries\color{errorcolor}{\#\# Error in file(con, "{}r"{}): cannot open the connection}}\end{kframe}
\end{knitrout}

\chapter{The R language: ``paragraphs'' and ``essays''}\label{chap:R:scripts}
\index{scripts}

\begin{VF}
An \Rlang script is simply a text file containing (almost) the same commands that you would enter on the command line of R.

\VA{Jim Lemon}{Kickstarting R}
\end{VF}

%\dictum[\href{https://cran.r-project.org/doc/contrib/Lemon-kickstart/}{Kickstarting R}]{An R script is simply a text file containing (almost) the same commands that you would enter on the command line of R.}\vskip2ex

\section{Aims of this chapter}

For those who have mainly used graphical user interfaces, understanding why and when scripts can help in communicating a certain data analysis protocol can be revelatory. As soon as a data analysis stops being trivial, describing the steps followed through a system of menus and dialogue boxes becomes extremely tedious.

Moreover, graphical user interfaces tend to be difficult to extend or improve in a way that keeps step-by-step instructions valid across program versions and operating systems.

Many times, exactly the same sequence of commands needs to be applied to different data sets, and scripts make both implementation and validation of such a requirement easy.

In this chapter, I will walk you through the use of \Rpgrm scripts, starting from an extremely simple script.

\section{Writing scripts}

In \Rlang language, the closest match to a natural language essay is a script. A script is built from multiple interconnected code statements needed to complete a given task. Simple statements can be combined into compound statements, which are the equivalent of natural language paragraphs. Scripts can vary from simple scripts containing only a few code statements, to complex scripts containing hundreds of code statements. In the rest of the present section I discuss how to write readable and reliable scripts and how to use them.

\subsection{What is a script?}\label{sec:script:what:is}
\index{scripts!definition}
A \textit{script} is a text file that contains (almost) the same commands that you would type at the console prompt. A true script is not, for example, an MS-Word file where you have pasted or typed some \Rlang commands. A script file has the following characteristics.
\begin{itemize}
  \item The script is a text file.
  \item The file contains valid \Rlang statements (including comments) and nothing else.
  \item Comments start at a \code{\#} and end at the end of the line.
  \item The \Rlang statements are in the file in the order that they must be executed.
  \item \Rlang scripts have file names ending in \texttt{.r} or \texttt{.R}.
\end{itemize}

It is good practice to write scripts so that they are self-contained. To make a script self-contained, one must include calls to \texttt{library()} to load the packages used, load or import data from files, perform the data analysis and display and/or save the results of the analysis. Such scripts can be used to apply the same analysis algorithm to other data and/or to reproduce the same analysis at a later time. Such scripts document all steps used for the analysis.



\subsection{How do we use a script?}\label{sec:script:using}
\index{scripts!sourcing}

A script can be ``sourced'' using function \Rfunction{source()}. If we have a text file called \texttt{my.first.script.r} containing the following text:
\begin{shaded}
\footnotesize
\begin{verbatim}
# this is my first R script
print(3 + 4)
\end{verbatim}
\end{shaded}

and then source this file:

\begin{knitrout}
\definecolor{shadecolor}{rgb}{0.969, 0.969, 0.969}\color{fgcolor}\begin{kframe}
\begin{alltt}
\hlkwd{source}\hlstd{(}\hlstr{"my.first.script.r"}\hlstd{)}
\end{alltt}
\begin{verbatim}
## [1] 7
\end{verbatim}
\end{kframe}
\end{knitrout}

The results of executing the statements contained in the file will appear in the console. The commands themselves are not shown (by default the sourced file is not \emph{echoed} to the console) and the results will not be printed unless you include explicit \Rfunction{print()} commands in the script. This applies in many cases also to plots---e.g., a figure created with \Rfunction{ggplot()} needs to be printed if we want it to be included in the output when the script is run. Adding a redundant \Rfunction{print()} is harmless.

From within \RStudio, if you have an \Rpgrm script open in the editor, there will be a ``source'' icon visible with an attached drop-down menu from which you can choose ``Source'' as described above, or ``Source with echo,'' or ``Source as local job'' for the script in the currently active editor tab.

When a script is \emph{sourced}, the output can be saved to a text file instead of being shown in the console. It is also easy to call \Rpgrm with the \Rlang script file as an argument directly at the operating system shell or command-interpreter prompt---and obviously also from shell scripts. The next two chunks show commands entered at the OS shell command prompt rather than at the \Rlang command prompt.
\begin{shaded}
\footnotesize
\begin{verbatim}
> RScript my.first.script.r
\end{verbatim}
\end{shaded}

You can open an operating system's \emph{shell} from the Tools menu in \RStudio, to run this command. The output will be printed to the shell console. If you would like to save the output to a file, use redirection using the operating system's syntax.
\begin{shaded}
\footnotesize
\begin{verbatim}
> RScript my.first.script.r > my.output.txt
\end{verbatim}
\end{shaded}

Sourcing is very useful when the script is ready, however, while developing a script, or sometimes when testing things, one usually wants to run (or \emph{execute}) one or a few statements at a time. This can be done using the ``run'' button\footnote{If you use a different IDE or editor with an \Rlang mode, the details will vary, but a run command will be usually available.} after either positioning the cursor in the line to be executed, or selecting the text that one would like to run (the selected text can be part of a line, a whole line, or a group of lines, as long as it is syntactically valid). The key-shortcut Ctrl-Enter is equivalent to pressing the ``run'' button in \RStudio.

\subsection{How to write a script}\label{sec:script:writing}
\index{scripts!writing}

As with any type of writing, different approaches may be preferred by different \Rlang users. In general, the approach used, or mix of approaches, will also depend on how confident you are that the statements will work as expected---you already know the best approach vs.\ you are exploring different alternatives.
\begin{description}
\item[If one is very familiar with similar problems] One would just create a new text file and write the whole thing in the editor, and then test it. This is rather unusual.
\item[If one is moderately familiar with the problem] One would write the script as above, but testing it, step by step, as one is writing it. This is usually what I do.
\item[If one is mostly playing around] Then if one is using \RStudio, one can type statements at the console prompt. As you should know by now, everything you run at the console is saved to the ``History.'' In \RStudio, the History is displayed in its own pane, and in this pane one can select any previous statement(s) and by clicking on a single icon, copy and paste them to either the \Rlang console prompt, or the cursor position in the editor pane. In this way one can build a script by copying and pasting from the history to your script file, the bits that have worked as you wanted.
\end{description}

\begin{playground}
By now you should be familiar enough with \Rlang to be able to write your own script.
\begin{enumerate}
  \item Create a new \Rpgrm script (in \RStudio, from the File menu, ``+'' icon, or by typing ``Ctrl + Shift + N'').
  \item Save the file as \texttt{my.second.script.r}.
  \item Use the editor pane in \RStudio to type some \Rpgrm commands and comments.
  \item \emph{Run} individual commands.
  \item \emph{Source} the whole file.
\end{enumerate}
\end{playground}

\subsection{The need to be understandable to people}\label{sec:script:readability}
\index{scripts!readability}

When you write a script, it is either because you want to document what you have done or you want re-use the script at a later time. In either case, the script itself although still meaningful for the computer, could become very obscure to you, and even more to someone seeing it for the first time. This must be avoided by spending time and effort on the writing style.

How does one achieve an understandable script or program?
\begin{itemize}
  \item Avoid the unusual. People using a certain programming language tend to use some implicit or explicit rules of style---style includes \textit{indentation} of statements, \textit{capitalization} of variable and function names. As a minimum try to be consistent with yourself.
  \item Use meaningful names for variables, and any other object. What is meaningful depends on the context. Depending on common use, a single letter may be more meaningful than a long word. However self-explanatory names are usually better: e.g.,  using \code{n.rows} and \code{n.cols} is much clearer than using \code{n1} and \code{n2} when dealing with a matrix of data. Probably \code{number.of.rows} and \code{number.of.columns} would make the script verbose, and take longer to type without gaining anything in return.
  \item How to make the words visible in names: traditionally in \Rlang one would use dots to separate the words and use only lower case. Some years ago, it became possible to use underscores. The use of underscores is quite common nowadays because in some contexts it is ``safer'', as in some situations a dot may have a special meaning. What we call ``camel case'' is only infrequently used in \Rlang programming but is common in other languages like Pascal. An example of camel case is \code{NumCols}.
\end{itemize}

\begin{playground}
Here is an example of bad style in a script. Read \href{https://google.github.io/styleguide/Rguide.xml}{Google's R Style Guide}%\footnote{This is just an example, similar, but not exactly the same style as the style I use myself.}
, and edit the code in the chunk below so that it becomes easier to read.





































































































































