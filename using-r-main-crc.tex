\documentclass[krantz2]{krantz}\usepackage{knitr}%,ChapterTOCs

%\usepackage[utf8]{inputenc}
\usepackage{color}

\usepackage{polyglossia}
\setdefaultlanguage[variant = british, ordinalmonthday = false]{english}

%\usepackage{gitinfo2} % remember to setup Git hooks

\usepackage{hologo}

\usepackage{csquotes}

\usepackage{graphicx}
\DeclareGraphicsExtensions{.jpg,.pdf,.png}

\usepackage{animate}

%\usepackage{microtype}
\usepackage[style=authoryear-comp,giveninits,sortcites,maxcitenames=2,%
    mincitenames=1,maxbibnames=10,minbibnames=10,backref,uniquename=mininit,%
    uniquelist=minyear,sortgiveninits=true,backend=biber]{biblatex}%,refsection=chapter

\newcommand{\href}[2]{\emph{#2} (\url{#1})}

%\usepackage[unicode,hyperindex,bookmarks,pdfview=FitB,%backref,
%            pdftitle={Learn R ...as you learnt your mother tongue},%
%            pdfkeywords={R, statistics, data analysis, plotting},%
%            pdfsubject={R},%
%            pdfauthor={Pedro J. Aphalo}%
%            ]{hyperref}

%\hypersetup{colorlinks,breaklinks,
%             urlcolor=blue,
%             linkcolor=blue,
%             citecolor=blue,
%             filecolor=blue,
%             menucolor=blue}

\usepackage{framed}

\usepackage{abbrev}
\usepackage{usingr}

\usepackage{imakeidx}

% this is to reduce spacing above and below verbatim, which is used by knitr
% to show returned values
\usepackage{etoolbox}
\makeatletter
\preto{\@verbatim}{\topsep=-5pt \partopsep=-4pt \itemsep=-2pt}
\makeatother

%%% Adjust graphic design

% New float "example" and corresponding "list of examples"
%\DeclareNewTOC[type=example,types=examples,float,counterwithin=chapter]{loe}
%\DeclareNewTOC[name=Box,listname=List of Text Boxes, type=example,types=examples,float,counterwithin=chapter,%
%]{lotxb}

% changing the style of float captions
%\addtokomafont{caption}{\sffamily\small}
%\setkomafont{captionlabel}{\sffamily\bfseries}
%\setcapindent{0em}

% finetuning tocs
%\makeatletter
%\renewcommand*\l@figure{\@dottedtocline{1}{0em}{2.6em}}
%\renewcommand*\l@table{\@dottedtocline{1}{0em}{2.6em}}
%\renewcommand*\l@example{\@dottedtocline{1}{0em}{2.3em}}
%\renewcommand{\@pnumwidth}{2.66em}
%\makeatother
%
%% add pdf bookmarks to tocs
%\makeatletter
%\BeforeTOCHead{%
%  \cleardoublepage
%    \edef\@tempa{%
%      \noexpand\pdfbookmark[0]{\list@fname}{\@currext}%
%    }\@tempa
%}

\setcounter{topnumber}{3}
\setcounter{bottomnumber}{3}
\setcounter{totalnumber}{4}
\renewcommand{\topfraction}{0.90}
\renewcommand{\bottomfraction}{0.90}
\renewcommand{\textfraction}{0.10}
\renewcommand{\floatpagefraction}{0.70}
\renewcommand{\dbltopfraction}{0.90}
\renewcommand{\dblfloatpagefraction}{0.70}

\addbibresource{rbooks.bib}
\addbibresource{references.bib}

\makeindex[title=General index]
\makeindex[name=rindex,title=Alphabetic index of \Rlang names]
\makeindex[name=rcatsidx,title=Index of \Rlang names by category]
\IfFileExists{upquote.sty}{\usepackage{upquote}}{}
\begin{document}

% customize chapter format:
%\KOMAoption{headings}{twolinechapter}
%\renewcommand*\chapterformat{\thechapter\autodot\hspace{1em}}

% customize dictum format:
%\setkomafont{dictumtext}{\itshape\small}
%\setkomafont{dictumauthor}{\normalfont}
%\renewcommand*\dictumwidth{0.7\linewidth}
%\renewcommand*\dictumauthorformat[1]{--- #1}
%\renewcommand*\dictumrule{}

%\extratitle{\vspace*{2\baselineskip}%
%             {\Huge\textsf{\textbf{Learn R}\\ \textsl{\huge\ldots as you learnt your mother tongue}}}}

\title{\Huge{\fontseries{ub}\sffamily Learn R\\{\Large\ldots as you learnt your mother tongue}}}

%\subtitle{Git hash: \gitAbbrevHash; Git date: \gitAuthorIsoDate}

\author{Pedro J. Aphalo}

\date{Helsinki, \today}

%\publishers{Draft, 95\% done\\Available through \href{https://leanpub.com/learnr}{Leanpub}}

%\uppertitleback{\copyright\ 2001--2017 by Pedro J. Aphalo\\
%Licensed under one of the \href{http://creativecommons.org/licenses/}{Creative Commons licenses} as indicated, or when not explicitly indicated, under the \href{http://creativecommons.org/licenses/by-sa/4.0/}{CC BY-SA 4.0 license}.}
%
%\lowertitleback{Typeset with \href{http://www.latex-project.org/}{\hologo{XeLaTeX}}\ in Lucida Bright and \textsf{Lucida Sans} using the KOMA-Script book class.\\
%The manuscript was written using \href{http://www.r-project.org/}{R} with package knitr. The manuscript was edited in \href{http://www.winedt.com/}{WinEdt} and \href{http://www.rstudio.com/}{RStudio}.
%The source files for the whole book are available at \url{https://bitbucket.org/aphalo/using-r}.}

%\frontmatter

% knitr setup

















% \thispagestyle{empty}
% \titleLL
% \clearpage

\frontmatter

\maketitle

\newpage

%\begin{titlingpage}
%  \maketitle
%\titleLL
%\end{titlingpage}

\setcounter{page}{7} %previous pages will be reserved for frontmatter to be added in later.
\tableofcontents
%\include{frontmatter/foreword}
\chapter*{Preface}

\begin{VF}
``Suppose that you want to teach the `cat' concept to a very young child. Do you explain that a cat is a relatively small, primarily carnivorous mammal with retractible claws, a distinctive sonic output, etc.? I'll bet not. You probably show the kid a lot of different cats, saying `kitty' each time, until it gets the idea. To put it more generally, generalizations are best made by abstraction from experience.''

\VA{R. P. Boas}{Can we make mathematics intelligible?}
\end{VF}

%\dictum[R. P. Boas (1981) Can we make mathematics intelligible?, \emph{American Mathematical Monthly} \textbf{88:} 727-731.]{"Suppose that you want to teach the `cat' concept to a very young child. Do you explain that a cat is a relatively small, primarily carnivorous mammal with retractible claws, a distinctive sonic output, etc.? I'll bet not. You probably show the kid a lot of different cats, saying `kitty' each time, until it gets the idea. To put it more generally, generalizations are best made by abstraction from experience."}


% Such pauses are not a miss use of our time. To learn a natural language we need to interact with other speakers, we need feedback. In the case of R, we can get feedback both from the outcomes from our utterances to the computer, and from other R users.}

\vspace{2ex}This book covers different aspects of the use of \Rpgrm. It is meant to be used as a tutorial complementing a reference book about \R, or the documentation that accompanies R and the many packages used in the examples. Explanations are rather short and terse, so as to encourage the development of a routine of exploration. This is not an arbitrary decision, this is the normal \emph{modus operandi} of most of us who use R regularly for a variety of different problems.

I do not discuss here statistics, just \Rpgrm as a tool and language for data manipulation and display. The idea is for you to learn the \Rpgrm language like children learn a language: they work-out what the rules are, simply by listening to people speak and trying to utter what they want to tell their parents. Instead of listening, you will read and execute on a computer \Rlang code statements, try your hand at telling \Rlang what you want it to compute. I do provide explanations and comments, but the idea of these book is mainly for you to use the numerous examples to find-out by yourself the overall patterns and coding philosophy behind the \Rlang language. Instead of parents being the sound board for your first utterances in \langname{R}, the computer will play this role. You will \emph{play} by modifying the examples, see how the computer responds, does \Rlang understand you or not?

When teaching I tend to lean towards challenging students rather than telling a simplified story. I do the same here, because it is what I prefer as a student, and how I learn best myself. Not everybody learns best with the same approach, for me the most limiting factor is for what I listen to, or read, to be in a way or another challenging or entertaining enough to keep my thoughts focused. This I achieve best when making an effort to understand the contents or to follow the thread or plot of a story. So, be warned, reading this book will be about exploring a new world, this book aims to be a travel guide, neither a traveler's account, nor a cookbook of R recipes.

Keep in mind that it is impossible to remember everything about \Rpgrm! \Rpgrm in a broad sense is vast because its capabilities can be expanded with independently developed packages. Learning to use \Rlang consists in learning the basics plus developing the skill of finding your way in \Rlang and its documentation.  In 2017 the number packages available for free in the Comprehensive R Archive Network (CRAN) broke the 10\,000 barrier. CRAN is the most important, but not only, public repository for R packages. How good a command of the \Rlang language and packages a user needs depends on the type activities to be carried out. This book attempts to train you in the use of the \Rlang language itself and some packages that provide extensions for data manipulation and graphical display which are broadly useful. Given the availability of numerous books on statistical analysis with \Rlang, here we will cover only the bare minimum. The same is true for package development in \Rlang. This book seats in-between, aiming at teaching programming in-the-small: the use \Rlang to automate the drudgery of data manipulation from raw data, through data exploration to the production of publication quality illustrations.

As with all ``rich'' languages there are many different ways of doing things in R, and there is no one-size-fits-all solution to a problem. There is always a compromise involved, usually between time spent by the user and processing time required in the computer. Many of the packages that are most popular nowadays did not exist when I started using R, and many of these packages make new approaches available. One could write many different \Rlang books with a given aim and still use substantially different ways of achieving the same results. In this book, I limit myself to packages that are currently popular and/or that I consider elegantly designed. I have in particular tried to limit myself to packages with similar design philosophies, especially in relation to their interfaces. What is elegant design, and in particular what is a friendly user interface depends strongly on each user's preferences and previous experience. Consequently, the contents of the book are strongly biased by my own preferences. I have tried to write examples in ways that execute fast without compromising readability. I encourage readers to take this book as a travel guide, as a starting point for exploring the very many packages, styles and approaches which I have not described.

I will appreciate suggestions for further examples, notification of errors and unclear sections. Many of the examples here have been collected from diverse sources over many years and because of this not all sources are acknowledged. If you recognize any example as yours or someone else's please let me know so that I can add a proper acknowledgement. I warmly thank the students that over the years have asked the questions and posed the problems that have helped me write this text and correct the mistakes and voids of previous versions. I have also received help on on-line forums and in person from numerous people, learnt from archived e-mail list messages, blog posts, books, articles, tutorials, webinars, and by struggling to solve some new problems on my own. In many ways this text owes much more to people who are not authors than to myself. However, as I am the one who has written this version and decided what to include and exclude, as author, I take full responsibility for any errors and inaccuracies.

I have been using \Rpgrm since around 1998 or 1999, but I am still constantly learning new things about \Rpgrm itself and \Rpgrm packages. With time it has replaced in my work as a researcher and teacher several other pieces of software: \pgrmname{SPSS}, \pgrmname{Systat}, \pgrmname{Origin}, \pgrmname{Excel}, and it has become a central piece of the tool set I use for producing lecture slides, notes, books and even web pages. This is to say that it is the most useful piece of software and programming language I have ever learnt to use. Of course, in time it will be replaced by something better, but at the moment it is the ``hot'' thing to learn for anybody with a need to analyse and display data.

\begin{framed}
\noindent\large%
\textbf{I encourage you to approach R, like a child approaches his or hers mother tongue when learning to speak:} Do not struggle, just play! If going gets difficult and frustrating, take a break! If you get a new insight, take a break to enjoy the victory!
\end{framed}

\newpage

\begin{framed}
\noindent
\textbf{Icons used to mark different content.} Throughout the book text boxes marked with icons present different types of information. First of all, we have \emph{playground} boxes indicated with \playicon\ which contain open-ended exercises---ideas and pieces of R code to play with at the R console. A few of these will require more time to grasp, and are indicated with \advplayicon. Boxes providing general information, usually not directly related to \langname{R} as a language, are indicated with \infoicon. Some boxes highlighted with \ilAttention\ give important bits of information that must be remembered when using \langname{R}---i.e.\ explain some unusual feature of the language. Finally, some boxes indicated by \ilAdvanced\ give in depth explanations, that may require you to spend time thinking, which en general can be skipped on first reading, but to which you should return at a later, and peaceful, time with a cup of coffee or tea.
\end{framed}
\newpage

%\newpage
%\begin{infobox}
%\noindent
%\textbf{Status as of 2016-11-23.} I have updated the manuscript to track package updates since the previous version uploaded six months ago, and added several examples of the new functionality added to packages \ggpmisc, \ggrepel, and \ggplot. I have written new sections on packages \viridis, \pkgname{gganimate}, \pkgname{ggstance}, \pkgname{ggbiplot}, \pkgname{ggforce}, \pkgname{ggtern} and \pkgname{ggalt}. Some of these sections are to be expanded, and additional sections are planned for other recently released packages.
%
%With respect to the chapter \textit{Storing and manipulating data with R} I have put it on hold, except for the introduction, until I can see a soon to be published book covering the same subject. Hadley Wickham has named the set of tools developed by him and his collaborators as \textit{tidyverse} to be described in the book titled \textit{R for Data Science} by Grolemund and Wickham (O'Reilly).
%
%An important update to \ggplot was released last week, and it includes changes to the behavior of some existing functions, specially faceting has become extensible through other packages. Several of the new facilities are described in the updated text and code included in this book and this pdf has been generated with up-to-date version of \ggplot and packages as available today from CRAN, except for \pkgname{ggtern} which was downloaded from Bitbucket minutes ago.
%
%The present update adds about 100 pages to the previous versions. I expect to upload a new update to this manuscript in one or two months time.
%
%\textbf{Status as of 2017-01-17.} Added ``playground'' exercises to the chapter describing \ggplot, and converted some of the examples earlier part of the main text into these playground items. Added icons to help readers quickly distinguish playground sections (\textcolor{blue}{\noticestd{"0055}}), information sections (\textcolor{blue}{\modpicts{"003D}}), warnings about things one needs to be specially aware of (\colorbox{yellow}{\typicons{"E136}}) and boxes with more advanced content that may require longer time/more effort to grasp (\typicons{"E04E}). Added to the sections \code{scales} and examples in the \ggplot chapter details about the use of colors in R and \ggplot2. Removed some redundant examples, and updated the section on \code{plotmath}. Added terms to the alphabetical index. Increased line-spacing to avoid uneven spacing with inline code bits.
%
%\textbf{Status as of 2017-02-09.} Wrote section on ggplot2 themes, and on using system- and Google fonts in ggpplots with the help of package \pkgname{showtext}. Expanded section on \ggplot's \code{annotation}, and revised some sections in the ``R scripts and Programming'' chapter. Started writing the data chapter. Wrote draft on writing and reading text files. Several other smaller edits to text and a few new examples.
%
%\textbf{Status as of 2017-02-14.} Wrote sections on reading and writing MS-Excel files, files from statistical programs such as SPSS, SyStat, etc., and NetCDF files. Also wrote sections on using URLs to directly read data, and on reading HTML and XML files directly, as well on using JSON to retrieve measured/logged data from IoT (internet of things) and similar intelligent physical sensors, micro-controller boards and sensor hubs with network access.
%
%\textbf{Status as of 2017-03-25.} Revised and expanded the chapter on plotting maps, adding a section on the manipulation and plotting of image data. Revised and expanded the chapter on extensions to \pkgname{ggplot2}, so that there are no longer empty sections. Wrote short chapter ``If and when R needs help''. Revised and expanded the ``Introduction'' chapter. Added index entries, and additional citations to literature.
%
%\textbf{Status as of 2017-04-04.} Revised and expanded the chapter on using \Rpgrm as a calculator. Revised and expanded the ``Scripts'' chapter. Minor edits to ``Functions'' chapter. Continued writing chapter on data, writing a section on R's native apply functions and added preliminary text for a pipes and tees section. Write intro to `tidyverse' and grammar of data manipulation. Added index entries, and a few additional citations to the literature. Spell checking.
%
%\textbf{Status as of 2017-04-08.} Completed writing first draft of chapter on data, writing all the previously missing sections on the ``grammar of data manipulation''. Wrote two extended examples in the same chapter. Add table listing several extensions to \pkgname{ggplot2} not described in the book.
%
%\textbf{Status as of 2017-04-13.} Revised all chapters correcting some spelling mistakes, adding some explanatory text and indexing all functions and operators used. Thoroughly revised the Introduction chapter and the Preface. Expanded section on bar plots (now bar and column plots). Revised section on tile plots. Expanded section on factors in chapter 2, adding examples of reordering of factor labels, and making clearer the difference between the labels of the levels and the levels themselves.
%
%\textbf{Status as of 2017-04-29.} Tested with R 3.4.0. Package \pkgname{gganimate} needs to be installed from Github as the updated version is not yet in CRAN. Function \code{gg\_animate()} has been renamed \code{gganimate().}
%
%\textbf{Status as of 2017-05-14.} Submitted package \pkgname{learnrbook} to CRAN. Revised code in the book
%to use this new package. Small fixes after more testing. Added examples of plotting and labeling based on fits with \code{method = "nls"}, including use of the new \code{ggpmisc::stat\_fit\_tidy()}.
%
%\textbf{Status as of 2017-06-11.} Added sections on R-code bench marking and profiling for performance optimization. Added also an example of explicit compilation of a function defined in the R language. Added section on functions \code{assign()}, \code{get()} and \code{mget()}.
%
%\textbf{Status as of 2017-08-12.} Various edits to all chapters. Expanded section on \pkgname{ggpmisc} to include the new functionality added in version 0.2.15.9002: \code{geom\_table} and \code{stat\_fit\_tb}. Added section on package \pkgname{ggbeeswarm}. Added sections on packages \pkgname{magick} and on using \pgrmname{ImageJ} from \Rpgrm. Improved indexing and cross references.
%
%\textbf{Status as of 2017-10-25.} Edited the chapter on using R as a calculator, adding examples on insertion and deletion of members of lists and vectors, and also of use of \code{gl()} and \code{reorder()}. Edited sections on scale limits and added new section on coordinate limits to explain more thoroughly their differences and uses in chapter on plotting with \pkgname{ggplot2}. Added a section on package \pkgname{ggsignif} to the chapter on extensions to \pkgname{ggplot2}. Expanded section on \pkgname{ggpmisc} in the same chapter describing new functionality added in version 0.2.16.
%\pkgname{ggplo2} $>=$ 2.2.1.9000 is required by the current development version of \pkgname{ggpmisc}.
%
%\textbf{Status as of 2017-10-30.}  Add section on using pipes with \code{ggplot()} and layers.
%\end{infobox} 
\listoffigures
\listoftables
%\include{frontmatter/contributor}
%\include{frontmatter/symbollist}

\mainmatter







% !Rnw root = appendix.main.Rnw



\chapter{The R language: ``paragraphs'' and ``essays''}\label{chap:R:scripts}
\index{scripts}

\begin{VF}
An \Rlang script is simply a text file containing (almost) the same commands that you would enter on the command line of R.

\VA{Jim Lemon}{Kickstarting R}
\end{VF}

%\dictum[\href{https://cran.r-project.org/doc/contrib/Lemon-kickstart/}{Kickstarting R}]{An R script is simply a text file containing (almost) the same commands that you would enter on the command line of R.}\vskip2ex

\section{Aims of this chapter}

In my experience, for those who have mainly used graphical user interfaces, understanding why and when scripts can help in communicating a certain data analysis protocol can be revelatory. As soon as a data analysis stops being trivial, describing the steps followed through a system of menus and dialogue boxes becomes extremely tedious.

It is also usually the case that graphical user interfaces tend to be difficult to extend or improve in a way that keeps step-by-step instructions valid across program versions and operating systems.

Many times exactly the same sequence of commands needs to be applied to different data sets, and scripts make both implementation and validation of such a requirement easy.

In this chapter I will walk you through the use of \Rpgrm scripts, starting from a extremely simple script.

\section{Writing scripts}

In \Rlang, the closest to a natural language essay is a script. A script is built from multiple interconnected code statements. Simple statements can be combined into compound statements, which are the equivalent of natural language paragraphs. Scripts can vary from simple scripts containing only a few code statements, to complex scripts containing hundreds of code statements. In the rest of the present section I discuss how to write readable and reliable scripts and how are they used.

\subsection{What is a script?}\label{sec:script:what:is}
\index{scripts!definition}
We call \textit{script} to a text file that contains (almost) the same commands that you would type at the console prompt. A true script is not for example an MS-Word file where you have pasted or typed some \Rlang commands. A script file has the following characteristics.
\begin{itemize}
  \item The script is a text file (ASCII or some other encoding e.g.\ UTF-8 that \Rpgrm uses in your locale).
  \item The file contains valid \Rlang statements (including comments) and nothing else.
  \item Comments start at a \code{\#} and end at the end of the line. (True end-of line as coded in the file, the editor may wrap lines or not at the edge of the screen).
  \item The \Rlang statements are in the file in the order that they must be executed.
  \item \Rlang scripts have file names ending in \texttt{.r} or \texttt{.R}.
\end{itemize}

It is good practice to write scripts so that are self-contained. To make a script self-contained, one must include calls to \texttt{library()} to load the packages used in addition to all the data-analysis commands. Such scripts can be used to generate the output of the analysis and/or to reproduce an earlier analysis.



\subsection{How do we use a scrip?}\label{sec:script:using}
\index{scripts!sourcing}

A script can be ``sourced'' using function \Rfunction{source()}. If we have a text file called \texttt{my.first.script.r} containing the following text:
\begin{shaded}
\footnotesize
\begin{verbatim}
# this is my first \Rlang script
print(3 + 4)
\end{verbatim}
\end{shaded}

And then source this file:

\begin{knitrout}\footnotesize
\definecolor{shadecolor}{rgb}{0.969, 0.969, 0.969}\color{fgcolor}\begin{kframe}
\begin{alltt}
\hlkwd{source}\hlstd{(}\hlstr{"my.first.script.r"}\hlstd{)}
\end{alltt}
\begin{verbatim}
## [1] 7
\end{verbatim}
\end{kframe}
\end{knitrout}

The results of executing the statements contained in the file will appear in the console. The commands themselves are not shown (by default the sourced file is not \emph{echoed} to the console) and the results will not be printed unless you include explicit \Rfunction{print()} commands in the script. This applies in many cases also to plots---e.g.\ A figure created with \Rfunction{ggplot()} needs to be printed if we want it to be included in the output when the script is run. Adding a redundant \Rfunction{print()} is harmless.

From within \RStudio, if you have an \Rpgrm script open in the editor, there will be a ``source'' icon visible with an attached drop-down menu from where you can choose ``Source'' as described above, or ``Source with echo'', or ``Source as local job'' for the script in the currently active editor tab.

When a script is \emph{sourced}, the output can be saved to a text file instead of being shown in the console. It is also easy to call \Rpgrm with the script file as argument directly at the operating system shell or command-interpreter prompt---and obviously also shell scripts.
\begin{shaded}
\footnotesize
\begin{verbatim}
RScript my.first.script.r
\end{verbatim}
\end{shaded}

You can open an operating system's \emph{shell} from the Tools menu in \RStudio, to run this command. The output will be printed to the shell console. If you would like to save the output to a file, use redirection using the operating system's syntax.
\begin{shaded}
\footnotesize
\begin{verbatim}
RScript my.first.script.r > my.output.txt
\end{verbatim}
\end{shaded}

Sourcing is very useful when the script is ready, however, while developing a script, or sometimes when testing things, one usually wants to run (or \emph{execute}) one or a few statements at a time. This can be done using the ``run'' button\footnote{If you use a different IDE or editor with an \Rlang mode, the details will vary, but a run command will be usually available.} after either positioning the cursor in the line to be executed, or selecting the text that one would like to run (the selected text can be part of a line, a whole line, or a group of lines, as long as it is syntactically valid). The key-shortcut Ctrl-Enter is equivalent to pressing the ``run'' button in \RStudio.

\subsection{How to write a script?}\label{sec:script:writing}
\index{scripts!writing}

As with any type of writing various approaches may be preferred by different persons. In general, the approach used, or mix of approaches will also depend on how confident you are that the statements will work as expected---you already know the best approach vs.\ you are exploring different alternatives.
\begin{description}
\item[If one is very familiar with similar problems] One would just create a new text file and write the whole thing in the editor, and then test it. This is rather unusual.
\item[If one is moderately familiar with the problem] One would write the script as above, but testing it, step by step as one is writing it. This is usually what I do.
\item[If one is mostly playing around] Then if one is using \RStudio, one can type statements at the console prompt. As you should know by now, everything you run at the console is saved to the ``History''. In \RStudio the History is displayed in its own pane, and in this pane one can select any previous statement(s) and by pressing a single button copy and paste them to either the \Rlang console prompt, or the cursor position in the editor pane. In this way one can build a script by copying and pasting from the history to your script file the bits that have worked as you wanted.
\end{description}

\begin{playground}
By now you should be familiar enough with \Rlang to be able to write your own script.
\begin{enumerate}
  \item Create a new \Rpgrm script (in \RStudio, from `File' menu, ``+'' button, or by typing ``Ctrl + Shift + N'').
  \item Save the file as \texttt{my.second.script.r}.
  \item Use the editor pane in \RStudio to type some \Rpgrm commands and comments.
  \item \emph{Run} individual commands.
  \item \emph{Source} the whole file.
\end{enumerate}
\end{playground}

\subsection{The need to be understandable to people}\label{sec:script:readability}
\index{scripts!readability}

When you write a script, it is either because you want to document what you have done or you want re-use the script at a later time. In either case, the script itself although still meaningful for the computer could become very obscure to you, and even more to someone seeing it for the first time. This must be avoided by spending time and effort on the writing style.

How does one achieve an understandable script or program?
\begin{itemize}
  \item Avoid the unusual. People using a certain programming language tend to use some implicit or explicit rules of style\footnote{Style includes \textit{indentation} of statements, \textit{capitalization} of variable and function names.}. As a minimum try to be consistent with yourself.
  \item Use meaningful names for variables, and any other object. What is meaningful depends on the context. Depending on common use a single letter may be more meaningful than a long word. However self explanatory names are usually better: e.g.\ using \code{n.rows} and \code{n.cols} is much clearer than using \code{n1} and \code{n2} when dealing with a matrix of data. Probably \code{number.of.rows} and \code{number.of.columns} would make the script verbose, take longer to type without gaining anything in return.
  \item How to make the words visible in names: traditionally in \Rlang one would use dots to separate the words and use only lower case. Some years ago, it became possible to use underscores. The use of underscores is quite common nowadays because in some contexts it is ``safer'' as in some situations a dot may have a special meaning. What we call ``camel case'' is only infrequently used in \Rlang programming but is common in other languages like Pascal. An example of camel case is \code{NumCols}. In some cases it can become a bit confusing as in \code{UVMean} or \code{UvMean}.
\end{itemize}

\begin{playground}
Here is an example of bad style in a script. Read \href{https://google.github.io/styleguide/Rguide.xml}{Google's R Style Guide}\footnote{This is just an example, similar, but not exactly the same style as the style I use myself.}, and edit the code in the chunck below so that it becomes easier to read.

\begin{knitrout}\footnotesize
\definecolor{shadecolor}{rgb}{0.969, 0.969, 0.969}\color{fgcolor}\begin{kframe}
\begin{alltt}
\hlstd{a} \hlkwb{<-} \hlnum{2} \hlcom{# height}
\hlstd{b} \hlkwb{<-} \hlnum{4} \hlcom{# length}
\hlstd{C} \hlkwb{<-}
    \hlstd{a} \hlopt{*}
\hlstd{b}
\hlstd{C} \hlkwb{->} \hlstd{variable}
      \hlkwd{print}\hlstd{(}
\hlstr{"area: "}\hlstd{, variable}
\hlstd{)}
\end{alltt}
\end{kframe}
\end{knitrout}
\end{playground}

The points discussed above already help a lot. However, one can go further in achieving the goal of human readability by interspersing explanations and code ``chunks'' and using all the facilities of typesetting, even of formatted maths formulas and equations, within the listing of the script. Furthermore, by including the results of the calculations and the code itself in a typeset report built automatically, we ensure that the results are indeed the result of running the code shown. This greatly contributes to data analysis reproducibility, which is becoming a widespread requirement for any data analysis both in academic research and in industry. It is possible not only to build whole books like this one, but also whole data-based web sites with these tools.

In the realm of programming, this approach is called literate programming\index{literate programming} and was first proposed by Donald Knuth \autocite{Knuth1984a} through his \pgrmname{WEB} system. In the case of \Rpgrm programming the first support of literate programming was through \pkgname{Sweave}, which has been mostly superseded by \pkgname{knitr} \autocite{Xie2013}. This package supports the use of Markdown or \LaTeX\ \autocite{Lamport1994} as markup language for the textual contents, and also can format and add syntax highlighting to code chunks. Markdown\index{Markdown}\index{Bookdown} language has been extended to make it easier to include \Rlang code---R Markdown (\url{http://rmarkdown.rstudio.com/}), and in addition suitable for typesetting large and complex documents (Bookdown), web sites (Blogdown), package vignettes (Pkgdown) \autocite{Xie2016,Xie2018}. The use of \pkgname{knitr} is well integrated into the \RStudio IDE.

This is not strictly an \Rlang programming subject, as it concerns programming in any language. On the other hand, this is an incredibly important skill to learn, but well described in other books and web sites cited in the previous paragraph. This whole book, including figures, has been generated using \pkgname{knitr} and the source scripts for the book are available through Bitbucket at \url{https://bitbucket.org/aphalo/learnr-book}.

\subsection{Debugging scripts}\label{sec:script:debug}
\index{scripts!debugging}

The use of the word \emph{bug} to describe a problem in computer hardware and software started in 1946 when a real bug, more precisely a moth, got between the contacts of a relay in an electromechanical computer causing it to malfunction and Grace Hooper described the first computer \emph{bug}. The use of the term bug in engineering predates the use in computer science, and consequently the first use of bug in computing did catch-on easily as it was memorable as it represented an earlier-used metaphor becoming real.

A suitable quotation from a letter written by Thomas Alva Edison 1878 (\autocite{Hughes2004}):
\begin{quote}
  It has been just so in all of my inventions. The first step is an intuition, and comes with a burst, then difficulties arise--this thing gives out and [it is] then that ``Bugs''--as such little faults and difficulties are called--show themselves and months of intense watching, study and labor are requisite before commercial success or failure is certainly reached.
\end{quote}

The quoted paragraph above, makes clear, that only very exceptionally any new design fully succeeds. The same applies to \Rlang scripts as well as any other non-trivial piece of computer code. From this it logically follows that testing and de-bugging are fundamental steps in the development of \Rlang scripts and packages. Debugging, as an activity, is outside the scope of this book. However, clear programming style and good documentation are indispensable for efficient testing and reuse.

Even for scripts used for analysing a single data set, we need to be confident that the algorithms and their implementation are valid, and able to return correct results. This is true both for scientific reports, expert data-based reports and any data analysis related to assessment of compliance with legislation or regulations. Of course, even in cases when we are not required to demonstrate validity, say for decision making purely internal to a private organization, we will still want to avoid costly mistakes.

The first step in producing reliable computer code is to accept that any code that we write needs to be tested and, if possible, validated. Another important step is to make sure that input is validated within the script and a suitable error produced for bad input (including valid input values falling outside the range that can be reliably handled by the script).

If during testing, or during normal use, a wrong value is returned by a calculation, or no value (e.g.\ the script crashes or triggers a fatal error), debugging consists in finding the cause of the problem. The cause can be either a mistake in the implementation of an algorithm, as well as in the algorithm itself. However, many apparent \emph{bugs} are caused by bad or missing handling of special cases like invalid input values, rounding errors, division by zero, etc.\ in which a program crashes instead of elegantly issuing a helpful error message.

Diagnosing the source of bugs is in most cases like detective work. One uses hunches based on common sense and experience to try to locate the lines of code causing the problem. One follows different \emph{leads} until the case is solved. In most cases at the very bottom we rely on some sort of divide and conquer strategy. For example, we may check the value returned by intermediate calculations until we locate the earliest code statement producing a wrong value. Another common case is when some input values trigger a bug. In such cases it is frequently best to start by testing if different ``cases'' of input lead to errors/crashes or not. Boundary input values are usually the telltale ones: e.g.\ for numbers, zero, negative and positive values, very large values, very small values, missing values (\code{NA}), vectors of length zero (\code{numeric()}), etc.

\begin{warningbox}
  \textbf{Error messages} When debugging keep in mind that in some cases a single bug can lead to a whole cascade of error messages. Do also keep in mind that typing mistakes, originating when code is entered through the keyboard, can break havock in a script: usually there is little correspondence between the number of error messages and the seriousness of the bug triggering them. When several errors are triggered, start by reading the error message printed first, as later errors can be an indirect consequence of earlier ones.
\end{warningbox}

There are special tools, called debuggers, available, and they help enormously. Debuggers allow one to step through the code, executing one statement at a time and at each pause allowing the user to inspect the objects present in the \Rlang environment and their values. It is even possible to execute additional statements, say to modify the value of a variable, while execution is paused. An \Rlang debugger is available within \RStudio and also through the \Rlang console.

When writing your first scripts, you will manage perfectly well, and learn more by running the script one line at a time and when needed temporarily inserting \code{print()} statements to ``look'' at how the value of variables changes at each step. A debugger, allows a lot more control, as one can ``step in'' and ``step out'' of functions definitions, set and unset break points where execution will stop, which is especially useful when developing \Rlang packages.

When reproducing the examples in this chapter, do keep this section in mind. In addition, if you get stuck trying to find the cause of a bug, do extend your search both to the most trivial of possible causes, and to the least likely ones (such as a bug in a package installed from CRAN or \Rlang itself). Of course, when suspecting a bug in code you have not written, it is wise to very carefully read the documentation, as the ``bug'' may be just in your understanding of what a certain piece of code is expected to do.  Also remember, as discussed on page \pageref{sec:getting:help}, you will find on-line ready-answersed questions to many of your likely problems and doubts. For example Googling for the text of an error message is usually well rewarded. Failing this, you can ask questions at, e.g., StackOverflow (\url{https://stackoverflow.com/}), after searching the site for possible existing answers.

\begin{warningbox}
When installing packages from other sources than CRAN (e.g.\ development versions from Github, Bitbucket or Rforge, or in-house packages) there is no warranty that conflicts will not happen. Packages (and their versions) released through CRAN are regularly checked for inter-compatibility, while packages released through other channels are usually checked against only a few packages.

Conflicts among packages can easily arise, for example when they use the same names for objects or functions. In addition, many packages use functions defined in packages in the \Rlang distribution itself or other independently developed packages by importing them. Updates to depended upon packages can ``break'' (make non-functional) the dependent packages or parts of them. The rigorous testing by CRAN detects in most cases such problems when package revisions are submitted, forcing package maintainers to fix problems before distribution through CRAN is possible. However, if you use other repositories, I recommend that you make sure that revised (especially if under development) versions do work with your own script, before their use in ``production'' (important) data analyses.
\end{warningbox}


\section{Control of execution flow}\label{sec:script:flow:control}
\index{control of execution flow}
We give the name \emph{control of execution statements} to those statements that allow the execution of sections of code when a certain dynamically computed condition is \code{TRUE}. Some of the control of execution flow statements, function like \emph{ON-OFF switches} for program statements. Others, allow statements to executed repeatedly while or until a condition is met, or until all members of a list or a vector are processed.

These \emph{control of execution statements} can be also used at the \Rlang console, but it usually awkward to do so as they can extend over several lines of text. In simple scripts the \emph{flow of execution} can be fixed and linear from the first to the last statement in the script. \emph{Control of execution statements} allow flexibility, as they allow conditional execution  and/or repeated execution of statements. The part of the script conditionally executed can be a simple or a compound code statement providing a lot of flexibility. As we will see next, a compound statement can include multiple simple or nested compound statements.

\subsection{Coumpound statements}
\index{compound code statements}\index{simple code statements}

\Rpgrm has two types of \emph{if}\index{if} statements, non-vectorized and vectorized. We will start with the non-vectorized one, which is similar to what is available in most other computer programming languages.

Before this we need to explain compound statements. Individual statements can be grouped into compound statements by enclosed them in curly braces.

\begin{knitrout}\footnotesize
\definecolor{shadecolor}{rgb}{0.969, 0.969, 0.969}\color{fgcolor}\begin{kframe}
\begin{alltt}
\hlkwd{print}\hlstd{(}\hlstr{"A"}\hlstd{)}
\end{alltt}
\begin{verbatim}
## [1] "A"
\end{verbatim}
\begin{alltt}
\hlstd{\{}
  \hlkwd{print}\hlstd{(}\hlstr{"B"}\hlstd{)}
  \hlkwd{print}\hlstd{(}\hlstr{"C"}\hlstd{)}
\hlstd{\}}
\end{alltt}
\begin{verbatim}
## [1] "B"
## [1] "C"
\end{verbatim}
\end{kframe}
\end{knitrout}

The grouping of the last two statements above is of no consequence by itself, but grouping becomes useful when used together with `control' constructs.

\subsection{Conditional execution}
\index{conditional execution}

Conditional execution allows handling different values, such as negative and non-negative values, differently in a script. It also allows switching ON and OFF parts of a script using a so-called \emph{flag} that can be manually set, preferable near the top of the script. Use of flags is specially important when switching between two ``modes'' affects multiple sections of code. A frequent use case is enabling and disabling printing of output from multiple statements scattered in over long script. Below I start by giving simple examples demonstrating how \emph{if} and \emph{if-else} statements work. Only afterwards I provide examples closer to how these statements are profitably used in \Rlang scripts.

\subsubsection[Non-vectorized \texttt{if}, \texttt{else} and \texttt{switch}]{Non-vectorized \code{if}, \code{else} and \code{switch}}
\qRcontrol{if()}\qRcontrol{if()\ldots else}%

The \code{if} construct ``decides'' depending on a \code{logical} value whether the next code statement is executed (if \code{TRUE}) or skipped (if \code{FALSE}).

\begin{knitrout}\footnotesize
\definecolor{shadecolor}{rgb}{0.969, 0.969, 0.969}\color{fgcolor}\begin{kframe}
\begin{alltt}
\hlstd{flag} \hlkwb{<-} \hlnum{TRUE}
\hlkwa{if} \hlstd{(flag)} \hlkwd{print}\hlstd{(}\hlstr{"Hello!"}\hlstd{)}
\end{alltt}
\begin{verbatim}
## [1] "Hello!"
\end{verbatim}
\end{kframe}
\end{knitrout}

\begin{playground}
Play with the code above by changing the value assigned to variable \code{flag}, \code{FALSE}, \code{NA}, and \code{logical(0)}.

In the example above we use variable \code{flag} as the \emph{condition}.

Nothing in the \Rlang language prevents this condition to be a \code{logical} constant. Explain why this use case is of no practical use.

\begin{knitrout}\footnotesize
\definecolor{shadecolor}{rgb}{0.969, 0.969, 0.969}\color{fgcolor}\begin{kframe}
\begin{alltt}
\hlkwa{if} \hlstd{(}\hlnum{TRUE}\hlstd{)} \hlkwd{print}\hlstd{(}\hlstr{"Hello!"}\hlstd{)}
\end{alltt}
\begin{verbatim}
## [1] "Hello!"
\end{verbatim}
\end{kframe}
\end{knitrout}
\end{playground}

Conditional execution is much more useful than what could be expected from the previous example, because the statement whose execution is being controlled can be a compound statement of almost any length or complexity. A very simple example follows.

\begin{knitrout}\footnotesize
\definecolor{shadecolor}{rgb}{0.969, 0.969, 0.969}\color{fgcolor}\begin{kframe}
\begin{alltt}
\hlstd{printing} \hlkwb{<-} \hlnum{TRUE}
\hlkwa{if} \hlstd{(printing) \{}
  \hlkwd{print}\hlstd{(}\hlstr{"A"}\hlstd{)}
  \hlkwd{print}\hlstd{(}\hlstr{"B"}\hlstd{)}
\hlstd{\}}
\end{alltt}
\begin{verbatim}
## [1] "A"
## [1] "B"
\end{verbatim}
\end{kframe}
\end{knitrout}

The condition passed as argument to \code{if}, enclosed in parentheses, can be anything yielding a \Rclass{logical} vector, however, as this condition is \emph{not} vectorized, only the first element will be used and a warning issued if longer than one.

\begin{knitrout}\footnotesize
\definecolor{shadecolor}{rgb}{0.969, 0.969, 0.969}\color{fgcolor}\begin{kframe}
\begin{alltt}
\hlstd{a} \hlkwb{<-} \hlnum{10.0}
\hlkwa{if} \hlstd{(a} \hlopt{<} \hlnum{0.0}\hlstd{)} \hlkwd{print}\hlstd{(}\hlstr{"'a' is negative"}\hlstd{)} \hlkwa{else} \hlkwd{print}\hlstd{(}\hlstr{"'a' is not negative"}\hlstd{)}
\end{alltt}
\begin{verbatim}
## [1] "'a' is not negative"
\end{verbatim}
\begin{alltt}
\hlkwd{print}\hlstd{(}\hlstr{"This is always printed"}\hlstd{)}
\end{alltt}
\begin{verbatim}
## [1] "This is always printed"
\end{verbatim}
\end{kframe}
\end{knitrout}

As can be seen above, the statement immediately following \code{if} is executed if the condition returns \code{TRUE} and that following \code{else} is executed if the condition returns \code{FALSE}. Statements after the conditionally executed \code{if} and \code{else} statements are always executed, independently of the value returned by the condition.

\begin{playground}
Play with the code in the chunk above by assigning different numeric vectors to \code{a}.
\end{playground}



\begin{explainbox}
Do you still remember the rules about continuation lines?

\begin{knitrout}\footnotesize
\definecolor{shadecolor}{rgb}{0.969, 0.969, 0.969}\color{fgcolor}\begin{kframe}
\begin{alltt}
\hlcom{# 1}
\hlstd{a} \hlkwb{<-} \hlnum{1}
\hlkwa{if} \hlstd{(a} \hlopt{<} \hlnum{0.0}\hlstd{)} \hlkwd{print}\hlstd{(}\hlstr{"'a' is negative"}\hlstd{)} \hlkwa{else} \hlkwd{print}\hlstd{(}\hlstr{"'a' is not negative"}\hlstd{)}
\end{alltt}
\begin{verbatim}
## [1] "'a' is not negative"
\end{verbatim}
\end{kframe}
\end{knitrout}

Why does the statement below (not evaluated here) trigger an error while the one above does not?

\begin{knitrout}\footnotesize
\definecolor{shadecolor}{rgb}{0.969, 0.969, 0.969}\color{fgcolor}\begin{kframe}
\begin{alltt}
\hlcom{# 2 (not evaluated here)}
\hlkwd{if} (a < 0.0) \hlkwd{print}(\hlstr{"\hlstr{'a'} is negative"})
else \hlkwd{print}(\hlstr{"\hlstr{'a'} is not negative"})
\end{alltt}
\end{kframe}
\end{knitrout}

How do the continuation line rules apply when we add curly braces as shown below.

\begin{knitrout}\footnotesize
\definecolor{shadecolor}{rgb}{0.969, 0.969, 0.969}\color{fgcolor}\begin{kframe}
\begin{alltt}
\hlcom{# 1}
\hlstd{a} \hlkwb{<-} \hlnum{1}
\hlkwa{if} \hlstd{(a} \hlopt{<} \hlnum{0.0}\hlstd{) \{}
    \hlkwd{print}\hlstd{(}\hlstr{"'a' is negative"}\hlstd{)}
  \hlstd{\}} \hlkwa{else} \hlstd{\{}
    \hlkwd{print}\hlstd{(}\hlstr{"'a' is not negative"}\hlstd{)}
  \hlstd{\}}
\end{alltt}
\begin{verbatim}
## [1] "'a' is not negative"
\end{verbatim}
\end{kframe}
\end{knitrout}

In the example above we enclosed a single statement between each pair of curly braces, but as these braces create compound statements, multiple statements could have been enclosed between each pair.
\end{explainbox}

\begin{playground}
Play with the use conditional execution, with both simple and compound statements, and also think how to combine \code{if} and \code{else} to select among more than two options.
\end{playground}

The value returned by any compound statement is in \Rlang the value returned by the last simple statement executed within the compound one. This means that we can assign the value returned by an \code{if} and \code{else} statement to a variable. This style is less frequently used, but occasionally can result in easier to understand scripts.

\begin{knitrout}\footnotesize
\definecolor{shadecolor}{rgb}{0.969, 0.969, 0.969}\color{fgcolor}\begin{kframe}
\begin{alltt}
\hlstd{a} \hlkwb{<-} \hlnum{1}
\hlstd{my.message} \hlkwb{<-}
  \hlkwa{if} \hlstd{(a} \hlopt{<} \hlnum{0.0}\hlstd{)} \hlstr{"'a' is negative"} \hlkwa{else} \hlstr{"'a' is not negative"}
\hlkwd{print}\hlstd{(my.message)}
\end{alltt}
\begin{verbatim}
## [1] "'a' is not negative"
\end{verbatim}
\end{kframe}
\end{knitrout}

\begin{playground}
Study the conversion rules between \Rclass{numeric} and \Rclass{logical} values, run each of the statements below, and explain the output based on how type conversions are interpreted, remembering the difference between \emph{floating-point numbers} as implemented in computers and \emph{real numbers} ($\mathbb{R}$) as defined in mathematics.

% chunk contains intentional error-triggering examples
\begin{knitrout}\footnotesize
\definecolor{shadecolor}{rgb}{0.969, 0.969, 0.969}\color{fgcolor}\begin{kframe}
\begin{alltt}
\hlkwa{if} \hlstd{(}\hlnum{0}\hlstd{)} \hlkwd{print}\hlstd{(}\hlstr{"hello"}\hlstd{)}
\hlkwa{if} \hlstd{(}\hlopt{-}\hlnum{1}\hlstd{)} \hlkwd{print}\hlstd{(}\hlstr{"hello"}\hlstd{)}
\hlkwa{if} \hlstd{(}\hlnum{0.01}\hlstd{)} \hlkwd{print}\hlstd{(}\hlstr{"hello"}\hlstd{)}
\hlkwa{if} \hlstd{(}\hlnum{1e-300}\hlstd{)} \hlkwd{print}\hlstd{(}\hlstr{"hello"}\hlstd{)}
\hlkwa{if} \hlstd{(}\hlnum{1e-323}\hlstd{)} \hlkwd{print}\hlstd{(}\hlstr{"hello"}\hlstd{)}
\hlkwa{if} \hlstd{(}\hlnum{1e-324}\hlstd{)} \hlkwd{print}\hlstd{(}\hlstr{"hello"}\hlstd{)}
\hlkwa{if} \hlstd{(}\hlnum{1e-500}\hlstd{)} \hlkwd{print}\hlstd{(}\hlstr{"hello"}\hlstd{)}
\hlkwa{if} \hlstd{(}\hlkwd{as.logical}\hlstd{(}\hlstr{"true"}\hlstd{))} \hlkwd{print}\hlstd{(}\hlstr{"hello"}\hlstd{)}
\hlkwa{if} \hlstd{(}\hlkwd{as.logical}\hlstd{(}\hlkwd{as.numeric}\hlstd{(}\hlstr{"1"}\hlstd{)))} \hlkwd{print}\hlstd{(}\hlstr{"hello"}\hlstd{)}
\hlkwa{if} \hlstd{(}\hlkwd{as.logical}\hlstd{(}\hlstr{"1"}\hlstd{))} \hlkwd{print}\hlstd{(}\hlstr{"hello"}\hlstd{)}
\hlkwa{if} \hlstd{(}\hlstr{"1"}\hlstd{)} \hlkwd{print}\hlstd{(}\hlstr{"hello"}\hlstd{)}
\end{alltt}
\end{kframe}
\end{knitrout}

Hint: If you need to refresh your understanding of the type conversion rules, see section \ref{sec:calc:type:conversion} on page \pageref{sec:calc:type:conversion}.
\end{playground}

In addition to \Rcontrol{if()} there is in \langname{R} a \Rcontrol{switch()} statement, that we describe here. It can be used to select among \emph{cases}, or several alternative statements, based on an expression evaluating to a \code{numeric} or a \code{character} value of length equal to one. The switch statement returns a value, the value returned by the statement corresponding to the matching switch value, or the default if there is no match and a default return value has been defined in the code. Both character values or numeric values can used.

\begin{knitrout}\footnotesize
\definecolor{shadecolor}{rgb}{0.969, 0.969, 0.969}\color{fgcolor}\begin{kframe}
\begin{alltt}
\hlstd{my.object} \hlkwb{<-} \hlstr{"two"}
\hlstd{b} \hlkwb{<-} \hlkwd{switch}\hlstd{(my.object,}
            \hlkwc{one} \hlstd{=} \hlnum{1}\hlstd{,}
            \hlkwc{two} \hlstd{=} \hlnum{1} \hlopt{/} \hlnum{2}\hlstd{,}
            \hlkwc{three} \hlstd{=} \hlnum{1} \hlopt{/} \hlnum{4}\hlstd{,}
            \hlnum{0}
\hlstd{)}
\hlstd{b}
\end{alltt}
\begin{verbatim}
## [1] 0.5
\end{verbatim}
\end{kframe}
\end{knitrout}

\begin{playground}
    Do play with the use of the switch statement. Look at the documentation for \code{switch()} using \code{help(switch)} and study the examples at the end of the help page.
\end{playground}

The \Rcontrol{switch()} statement can substitute for chained \code{if else} statements when all the conditions are comparisons against different constant values, resulting in more concise and clear code.

\subsubsection[Vectorized \texttt{ifelse()}]{Vectorized \code{ifelse()}}
\index{vectorized ifelse}
Vectorized \emph{if} is a peculiarity of the \Rlang language, but very useful for writing concise code that may execute faster than logically equivalent but not vectorized code.
Vectorized conditional execution is coded by means of \emph{function} \Rcontrol{ifelse()} (written as a single word). This function takes three arguments: a \code{logical} vector to parameter (\code{test}), a result vector for TRUE cases (\code{yes}), a result vector for FALSE false (\code{no}). All three can be any \Rlang statement supplying the necessary argument as their return value. In the case of vectors passed as arguments to parameters \code{yes} and \code{no}, recycling will take place if they are shorter than the logical vector passed as argument to \code{test}. No recycling ever applies to \code{test}, even if \code{yes} and/or \code{no} are longer than \code{test}. It is customary to pass arguments to \code{ifelse} by position. We give a first example with named arguments to clarify the use of the function.

\begin{knitrout}\footnotesize
\definecolor{shadecolor}{rgb}{0.969, 0.969, 0.969}\color{fgcolor}\begin{kframe}
\begin{alltt}
\hlstd{my.test} \hlkwb{<-} \hlkwd{c}\hlstd{(}\hlnum{TRUE}\hlstd{,} \hlnum{FALSE}\hlstd{,} \hlnum{TRUE}\hlstd{,} \hlnum{TRUE}\hlstd{)}
\hlkwd{ifelse}\hlstd{(}\hlkwc{test} \hlstd{= my.test,} \hlkwc{yes} \hlstd{=} \hlnum{1}\hlstd{,} \hlkwc{no} \hlstd{=} \hlopt{-}\hlnum{1}\hlstd{)}
\end{alltt}
\begin{verbatim}
## [1]  1 -1  1  1
\end{verbatim}
\end{kframe}
\end{knitrout}

In practice, the most common idiom is to have as argument passed to \code{test} the result of a comparison calculated on-the-fly. In the first example we compute the absolute values for a vector, equivalent to that returned by \Rlang function \code{abs()}.

\begin{knitrout}\footnotesize
\definecolor{shadecolor}{rgb}{0.969, 0.969, 0.969}\color{fgcolor}\begin{kframe}
\begin{alltt}
\hlstd{nums} \hlkwb{<-} \hlopt{-}\hlnum{3}\hlopt{:+}\hlnum{3}
\hlkwd{ifelse}\hlstd{(nums} \hlopt{<} \hlnum{0}\hlstd{,} \hlopt{-}\hlstd{nums, nums)}
\end{alltt}
\begin{verbatim}
## [1] 3 2 1 0 1 2 3
\end{verbatim}
\end{kframe}
\end{knitrout}

\begin{playground}
Some additional examples to play with, with a few surprises. Study the examples below until understand why returned values are what they are. In addition create your own examples to test other possible cases. In other words, play with the code until you fully understand how \code{ifelse} works.

\begin{knitrout}\footnotesize
\definecolor{shadecolor}{rgb}{0.969, 0.969, 0.969}\color{fgcolor}\begin{kframe}
\begin{alltt}
\hlstd{a} \hlkwb{<-} \hlnum{1}\hlopt{:}\hlnum{10}
\hlkwd{ifelse}\hlstd{(a} \hlopt{>} \hlnum{5}\hlstd{,} \hlnum{1}\hlstd{,} \hlopt{-}\hlnum{1}\hlstd{)}
\hlkwd{ifelse}\hlstd{(a} \hlopt{>} \hlnum{5}\hlstd{, a} \hlopt{+} \hlnum{1}\hlstd{, a} \hlopt{-} \hlnum{1}\hlstd{)}
\hlkwd{ifelse}\hlstd{(}\hlkwd{any}\hlstd{(a} \hlopt{>} \hlnum{5}\hlstd{), a} \hlopt{+} \hlnum{1}\hlstd{, a} \hlopt{-} \hlnum{1}\hlstd{)} \hlcom{# tricky}
\hlkwd{ifelse}\hlstd{(}\hlkwd{logical}\hlstd{(}\hlnum{0}\hlstd{), a} \hlopt{+} \hlnum{1}\hlstd{, a} \hlopt{-} \hlnum{1}\hlstd{)} \hlcom{# even more tricky}
\hlkwd{ifelse}\hlstd{(}\hlnum{NA}\hlstd{, a} \hlopt{+} \hlnum{1}\hlstd{, a} \hlopt{-} \hlnum{1}\hlstd{)} \hlcom{# as expected}
\end{alltt}
\end{kframe}
\end{knitrout}
Hint: If you need to refresh your understanding of \code{logical} values and Boolean algebra see section \ref{sec:calc:boolean} on page \pageref{sec:calc:boolean}.
\end{playground}

\begin{warningbox}
In the case of \Rcontrol{ifelse()}, the length of the returned value is determined by the length of the logical vector passed as argument to its first formal parameter (named \code{test})! A frequent mistake is to use a condition that returns a \code{logical} of length one, expecting that it will be recycled because arguments passed to the other formal parameters (named \code{yes} and \code{no}) are longer. However, no recycling will take place, resulting in a returned value of length one, with the remaining of the vectors being discarded. Do try this by yourself, using logical vectors of different lengths. You can start with the examples below, making sure you understand why the returned values are what they are.

\begin{knitrout}\footnotesize
\definecolor{shadecolor}{rgb}{0.969, 0.969, 0.969}\color{fgcolor}\begin{kframe}
\begin{alltt}
\hlkwd{ifelse}\hlstd{(}\hlnum{TRUE}\hlstd{,} \hlnum{1}\hlopt{:}\hlnum{5}\hlstd{,} \hlopt{-}\hlnum{5}\hlopt{:-}\hlnum{1}\hlstd{)}
\end{alltt}
\begin{verbatim}
## [1] 1
\end{verbatim}
\begin{alltt}
\hlkwd{ifelse}\hlstd{(}\hlnum{FALSE}\hlstd{,} \hlnum{1}\hlopt{:}\hlnum{5}\hlstd{,} \hlopt{-}\hlnum{5}\hlopt{:-}\hlnum{1}\hlstd{)}
\end{alltt}
\begin{verbatim}
## [1] -5
\end{verbatim}
\begin{alltt}
\hlkwd{ifelse}\hlstd{(}\hlkwd{c}\hlstd{(}\hlnum{TRUE}\hlstd{,} \hlnum{FALSE}\hlstd{),} \hlnum{1}\hlopt{:}\hlnum{5}\hlstd{,} \hlopt{-}\hlnum{5}\hlopt{:-}\hlnum{1}\hlstd{)}
\end{alltt}
\begin{verbatim}
## [1]  1 -4
\end{verbatim}
\begin{alltt}
\hlkwd{ifelse}\hlstd{(}\hlkwd{c}\hlstd{(}\hlnum{FALSE}\hlstd{,} \hlnum{TRUE}\hlstd{),} \hlnum{1}\hlopt{:}\hlnum{5}\hlstd{,} \hlopt{-}\hlnum{5}\hlopt{:-}\hlnum{1}\hlstd{)}
\end{alltt}
\begin{verbatim}
## [1] -5  2
\end{verbatim}
\begin{alltt}
\hlkwd{ifelse}\hlstd{(}\hlkwd{c}\hlstd{(}\hlnum{FALSE}\hlstd{,} \hlnum{TRUE}\hlstd{),} \hlnum{1}\hlopt{:}\hlnum{5}\hlstd{,} \hlnum{0}\hlstd{)}
\end{alltt}
\begin{verbatim}
## [1] 0 2
\end{verbatim}
\end{kframe}
\end{knitrout}
\end{warningbox}

\begin{playground}
Write, using \Rcontrol{ifelse()}, a single statement to combine numbers from the two vectors \code{a} and \code{b} into a result vector \code{d}, based on whether the corresponding value in vector \code{c} is the character \code{"a"} or \code{"b"}. Then print vector \code{d} to make the result visible.

\begin{knitrout}\footnotesize
\definecolor{shadecolor}{rgb}{0.969, 0.969, 0.969}\color{fgcolor}\begin{kframe}
\begin{alltt}
\hlstd{a} \hlkwb{<-} \hlopt{-}\hlnum{10}\hlopt{:-}\hlnum{1}
\hlstd{b} \hlkwb{<-} \hlopt{+}\hlnum{1}\hlopt{:}\hlnum{10}
\hlstd{c} \hlkwb{<-} \hlkwd{c}\hlstd{(}\hlkwd{rep}\hlstd{(}\hlstr{"a"}\hlstd{,} \hlnum{5}\hlstd{),} \hlkwd{rep}\hlstd{(}\hlstr{"b"}\hlstd{,} \hlnum{5}\hlstd{))}
\hlcom{# your code}
\end{alltt}
\end{kframe}
\end{knitrout}

If you do not understand how the three vectors are built, or you cannot guess the values they contain by reading the code, print them, and play with the arguments, until you have clear what each parameter does. Also use \code{help(rep)} and/or \code{help(ifelse)} to access the documentation.
\end{playground}

\subsection{Iteration}
\index{loops|seealso{iteration}}\index{for loop}\index{iteration!for loop}\qRloop{for}
We give the name \emph{iteration} to the process of repetitive execution of a program statement (simple or compound)---e.g.\ \emph{computed by iteration}. We use the same word, iteration, also to name each one of these repetitions of the execution of a statement--e.g.\ the second iteration.

The section of computer code being executed multiple times, conforms a loop (a closed path). Most loops contain a condition that determines when execution will continue outside the loop. The most frequently used type of loop is a \code{for} loop. These loops work in \Rlang on lists or vectors of values to act upon.

\begin{knitrout}\footnotesize
\definecolor{shadecolor}{rgb}{0.969, 0.969, 0.969}\color{fgcolor}\begin{kframe}
\begin{alltt}
\hlstd{b} \hlkwb{<-} \hlnum{0}
\hlkwa{for} \hlstd{(a} \hlkwa{in} \hlnum{1}\hlopt{:}\hlnum{5}\hlstd{) b} \hlkwb{<-} \hlstd{b} \hlopt{+} \hlstd{a}
\hlstd{b}
\end{alltt}
\begin{verbatim}
## [1] 15
\end{verbatim}
\begin{alltt}
\hlstd{b} \hlkwb{<-} \hlkwd{sum}\hlstd{(}\hlnum{1}\hlopt{:}\hlnum{5}\hlstd{)} \hlcom{# built-in function (faster)}
\hlstd{b}
\end{alltt}
\begin{verbatim}
## [1] 15
\end{verbatim}
\end{kframe}
\end{knitrout}

Here the statement \code{b <- b + a} is executed five times, with variable \code{a} sequentially taking each of the values in \code{1:5}. Instead of a simple statement used here, also a compound statement could have been used for the body of the \code{for} loop.

\begin{warningbox}
It is important to note that a list or vector of length zero is a valid argument to \code{for()}, that triggers no error, but skips the statements in the loop body.
\end{warningbox}

Some examples of use of \code{for} loops --- and of how to avoid their use.

\begin{knitrout}\footnotesize
\definecolor{shadecolor}{rgb}{0.969, 0.969, 0.969}\color{fgcolor}\begin{kframe}
\begin{alltt}
\hlstd{a} \hlkwb{<-} \hlkwd{c}\hlstd{(}\hlnum{1}\hlstd{,} \hlnum{4}\hlstd{,} \hlnum{3}\hlstd{,} \hlnum{6}\hlstd{,} \hlnum{8}\hlstd{)}
\hlkwa{for}\hlstd{(x} \hlkwa{in} \hlstd{a) x}\hlopt{*}\hlnum{2} \hlcom{# result is lost}
\hlkwa{for}\hlstd{(x} \hlkwa{in} \hlstd{a)} \hlkwd{print}\hlstd{(x}\hlopt{*}\hlnum{2}\hlstd{)} \hlcom{# print is needed!}
\end{alltt}
\begin{verbatim}
## [1] 2
## [1] 8
## [1] 6
## [1] 12
## [1] 16
\end{verbatim}
\begin{alltt}
\hlstd{b} \hlkwb{<-} \hlkwa{for}\hlstd{(x} \hlkwa{in} \hlstd{a) x}\hlopt{*}\hlnum{2} \hlcom{# does not work as expected, but triggers no error}
\hlstd{b}
\end{alltt}
\begin{verbatim}
## NULL
\end{verbatim}
\begin{alltt}
\hlkwa{for}\hlstd{(x} \hlkwa{in} \hlstd{a) b} \hlkwb{<-} \hlstd{x}\hlopt{*}\hlnum{2} \hlcom{# a bit of a surprise, as b is not a vector!}

\hlstd{b} \hlkwb{<-} \hlkwd{numeric}\hlstd{()}
\hlkwa{for}\hlstd{(i} \hlkwa{in} \hlkwd{seq}\hlstd{(}\hlkwc{along.with} \hlstd{= a)) \{}
  \hlstd{b[i]} \hlkwb{<-} \hlstd{a[i]}\hlopt{^}\hlnum{2}
  \hlkwd{print}\hlstd{(b)}
\hlstd{\}}
\end{alltt}
\begin{verbatim}
## [1] 1
## [1]  1 16
## [1]  1 16  9
## [1]  1 16  9 36
## [1]  1 16  9 36 64
\end{verbatim}
\begin{alltt}
\hlstd{b} \hlcom{# is a vector!}
\end{alltt}
\begin{verbatim}
## [1]  1 16  9 36 64
\end{verbatim}
\begin{alltt}
\hlcom{# a bit faster if we first allocate a vector of the required length}
\hlstd{b} \hlkwb{<-} \hlkwd{numeric}\hlstd{(}\hlkwd{length}\hlstd{(a))}
\hlkwa{for}\hlstd{(i} \hlkwa{in} \hlkwd{seq}\hlstd{(}\hlkwc{along.with} \hlstd{= a)) \{}
  \hlstd{b[i]} \hlkwb{<-} \hlstd{a[i]}\hlopt{^}\hlnum{2}
  \hlkwd{print}\hlstd{(b)}
\hlstd{\}}
\end{alltt}
\begin{verbatim}
## [1] 1 0 0 0 0
## [1]  1 16  0  0  0
## [1]  1 16  9  0  0
## [1]  1 16  9 36  0
## [1]  1 16  9 36 64
\end{verbatim}
\begin{alltt}
\hlstd{b} \hlcom{# is a vector!}
\end{alltt}
\begin{verbatim}
## [1]  1 16  9 36 64
\end{verbatim}
\begin{alltt}
\hlcom{# vectorization is simplest and fastest}
\hlstd{b} \hlkwb{<-} \hlstd{a}\hlopt{^}\hlnum{2}
\hlstd{b}
\end{alltt}
\begin{verbatim}
## [1]  1 16  9 36 64
\end{verbatim}
\end{kframe}
\end{knitrout}

\code{seq(along.with = a)} builds a new numeric vector with a sequence of the same length as vector \code{a}, passed as argument to parameter \code{along.width}---the default for parameter \code{from} is 1.

\begin{playground}\label{box:play:forloop}
Look at the results from the above examples, and try to understand where does the returned value come from in each case. In the code chunk above, \Rfunction{print()} is used within the \emph{loop} to make intermediate values visible. You can add additional \code{print()} statements to visualize other variables such as \code{i} or run parts of the code, such as \code{seq(along.with = a)}, by themselves.

In this case, the code examples are valid, but the same approach can be used for debugging syntactically correct code that does not return the expected results, either for every input value, or with a specific value as input.
\end{playground}

\begin{advplayground}
In the examples above we show the use of \code{seq()} passing a vector as argument to its parameter \code{along.with}. This approach is much better than using the not exactly equivalent call to \code{seq()} based on the length of the vector, or its short version using operator \code{:}.

\begin{knitrout}\footnotesize
\definecolor{shadecolor}{rgb}{0.969, 0.969, 0.969}\color{fgcolor}\begin{kframe}
\begin{alltt}
\hlstd{a} \hlkwb{<-} \hlkwd{c}\hlstd{(}\hlnum{1}\hlstd{,} \hlnum{4}\hlstd{,} \hlnum{3}\hlstd{,} \hlnum{6}\hlstd{,} \hlnum{8}\hlstd{)}
\hlcom{# a <- numeric(0)}

\hlstd{b} \hlkwb{<-} \hlkwd{numeric}\hlstd{(}\hlkwd{length}\hlstd{(a))}
\hlkwa{for}\hlstd{(i} \hlkwa{in} \hlkwd{seq}\hlstd{(}\hlkwc{along.with} \hlstd{= a)) \{}
  \hlstd{b[i]} \hlkwb{<-} \hlstd{a[i]}\hlopt{^}\hlnum{2}
\hlstd{\}}
\hlkwd{print}\hlstd{(b)}
\end{alltt}
\begin{verbatim}
## [1]  1 16  9 36 64
\end{verbatim}
\begin{alltt}
\hlstd{c} \hlkwb{<-} \hlkwd{numeric}\hlstd{(}\hlkwd{length}\hlstd{(a))}
\hlkwa{for}\hlstd{(i} \hlkwa{in} \hlnum{1}\hlopt{:}\hlkwd{length}\hlstd{(a)) \{}
  \hlstd{c[i]} \hlkwb{<-} \hlstd{a[i]}\hlopt{^}\hlnum{2}
\hlstd{\}}
\hlkwd{print}\hlstd{(c)}
\end{alltt}
\begin{verbatim}
## [1]  1 16  9 36 64
\end{verbatim}
\end{kframe}
\end{knitrout}

With \code{a} of length 1 or longer, the statements are equivalent, but when \code{a} has length zero the two statements are no longer equivalent. Run the statements above, after un-commenting the second definition of \code{a} and try to understand \emph{why} they behave as they do.
\end{advplayground}

\Rloop{while} loops\index{iteration!while loop} are quite frequently also useful. Instead of a list or vector, they take a logical argument, which is usually an expression, but which can also be a variable.

In this example we use the result of the previous iteration in the current one. In this example you can also see, that it is allowed to put more than one statement in a single line, in which case the statements should be separated by a semicolon (;).

\begin{knitrout}\footnotesize
\definecolor{shadecolor}{rgb}{0.969, 0.969, 0.969}\color{fgcolor}\begin{kframe}
\begin{alltt}
\hlstd{a} \hlkwb{<-} \hlnum{2}
\hlkwa{while} \hlstd{(a} \hlopt{<} \hlnum{50}\hlstd{) \{}\hlkwd{print}\hlstd{(a); a} \hlkwb{<-} \hlstd{a}\hlopt{^}\hlnum{2}\hlstd{\}}
\end{alltt}
\begin{verbatim}
## [1] 2
## [1] 4
## [1] 16
\end{verbatim}
\begin{alltt}
\hlkwd{print}\hlstd{(a)}
\end{alltt}
\begin{verbatim}
## [1] 256
\end{verbatim}
\end{kframe}
\end{knitrout}

\begin{playground}
Make sure that you understand why the final value of \code{a} is larger than 50.
\end{playground}


\begin{advplayground}
The statements above can be simplified to:

\begin{knitrout}\footnotesize
\definecolor{shadecolor}{rgb}{0.969, 0.969, 0.969}\color{fgcolor}\begin{kframe}
\begin{alltt}
\hlstd{a} \hlkwb{<-} \hlnum{2}
\hlkwd{print}\hlstd{(a)}
\hlkwa{while} \hlstd{(a} \hlopt{<} \hlnum{50}\hlstd{) \{}\hlkwd{print}\hlstd{(a} \hlkwb{<-} \hlstd{a}\hlopt{^}\hlnum{2}\hlstd{)\}}
\end{alltt}
\end{kframe}
\end{knitrout}

1) Explain why this works, and how it relates to the support in \Rlang of \emph{chained} assignments to several variables within a single statement like the one below.

\begin{knitrout}\footnotesize
\definecolor{shadecolor}{rgb}{0.969, 0.969, 0.969}\color{fgcolor}\begin{kframe}
\begin{alltt}
\hlstd{a} \hlkwb{<-} \hlstd{b} \hlkwb{<-} \hlstd{c} \hlkwb{<-} \hlnum{1}\hlopt{:}\hlnum{5}
\hlstd{a}
\hlstd{b}
\hlstd{c}
\end{alltt}
\end{kframe}
\end{knitrout}

2) Explain why \code{print(a)} has been moved before \code{while()}. Hint: experiment if necessary.
\end{advplayground}

The \Rloop{repeat}\index{iteration!repeat loop} construct is seldom used, but adds flexibility as \Rcontrol{break()}, which is used to \emph{break out of the loop} can be located in any place within the body of the loop. To achieve conditional end of iterations the \Rcontrol{break()} function must be called from within the statement of an \code{if} or \code{else} condition.

\begin{knitrout}\footnotesize
\definecolor{shadecolor}{rgb}{0.969, 0.969, 0.969}\color{fgcolor}\begin{kframe}
\begin{alltt}
\hlstd{a} \hlkwb{<-} \hlnum{2}
\hlkwa{repeat}\hlstd{\{}
  \hlkwd{print}\hlstd{(a)}
  \hlstd{a} \hlkwb{<-} \hlstd{a}\hlopt{^}\hlnum{2}
  \hlkwa{if} \hlstd{(a} \hlopt{>} \hlnum{50}\hlstd{) \{}\hlkwd{print}\hlstd{(a);} \hlkwa{break}\hlstd{()\}}
\hlstd{\}}
\end{alltt}
\begin{verbatim}
## [1] 2
## [1] 4
## [1] 16
## [1] 256
\end{verbatim}
\begin{alltt}
\hlcom{# or more elegantly}
\hlstd{a} \hlkwb{<-} \hlnum{2}
\hlkwa{repeat}\hlstd{\{}
  \hlkwd{print}\hlstd{(a)}
  \hlkwa{if} \hlstd{(a} \hlopt{>} \hlnum{50}\hlstd{)} \hlkwa{break}\hlstd{()}
  \hlstd{a} \hlkwb{<-} \hlstd{a}\hlopt{^}\hlnum{2}
\hlstd{\}}
\end{alltt}
\begin{verbatim}
## [1] 2
## [1] 4
## [1] 16
## [1] 256
\end{verbatim}
\end{kframe}
\end{knitrout}

\begin{playground}
Please, explain why the examples above return the values they do. Use the approach of adding \Rfunction{print()} statements, as described on page \pageref{box:play:forloop}.
\end{playground}

\subsection{Explicit loops can be slow in \Rlang}\label{sec:loops:slow}
\index{vectorization}\index{recycling of arguments}\index{iteration}\index{loops!faster alternatives|(}
If you have written programs in other languages, it would feel to you natural to use loops (for, repeat while, repeat until) for many of the things for which we have been using vectorization. When using the \Rlang language it is best to use vectorization whenever possible, because it keeps scripts shorter and easier to understand (at least for those with experience in \Rlang). However, there is another very important reason: execution speed. The reason behind this is that \Rlang is an interpreted language. In current versions of \Rpgrm it is possible to byte-compile functions and loops may be compiled on-the-fly, which relieves part of the burden of repeated interpretation. However, even byte-compiled loops are usually slower to execute than vectorized functions.

Execution speed needs to be balanced against the effort invested in writing faster code. However, using vectorization and specific \Rlang functions requires little effort once we are familiar with them. The simplest way of measuring the execution time of an \Rlang expression is to use function \Rfunction{system.time()}. However, the returned time is in seconds and consequently the expression must take long enough to execute for the returned time to have useful resolution. See package \pkgname{microbenchmark} for tools for benchmarking code with better time resolution.\qRloop{for}

\begin{knitrout}\footnotesize
\definecolor{shadecolor}{rgb}{0.969, 0.969, 0.969}\color{fgcolor}\begin{kframe}
\begin{alltt}
\hlkwd{system.time}\hlstd{(\{a} \hlkwb{<-} \hlkwd{numeric}\hlstd{()}
            \hlkwa{for} \hlstd{(i} \hlkwa{in} \hlnum{1}\hlopt{:}\hlnum{1000000}\hlstd{) \{}
              \hlstd{a[i]} \hlkwb{<-} \hlstd{i} \hlopt{/} \hlnum{1000}
              \hlstd{\}}
            \hlstd{\})}
\end{alltt}
\begin{verbatim}
##    user  system elapsed 
##    0.22    0.03    0.25
\end{verbatim}
\end{kframe}
\end{knitrout}

\begin{explainbox}
Whenever working with large data sets, or many similar data sets, we will need to take performance into account. As vectorization usually also makes code simpler, it is good style to use it whenever possible. For operations that are frequently used, \Rlang includes specific functions. It is thus important to consider not only vectorization using subscripts but also check for the availability of performance optimized functions available. The results from running the code examples in this box are not included, they are the same for all chunks. Here we are interested in the execution time, and we leave this as an exercise.

\begin{knitrout}\footnotesize
\definecolor{shadecolor}{rgb}{0.969, 0.969, 0.969}\color{fgcolor}\begin{kframe}
\begin{alltt}
\hlstd{a} \hlkwb{<-} \hlkwd{rnorm}\hlstd{(}\hlnum{10}\hlopt{^}\hlnum{7}\hlstd{)} \hlcom{# 10 000 0000 pseudo-random numbers}
\end{alltt}
\end{kframe}
\end{knitrout}

\begin{knitrout}\footnotesize
\definecolor{shadecolor}{rgb}{0.969, 0.969, 0.969}\color{fgcolor}\begin{kframe}
\begin{alltt}
\hlcom{# b <- numeric()}
\hlstd{b} \hlkwb{<-} \hlkwd{numeric}\hlstd{(}\hlkwd{length}\hlstd{(a)}\hlopt{-}\hlnum{1}\hlstd{)} \hlcom{# pre-allocate memory}
\hlstd{i} \hlkwb{<-} \hlnum{1}
\hlkwa{while} \hlstd{(i} \hlopt{<} \hlkwd{length}\hlstd{(a)) \{}
  \hlstd{b[i]} \hlkwb{<-} \hlstd{a[i]}\hlopt{^}\hlnum{2}
  \hlkwd{print}\hlstd{(b)}
  \hlstd{i} \hlkwb{<-} \hlstd{i} \hlopt{+} \hlnum{1}
\hlstd{\}}
\hlstd{b}
\end{alltt}
\end{kframe}
\end{knitrout}

\begin{knitrout}\footnotesize
\definecolor{shadecolor}{rgb}{0.969, 0.969, 0.969}\color{fgcolor}\begin{kframe}
\begin{alltt}
\hlcom{# b <- numeric()}
\hlstd{b} \hlkwb{<-} \hlkwd{numeric}\hlstd{(}\hlkwd{length}\hlstd{(a)}\hlopt{-}\hlnum{1}\hlstd{)} \hlcom{# pre-allocate memory}
\hlkwa{for}\hlstd{(i} \hlkwa{in} \hlkwd{seq}\hlstd{(}\hlkwc{along.with} \hlstd{= b)) \{}
  \hlstd{b[i]} \hlkwb{<-} \hlstd{a[i}\hlopt{+}\hlnum{1}\hlstd{]} \hlopt{-} \hlstd{a[i]}
  \hlkwd{print}\hlstd{(b)}
\hlstd{\}}
\end{alltt}
\end{kframe}
\end{knitrout}

\begin{knitrout}\footnotesize
\definecolor{shadecolor}{rgb}{0.969, 0.969, 0.969}\color{fgcolor}\begin{kframe}
\begin{alltt}
\hlcom{# although in this case there were alternatives, there}
\hlcom{# are other cases when we need to use indexes explicitly}
\hlstd{b} \hlkwb{<-} \hlstd{a[}\hlnum{2}\hlopt{:}\hlkwd{length}\hlstd{(a)]} \hlopt{-} \hlstd{a[}\hlnum{1}\hlopt{:}\hlkwd{length}\hlstd{(a)}\hlopt{-}\hlnum{1}\hlstd{]}
\hlstd{b}
\end{alltt}
\end{kframe}
\end{knitrout}

\begin{knitrout}\footnotesize
\definecolor{shadecolor}{rgb}{0.969, 0.969, 0.969}\color{fgcolor}\begin{kframe}
\begin{alltt}
\hlcom{# or even better}
\hlstd{b} \hlkwb{<-} \hlkwd{diff}\hlstd{(a)}
\hlstd{b}
\end{alltt}
\end{kframe}
\end{knitrout}

Execution time can be obtained with \Rfunction{system.time()}. For a vector of ten million numbers, the \code{for} loop above takes 1.1~s and the equivalent \code{while} loop 2.0~s, the vectorized statement using indexing takes 0.2~s and function \code{diff()} takes 0.1~s. The \code{for} loop without pre-allocation of memory to \code{b} takes 3.6~s, and the equivalent while loop 4.7~s---i.e.\ the fastest execution time was more than 40 times faster than the slowest one. (Times for \Rpgrm 3.5.1 on a laptop with Winodws 10 x64.)

\end{explainbox}
\index{loops!faster alternatives|)}

\subsection{Nesting of loops}\label{sec:nested:loops}
\index{iteration!nesting of loops}\index{nested iteration loops}\index{loops!nested}

All the execution-flow control statements seen above can be nested. We will show an example with two \code{for} loops. We first need a matrix of data to work with:

\begin{knitrout}\footnotesize
\definecolor{shadecolor}{rgb}{0.969, 0.969, 0.969}\color{fgcolor}\begin{kframe}
\begin{alltt}
\hlstd{A} \hlkwb{<-} \hlkwd{matrix}\hlstd{(}\hlnum{1}\hlopt{:}\hlnum{50}\hlstd{,} \hlnum{10}\hlstd{)}
\hlstd{A}
\end{alltt}
\begin{verbatim}
##       [,1] [,2] [,3] [,4] [,5]
##  [1,]    1   11   21   31   41
##  [2,]    2   12   22   32   42
##  [3,]    3   13   23   33   43
##  [4,]    4   14   24   34   44
##  [5,]    5   15   25   35   45
##  [6,]    6   16   26   36   46
##  [7,]    7   17   27   37   47
##  [8,]    8   18   28   38   48
##  [9,]    9   19   29   39   49
## [10,]   10   20   30   40   50
\end{verbatim}
\end{kframe}
\end{knitrout}

\begin{knitrout}\footnotesize
\definecolor{shadecolor}{rgb}{0.969, 0.969, 0.969}\color{fgcolor}\begin{kframe}
\begin{alltt}
\hlstd{row.sum} \hlkwb{<-} \hlkwd{numeric}\hlstd{()}
\hlkwa{for} \hlstd{(i} \hlkwa{in} \hlnum{1}\hlopt{:}\hlkwd{nrow}\hlstd{(A)) \{}
  \hlstd{row.sum[i]} \hlkwb{<-} \hlnum{0}
  \hlkwa{for} \hlstd{(j} \hlkwa{in} \hlnum{1}\hlopt{:}\hlkwd{ncol}\hlstd{(A))}
    \hlstd{row.sum[i]} \hlkwb{<-} \hlstd{row.sum[i]} \hlopt{+} \hlstd{A[i, j]}
\hlstd{\}}
\hlkwd{print}\hlstd{(row.sum)}
\end{alltt}
\begin{verbatim}
##  [1] 105 110 115 120 125 130 135 140 145 150
\end{verbatim}
\end{kframe}
\end{knitrout}

The code above is very general, it will work with any two dimensional matrix with at least one column and one row. However, sometimes we need more specific calculations. \code{A[1, 2]} selects one cell in the matrix, the one on the first row of the second column. \code{A[1, ]} selects row one, and  \code{A[ , 2]} selects column two. In the example above the value of \code{i} changes for each iteration of the outer loop. The value of \code{j} changes for each iteration of the inner loop, and the inner loop is run in full for each iteration of the outer loop. The inner loop index \code{j} changes fastest.

\begin{playground}
1) modify the example above to add up only the first three columns of \code{A}, 2) modify the example above to add the last three columns of \code{A}.

Will the code you wrote continue working as expected if the number of rows in \code{A} changed? and what if the number of columns in \code{A} changed, and the required results still needed to be calculated for relative positions? What would happen if \code{A} had fewer than three columns? Try to think first what to expect based on the code you wrote. Then create matrices of different sizes and test your code. After that think how to improve the code, so that wrong results are not produced.
\end{playground}

\begin{explainbox}
If the total number of iterations is large and the code executed at each iteration runs fast, the overhead added by the loop code can make a big contribution to the total running time of a script.
When dealing with nested loops, as the inner loop is executed most frequently, this is the best place to look for ways of reducing execution time. In this example vectorization can be achieved easily for the inner loop, as \Rlang has a function \code{sum()} which returns the sum of a vector passed as its argument. Replacing the inner loop by an efficient function can be expected to improve performance significantly. See section \ref{sec:performance:tuning} on page \pageref{sec:performance:tuning} for a brief description of tools for measuring performance and finding the bottlenecks that may be limiting it.

\begin{knitrout}\footnotesize
\definecolor{shadecolor}{rgb}{0.969, 0.969, 0.969}\color{fgcolor}\begin{kframe}
\begin{alltt}
\hlstd{row.sum} \hlkwb{<-} \hlkwd{numeric}\hlstd{(}\hlkwd{nrow}\hlstd{(A))} \hlcom{# faster}
\hlkwa{for} \hlstd{(i} \hlkwa{in} \hlnum{1}\hlopt{:}\hlkwd{nrow}\hlstd{(A)) \{}
  \hlstd{row.sum[i]} \hlkwb{<-} \hlkwd{sum}\hlstd{(A[i, ])}
\hlstd{\}}
\hlkwd{print}\hlstd{(row.sum)}
\end{alltt}
\begin{verbatim}
##  [1] 105 110 115 120 125 130 135 140 145 150
\end{verbatim}
\end{kframe}
\end{knitrout}

\code{A[i, ]} selects row \code{i} and all columns. (In \Rlang the row index comes first. See section \ref{sec:calc:indexing} on page \pageref{sec:calc:indexing} for a detailed explanation.)

Both\index{apply functions} explicit loops can be eliminated if we use an \emph{apply} function, such as \Rloop{apply()}, \Rloop{lapply()} or \Rloop{sapply()}, in place of the outer \code{for} loop. See section \ref{sec:data:apply} on page \pageref{sec:data:apply} for details on the use of the different \emph{apply} functions.

\begin{knitrout}\footnotesize
\definecolor{shadecolor}{rgb}{0.969, 0.969, 0.969}\color{fgcolor}\begin{kframe}
\begin{alltt}
\hlstd{row.sum} \hlkwb{<-} \hlkwd{apply}\hlstd{(A,} \hlkwc{MARGIN} \hlstd{=} \hlnum{1}\hlstd{, sum)} \hlcom{# MARGIN=1 inidcates rows}
\hlkwd{print}\hlstd{(row.sum)}
\end{alltt}
\begin{verbatim}
##  [1] 105 110 115 120 125 130 135 140 145 150
\end{verbatim}
\end{kframe}
\end{knitrout}
There are several variants of \emph{apply} functions available, both in base \Rlang and exported by contributed packages. See section \ref{sec:data:apply} for details on the use of several of the later ones.

Calculating row sums is a frequent operation, so \Rlang has a built-in function for this. As earlier with \code{diff()} it is always worthwhile to check if there is an existing \Rlang function capable of doing the computations we need. In this case using \code{rowSums()} simplifies the nested loops into a single function code also optimizing performance.

\begin{knitrout}\footnotesize
\definecolor{shadecolor}{rgb}{0.969, 0.969, 0.969}\color{fgcolor}\begin{kframe}
\begin{alltt}
\hlkwd{rowSums}\hlstd{(A)}
\end{alltt}
\begin{verbatim}
##  [1] 105 110 115 120 125 130 135 140 145 150
\end{verbatim}
\end{kframe}
\end{knitrout}

\end{explainbox}

\begin{playground}
1) How would you change this last example, so that only the last three columns are added up? (Think about use of subscripts to select a part of the matrix.)

2) To obtain column sums one could modify the nested loops, transpose the matrix and use \code{rowSums()}, or look up if there is in \Rlang a function for this operation. A good place to start is with \code{help(rowSums)} as similar functions may share the same help page, or at least be listed in the ``See also'' section. Do try this, and explore other help pages in search for some function you may find useful in the analysis of your own data.
\end{playground}

\section[Apply functions]{\emph{Apply} functions}\label{sec:data:apply}

\emph{Apply}\index{apply functions}\index{loops!faster alternatives}
 functions apply functions to elements in a collection of \Rlang objects. These collections can be vectors, lists, data frames, matrices or arrays. As long as the operations to be applied are \emph{independent---i.e.\ the results from one iteration are not used in another iteration, and each iteration refers to only one member of the collection of objects---} apply functions can replace \code{for}, \code{while} or \code{repeat} loops.

Functions in the \emph{apply} family allow us to call an \Rlang function for each element of a vector, matrix or array without writing an explicit loop in our script.\qRloop{apply()}

\begin{knitrout}\footnotesize
\definecolor{shadecolor}{rgb}{0.969, 0.969, 0.969}\color{fgcolor}\begin{kframe}
\begin{alltt}
\hlstd{x} \hlkwb{<-} \hlkwd{matrix}\hlstd{(}\hlkwd{runif}\hlstd{(}\hlnum{100}\hlstd{),} \hlkwc{ncol} \hlstd{=} \hlnum{10}\hlstd{)}
\hlkwd{apply}\hlstd{(x,} \hlkwc{MARGIN} \hlstd{=} \hlnum{1}\hlstd{,} \hlkwc{FUN} \hlstd{= mean)}
\end{alltt}
\begin{verbatim}
##  [1] 0.4590021 0.4765083 0.3984682 0.5111989 0.4422277 0.4037644 0.6795558
##  [8] 0.3933792 0.5508738 0.5450010
\end{verbatim}
\end{kframe}
\end{knitrout}

A constrain on the function to be applied is that the elements will be passed by apply functions to the first argument as demonstrated in the code chunk below.

\begin{knitrout}\footnotesize
\definecolor{shadecolor}{rgb}{0.969, 0.969, 0.969}\color{fgcolor}\begin{kframe}
\begin{alltt}
\hlcom{# not run}
\hlstd{my.mean} \hlkwb{<-} \hlkwa{function}\hlstd{(}\hlkwc{na.rm} \hlstd{=} \hlnum{FALSE}\hlstd{,} \hlkwc{x}\hlstd{) \{}
  \hlkwd{mean}\hlstd{(x, na.rm)}
\hlstd{\}}
\hlkwd{apply}\hlstd{(x,} \hlkwc{MARGIN} \hlstd{=} \hlnum{1}\hlstd{,} \hlkwc{FUN} \hlstd{= my.mean)} \hlcom{# triggers error!!}
\end{alltt}
\end{kframe}
\end{knitrout}

One or more additional named arguments can be passed to the function to be applied, but these are invariant.

\begin{knitrout}\footnotesize
\definecolor{shadecolor}{rgb}{0.969, 0.969, 0.969}\color{fgcolor}\begin{kframe}
\begin{alltt}
\hlkwd{apply}\hlstd{(x,} \hlkwc{MARGIN} \hlstd{=} \hlnum{1}\hlstd{,} \hlkwc{FUN} \hlstd{= mean,} \hlkwc{trim} \hlstd{=} \hlnum{0.1}\hlstd{)}
\end{alltt}
\begin{verbatim}
##  [1] 0.4440470 0.4741493 0.3981852 0.5108974 0.4261187 0.3979126 0.7197825
##  [8] 0.3655417 0.5667565 0.5501504
\end{verbatim}
\end{kframe}
\end{knitrout}

\begin{playground}
Look up the help pages for \Rloop{apply()} and \code{mean()} and study them until you understand how additional arguments can be passed to the applied function. Can you guess why \Rloop{apply()} was designed to have parameter names fully in upper case, something very unusual for \Rlang code style?
\end{playground}

The various base \Rlang \emph{apply} functions differ on the class of the returned value and on the class of the argument expected for their \code{X} parameter: \Rloop{apply()} expects a \code{matrix} or \code{array} as argument, or an argument like a \code{data.frame} which can be converted to a matrix or array. \Rloop{apply()} returns an array or a list or a vector depending on the size, and consistency in length and class among the values returned by the applied function. \Rloop{lapply()} and \Rloop{sapply()} expect a \code{vector} or \code{list} as argument passed through \code{X}. \Rloop{lapply()} returns a \code{list} or an \code{array}; and \Rloop{vapply()} always \emph{simplifies} its returned value into a vector, while \Rloop{sapply()} does the simplification according to the argument passed to its \code{simplify} parameter. All these \emph{apply} functions can be used to apply any \Rlang function that returns a value of the same or a different class as its argument. In the case of \Rloop{apply()} and \Rloop{lapply()} not even the length of the values returned for each member of the collection passed as argument, needs to be consistent. In summary, \Rloop{apply()} is used to apply a function to the elements of an object that has \emph{dimensions} defined, and \Rloop{lapply()} and \Rloop{sapply()} to apply a function to the members of and object without dimensions, such as a vector.

We next exemplify the use of \Rloop{lapply()}, \Rloop{sapply()} and \Rloop{vapply()} given that the argument for \code{X} is for these functions a vector or list. In this example we apply a user defined function.

\begin{knitrout}\footnotesize
\definecolor{shadecolor}{rgb}{0.969, 0.969, 0.969}\color{fgcolor}\begin{kframe}
\begin{alltt}
\hlkwd{set.seed}\hlstd{(}\hlnum{123456}\hlstd{)} \hlcom{# so that a.vector does not change}
\hlstd{a.vector} \hlkwb{<-} \hlkwd{runif}\hlstd{(}\hlnum{6}\hlstd{)} \hlcom{# A short vector as input to keep output short}
\hlstd{my.fun} \hlkwb{<-} \hlkwa{function}\hlstd{(}\hlkwc{x}\hlstd{,} \hlkwc{k}\hlstd{) \{}\hlkwd{log}\hlstd{(x)} \hlopt{+} \hlstd{k\}}
\hlstd{z} \hlkwb{<-} \hlkwd{lapply}\hlstd{(}\hlkwc{X} \hlstd{= a.vector,} \hlkwc{FUN} \hlstd{= my.fun,} \hlkwc{k} \hlstd{=} \hlnum{5}\hlstd{)}
\hlkwd{class}\hlstd{(z)}
\end{alltt}
\begin{verbatim}
## [1] "list"
\end{verbatim}
\begin{alltt}
\hlkwd{dim}\hlstd{(z)}
\end{alltt}
\begin{verbatim}
## NULL
\end{verbatim}
\begin{alltt}
\hlstd{z}
\end{alltt}
\begin{verbatim}
## [[1]]
## [1] 4.774083
## 
## [[2]]
## [1] 4.71706
## 
## [[3]]
## [1] 4.061606
## 
## [[4]]
## [1] 3.925758
## 
## [[5]]
## [1] 3.981937
## 
## [[6]]
## [1] 3.382251
\end{verbatim}
\end{kframe}
\end{knitrout}

\begin{knitrout}\footnotesize
\definecolor{shadecolor}{rgb}{0.969, 0.969, 0.969}\color{fgcolor}\begin{kframe}
\begin{alltt}
\hlstd{z} \hlkwb{<-} \hlkwd{sapply}\hlstd{(}\hlkwc{X} \hlstd{= a.vector,} \hlkwc{FUN} \hlstd{= my.fun,} \hlkwc{k} \hlstd{=} \hlnum{5}\hlstd{)}
\hlkwd{class}\hlstd{(z)}
\end{alltt}
\begin{verbatim}
## [1] "numeric"
\end{verbatim}
\begin{alltt}
\hlkwd{dim}\hlstd{(z)}
\end{alltt}
\begin{verbatim}
## NULL
\end{verbatim}
\begin{alltt}
\hlstd{z}
\end{alltt}
\begin{verbatim}
## [1] 4.774083 4.717060 4.061606 3.925758 3.981937 3.382251
\end{verbatim}
\end{kframe}
\end{knitrout}

\begin{knitrout}\footnotesize
\definecolor{shadecolor}{rgb}{0.969, 0.969, 0.969}\color{fgcolor}\begin{kframe}
\begin{alltt}
\hlstd{z} \hlkwb{<-} \hlkwd{sapply}\hlstd{(}\hlkwc{X} \hlstd{= a.vector,} \hlkwc{FUN} \hlstd{= my.fun,} \hlkwc{k} \hlstd{=} \hlnum{5}\hlstd{,} \hlkwc{simplify} \hlstd{=} \hlnum{FALSE}\hlstd{)}
\hlkwd{class}\hlstd{(z)}
\end{alltt}
\begin{verbatim}
## [1] "list"
\end{verbatim}
\begin{alltt}
\hlkwd{dim}\hlstd{(z)}
\end{alltt}
\begin{verbatim}
## NULL
\end{verbatim}
\begin{alltt}
\hlstd{z}
\end{alltt}
\begin{verbatim}
## [[1]]
## [1] 4.774083
## 
## [[2]]
## [1] 4.71706
## 
## [[3]]
## [1] 4.061606
## 
## [[4]]
## [1] 3.925758
## 
## [[5]]
## [1] 3.981937
## 
## [[6]]
## [1] 3.382251
\end{verbatim}
\end{kframe}
\end{knitrout}

Anonymous functions can be defined on the fly and passed to \code{FUN}, allowing us to re-write the examples above more concisely.

\begin{knitrout}\footnotesize
\definecolor{shadecolor}{rgb}{0.969, 0.969, 0.969}\color{fgcolor}\begin{kframe}
\begin{alltt}
\hlkwd{sapply}\hlstd{(}\hlkwc{X} \hlstd{= a.vector,} \hlkwc{FUN} \hlstd{=} \hlkwa{function}\hlstd{(}\hlkwc{x}\hlstd{,} \hlkwc{k}\hlstd{) \{}\hlkwd{log}\hlstd{(x)} \hlopt{+} \hlstd{k\},} \hlkwc{k} \hlstd{=} \hlnum{5}\hlstd{)}
\end{alltt}
\begin{verbatim}
## [1] 4.774083 4.717060 4.061606 3.925758 3.981937 3.382251
\end{verbatim}
\end{kframe}
\end{knitrout}

Of course, as discussed in section \ref{sec:loops:slow} on page \pageref{sec:loops:slow}, when suitable vectorized functions are available, their use results in fast execution and the simplest code. On the other hand, \emph{apply} functions execute much faster than equivalent \code{for()} loops.

\begin{knitrout}\footnotesize
\definecolor{shadecolor}{rgb}{0.969, 0.969, 0.969}\color{fgcolor}\begin{kframe}
\begin{alltt}
\hlkwd{log}\hlstd{(a.vector)} \hlopt{+} \hlnum{5}
\end{alltt}
\begin{verbatim}
## [1] 4.774083 4.717060 4.061606 3.925758 3.981937 3.382251
\end{verbatim}
\end{kframe}
\end{knitrout}

\begin{warningbox}
In \Rlang the different dimensions such as rows and columns in a matrix over which it is summarized are called \emph{margins}. If the function is applied over rows, we say that we operate on the first margin, and if the function is applied over columns, over the second margin. Array can have more dimensions, and consequently more margins. In the case of arrays with more than two dimensions, it is possible and useful to apply functions over multiple margins at once.
\end{warningbox}

In the examples below we use function \Rloop{apply()} which takes a matrix or array as argument for \code{X}. The argument passed to \code{MARGIN} determines, the dimension along which the matrix or array passed to \code{X} will be split before passing it as argument to the function passed through \code{FUN}. In the example below we get either row- or column means. In these examples, \Rfunction{sum()} is passed a vector, for each row or each column of the matrix. As function \Rfunction{sum()} returns a single value independently of the length of its argument, instead of a matrix, the returned value is a vector. More generally, an array with fewer dimensions than the argument passed to \code{X}.

\begin{knitrout}\footnotesize
\definecolor{shadecolor}{rgb}{0.969, 0.969, 0.969}\color{fgcolor}\begin{kframe}
\begin{alltt}
\hlkwd{set.seed}\hlstd{(}\hlnum{123456}\hlstd{)}
\hlstd{a.mat} \hlkwb{<-} \hlkwd{matrix}\hlstd{(}\hlkwd{runif}\hlstd{(}\hlnum{10}\hlstd{),} \hlkwc{ncol} \hlstd{=} \hlnum{2}\hlstd{)}
\hlstd{row.means} \hlkwb{<-} \hlkwd{apply}\hlstd{(}\hlkwc{X} \hlstd{= a.mat,} \hlkwc{MARGIN} \hlstd{=} \hlnum{1}\hlstd{,} \hlkwc{FUN} \hlstd{= mean,} \hlkwc{na.rm} \hlstd{=} \hlnum{TRUE}\hlstd{)}
\hlkwd{class}\hlstd{(row.means)}
\end{alltt}
\begin{verbatim}
## [1] "numeric"
\end{verbatim}
\begin{alltt}
\hlkwd{dim}\hlstd{(row.means)}
\end{alltt}
\begin{verbatim}
## NULL
\end{verbatim}
\begin{alltt}
\hlstd{row.means}
\end{alltt}
\begin{verbatim}
## [1] 0.4980645 0.6442115 0.2438910 0.6647018 0.2644318
\end{verbatim}
\begin{alltt}
\hlstd{col.means} \hlkwb{<-} \hlkwd{apply}\hlstd{(}\hlkwc{X} \hlstd{= a.mat,} \hlkwc{MARGIN} \hlstd{=} \hlnum{2}\hlstd{,} \hlkwc{FUN} \hlstd{= mean,} \hlkwc{na.rm} \hlstd{=} \hlnum{TRUE}\hlstd{)}
\hlkwd{class}\hlstd{(col.means)}
\end{alltt}
\begin{verbatim}
## [1] "numeric"
\end{verbatim}
\begin{alltt}
\hlkwd{dim}\hlstd{(col.means)}
\end{alltt}
\begin{verbatim}
## NULL
\end{verbatim}
\begin{alltt}
\hlstd{col.means}
\end{alltt}
\begin{verbatim}
## [1] 0.5290912 0.3970291
\end{verbatim}
\end{kframe}
\end{knitrout}

\begin{explainbox}
If we apply a function that returns a value of the same length as its input, then the dimensions of the value returned by \Rloop{apply()} are the same as those of its input. We use in the next examples a ``no-op'' function that returns its argument unchanged, so that input and output can be easily compared.

\begin{knitrout}\footnotesize
\definecolor{shadecolor}{rgb}{0.969, 0.969, 0.969}\color{fgcolor}\begin{kframe}
\begin{alltt}
\hlkwd{set.seed}\hlstd{(}\hlnum{123456}\hlstd{)}
\hlstd{a.mat} \hlkwb{<-} \hlkwd{matrix}\hlstd{(}\hlnum{1}\hlopt{:}\hlnum{10}\hlstd{,} \hlkwc{ncol} \hlstd{=} \hlnum{2}\hlstd{)}
\hlstd{no_op.fun} \hlkwb{<-} \hlkwa{function}\hlstd{(}\hlkwc{x}\hlstd{) \{x\}}
\hlstd{b.mat} \hlkwb{<-} \hlkwd{apply}\hlstd{(}\hlkwc{X} \hlstd{= a.mat,} \hlkwc{MARGIN} \hlstd{=} \hlnum{2}\hlstd{,} \hlkwc{FUN} \hlstd{= no_op.fun)}
\hlkwd{class}\hlstd{(b.mat)}
\end{alltt}
\begin{verbatim}
## [1] "matrix"
\end{verbatim}
\begin{alltt}
\hlkwd{dim}\hlstd{(b.mat)}
\end{alltt}
\begin{verbatim}
## [1] 5 2
\end{verbatim}
\begin{alltt}
\hlstd{b.mat}
\end{alltt}
\begin{verbatim}
##      [,1] [,2]
## [1,]    1    6
## [2,]    2    7
## [3,]    3    8
## [4,]    4    9
## [5,]    5   10
\end{verbatim}
\begin{alltt}
\hlkwd{t}\hlstd{(b.mat)}
\end{alltt}
\begin{verbatim}
##      [,1] [,2] [,3] [,4] [,5]
## [1,]    1    2    3    4    5
## [2,]    6    7    8    9   10
\end{verbatim}
\end{kframe}
\end{knitrout}

In the chunk above we passed \code{MARGIN = 2}, but if we pass \code{MARGIN = 1}, we get a return value that is transposed! To restore the original layout of the matrix we can transpose the result with function \Rfunction{t()}.

\begin{knitrout}\footnotesize
\definecolor{shadecolor}{rgb}{0.969, 0.969, 0.969}\color{fgcolor}\begin{kframe}
\begin{alltt}
\hlstd{b.mat} \hlkwb{<-} \hlkwd{apply}\hlstd{(}\hlkwc{X} \hlstd{= a.mat,} \hlkwc{MARGIN} \hlstd{=} \hlnum{1}\hlstd{,} \hlkwc{FUN} \hlstd{= no_op.fun)}
\hlkwd{class}\hlstd{(b.mat)}
\end{alltt}
\begin{verbatim}
## [1] "matrix"
\end{verbatim}
\begin{alltt}
\hlkwd{dim}\hlstd{(b.mat)}
\end{alltt}
\begin{verbatim}
## [1] 2 5
\end{verbatim}
\begin{alltt}
\hlstd{b.mat}
\end{alltt}
\begin{verbatim}
##      [,1] [,2] [,3] [,4] [,5]
## [1,]    1    2    3    4    5
## [2,]    6    7    8    9   10
\end{verbatim}
\end{kframe}
\end{knitrout}

Of course, these two toy examples, are something that can, and should be always avoided, as vectorization allows us to directly apply the function to the whole matrix.

\begin{knitrout}\footnotesize
\definecolor{shadecolor}{rgb}{0.969, 0.969, 0.969}\color{fgcolor}\begin{kframe}
\begin{alltt}
\hlstd{b.mat} \hlkwb{<-} \hlkwd{no_op.fun}\hlstd{(a.mat)}
\end{alltt}
\end{kframe}
\end{knitrout}

A more realistic example, but difficult to grasp without seeing the toy examples shown above, is when we apply a function that returns a value of a different length than its input, but longer than one. If this length is consistent, an array with matching dimensions is returned, but again with the original columns as rows. What happens is that by using \Rloop{apply()} one dimension of the original matrix or array disappears, as we apply the function over it. Consequently, given how matrices are stored in R, when the column dimension disappears, the row dimension becomes the new column dimension. After this, the elements of the vectors returned by the applied function, are stored along rows.

\begin{knitrout}\footnotesize
\definecolor{shadecolor}{rgb}{0.969, 0.969, 0.969}\color{fgcolor}\begin{kframe}
\begin{alltt}
\hlkwd{set.seed}\hlstd{(}\hlnum{123456}\hlstd{)}
\hlstd{a.mat} \hlkwb{<-} \hlkwd{matrix}\hlstd{(}\hlkwd{runif}\hlstd{(}\hlnum{6}\hlstd{),} \hlkwc{ncol} \hlstd{=} \hlnum{2}\hlstd{)}
\hlstd{mean_and_sd} \hlkwb{<-} \hlkwa{function}\hlstd{(}\hlkwc{x}\hlstd{,} \hlkwc{na.rm} \hlstd{=} \hlnum{FALSE}\hlstd{) \{}
       \hlkwd{c}\hlstd{(}\hlkwd{mean}\hlstd{(x,} \hlkwc{na.rm} \hlstd{= na.rm),}  \hlkwd{sd}\hlstd{(x,} \hlkwc{na.rm} \hlstd{= na.rm))}
    \hlstd{\}}
\hlstd{c.mat} \hlkwb{<-} \hlkwd{apply}\hlstd{(}\hlkwc{X} \hlstd{= a.mat,} \hlkwc{MARGIN} \hlstd{=} \hlnum{1}\hlstd{,} \hlkwc{FUN} \hlstd{= mean_and_sd,} \hlkwc{na.rm} \hlstd{=} \hlnum{TRUE}\hlstd{)}
\hlkwd{class}\hlstd{(c.mat)}
\end{alltt}
\begin{verbatim}
## [1] "matrix"
\end{verbatim}
\begin{alltt}
\hlkwd{dim}\hlstd{(c.mat)}
\end{alltt}
\begin{verbatim}
## [1] 2 3
\end{verbatim}
\begin{alltt}
\hlstd{c.mat}
\end{alltt}
\begin{verbatim}
##           [,1]      [,2]      [,3]
## [1,] 0.5696705 0.5574296 0.2948002
## [2,] 0.3226016 0.2773775 0.1364086
\end{verbatim}
\end{kframe}
\end{knitrout}

In this case, calling the user-defined function with the whole matrix as argument is not equivalent as it would operate on the whole matrix at once. Of course, a \code{for} loop stepping through the rows would do the job, but more slowly.

\begin{knitrout}\footnotesize
\definecolor{shadecolor}{rgb}{0.969, 0.969, 0.969}\color{fgcolor}\begin{kframe}
\begin{alltt}
\hlkwd{mean_and_sd}\hlstd{(a.mat)}
\end{alltt}
\begin{verbatim}
## [1] 0.4739668 0.2433389
\end{verbatim}
\end{kframe}
\end{knitrout}
\end{explainbox}

Function \Rloop{vapply()} is safer as the mode of returned values is enforced. Here is a possible way of obtaining means and variances across member vectors at each vector index position from a list of vectors. These could be called \emph{parallel} means and variances. The argument passed to \code{FUN.VALUE} provides a template for the type of the return value and its organization into rows and columns. Notice that the rows in the output are now named according to the names in \code{FUN.VALUE}. (We use \code{lapply()} to create the object \code{a.list} containing artificial data.)

\begin{knitrout}\footnotesize
\definecolor{shadecolor}{rgb}{0.969, 0.969, 0.969}\color{fgcolor}\begin{kframe}
\begin{alltt}
\hlstd{mean_and_sd} \hlkwb{<-} \hlkwa{function}\hlstd{(}\hlkwc{x}\hlstd{,} \hlkwc{na.rm} \hlstd{=} \hlnum{FALSE}\hlstd{) \{}
       \hlkwd{c}\hlstd{(}\hlkwd{mean}\hlstd{(x,} \hlkwc{na.rm} \hlstd{= na.rm),}  \hlkwd{sd}\hlstd{(x,} \hlkwc{na.rm} \hlstd{= na.rm))}
    \hlstd{\}}
\hlkwd{set.seed}\hlstd{(}\hlnum{123456}\hlstd{)}
\hlstd{a.list} \hlkwb{<-} \hlkwd{lapply}\hlstd{(}\hlkwd{rep}\hlstd{(}\hlnum{4}\hlstd{,} \hlnum{5}\hlstd{), runif)}
\hlstd{values} \hlkwb{<-} \hlkwd{vapply}\hlstd{(}\hlkwc{X} \hlstd{= a.list,}
                 \hlkwc{FUN} \hlstd{= mean_and_sd,}
                 \hlkwc{FUN.VALUE} \hlstd{=} \hlkwd{c}\hlstd{(}\hlkwc{mean} \hlstd{=} \hlnum{0}\hlstd{,} \hlkwc{sd} \hlstd{=} \hlnum{0}\hlstd{),}
                 \hlkwc{na.rm} \hlstd{=} \hlnum{TRUE}\hlstd{)}
\hlkwd{class}\hlstd{(values)}
\end{alltt}
\begin{verbatim}
## [1] "matrix"
\end{verbatim}
\begin{alltt}
\hlkwd{dim}\hlstd{(values)}
\end{alltt}
\begin{verbatim}
## [1] 2 5
\end{verbatim}
\begin{alltt}
\hlstd{values}
\end{alltt}
\begin{verbatim}
##           [,1]      [,2]      [,3]       [,4]      [,5]
## mean 0.5710404 0.2977558 0.6367999 0.91898789 0.5471123
## sd   0.2378469 0.1920334 0.3517777 0.05090665 0.3316651
\end{verbatim}
\end{kframe}
\end{knitrout}

In all examples above we have used ordinary functions. Operators in \Rlang are functions with two formal parameters which can be called using infix notation in expressions---i.e. \code{a + b}. By back-quoting their names they can be called using the same syntax as for ordinary functions, and consequently also passed to the \code{FUN} parameter of apply functions. A toy example, equivalent to the vectorized operation \code{a.vector + 5} follows. We enclosed operator \code{+} in back ticks (\code{`}).

\begin{knitrout}\footnotesize
\definecolor{shadecolor}{rgb}{0.969, 0.969, 0.969}\color{fgcolor}\begin{kframe}
\begin{alltt}
\hlkwd{set.seed}\hlstd{(}\hlnum{123456}\hlstd{)} \hlcom{# so that a.vector does not change}
\hlstd{a.vector} \hlkwb{<-} \hlkwd{runif}\hlstd{(}\hlnum{10}\hlstd{)}
\hlstd{z} \hlkwb{<-} \hlkwd{sapply}\hlstd{(}\hlkwc{X} \hlstd{= a.vector,} \hlkwc{FUN} \hlstd{= `+`,} \hlkwc{e2} \hlstd{=} \hlnum{5}\hlstd{)}
\hlkwd{class}\hlstd{(z)}
\end{alltt}
\begin{verbatim}
## [1] "numeric"
\end{verbatim}
\begin{alltt}
\hlkwd{is.vector}\hlstd{(z)}
\end{alltt}
\begin{verbatim}
## [1] TRUE
\end{verbatim}
\begin{alltt}
\hlstd{z}
\end{alltt}
\begin{verbatim}
##  [1] 5.797784 5.753565 5.391256 5.341557 5.361294 5.198345 5.534858
##  [8] 5.096526 5.987847 5.167569
\end{verbatim}
\end{kframe}
\end{knitrout}

\begin{explainbox}
\textbf{Apply functions vs.\ loop constructs} Apply functions can only sometimes replace explicit loops as they lack in flexibility, which is what allows them to be executed faster. We will give some typical examples where apply functions are not usable. First case is the accumulation pattern, where we ``walk'' through a collection storing a partial result between iterations.

\begin{knitrout}\footnotesize
\definecolor{shadecolor}{rgb}{0.969, 0.969, 0.969}\color{fgcolor}\begin{kframe}
\begin{alltt}
\hlkwd{set.seed}\hlstd{(}\hlnum{123456}\hlstd{)}
\hlstd{a.vector} \hlkwb{<-} \hlkwd{runif}\hlstd{(}\hlnum{20}\hlstd{)}
\hlstd{total} \hlkwb{<-} \hlnum{0}
\hlkwa{for} \hlstd{(i} \hlkwa{in} \hlkwd{seq}\hlstd{(}\hlkwc{along.with} \hlstd{= a.vector)) \{}
  \hlstd{total} \hlkwb{<-} \hlstd{total} \hlopt{+} \hlstd{a.vector[i]}
  \hlstd{\}}
\hlstd{total}
\end{alltt}
\begin{verbatim}
## [1] 11.88678
\end{verbatim}
\end{kframe}
\end{knitrout}

Although the loop above cannot the replaced by a statement based on an \emph{apply} function, it can be replaced by the summation function \Rfunction{sum()} from base R.

\begin{knitrout}\footnotesize
\definecolor{shadecolor}{rgb}{0.969, 0.969, 0.969}\color{fgcolor}\begin{kframe}
\begin{alltt}
\hlkwd{set.seed}\hlstd{(}\hlnum{123456}\hlstd{)}
\hlstd{a.vector} \hlkwb{<-} \hlkwd{runif}\hlstd{(}\hlnum{20}\hlstd{)}
\hlstd{total} \hlkwb{<-} \hlkwd{sum}\hlstd{(a.vector)}
\hlstd{total}
\end{alltt}
\begin{verbatim}
## [1] 11.88678
\end{verbatim}
\end{kframe}
\end{knitrout}

Another frequent pattern are operations, at each iteration, on a subset composed by two or more consecutive elements of the collection. The simplest and probably most frequent calculation of this kind is the calculation of differences between successive members.

\begin{knitrout}\footnotesize
\definecolor{shadecolor}{rgb}{0.969, 0.969, 0.969}\color{fgcolor}\begin{kframe}
\begin{alltt}
\hlkwd{set.seed}\hlstd{(}\hlnum{123456}\hlstd{)}
\hlstd{a.vector} \hlkwb{<-} \hlkwd{runif}\hlstd{(}\hlnum{20}\hlstd{)}
\hlstd{b.vector} \hlkwb{<-} \hlkwd{numeric}\hlstd{(}\hlkwd{length}\hlstd{(a.vector)} \hlopt{-} \hlnum{1}\hlstd{)}
\hlkwa{for} \hlstd{(i} \hlkwa{in} \hlkwd{seq}\hlstd{(}\hlkwc{along.with} \hlstd{= b.vector)) \{}
  \hlstd{b.vector[i]} \hlkwb{<-} \hlstd{a.vector[i} \hlopt{+} \hlnum{1}\hlstd{]} \hlopt{-} \hlstd{a.vector[i]}
  \hlstd{\}}
\hlstd{b.vector}
\end{alltt}
\begin{verbatim}
##  [1] -0.04421923 -0.36230941 -0.04969899  0.01973741 -0.16294938
##  [6]  0.33651323 -0.43833172  0.89132070 -0.82027747  0.63041965
## [11] -0.20419511  0.31151599 -0.02446136  0.11298790 -0.09788022
## [16] -0.01731298 -0.68103760  0.13738785  0.44221272
\end{verbatim}
\end{kframe}
\end{knitrout}

In this case, we can use \code{diff()} instead of an explicit loop.

\begin{knitrout}\footnotesize
\definecolor{shadecolor}{rgb}{0.969, 0.969, 0.969}\color{fgcolor}\begin{kframe}
\begin{alltt}
\hlstd{b.vector} \hlkwb{<-} \hlkwd{diff}\hlstd{(a.vector)}
\hlstd{b.vector}
\end{alltt}
\begin{verbatim}
##  [1] -0.04421923 -0.36230941 -0.04969899  0.01973741 -0.16294938
##  [6]  0.33651323 -0.43833172  0.89132070 -0.82027747  0.63041965
## [11] -0.20419511  0.31151599 -0.02446136  0.11298790 -0.09788022
## [16] -0.01731298 -0.68103760  0.13738785  0.44221272
\end{verbatim}
\end{kframe}
\end{knitrout}

Cumulation of values along a vector is another frequent operation for which \Rlang provides functions that can replace explicit \code{for} loops resulting in faster execution. They are \code{cummax()}, \code{cummin()}, \code{cumprod()} and \code{cumsum()}.

\end{explainbox}

\section{Object names and character strings}

In\index{object names}\index{object names!as character strings} all assignment examples before this section, we have given the object names to be assigned to, as part of expressions. In other words, the names are ``decided'' as part of the code, rather that at run time. In scripts or packages, the object name to be assigned to may need to be decided at run time and available only as a character string. In this case function \Rfunction{assign()} must be used instead of the operators \code{<-} or \code{->}. The statements below demonstrate this.

First using a \code{character} constant

\begin{knitrout}\footnotesize
\definecolor{shadecolor}{rgb}{0.969, 0.969, 0.969}\color{fgcolor}\begin{kframe}
\begin{alltt}
\hlkwd{assign}\hlstd{(}\hlstr{"a"}\hlstd{,} \hlnum{9.99}\hlstd{)}
\hlstd{a}
\end{alltt}
\begin{verbatim}
## [1] 9.99
\end{verbatim}
\end{kframe}
\end{knitrout}
next using a \code{character} value stored in a variable.

\begin{knitrout}\footnotesize
\definecolor{shadecolor}{rgb}{0.969, 0.969, 0.969}\color{fgcolor}\begin{kframe}
\begin{alltt}
\hlstd{name.of.var} \hlkwb{<-} \hlstr{"b"}
\hlkwd{assign}\hlstd{(name.of.var,} \hlnum{9.99}\hlstd{)}
\hlstd{b}
\end{alltt}
\begin{verbatim}
## [1] 9.99
\end{verbatim}
\end{kframe}
\end{knitrout}

The two toy examples above do not demonstrate why one may want to use \Rfunction{assign()}. Common situations where we may want to use character strings to store (future or existing) object names are 1) when we allow users to provide names for objects either interactively or as \code{character} data, 2) when in a loop we transverse a vector or list of object names, or 3) we construct at runtime object names from multiple character strings based on data or settings.

Here the \code{character} values are the result of a computation.

\begin{knitrout}\footnotesize
\definecolor{shadecolor}{rgb}{0.969, 0.969, 0.969}\color{fgcolor}\begin{kframe}
\begin{alltt}
\hlkwa{for} \hlstd{(i} \hlkwa{in} \hlnum{1}\hlopt{:}\hlnum{5}\hlstd{) \{}
   \hlkwd{assign}\hlstd{(}\hlkwd{paste}\hlstd{(}\hlstr{"zz_"}\hlstd{, i,} \hlkwc{sep} \hlstd{=} \hlstr{""}\hlstd{), i}\hlopt{^}\hlnum{2}\hlstd{)}
\hlstd{\}}
\hlkwd{ls}\hlstd{(}\hlkwc{pattern} \hlstd{=} \hlstr{"zz_*"}\hlstd{)}
\end{alltt}
\begin{verbatim}
## [1] "zz_1" "zz_2" "zz_3" "zz_4" "zz_5"
\end{verbatim}
\end{kframe}
\end{knitrout}

The complementary operation of \emph{assigning} a name to an object is to \emph{get} an object when we have available its name as a character string. The corresponding function is \Rfunction{get()}.

\begin{knitrout}\footnotesize
\definecolor{shadecolor}{rgb}{0.969, 0.969, 0.969}\color{fgcolor}\begin{kframe}
\begin{alltt}
\hlstd{a} \hlkwb{<-} \hlnum{555}
\hlkwd{get}\hlstd{(}\hlstr{"a"}\hlstd{)}
\end{alltt}
\begin{verbatim}
## [1] 555
\end{verbatim}
\end{kframe}
\end{knitrout}

This can be the case, for example, when we import data from a text file and we want to name the object according to the name of the file on disk.

If we have available a character vector containing object names and we want to create a list containing these objects we can use function \Rfunction{mget()}. In the example below we use function \code{ls()} to obtain a character vector of object names matching a specific pattern and then read all these objects into a list.

\begin{knitrout}\footnotesize
\definecolor{shadecolor}{rgb}{0.969, 0.969, 0.969}\color{fgcolor}\begin{kframe}
\begin{alltt}
\hlstd{obj_names} \hlkwb{<-} \hlkwd{ls}\hlstd{(}\hlkwc{pattern} \hlstd{=} \hlstr{"zz_*"}\hlstd{)}
\hlstd{obj_lst} \hlkwb{<-} \hlkwd{mget}\hlstd{(obj_names)}
\hlkwd{str}\hlstd{(obj_lst)}
\end{alltt}
\begin{verbatim}
## List of 5
##  $ zz_1: num 1
##  $ zz_2: num 4
##  $ zz_3: num 9
##  $ zz_4: num 16
##  $ zz_5: num 25
\end{verbatim}
\end{kframe}
\end{knitrout}

\begin{advplayground}
Think of possible uses of functions \Rfunction{assign()}, \Rfunction{get()} and \Rfunction{mget()} in scripts you use or could use to analyze your own data (or from other sources). Write a script to implement this, and iteratively test and revise this script until the result produced by the script matches your expectations.
\end{advplayground}

\section{The multiple faces of loops}

\ilAdvanced\ To close this chapter I will mention some advanced aspects of the \Rlang language that can be useful when writing complex scrips. In the same way as we can assign names to \code{numeric}, \code{character} and other types of objects we can assign functions and expressions. We can also create lists of functions and/or expressions. The \Rlang language has a very consistent grammar, with all lists and vectors behaving in the same way. The implication of this is that we can assign different functions or expressions to a given name, and consequently it is possible to write loops over lists of functions or expressions.

In this first example we use a character vector of function names, and use function \Rfunction{do.call()} as it accepts either character strings or function names as its first argument.

\begin{knitrout}\footnotesize
\definecolor{shadecolor}{rgb}{0.969, 0.969, 0.969}\color{fgcolor}\begin{kframe}
\begin{alltt}
\hlstd{x} \hlkwb{<-} \hlkwd{rnorm}\hlstd{(}\hlnum{10}\hlstd{)}
\hlstd{results} \hlkwb{<-} \hlkwd{list}\hlstd{()}
\hlstd{fun.names} \hlkwb{<-} \hlkwd{c}\hlstd{(}\hlstr{"mean"}\hlstd{,} \hlstr{"max"}\hlstd{,} \hlstr{"min"}\hlstd{)}
\hlkwa{for} \hlstd{(f} \hlkwa{in} \hlstd{fun.names) \{}
   \hlstd{results[[f]]} \hlkwb{<-} \hlkwd{do.call}\hlstd{(f,} \hlkwd{list}\hlstd{(x))}
   \hlstd{\}}
\hlstd{results}
\end{alltt}
\begin{verbatim}
## $mean
## [1] 0.254283
## 
## $max
## [1] 1.668211
## 
## $min
## [1] -1.11395
\end{verbatim}
\end{kframe}
\end{knitrout}

When traversing a list or vector of functions in a loop we face the problem that we cannot access the original names of the functions as what is stored in the list are the definitions of the functions. We can do the same calculations returning the values in a vector. In this case we could have used \Rfunction{do.call()} as above but is simpler to call them directly using brackets on the loop variable \code{f}.

\begin{knitrout}\footnotesize
\definecolor{shadecolor}{rgb}{0.969, 0.969, 0.969}\color{fgcolor}\begin{kframe}
\begin{alltt}
\hlstd{results} \hlkwb{<-} \hlkwd{numeric}\hlstd{()}
\hlstd{funs} \hlkwb{<-} \hlkwd{list}\hlstd{(mean, max, min)}
\hlkwa{for} \hlstd{(f} \hlkwa{in} \hlstd{funs) \{}
   \hlstd{results} \hlkwb{<-} \hlkwd{c}\hlstd{(results,} \hlkwd{f}\hlstd{(x))}
   \hlstd{\}}
\hlstd{results}
\end{alltt}
\begin{verbatim}
## [1]  0.254283  1.668211 -1.113950
\end{verbatim}
\end{kframe}
\end{knitrout}

We can use a named list of functions or a vector of names of functions to gain full control of the naming of the results.

\begin{knitrout}\footnotesize
\definecolor{shadecolor}{rgb}{0.969, 0.969, 0.969}\color{fgcolor}\begin{kframe}
\begin{alltt}
\hlstd{results} \hlkwb{<-} \hlkwd{list}\hlstd{()}
\hlstd{funs} \hlkwb{<-} \hlkwd{list}\hlstd{(}\hlkwc{average} \hlstd{= mean,} \hlkwc{maximum} \hlstd{= max,} \hlkwc{minimum} \hlstd{= min)}
\hlkwa{for} \hlstd{(f} \hlkwa{in} \hlkwd{names}\hlstd{(funs)) \{}
   \hlstd{results[[f]]} \hlkwb{<-} \hlstd{funs[[f]](x)}
   \hlstd{\}}
\hlstd{results}
\end{alltt}
\begin{verbatim}
## $average
## [1] 0.254283
## 
## $maximum
## [1] 1.668211
## 
## $minimum
## [1] -1.11395
\end{verbatim}
\end{kframe}
\end{knitrout}

Next is an example using model formulas. We cannot pass to \Rfunction{anova()} a list of fitted models, as it expects them as separate arguments, but we can get around this problem using function \Rfunction{do.call()} once again. Function \Rfunction{do.call()} translates the names of members of the list passed as its second argument as names of the arguments to pass to its first argument. \Rfunction{anova()} expects nameless arguments so we need to remove the names.

\begin{knitrout}\footnotesize
\definecolor{shadecolor}{rgb}{0.969, 0.969, 0.969}\color{fgcolor}\begin{kframe}
\begin{alltt}
\hlstd{my.data} \hlkwb{<-} \hlkwd{data.frame}\hlstd{(}\hlkwc{x} \hlstd{=} \hlnum{1}\hlopt{:}\hlnum{10}\hlstd{,} \hlkwc{y} \hlstd{=} \hlnum{1}\hlopt{:}\hlnum{10} \hlopt{+} \hlkwd{rnorm}\hlstd{(}\hlnum{10}\hlstd{,} \hlnum{1}\hlstd{,} \hlnum{0.1}\hlstd{))}
\hlstd{results} \hlkwb{<-} \hlkwd{list}\hlstd{()}
\hlstd{models} \hlkwb{<-} \hlkwd{list}\hlstd{(}\hlkwc{linear} \hlstd{= y} \hlopt{~} \hlstd{x,} \hlkwc{linear.orig} \hlstd{= y} \hlopt{~} \hlstd{x} \hlopt{-} \hlnum{1}\hlstd{,} \hlkwc{quadratic} \hlstd{= y} \hlopt{~} \hlstd{x} \hlopt{+} \hlkwd{I}\hlstd{(x}\hlopt{^}\hlnum{2}\hlstd{))}
\hlkwa{for} \hlstd{(m} \hlkwa{in} \hlkwd{names}\hlstd{(models)) \{}
   \hlstd{results[[m]]} \hlkwb{<-} \hlkwd{lm}\hlstd{(models[[m]],} \hlkwc{data} \hlstd{= my.data)}
   \hlstd{\}}
\hlkwd{do.call}\hlstd{(anova,} \hlkwd{unname}\hlstd{(results))}
\end{alltt}
\begin{verbatim}
## Analysis of Variance Table
## 
## Model 1: y ~ x
## Model 2: y ~ x - 1
## Model 3: y ~ x + I(x^2)
##   Res.Df     RSS Df Sum of Sq      F    Pr(>F)    
## 1      8 0.04066                                  
## 2      9 2.24661 -1   -2.2060 400.67 1.944e-07 ***
## 3      7 0.03854  2    2.2081 200.53 6.612e-07 ***
## ---
## Signif. codes:  0 '***' 0.001 '**' 0.01 '*' 0.05 '.' 0.1 ' ' 1
\end{verbatim}
\end{kframe}
\end{knitrout}

If we had no further use for \code{results} we could simply build a list with nameless members by using positional indexing.

\begin{knitrout}\footnotesize
\definecolor{shadecolor}{rgb}{0.969, 0.969, 0.969}\color{fgcolor}\begin{kframe}
\begin{alltt}
\hlstd{results} \hlkwb{<-} \hlkwd{list}\hlstd{()}
\hlstd{models} \hlkwb{<-} \hlkwd{list}\hlstd{(y} \hlopt{~} \hlstd{x, y} \hlopt{~} \hlstd{x} \hlopt{-} \hlnum{1}\hlstd{, y} \hlopt{~} \hlstd{x} \hlopt{+} \hlkwd{I}\hlstd{(x}\hlopt{^}\hlnum{2}\hlstd{))}
\hlkwa{for} \hlstd{(i} \hlkwa{in} \hlkwd{seq}\hlstd{(}\hlkwc{along.with} \hlstd{= models)) \{}
   \hlstd{results[[i]]} \hlkwb{<-} \hlkwd{lm}\hlstd{(models[[i]],} \hlkwc{data} \hlstd{= my.data)}
   \hlstd{\}}
\hlkwd{do.call}\hlstd{(anova, results)}
\end{alltt}
\begin{verbatim}
## Analysis of Variance Table
## 
## Model 1: y ~ x
## Model 2: y ~ x - 1
## Model 3: y ~ x + I(x^2)
##   Res.Df     RSS Df Sum of Sq      F    Pr(>F)    
## 1      8 0.04066                                  
## 2      9 2.24661 -1   -2.2060 400.67 1.944e-07 ***
## 3      7 0.03854  2    2.2081 200.53 6.612e-07 ***
## ---
## Signif. codes:  0 '***' 0.001 '**' 0.01 '*' 0.05 '.' 0.1 ' ' 1
\end{verbatim}
\end{kframe}
\end{knitrout}

In chapter \ref{chap:R:functions} we will learn how to define new functions and classes---i.e.\ how to add ``new'' words to the \Rlang language, words that we can use both in our scripts and in interactive sessions at the \Rpgrm console. 











\chapter{Further reading about R}\label{chap:R:readings}

\begin{VF}
Before you become too entranced with gorgeous gadgets and mesmerizing video displays, let me remind you that information is not knowledge, knowledge is not wisdom, and wisdom is not foresight. Each grows out of the other, and we need them all.

\VA{Arthur C. Clarke}{Official website at \url{http://arthurcclarke.org}}
\end{VF}

%\dictum[Arthur C. Clarke]{Before you become too entranced with gorgeous gadgets and mesmerizing video displays, let me remind you that information is not knowledge, knowledge is not wisdom, and wisdom is not foresight. Each grows out of the other, and we need them all.}\vskip2ex

\begin{warningbox}
  This list will be expanded and more importantly reorganized and short comments added for book or group of books.
\end{warningbox}

\section{Introductory texts}

\cite{Allerhand2011,Dalgaard2008,Zuur2009,Teetor2011,Peng2017,Paradis2005,Peng2016}

\section{Texts on specific aspects}

\cite{Chang2013,Fox2002,Fox2010,Faraway2004,Faraway2006,Everitt2011,Wickham2017}

\section{Advanced texts}

\cite{Xie2013,Chambers2016,Wickham2015,Wickham2014advanced,Wickham2016,Pinheiro2000,Murrell2011,Matloff2011,Ihaka1996,Venables2000}

\section{Texts for S/R wisdom}

\cite{Burns1998,Burns2011,Burns2012,Bentley1986,Bentley1988}

\backmatter

\printbibliography

\printindex

\printindex[rcatsidx]

\printindex[rindex]

\end{document}

\appendix

\chapter{Build information}

\begin{knitrout}\footnotesize
\definecolor{shadecolor}{rgb}{0.969, 0.969, 0.969}\color{fgcolor}\begin{kframe}
\begin{alltt}
\hlkwd{Sys.info}\hlstd{()}
\end{alltt}
\begin{verbatim}
##        sysname        release        version       nodename        machine 
##      "Windows"       "10 x64"  "build 18362"        "MUSTI"       "x86-64" 
##          login           user effective_user 
##       "aphalo"       "aphalo"       "aphalo"
\end{verbatim}
\end{kframe}
\end{knitrout}



\begin{knitrout}\footnotesize
\definecolor{shadecolor}{rgb}{0.969, 0.969, 0.969}\color{fgcolor}\begin{kframe}
\begin{alltt}
\hlkwd{sessionInfo}\hlstd{()}
\end{alltt}
\begin{verbatim}
## R version 3.6.1 (2019-07-05)
## Platform: x86_64-w64-mingw32/x64 (64-bit)
## Running under: Windows 10 x64 (build 18362)
## 
## Matrix products: default
## 
## locale:
## [1] LC_COLLATE=English_United Kingdom.1252 
## [2] LC_CTYPE=English_United Kingdom.1252   
## [3] LC_MONETARY=English_United Kingdom.1252
## [4] LC_NUMERIC=C                           
## [5] LC_TIME=English_United Kingdom.1252    
## 
## attached base packages:
## [1] tools     stats     graphics  grDevices utils     datasets  methods  
## [8] base     
## 
## other attached packages:
## [1] svglite_1.2.2 stringr_1.4.0 knitr_1.23   
## 
## loaded via a namespace (and not attached):
## [1] compiler_3.6.1 magrittr_1.5   Rcpp_1.0.1     gdtools_0.1.9 
## [5] stringi_1.4.3  highr_0.8      xfun_0.8       evaluate_0.14
\end{verbatim}
\end{kframe}
\end{knitrout}

%

\end{document}


