\documentclass[krantz2]{krantz}\usepackage{knitr}%,ChapterTOCs

%\usepackage[utf8]{inputenc}
\usepackage{color}

\usepackage{polyglossia}
\setdefaultlanguage[variant = british, ordinalmonthday = false]{english}

%\usepackage{gitinfo2} % remember to setup Git hooks

\usepackage{hologo}

\usepackage{csquotes}

\usepackage{graphicx}
\DeclareGraphicsExtensions{.jpg,.pdf,.png}

\usepackage{animate}

%\usepackage{microtype}
\usepackage[style=authoryear-comp,giveninits,sortcites,maxcitenames=2,%
    mincitenames=1,maxbibnames=10,minbibnames=10,backref,uniquename=mininit,%
    uniquelist=minyear,sortgiveninits=true,backend=biber]{biblatex}%,refsection=chapter

\newcommand{\href}[2]{\emph{#2} (\url{#1})}

%\usepackage[unicode,hyperindex,bookmarks,pdfview=FitB,%backref,
%            pdftitle={Learn R ...as you learnt your mother tongue},%
%            pdfkeywords={R, statistics, data analysis, plotting},%
%            pdfsubject={R},%
%            pdfauthor={Pedro J. Aphalo}%
%            ]{hyperref}

%\hypersetup{colorlinks,breaklinks,
%             urlcolor=blue,
%             linkcolor=blue,
%             citecolor=blue,
%             filecolor=blue,
%             menucolor=blue}

\usepackage{framed}

\usepackage{abbrev}
\usepackage{usingr}

\usepackage{imakeidx}

% this is to reduce spacing above and below verbatim, which is used by knitr
% to show returned values
\usepackage{etoolbox}
\makeatletter
\preto{\@verbatim}{\topsep=-5pt \partopsep=-4pt \itemsep=-2pt}
\makeatother

%%% Adjust graphic design

% New float "example" and corresponding "list of examples"
%\DeclareNewTOC[type=example,types=examples,float,counterwithin=chapter]{loe}
%\DeclareNewTOC[name=Box,listname=List of Text Boxes, type=example,types=examples,float,counterwithin=chapter,%
%]{lotxb}

% changing the style of float captions
%\addtokomafont{caption}{\sffamily\small}
%\setkomafont{captionlabel}{\sffamily\bfseries}
%\setcapindent{0em}

% finetuning tocs
%\makeatletter
%\renewcommand*\l@figure{\@dottedtocline{1}{0em}{2.6em}}
%\renewcommand*\l@table{\@dottedtocline{1}{0em}{2.6em}}
%\renewcommand*\l@example{\@dottedtocline{1}{0em}{2.3em}}
%\renewcommand{\@pnumwidth}{2.66em}
%\makeatother
%
%% add pdf bookmarks to tocs
%\makeatletter
%\BeforeTOCHead{%
%  \cleardoublepage
%    \edef\@tempa{%
%      \noexpand\pdfbookmark[0]{\list@fname}{\@currext}%
%    }\@tempa
%}

\setcounter{topnumber}{3}
\setcounter{bottomnumber}{3}
\setcounter{totalnumber}{4}
\renewcommand{\topfraction}{0.90}
\renewcommand{\bottomfraction}{0.90}
\renewcommand{\textfraction}{0.10}
\renewcommand{\floatpagefraction}{0.70}
\renewcommand{\dbltopfraction}{0.90}
\renewcommand{\dblfloatpagefraction}{0.70}

\addbibresource{rbooks.bib}
\addbibresource{references.bib}

\makeindex
\IfFileExists{upquote.sty}{\usepackage{upquote}}{}
\begin{document}

% customize chapter format:
%\KOMAoption{headings}{twolinechapter}
%\renewcommand*\chapterformat{\thechapter\autodot\hspace{1em}}

% customize dictum format:
%\setkomafont{dictumtext}{\itshape\small}
%\setkomafont{dictumauthor}{\normalfont}
%\renewcommand*\dictumwidth{0.7\linewidth}
%\renewcommand*\dictumauthorformat[1]{--- #1}
%\renewcommand*\dictumrule{}

%\extratitle{\vspace*{2\baselineskip}%
%             {\Huge\textsf{\textbf{Learn R}\\ \textsl{\huge\ldots as you learnt your mother tongue}}}}

\title{\Huge{\fontseries{ub}\sffamily Learn R\\{\Large\ldots as you learnt your mother tongue}}}

%\subtitle{Git hash: \gitAbbrevHash; Git date: \gitAuthorIsoDate}

\author{Pedro J. Aphalo}

\date{Helsinki, \today}

%\publishers{Draft, 95\% done\\Available through \href{https://leanpub.com/learnr}{Leanpub}}

%\uppertitleback{\copyright\ 2001--2017 by Pedro J. Aphalo\\
%Licensed under one of the \href{http://creativecommons.org/licenses/}{Creative Commons licenses} as indicated, or when not explicitly indicated, under the \href{http://creativecommons.org/licenses/by-sa/4.0/}{CC BY-SA 4.0 license}.}
%
%\lowertitleback{Typeset with \href{http://www.latex-project.org/}{\hologo{XeLaTeX}}\ in Lucida Bright and \textsf{Lucida Sans} using the KOMA-Script book class.\\
%The manuscript was written using \href{http://www.r-project.org/}{R} with package knitr. The manuscript was edited in \href{http://www.winedt.com/}{WinEdt} and \href{http://www.rstudio.com/}{RStudio}.
%The source files for the whole book are available at \url{https://bitbucket.org/aphalo/using-r}.}

%\frontmatter

% knitr setup















% \thispagestyle{empty}
% \titleLL
% \clearpage

\frontmatter

\maketitle

%\begin{titlingpage}
%  \maketitle
%\titleLL
%\end{titlingpage}

\setcounter{page}{7} %previous pages will be reserved for frontmatter to be added in later.
\tableofcontents
%\include{frontmatter/foreword}
\chapter*{Preface}

\begin{VF}
``Suppose that you want to teach the `cat' concept to a very young child. Do you explain that a cat is a relatively small, primarily carnivorous mammal with retractible claws, a distinctive sonic output, etc.? I'll bet not. You probably show the kid a lot of different cats, saying `kitty' each time, until it gets the idea. To put it more generally, generalizations are best made by abstraction from experience.''

\VA{R. P. Boas}{Can we make mathematics intelligible?}
\end{VF}

%\dictum[R. P. Boas (1981) Can we make mathematics intelligible?, \emph{American Mathematical Monthly} \textbf{88:} 727-731.]{"Suppose that you want to teach the `cat' concept to a very young child. Do you explain that a cat is a relatively small, primarily carnivorous mammal with retractible claws, a distinctive sonic output, etc.? I'll bet not. You probably show the kid a lot of different cats, saying `kitty' each time, until it gets the idea. To put it more generally, generalizations are best made by abstraction from experience."}


% Such pauses are not a miss use of our time. To learn a natural language we need to interact with other speakers, we need feedback. In the case of R, we can get feedback both from the outcomes from our utterances to the computer, and from other R users.}

\vspace{2ex}This book covers different aspects of the use of \Rpgrm. It is meant to be used as a tutorial complementing a reference book about \R, or the documentation that accompanies R and the many packages used in the examples. Explanations are rather short and terse, so as to encourage the development of a routine of exploration. This is not an arbitrary decision, this is the normal \emph{modus operandi} of most of us who use R regularly for a variety of different problems.

I do not discuss here statistics, just \Rpgrm as a tool and language for data manipulation and display. The idea is for you to learn the \Rpgrm language like children learn a language: they work-out what the rules are, simply by listening to people speak and trying to utter what they want to tell their parents. Instead of listening, you will read and execute on a computer \Rlang code statements, try your hand at telling \Rlang what you want it to compute. I do provide explanations and comments, but the idea of these book is mainly for you to use the numerous examples to find-out by yourself the overall patterns and coding philosophy behind the \Rlang language. Instead of parents being the sound board for your first utterances in \langname{R}, the computer will play this role. You will \emph{play} by modifying the examples, see how the computer responds, does \Rlang understand you or not?

When teaching I tend to lean towards challenging students rather than telling a simplified story. I do the same here, because it is what I prefer as a student, and how I learn best myself. Not everybody learns best with the same approach, for me the most limiting factor is for what I listen to, or read, to be in a way or another challenging or entertaining enough to keep my thoughts focused. This I achieve best when making an effort to understand the contents or to follow the thread or plot of a story. So, be warned, reading this book will be about exploring a new world, this book aims to be a travel guide, neither a traveler's account, nor a cookbook of R recipes.

Keep in mind that it is impossible to remember everything about \Rpgrm! \Rpgrm in a broad sense is vast because its capabilities can be expanded with independently developed packages. Learning to use \Rlang consists in learning the basics plus developing the skill of finding your way in \Rlang and its documentation.  In 2017 the number packages available for free in the Comprehensive R Archive Network (CRAN) broke the 10\,000 barrier. CRAN is the most important, but not only, public repository for R packages. How good a command of the \Rlang language and packages a user needs depends on the type activities to be carried out. This book attempts to train you in the use of the \Rlang language itself and some packages that provide extensions for data manipulation and graphical display which are broadly useful. Given the availability of numerous books on statistical analysis with \Rlang, here we will cover only the bare minimum. The same is true for package development in \Rlang. This book seats in-between, aiming at teaching programming in-the-small: the use \Rlang to automate the drudgery of data manipulation from raw data, through data exploration to the production of publication quality illustrations.

As with all ``rich'' languages there are many different ways of doing things in R, and there is no one-size-fits-all solution to a problem. There is always a compromise involved, usually between time spent by the user and processing time required in the computer. Many of the packages that are most popular nowadays did not exist when I started using R, and many of these packages make new approaches available. One could write many different \Rlang books with a given aim and still use substantially different ways of achieving the same results. In this book, I limit myself to packages that are currently popular and/or that I consider elegantly designed. I have in particular tried to limit myself to packages with similar design philosophies, especially in relation to their interfaces. What is elegant design, and in particular what is a friendly user interface depends strongly on each user's preferences and previous experience. Consequently, the contents of the book are strongly biased by my own preferences. I have tried to write examples in ways that execute fast without compromising readability. I encourage readers to take this book as a travel guide, as a starting point for exploring the very many packages, styles and approaches which I have not described.

I will appreciate suggestions for further examples, notification of errors and unclear sections. Many of the examples here have been collected from diverse sources over many years and because of this not all sources are acknowledged. If you recognize any example as yours or someone else's please let me know so that I can add a proper acknowledgement. I warmly thank the students that over the years have asked the questions and posed the problems that have helped me write this text and correct the mistakes and voids of previous versions. I have also received help on on-line forums and in person from numerous people, learnt from archived e-mail list messages, blog posts, books, articles, tutorials, webinars, and by struggling to solve some new problems on my own. In many ways this text owes much more to people who are not authors than to myself. However, as I am the one who has written this version and decided what to include and exclude, as author, I take full responsibility for any errors and inaccuracies.

I have been using \Rpgrm since around 1998 or 1999, but I am still constantly learning new things about \Rpgrm itself and \Rpgrm packages. With time it has replaced in my work as a researcher and teacher several other pieces of software: \pgrmname{SPSS}, \pgrmname{Systat}, \pgrmname{Origin}, \pgrmname{Excel}, and it has become a central piece of the tool set I use for producing lecture slides, notes, books and even web pages. This is to say that it is the most useful piece of software and programming language I have ever learnt to use. Of course, in time it will be replaced by something better, but at the moment it is the ``hot'' thing to learn for anybody with a need to analyse and display data.

\begin{framed}
\noindent\large%
\textbf{I encourage you to approach R, like a child approaches his or hers mother tongue when learning to speak:} Do not struggle, just play! If going gets difficult and frustrating, take a break! If you get a new insight, take a break to enjoy the victory!
\end{framed}

\newpage

\begin{framed}
\noindent
\textbf{Icons used to mark different content.} Throughout the book text boxes marked with icons present different types of information. First of all, we have \emph{playground} boxes indicated with \playicon\ which contain open-ended exercises---ideas and pieces of R code to play with at the R console. A few of these will require more time to grasp, and are indicated with \advplayicon. Boxes providing general information, usually not directly related to \langname{R} as a language, are indicated with \infoicon. Some boxes highlighted with \ilAttention\ give important bits of information that must be remembered when using \langname{R}---i.e.\ explain some unusual feature of the language. Finally, some boxes indicated by \ilAdvanced\ give in depth explanations, that may require you to spend time thinking, which en general can be skipped on first reading, but to which you should return at a later, and peaceful, time with a cup of coffee or tea.
\end{framed}
\newpage

%\newpage
%\begin{infobox}
%\noindent
%\textbf{Status as of 2016-11-23.} I have updated the manuscript to track package updates since the previous version uploaded six months ago, and added several examples of the new functionality added to packages \ggpmisc, \ggrepel, and \ggplot. I have written new sections on packages \viridis, \pkgname{gganimate}, \pkgname{ggstance}, \pkgname{ggbiplot}, \pkgname{ggforce}, \pkgname{ggtern} and \pkgname{ggalt}. Some of these sections are to be expanded, and additional sections are planned for other recently released packages.
%
%With respect to the chapter \textit{Storing and manipulating data with R} I have put it on hold, except for the introduction, until I can see a soon to be published book covering the same subject. Hadley Wickham has named the set of tools developed by him and his collaborators as \textit{tidyverse} to be described in the book titled \textit{R for Data Science} by Grolemund and Wickham (O'Reilly).
%
%An important update to \ggplot was released last week, and it includes changes to the behavior of some existing functions, specially faceting has become extensible through other packages. Several of the new facilities are described in the updated text and code included in this book and this pdf has been generated with up-to-date version of \ggplot and packages as available today from CRAN, except for \pkgname{ggtern} which was downloaded from Bitbucket minutes ago.
%
%The present update adds about 100 pages to the previous versions. I expect to upload a new update to this manuscript in one or two months time.
%
%\textbf{Status as of 2017-01-17.} Added ``playground'' exercises to the chapter describing \ggplot, and converted some of the examples earlier part of the main text into these playground items. Added icons to help readers quickly distinguish playground sections (\textcolor{blue}{\noticestd{"0055}}), information sections (\textcolor{blue}{\modpicts{"003D}}), warnings about things one needs to be specially aware of (\colorbox{yellow}{\typicons{"E136}}) and boxes with more advanced content that may require longer time/more effort to grasp (\typicons{"E04E}). Added to the sections \code{scales} and examples in the \ggplot chapter details about the use of colors in R and \ggplot2. Removed some redundant examples, and updated the section on \code{plotmath}. Added terms to the alphabetical index. Increased line-spacing to avoid uneven spacing with inline code bits.
%
%\textbf{Status as of 2017-02-09.} Wrote section on ggplot2 themes, and on using system- and Google fonts in ggpplots with the help of package \pkgname{showtext}. Expanded section on \ggplot's \code{annotation}, and revised some sections in the ``R scripts and Programming'' chapter. Started writing the data chapter. Wrote draft on writing and reading text files. Several other smaller edits to text and a few new examples.
%
%\textbf{Status as of 2017-02-14.} Wrote sections on reading and writing MS-Excel files, files from statistical programs such as SPSS, SyStat, etc., and NetCDF files. Also wrote sections on using URLs to directly read data, and on reading HTML and XML files directly, as well on using JSON to retrieve measured/logged data from IoT (internet of things) and similar intelligent physical sensors, micro-controller boards and sensor hubs with network access.
%
%\textbf{Status as of 2017-03-25.} Revised and expanded the chapter on plotting maps, adding a section on the manipulation and plotting of image data. Revised and expanded the chapter on extensions to \pkgname{ggplot2}, so that there are no longer empty sections. Wrote short chapter ``If and when R needs help''. Revised and expanded the ``Introduction'' chapter. Added index entries, and additional citations to literature.
%
%\textbf{Status as of 2017-04-04.} Revised and expanded the chapter on using \Rpgrm as a calculator. Revised and expanded the ``Scripts'' chapter. Minor edits to ``Functions'' chapter. Continued writing chapter on data, writing a section on R's native apply functions and added preliminary text for a pipes and tees section. Write intro to `tidyverse' and grammar of data manipulation. Added index entries, and a few additional citations to the literature. Spell checking.
%
%\textbf{Status as of 2017-04-08.} Completed writing first draft of chapter on data, writing all the previously missing sections on the ``grammar of data manipulation''. Wrote two extended examples in the same chapter. Add table listing several extensions to \pkgname{ggplot2} not described in the book.
%
%\textbf{Status as of 2017-04-13.} Revised all chapters correcting some spelling mistakes, adding some explanatory text and indexing all functions and operators used. Thoroughly revised the Introduction chapter and the Preface. Expanded section on bar plots (now bar and column plots). Revised section on tile plots. Expanded section on factors in chapter 2, adding examples of reordering of factor labels, and making clearer the difference between the labels of the levels and the levels themselves.
%
%\textbf{Status as of 2017-04-29.} Tested with R 3.4.0. Package \pkgname{gganimate} needs to be installed from Github as the updated version is not yet in CRAN. Function \code{gg\_animate()} has been renamed \code{gganimate().}
%
%\textbf{Status as of 2017-05-14.} Submitted package \pkgname{learnrbook} to CRAN. Revised code in the book
%to use this new package. Small fixes after more testing. Added examples of plotting and labeling based on fits with \code{method = "nls"}, including use of the new \code{ggpmisc::stat\_fit\_tidy()}.
%
%\textbf{Status as of 2017-06-11.} Added sections on R-code bench marking and profiling for performance optimization. Added also an example of explicit compilation of a function defined in the R language. Added section on functions \code{assign()}, \code{get()} and \code{mget()}.
%
%\textbf{Status as of 2017-08-12.} Various edits to all chapters. Expanded section on \pkgname{ggpmisc} to include the new functionality added in version 0.2.15.9002: \code{geom\_table} and \code{stat\_fit\_tb}. Added section on package \pkgname{ggbeeswarm}. Added sections on packages \pkgname{magick} and on using \pgrmname{ImageJ} from \Rpgrm. Improved indexing and cross references.
%
%\textbf{Status as of 2017-10-25.} Edited the chapter on using R as a calculator, adding examples on insertion and deletion of members of lists and vectors, and also of use of \code{gl()} and \code{reorder()}. Edited sections on scale limits and added new section on coordinate limits to explain more thoroughly their differences and uses in chapter on plotting with \pkgname{ggplot2}. Added a section on package \pkgname{ggsignif} to the chapter on extensions to \pkgname{ggplot2}. Expanded section on \pkgname{ggpmisc} in the same chapter describing new functionality added in version 0.2.16.
%\pkgname{ggplo2} $>=$ 2.2.1.9000 is required by the current development version of \pkgname{ggpmisc}.
%
%\textbf{Status as of 2017-10-30.}  Add section on using pipes with \code{ggplot()} and layers.
%\end{infobox} 
\listoffigures
\listoftables
%\include{frontmatter/contributor}
%\include{frontmatter/symbollist}

\mainmatter








% !Rnw root = appendix.main.Rnw


\chapter{The R language: Statistics}\label{chap:R:functions}

\begin{VF}
The purpose of computing is insight, not numbers.

\VA{Richard W. Hamming}{Numerical Methods for Scientists and Engineers, 1962}
\end{VF}

\section{Aims of this chapter}

In this chapter you will learn the approach used in \Rlang for fitting models and doing tests of significance. We will use linear models, \emph{t}-test and linear correlation as examples. In this chapter I aim at explaining how to specify models, contrasts and data used, and how to access different components of the returned values.

This chapter is designed to give the reader only a quick introduction to Statistics in base \Rpgrm, as there are many good texts on the use of \Rpgrm for different kinds of statistical analyses. Three good examples of books with a broad scope and of moderate size are \citetitle{Dalgaard2008} \autocite{Dalgaard2008}, \citetitle{Everitt2009} \autocite{Everitt2009} and \citetitle{??} \autocite{??}. The book \citetitle{Crawley2012} \autocite{Crawley2012} is comprehensive in the scope of statistical procedures described. Furthermore, many of base \R's functions are specific to different statistical procedures, maths and calculus, that transcend the description of \Rlang as a programming language. The use of \Rpgrm for the analysis of different kinds of data and using different methods is covered by a vast bibliography, to which we provide some pointers in chapter \ref{chap:R:readings} on page \pageref{chap:R:readings}.

Along the chapter, I will show occasionally the equivalent of \Rlang code in mathematical notation. If you are not familiar with the mathematical notation, you can safely ignore it, as long as you understand the \Rlang code.

\section{Correlation}

Both parametric and non-parametric robust methods for the estimation of the (linear) correlation between pairs of variables can be easily computed in \Rlang.

\subsection{Pearson's $r$}

\begin{knitrout}\footnotesize
\definecolor{shadecolor}{rgb}{0.969, 0.969, 0.969}\color{fgcolor}\begin{kframe}
\begin{alltt}
\hlkwd{cor}\hlstd{(cars)}
\end{alltt}
\begin{verbatim}
##           speed      dist
## speed 1.0000000 0.8068949
## dist  0.8068949 1.0000000
\end{verbatim}
\end{kframe}
\end{knitrout}

\begin{knitrout}\footnotesize
\definecolor{shadecolor}{rgb}{0.969, 0.969, 0.969}\color{fgcolor}\begin{kframe}
\begin{alltt}
\hlkwd{cor}\hlstd{(}\hlkwc{x} \hlstd{= cars}\hlopt{$}\hlstd{speed,} \hlkwc{y} \hlstd{= cars}\hlopt{$}\hlstd{dist)}
\end{alltt}
\begin{verbatim}
## [1] 0.8068949
\end{verbatim}
\end{kframe}
\end{knitrout}

\begin{knitrout}\footnotesize
\definecolor{shadecolor}{rgb}{0.969, 0.969, 0.969}\color{fgcolor}\begin{kframe}
\begin{alltt}
\hlkwd{cor.test}\hlstd{(}\hlkwc{x} \hlstd{= cars}\hlopt{$}\hlstd{speed,} \hlkwc{y} \hlstd{= cars}\hlopt{$}\hlstd{dist)}
\end{alltt}
\begin{verbatim}
## 
## 	Pearson's product-moment correlation
## 
## data:  cars$speed and cars$dist
## t = 9.464, df = 48, p-value = 1.49e-12
## alternative hypothesis: true correlation is not equal to 0
## 95 percent confidence interval:
##  0.6816422 0.8862036
## sample estimates:
##       cor 
## 0.8068949
\end{verbatim}
\end{kframe}
\end{knitrout}

\subsection{Kendall's $\tau$ and Spearman's $\rho$}

We use the same functions as for Pearson's $r$ but explicitly request the use of one of these methods.

\begin{knitrout}\footnotesize
\definecolor{shadecolor}{rgb}{0.969, 0.969, 0.969}\color{fgcolor}\begin{kframe}
\begin{alltt}
\hlkwd{cor}\hlstd{(}\hlkwc{x} \hlstd{= cars}\hlopt{$}\hlstd{speed,} \hlkwc{y} \hlstd{= cars}\hlopt{$}\hlstd{dist,} \hlkwc{method} \hlstd{=} \hlstr{"kendall"}\hlstd{)}
\end{alltt}
\begin{verbatim}
## [1] 0.6689901
\end{verbatim}
\begin{alltt}
\hlkwd{cor}\hlstd{(}\hlkwc{x} \hlstd{= cars}\hlopt{$}\hlstd{speed,} \hlkwc{y} \hlstd{= cars}\hlopt{$}\hlstd{dist,} \hlkwc{method} \hlstd{=} \hlstr{"spearman"}\hlstd{)}
\end{alltt}
\begin{verbatim}
## [1] 0.8303568
\end{verbatim}
\end{kframe}
\end{knitrout}

Function \code{cor.test()} allows the choice of the method using the same syntax as for \code{cor()}.

\section{Fitting linear models}
\index{models!linear|see{linear models}}
\index{linear models|(}
\index{LM|see{linear models}}

In \Rlang, the models to be fitted are described by ``model formulas'' such \verb|y ~ x| which we read as $y$ is explained by $x$. Model `formulas' are used in different contexts: fitting of models, plotting, and tests like $t$-test. The syntax of model formulas is consistent throughout base \Rlang and numerous independently developed packages. However, their use is not universal, and several packages extend the basic syntax to allow the description of specific types of models.

As most things in \Rlang model formulas can be stored in variables. In addition, contrary to the usual behaviour of other statistical software, the result of a model fit is returned as an object, containing the different components of the fit. Once the model has been fitted, different methods allow us to extract parts and/or further manipulate the result of fitting the model. Most of these methods have implementations for model fit objects for many different types of statistical models. Consequently what is described in this chapter using linear models as examples, also applies in many respects to the fit of classes of models not described here.

The \Rlang function \Rfunction{lm()} is used next to fit linear models. If the explanatory variable is continuous, the fit is a regression. If the explanatory variable is a factor, the fit is an analysis of variance (ANOVA) in broad terms. However, there is another meaning of ANOVA, referring only to the tests of significance rather to an approach to model fitting. Consequently, rather confusingly, results for tests of significance for fitted parameter estimates can both in the case of regression and ANOVA, be presented in an ANOVA table. In this second, stricter meaning, ANOVA means a test of significance based on the ratio between two variances.

\begin{warningbox}
  If you do not clearly remember the difference between numeric vectors and factors, or how they can be created, please, revisit Chapter \ref{chap:R:as:calc} on page \pageref{chap:R:as:calc}.
\end{warningbox}

\subsection{Regression}
\index{linear regression}\index{linear models!linear regression}
In the example immediately below, \code{speed} is a continuous numeric variable. In the ANOVA table calculated for the model fit, in this case a linear regression, we can see that the term for \code{speed} has only one degree of freedom (df) for the denominator.

We first load and explore the data to which we will fit a linear model, data that is included in \Rpgrm. These data consist in stopping distances for cars moving at different speeds as described in the documentation available by entering \code{help(cars)}).
\label{xmpl:fun:lm:fm1}

\begin{knitrout}\footnotesize
\definecolor{shadecolor}{rgb}{0.969, 0.969, 0.969}\color{fgcolor}\begin{kframe}
\begin{alltt}
\hlkwd{data}\hlstd{(cars)}
\hlkwd{plot}\hlstd{(cars)}
\hlkwd{is.factor}\hlstd{(cars}\hlopt{$}\hlstd{speed)}
\end{alltt}
\begin{verbatim}
## [1] FALSE
\end{verbatim}
\begin{alltt}
\hlkwd{is.numeric}\hlstd{(cars}\hlopt{$}\hlstd{speed)}
\end{alltt}
\begin{verbatim}
## [1] TRUE
\end{verbatim}
\end{kframe}

{\centering \includegraphics[width=.76\textwidth]{figure/pos-models-1z-1} 

}



\end{knitrout}

We then fit the simple linear model $y = \alpha \cdot 1 + \beta \cdot x$ where $y$ corresponds to stopping distance (\code{dist}) and $x$ to initial speed (\code{speed}). Such a model is formulated in \Rlang as \verb|dist ~ 1 + speed|. We save the fitted model as \code{fm1} (a mnemonic for fitted-model one).%
\begin{knitrout}\footnotesize
\definecolor{shadecolor}{rgb}{0.969, 0.969, 0.969}\color{fgcolor}\begin{kframe}
\begin{alltt}
\hlstd{fm1} \hlkwb{<-} \hlkwd{lm}\hlstd{(dist} \hlopt{~} \hlnum{1} \hlopt{+} \hlstd{speed,} \hlkwc{data}\hlstd{=cars)}
\end{alltt}
\end{kframe}
\end{knitrout}

The next step is diagnosis of the fit. Are assumptions of the linear model procedure used reasonably fulfilled? In \Rlang it is most common to use plots to this end. We show here only one of the four plots normally produced. This quantile vs.\ quantile plot allows to assess how much the residuals deviate from being normally distributed.

\begin{knitrout}\footnotesize
\definecolor{shadecolor}{rgb}{0.969, 0.969, 0.969}\color{fgcolor}\begin{kframe}
\begin{alltt}
\hlkwd{plot}\hlstd{(fm1,} \hlkwc{which} \hlstd{=} \hlnum{2}\hlstd{)}
\end{alltt}
\end{kframe}

{\centering \includegraphics[width=.76\textwidth]{figure/pos-models-1a-1} 

}



\end{knitrout}

In the case of a regression, calling \Rfunction{summary()} with the fitted model object as argument is most useful as it provides a table of coefficient estimates and their errors. 
\begin{knitrout}\footnotesize
\definecolor{shadecolor}{rgb}{0.969, 0.969, 0.969}\color{fgcolor}\begin{kframe}
\begin{alltt}
\hlkwd{summary}\hlstd{(fm1)}
\end{alltt}
\begin{verbatim}
## 
## Call:
## lm(formula = dist ~ 1 + speed, data = cars)
## 
## Residuals:
##     Min      1Q  Median      3Q     Max 
## -29.069  -9.525  -2.272   9.215  43.201 
## 
## Coefficients:
##             Estimate Std. Error t value Pr(>|t|)    
## (Intercept) -17.5791     6.7584  -2.601   0.0123 *  
## speed         3.9324     0.4155   9.464 1.49e-12 ***
## ---
## Signif. codes:  0 '***' 0.001 '**' 0.01 '*' 0.05 '.' 0.1 ' ' 1
## 
## Residual standard error: 15.38 on 48 degrees of freedom
## Multiple R-squared:  0.6511,	Adjusted R-squared:  0.6438 
## F-statistic: 89.57 on 1 and 48 DF,  p-value: 1.49e-12
\end{verbatim}
\end{kframe}
\end{knitrout}

Let's look at the summary, section by section. Under ``Call:'' we find, \verb|dist ~ 1 + speed| or the specification of the model fitted, plus the data used. Under ``Residuals:'' we find the extremes, quartiles and median of the residuals, or deviations between observations and the fitted line. Under ``Coefficients:'' we find the estimates of the model parameters and their variation plus corresponding $t$-tests. At the end of summary there is information on degrees of freedom and overall coefficient of determination ($R^2$).

If we return to the model formulation, we can now replace $\alpha$ and $\beta$ by the estimates obtaining $y = -17.6 + 3.93 x$. Given the nature of the problem, we \emph{know based on first principles} that stopping distance must be zero when speed is zero. This suggests that we should not estimate the value of $\alpha$ but instead set $\alpha = 0$, or in other words fit the model $y = \beta \cdot x$.

However, in \Rlang models the intercept is always implicitly included, so the model fitted above can be formulated as \verb|dist ~ speed|---i.e.\ a missing \code{+1} does not change the model. To `remove' the intercept from the earlier model we need to use \verb|dist ~ speed - 1|, resulting in the fitting of a straight line through the origin ($x = 0$, $y = 0$).

\begin{knitrout}\footnotesize
\definecolor{shadecolor}{rgb}{0.969, 0.969, 0.969}\color{fgcolor}\begin{kframe}
\begin{alltt}
\hlstd{fm2} \hlkwb{<-} \hlkwd{lm}\hlstd{(dist} \hlopt{~} \hlstd{speed} \hlopt{-} \hlnum{1}\hlstd{,} \hlkwc{data}\hlstd{=cars)}
\hlkwd{summary}\hlstd{(fm2)}
\end{alltt}
\begin{verbatim}
## 
## Call:
## lm(formula = dist ~ speed - 1, data = cars)
## 
## Residuals:
##     Min      1Q  Median      3Q     Max 
## -26.183 -12.637  -5.455   4.590  50.181 
## 
## Coefficients:
##       Estimate Std. Error t value Pr(>|t|)    
## speed   2.9091     0.1414   20.58   <2e-16 ***
## ---
## Signif. codes:  0 '***' 0.001 '**' 0.01 '*' 0.05 '.' 0.1 ' ' 1
## 
## Residual standard error: 16.26 on 49 degrees of freedom
## Multiple R-squared:  0.8963,	Adjusted R-squared:  0.8942 
## F-statistic: 423.5 on 1 and 49 DF,  p-value: < 2.2e-16
\end{verbatim}
\end{kframe}
\end{knitrout}

Now there is no estimate for the intercept in the summary, only an estimate for the slope. The equation of the second fitted model is $y = 2.91 x$, and from the residuals, it can be seen that it is inadequate, as the straight line does not follow the curvature of the relationship between \code{dist} and \code{speed}.

\begin{knitrout}\footnotesize
\definecolor{shadecolor}{rgb}{0.969, 0.969, 0.969}\color{fgcolor}\begin{kframe}
\begin{alltt}
\hlkwd{plot}\hlstd{(fm2,} \hlkwc{which} \hlstd{=} \hlnum{1}\hlstd{)}
\end{alltt}
\end{kframe}

{\centering \includegraphics[width=.76\textwidth]{figure/pos-models-2a-1} 

}



\end{knitrout}

\begin{playground}
You will now fit a second degree polynomial\index{linear models!polynomial regression}\index{polynomial regression}, a different linear model: $y = \alpha \cdot 1 + \beta_1 \cdot x + \beta_2 \cdot x^2$. The function used is the same as for linear regression, \Rfunction{lm()}. We only need to alter the formulation of the model. The identity function \Rfunction{I()} is used to protect its argument from being interpreted as part of the model formula. Instead, its argument is evaluated beforehand and the result is used as the, in this case second, explanatory variable.

\begin{knitrout}\footnotesize
\definecolor{shadecolor}{rgb}{0.969, 0.969, 0.969}\color{fgcolor}\begin{kframe}
\begin{alltt}
\hlstd{fm3} \hlkwb{<-} \hlkwd{lm}\hlstd{(dist} \hlopt{~} \hlstd{speed} \hlopt{+} \hlkwd{I}\hlstd{(speed}\hlopt{^}\hlnum{2}\hlstd{),} \hlkwc{data} \hlstd{= cars)}
\hlkwd{plot}\hlstd{(fm3,} \hlkwc{which} \hlstd{=} \hlnum{3}\hlstd{)}
\hlkwd{summary}\hlstd{(fm3)}
\hlkwd{anova}\hlstd{(fm3)}
\end{alltt}
\end{kframe}
\end{knitrout}

The ``same'' fit using an orthogonal polynomial. Higher degrees can be obtained by supplying as second argument to \Rfunction{poly()} a different positive integer value.

\begin{knitrout}\footnotesize
\definecolor{shadecolor}{rgb}{0.969, 0.969, 0.969}\color{fgcolor}\begin{kframe}
\begin{alltt}
\hlstd{fm3a} \hlkwb{<-} \hlkwd{lm}\hlstd{(dist} \hlopt{~} \hlkwd{poly}\hlstd{(speed,} \hlnum{2}\hlstd{),} \hlkwc{data}\hlstd{=cars)}
\hlkwd{summary}\hlstd{(fm3a)}
\hlkwd{anova}\hlstd{(fm3a)}
\end{alltt}
\end{kframe}
\end{knitrout}

We can also compare two model fits using \Rfunction{anova()}, to test whether one of models describes the data better than the other.

\begin{knitrout}\footnotesize
\definecolor{shadecolor}{rgb}{0.969, 0.969, 0.969}\color{fgcolor}\begin{kframe}
\begin{alltt}
\hlkwd{anova}\hlstd{(fm2, fm1)}
\end{alltt}
\end{kframe}
\end{knitrout}

Or three or more models. But be careful, as the order of the arguments matters.

\begin{knitrout}\footnotesize
\definecolor{shadecolor}{rgb}{0.969, 0.969, 0.969}\color{fgcolor}\begin{kframe}
\begin{alltt}
\hlkwd{anova}\hlstd{(fm2, fm1, fm3, fm3a)}
\end{alltt}
\end{kframe}
\end{knitrout}

We can use different criteria to choose the best model: significance based on $P$-values or information criteria (AIC, BIC). AIC (Akaike's ‘An Information Criterion’) and BIC (= SBC, Schwarz's Bayesian criterion) that penalize the resulting `goodness' based on the number of parameters in the fitted model. In the case of AIC and BIC, a smaller value is better, and values returned can be either positive or negative, in which case more negative is better.\qRfunction{BIC()}\qRfunction{AIC()}

\begin{knitrout}\footnotesize
\definecolor{shadecolor}{rgb}{0.969, 0.969, 0.969}\color{fgcolor}\begin{kframe}
\begin{alltt}
\hlkwd{BIC}\hlstd{(fm2, fm1, fm3, fm3a)}
\hlkwd{AIC}\hlstd{(fm2, fm1, fm3, fm3a)}
\end{alltt}
\end{kframe}
\end{knitrout}

Once you have run the code in the chunks above, you will be able see that these three criteria not necessarily agree on which is the ``best'' model. Find in the output $p$-values from ANOVA, BIC and AIC values, for the different models and conclude on which model is favoured by each of the three criteria. In addition you will notice that the two different formulations of the quadratic polynomial are equivalent.

\end{playground}

\subsection{Analysis of variance, ANOVA}\label{sec:anova}
\index{analysis of variance}\index{linear models!analysis of variance}
\index{ANOVA|see{analysis of variance}}
We use as the \code{InsectSpray} data set, giving insect counts in plots sprayed with different insecticides. In these data \code{spray} is a factor with six levels.%
\label{xmpl:fun:lm:fm4}

The call is exactly the same as the one for linear regression, only the names of the variables and data frame are different. What determines that this is an ANOVA is that \code{spray}, the explanatory variable, is a \code{factor}.

\begin{knitrout}\footnotesize
\definecolor{shadecolor}{rgb}{0.969, 0.969, 0.969}\color{fgcolor}\begin{kframe}
\begin{alltt}
\hlkwd{data}\hlstd{(InsectSpray)}
\end{alltt}


{\ttfamily\noindent\color{warningcolor}{\#\# Warning in data(InsectSpray): data set 'InsectSpray' not found}}\begin{alltt}
\hlkwd{is.factor}\hlstd{(InsectSprays}\hlopt{$}\hlstd{spray)}
\end{alltt}
\begin{verbatim}
## [1] TRUE
\end{verbatim}
\begin{alltt}
\hlkwd{is.numeric}\hlstd{(InsectSprays}\hlopt{$}\hlstd{spray)}
\end{alltt}
\begin{verbatim}
## [1] FALSE
\end{verbatim}
\end{kframe}
\end{knitrout}

\begin{knitrout}\footnotesize
\definecolor{shadecolor}{rgb}{0.969, 0.969, 0.969}\color{fgcolor}\begin{kframe}
\begin{alltt}
\hlstd{fm4} \hlkwb{<-} \hlkwd{lm}\hlstd{(count} \hlopt{~} \hlstd{spray,} \hlkwc{data} \hlstd{= InsectSprays)}
\end{alltt}
\end{kframe}
\end{knitrout}

\begin{knitrout}\footnotesize
\definecolor{shadecolor}{rgb}{0.969, 0.969, 0.969}\color{fgcolor}\begin{kframe}
\begin{alltt}
\hlkwd{plot}\hlstd{(fm4,} \hlkwc{which} \hlstd{=} \hlnum{3}\hlstd{)}
\end{alltt}
\end{kframe}

{\centering \includegraphics[width=.76\textwidth]{figure/pos-model-6a-1} 

}



\end{knitrout}

\begin{knitrout}\footnotesize
\definecolor{shadecolor}{rgb}{0.969, 0.969, 0.969}\color{fgcolor}\begin{kframe}
\begin{alltt}
\hlkwd{anova}\hlstd{(fm4)}
\end{alltt}
\begin{verbatim}
## Analysis of Variance Table
## 
## Response: count
##           Df Sum Sq Mean Sq F value    Pr(>F)    
## spray      5 2668.8  533.77  34.702 < 2.2e-16 ***
## Residuals 66 1015.2   15.38                      
## ---
## Signif. codes:  0 '***' 0.001 '**' 0.01 '*' 0.05 '.' 0.1 ' ' 1
\end{verbatim}
\end{kframe}
\end{knitrout}

In ANOVA we are normally mainly interested in testing hypotheses, and \Rfunction{anova()} provides the most interesting output. Function \Rfunction{summary()} can be used to extract the estimates but usually the default contrasts and corresponding $P$-values are for hypotheses that have little or no interest.

\subsection{Analysis of covariance, ANCOVA}
\index{analysis of covariance}\index{linear models!analysis of covariance}
\index{ANCOVA|see{analysis of covariance}}

When a linear model includes both explanatory factors and continuous explanatory variables, we may call it \emph{analysis of covariance} (ANCOVA). The formula syntax is the same for all linear models, what determines the type of analysis is the nature of the explanatory variable(s). Conceptually a factor (an unordered categorical variable) is very different from a continuous variable.

\begin{playground}
There are additional methods that can be used to extract information from model fits. As objects containing the results of model fitting belong to different classes reflecting the type of model, methods with the same name can co-exist and selected invisibly to the user.\qRfunction{coef()}\qRfunction{resid()}

\begin{knitrout}\footnotesize
\definecolor{shadecolor}{rgb}{0.969, 0.969, 0.969}\color{fgcolor}\begin{kframe}
\begin{alltt}
\hlkwd{class}\hlstd{(fm1)}
\hlkwd{coef}\hlstd{(fm1)}
\hlkwd{resid}\hlstd{(fm1)}
\end{alltt}
\end{kframe}
\end{knitrout}

Do explore these methods if you know enough about statistics to recognize them. In the case of prediction with \Rfunction{predict()} new data needs to be supplied as argument as in the case of linear models no default is provided for this parameter, which in other cases are the data used in the fit. Be aware that the name of the variable (= column) within the new data frame needs to match the name of the explanatory variable in the data to which the model was fit, in our case \code{speed}.

\begin{knitrout}\footnotesize
\definecolor{shadecolor}{rgb}{0.969, 0.969, 0.969}\color{fgcolor}\begin{kframe}
\begin{alltt}
\hlstd{my.new.data} \hlkwb{<-} \hlkwd{data.frame}\hlstd{(}\hlkwc{speed} \hlstd{=} \hlnum{5}\hlopt{:}\hlnum{10}\hlstd{)}
\end{alltt}
\end{kframe}
\end{knitrout}

\begin{knitrout}\footnotesize
\definecolor{shadecolor}{rgb}{0.969, 0.969, 0.969}\color{fgcolor}\begin{kframe}
\begin{alltt}
\hlkwd{predict}\hlstd{(fm1,} \hlkwc{newdata} \hlstd{= my.new.data)}
\end{alltt}
\end{kframe}
\end{knitrout}

\begin{knitrout}\footnotesize
\definecolor{shadecolor}{rgb}{0.969, 0.969, 0.969}\color{fgcolor}\begin{kframe}
\begin{alltt}
\hlkwd{predict}\hlstd{(fm1,} \hlkwc{newdata} \hlstd{= my.new.data,} \hlkwc{interval} \hlstd{=} \hlstr{"confidence"}\hlstd{,} \hlkwc{level} \hlstd{=} \hlnum{0.9}\hlstd{)}
\end{alltt}
\end{kframe}
\end{knitrout}

Some other components of the fitted model object can be extracted with the usual syntax for lists, once we discover the names of the components. As shown earlier, we can use \code{str()} or \code{names()} to this end.

\begin{knitrout}\footnotesize
\definecolor{shadecolor}{rgb}{0.969, 0.969, 0.969}\color{fgcolor}\begin{kframe}
\begin{alltt}
\hlkwd{names}\hlstd{(fm1)}
\end{alltt}
\end{kframe}
\end{knitrout}

\begin{knitrout}\footnotesize
\definecolor{shadecolor}{rgb}{0.969, 0.969, 0.969}\color{fgcolor}\begin{kframe}
\begin{alltt}
\hlstd{fm1}\hlopt{$}\hlstd{df.residual}
\end{alltt}
\end{kframe}
\end{knitrout}

To see the whole structure of the fitted model object we can use \code{str()}, which reveals the nesting of members and all their \emph{attributes} (described in section \ref{sec:calc:attributes} on page \pageref{sec:calc:attributes}). In this case attributes contain very important information about the fit.

\begin{knitrout}\footnotesize
\definecolor{shadecolor}{rgb}{0.969, 0.969, 0.969}\color{fgcolor}\begin{kframe}
\begin{alltt}
\hlkwd{str}\hlstd{(fm1,} \hlkwc{max.level} \hlstd{=} \hlnum{1}\hlstd{)}
\end{alltt}
\end{kframe}
\end{knitrout}

\end{playground}
\index{linear models|)}

\section{Generalized linear models}
\index{generalized linear models|(}\index{models!generalized linear|see{generalized linear models}}
\index{GLM|see{generalized linear models}}

Linear models make the assumption of normally distributed residuals. Generalized linear models, fitted with function \Rfunction{glm()} are more flexible, and allow the assumed distribution to be selected as well as the link function.
For the analysis of the \code{InsectSpray} data set, above (section \ref{sec:anova} on page \pageref{sec:anova}) the Normal distribution is not a good approximation as count data deviates from it. This was visible in the quantile--quantile plot above.

For count data GLMs provide a better alternative. In the example below we fit the same model as above, but we assume a quasi-Poisson distribution instead of the Normal.

\begin{knitrout}\footnotesize
\definecolor{shadecolor}{rgb}{0.969, 0.969, 0.969}\color{fgcolor}\begin{kframe}
\begin{alltt}
\hlstd{fm10} \hlkwb{<-} \hlkwd{glm}\hlstd{(count} \hlopt{~} \hlstd{spray,} \hlkwc{data} \hlstd{= InsectSprays,} \hlkwc{family} \hlstd{= quasipoisson)}
\hlkwd{anova}\hlstd{(fm10,} \hlkwc{test} \hlstd{=} \hlstr{"F"}\hlstd{)}
\end{alltt}
\begin{verbatim}
## Analysis of Deviance Table
## 
## Model: quasipoisson, link: log
## 
## Response: count
## 
## Terms added sequentially (first to last)
## 
## 
##       Df Deviance Resid. Df Resid. Dev      F    Pr(>F)    
## NULL                     71     409.04                     
## spray  5   310.71        66      98.33 41.216 < 2.2e-16 ***
## ---
## Signif. codes:  0 '***' 0.001 '**' 0.01 '*' 0.05 '.' 0.1 ' ' 1
\end{verbatim}
\end{kframe}
\end{knitrout}

\begin{knitrout}\footnotesize
\definecolor{shadecolor}{rgb}{0.969, 0.969, 0.969}\color{fgcolor}\begin{kframe}
\begin{alltt}
\hlkwd{plot}\hlstd{(fm10,} \hlkwc{which} \hlstd{=} \hlnum{3}\hlstd{)}
\end{alltt}
\end{kframe}

{\centering \includegraphics[width=.76\textwidth]{figure/pos-model-11-1} 

}



\end{knitrout}
\index{generalized linear models|(}
\section{Non-linear regression}
\index{non-linear models|(}\index{models!non-linear|see{non-linear models}}

Function \Rfunction{nls} is \Rlang's workhorse for fitting non-linear models. By \emph{non-linear} it is meant not linear \emph{in the parameters} whose value is being estimated through fitting the model to data. This is different from the shape of the function when plotted---i.e.\ polynomials of all degrees are linear models.

\index{non-linear models|)}

\section{Model formulas}
  In the examples above we fitted simple models. More complex ones can be easily formulated using the same syntax. First of all one can avoid use of \code{*} and explicitly define all individual main effects and interactions. The syntax implemented in base \Rlang allows grouping by means of parentheses, so it is also possible to use grouping to exclude some interactions.

\begin{explainbox}
  Here we show some examples of models in mathematical notation together with the equivalent formulation using R's syntax.

  MISSING!

\end{explainbox}

The same symbols as for arithmetic operators are used for model formulas. Within a formula, symbols are interpreted according to formula syntax. When we mean an arithmetic operation that could be interpreted as being part of the model formula we need to ``protect'' it by means of the identity function \Rfunction{I()}. The next two examples define formulas for models with only one explanatory variable. With formulas like these the explanatory variable will be computed on the fly when fitting the model to data. In the first case below we need to explicitly protect the addition of the two variables into their sum, because otherwise they would be interpreted as two explanatory variables in the model. In the second case \Rfunction{log()} cannot be interpreted as part of the model formula, and consequently does no require additional protection, neither does its argument.

\begin{knitrout}\footnotesize
\definecolor{shadecolor}{rgb}{0.969, 0.969, 0.969}\color{fgcolor}\begin{kframe}
\begin{alltt}
\hlstd{y} \hlopt{~} \hlkwd{I}\hlstd{(x1} \hlopt{+} \hlstd{x2)}
\hlstd{y} \hlopt{~} \hlkwd{log}\hlstd{(x1} \hlopt{+} \hlstd{x2)}
\end{alltt}
\end{kframe}
\end{knitrout}

\Rlang's formula syntax allows alternative ways for specifying interaction terms. They allow ``abbreviated'' ways of entering formulas, which for complex experimental designs saves typing and can improve clarity. As seen above operator \code{*} saves us from having to explicitly indicate all the interaction terms in a full factorial model.

\begin{knitrout}\footnotesize
\definecolor{shadecolor}{rgb}{0.969, 0.969, 0.969}\color{fgcolor}\begin{kframe}
\begin{alltt}
\hlstd{y} \hlopt{~} \hlstd{x1} \hlopt{+} \hlstd{x2} \hlopt{+} \hlstd{x3} \hlopt{+} \hlstd{x1}\hlopt{:}\hlstd{x2} \hlopt{+} \hlstd{x1}\hlopt{:}\hlstd{x3} \hlopt{+} \hlstd{x2}\hlopt{:}\hlstd{x3} \hlopt{+} \hlstd{x1}\hlopt{:}\hlstd{x2}\hlopt{:}\hlstd{x3}
\end{alltt}
\end{kframe}
\end{knitrout}

Can be replaced by a concise equivalent.

\begin{knitrout}\footnotesize
\definecolor{shadecolor}{rgb}{0.969, 0.969, 0.969}\color{fgcolor}\begin{kframe}
\begin{alltt}
\hlstd{y} \hlopt{~} \hlstd{x1} \hlopt{*} \hlstd{x2} \hlopt{*} \hlstd{x3}
\end{alltt}
\end{kframe}
\end{knitrout}

When the model to be specified does not include all possible interaction terms, we can combine the concise notation with parentheses.

\begin{knitrout}\footnotesize
\definecolor{shadecolor}{rgb}{0.969, 0.969, 0.969}\color{fgcolor}\begin{kframe}
\begin{alltt}
\hlstd{y} \hlopt{~} \hlstd{x1} \hlopt{+} \hlstd{(x2} \hlopt{*} \hlstd{x3)}
\hlstd{y} \hlopt{~} \hlstd{x1} \hlopt{+} \hlstd{x2} \hlopt{+} \hlstd{x3} \hlopt{+} \hlstd{x2}\hlopt{:}\hlstd{x3}
\end{alltt}
\end{kframe}
\end{knitrout}

That the two model formulas above are equivalent, can be seen using \code{term()}

\begin{knitrout}\footnotesize
\definecolor{shadecolor}{rgb}{0.969, 0.969, 0.969}\color{fgcolor}\begin{kframe}
\begin{alltt}
\hlkwd{terms}\hlstd{(y} \hlopt{~} \hlstd{x1} \hlopt{+} \hlstd{(x2} \hlopt{*} \hlstd{x3))}
\end{alltt}
\begin{verbatim}
## y ~ x1 + (x2 * x3)
## attr(,"variables")
## list(y, x1, x2, x3)
## attr(,"factors")
##    x1 x2 x3 x2:x3
## y   0  0  0     0
## x1  1  0  0     0
## x2  0  1  0     1
## x3  0  0  1     1
## attr(,"term.labels")
## [1] "x1"    "x2"    "x3"    "x2:x3"
## attr(,"order")
## [1] 1 1 1 2
## attr(,"intercept")
## [1] 1
## attr(,"response")
## [1] 1
## attr(,".Environment")
## <environment: R_GlobalEnv>
\end{verbatim}
\end{kframe}
\end{knitrout}

\begin{knitrout}\footnotesize
\definecolor{shadecolor}{rgb}{0.969, 0.969, 0.969}\color{fgcolor}\begin{kframe}
\begin{alltt}
\hlstd{y} \hlopt{~} \hlstd{x1} \hlopt{*} \hlstd{(x2} \hlopt{+} \hlstd{x3)}
\hlstd{y} \hlopt{~} \hlstd{x1} \hlopt{+} \hlstd{x2} \hlopt{+} \hlstd{x3} \hlopt{+} \hlstd{x1}\hlopt{:}\hlstd{x2} \hlopt{+} \hlstd{x1}\hlopt{:}\hlstd{x3}
\end{alltt}
\end{kframe}
\end{knitrout}

\begin{knitrout}\footnotesize
\definecolor{shadecolor}{rgb}{0.969, 0.969, 0.969}\color{fgcolor}\begin{kframe}
\begin{alltt}
\hlkwd{terms}\hlstd{(y} \hlopt{~} \hlstd{x1} \hlopt{*} \hlstd{(x2} \hlopt{+} \hlstd{x3))}
\end{alltt}
\begin{verbatim}
## y ~ x1 * (x2 + x3)
## attr(,"variables")
## list(y, x1, x2, x3)
## attr(,"factors")
##    x1 x2 x3 x1:x2 x1:x3
## y   0  0  0     0     0
## x1  1  0  0     1     1
## x2  0  1  0     1     0
## x3  0  0  1     0     1
## attr(,"term.labels")
## [1] "x1"    "x2"    "x3"    "x1:x2" "x1:x3"
## attr(,"order")
## [1] 1 1 1 2 2
## attr(,"intercept")
## [1] 1
## attr(,"response")
## [1] 1
## attr(,".Environment")
## <environment: R_GlobalEnv>
\end{verbatim}
\end{kframe}
\end{knitrout}

The \code{\textasciicircum{}} operator can be used to limit the order of the interaction terms included in a formula.

\begin{knitrout}\footnotesize
\definecolor{shadecolor}{rgb}{0.969, 0.969, 0.969}\color{fgcolor}\begin{kframe}
\begin{alltt}
\hlstd{y} \hlopt{~} \hlstd{(x1} \hlopt{+} \hlstd{x2} \hlopt{+} \hlstd{x3)}\hlopt{^}\hlnum{2}
\hlstd{y} \hlopt{~} \hlstd{x1} \hlopt{+} \hlstd{x2} \hlopt{+} \hlstd{x3} \hlopt{+} \hlstd{x1}\hlopt{:}\hlstd{x2} \hlopt{+} \hlstd{x1}\hlopt{:}\hlstd{x3} \hlopt{+} \hlstd{x2}\hlopt{:}\hlstd{x3}
\end{alltt}
\end{kframe}
\end{knitrout}

\begin{knitrout}\footnotesize
\definecolor{shadecolor}{rgb}{0.969, 0.969, 0.969}\color{fgcolor}\begin{kframe}
\begin{alltt}
\hlkwd{terms}\hlstd{(y} \hlopt{~} \hlstd{(x1} \hlopt{+} \hlstd{x2} \hlopt{+} \hlstd{x3)}\hlopt{^}\hlnum{2}\hlstd{)}
\end{alltt}
\begin{verbatim}
## y ~ (x1 + x2 + x3)^2
## attr(,"variables")
## list(y, x1, x2, x3)
## attr(,"factors")
##    x1 x2 x3 x1:x2 x1:x3 x2:x3
## y   0  0  0     0     0     0
## x1  1  0  0     1     1     0
## x2  0  1  0     1     0     1
## x3  0  0  1     0     1     1
## attr(,"term.labels")
## [1] "x1"    "x2"    "x3"    "x1:x2" "x1:x3" "x2:x3"
## attr(,"order")
## [1] 1 1 1 2 2 2
## attr(,"intercept")
## [1] 1
## attr(,"response")
## [1] 1
## attr(,".Environment")
## <environment: R_GlobalEnv>
\end{verbatim}
\end{kframe}
\end{knitrout}

\begin{playground}
 For operator \code{\textasciicircum{}} to behave as expected its first operand should be a formula with no interactions!  Compare the result of expanding these two formulas.

\begin{knitrout}\footnotesize
\definecolor{shadecolor}{rgb}{0.969, 0.969, 0.969}\color{fgcolor}\begin{kframe}
\begin{alltt}
\hlstd{y} \hlopt{~} \hlstd{(x1} \hlopt{+} \hlstd{x2} \hlopt{+} \hlstd{x3)}\hlopt{^}\hlnum{2}
\hlstd{y} \hlopt{~} \hlstd{(x1} \hlopt{*} \hlstd{x2} \hlopt{*} \hlstd{x3)}\hlopt{^}\hlnum{2}
\end{alltt}
\end{kframe}
\end{knitrout}

\end{playground}

Operator \code{\%in\%} can also be used as a shortcut to including only some of all the possible interaction terms in a formula.

\begin{knitrout}\footnotesize
\definecolor{shadecolor}{rgb}{0.969, 0.969, 0.969}\color{fgcolor}\begin{kframe}
\begin{alltt}
\hlstd{y} \hlopt{~} \hlstd{x1} \hlopt{+} \hlstd{x2} \hlopt{+} \hlstd{x1} \hlopt \hlstd{x2}
\end{alltt}
\end{kframe}
\end{knitrout}

\begin{knitrout}\footnotesize
\definecolor{shadecolor}{rgb}{0.969, 0.969, 0.969}\color{fgcolor}\begin{kframe}
\begin{alltt}
\hlkwd{terms}\hlstd{(y} \hlopt{~} \hlstd{x1} \hlopt{+} \hlstd{x2} \hlopt{+} \hlstd{x1} \hlopt \hlstd{x2)}
\end{alltt}
\begin{verbatim}
## y ~ x1 + x2 + x1 %in% x2
## attr(,"variables")
## list(y, x1, x2)
## attr(,"factors")
##    x1 x2 x1:x2
## y   0  0     0
## x1  1  0     1
## x2  0  1     1
## attr(,"term.labels")
## [1] "x1"    "x2"    "x1:x2"
## attr(,"order")
## [1] 1 1 2
## attr(,"intercept")
## [1] 1
## attr(,"response")
## [1] 1
## attr(,".Environment")
## <environment: R_GlobalEnv>
\end{verbatim}
\end{kframe}
\end{knitrout}

\begin{playground}
The following examples of the use of formulas in ANOVA, plus your own variations on the same theme will help understand the syntax of model formulas.

\begin{knitrout}\footnotesize
\definecolor{shadecolor}{rgb}{0.969, 0.969, 0.969}\color{fgcolor}\begin{kframe}
\begin{alltt}
\hlkwd{data}\hlstd{(npk)}
\hlkwd{anova}\hlstd{(}\hlkwd{lm}\hlstd{(yield} \hlopt{~} \hlstd{N} \hlopt{*} \hlstd{P} \hlopt{*} \hlstd{K,} \hlkwc{data} \hlstd{= npk))}
\end{alltt}
\end{kframe}
\end{knitrout}

\begin{knitrout}\footnotesize
\definecolor{shadecolor}{rgb}{0.969, 0.969, 0.969}\color{fgcolor}\begin{kframe}
\begin{alltt}
\hlkwd{anova}\hlstd{(}\hlkwd{lm}\hlstd{(yield} \hlopt{~} \hlstd{(N} \hlopt{+} \hlstd{P} \hlopt{+} \hlstd{K)}\hlopt{^}\hlnum{2}\hlstd{,} \hlkwc{data} \hlstd{= npk))}
\end{alltt}
\end{kframe}
\end{knitrout}

\begin{knitrout}\footnotesize
\definecolor{shadecolor}{rgb}{0.969, 0.969, 0.969}\color{fgcolor}\begin{kframe}
\begin{alltt}
\hlkwd{anova}\hlstd{(}\hlkwd{lm}\hlstd{(yield} \hlopt{~} \hlstd{N} \hlopt{+} \hlstd{P} \hlopt{+} \hlstd{K} \hlopt{+} \hlstd{P} \hlopt \hlstd{N} \hlopt{+} \hlstd{K} \hlopt \hlstd{N,} \hlkwc{data} \hlstd{= npk))}
\end{alltt}
\end{kframe}
\end{knitrout}

\begin{knitrout}\footnotesize
\definecolor{shadecolor}{rgb}{0.969, 0.969, 0.969}\color{fgcolor}\begin{kframe}
\begin{alltt}
\hlkwd{anova}\hlstd{(}\hlkwd{lm}\hlstd{(yield} \hlopt{~} \hlstd{N} \hlopt{+} \hlstd{P} \hlopt{+} \hlstd{K} \hlopt{+} \hlstd{N} \hlopt \hlstd{P} \hlopt{+} \hlstd{K} \hlopt \hlstd{P,} \hlkwc{data} \hlstd{= npk))}
\end{alltt}
\end{kframe}
\end{knitrout}

\end{playground}

Nesting of factors results in the computation of additional error terms, with different degrees of freedom. Whether nesting exists or not is a property of an experiment. It is decided as part of the design of the experiment based on the mechanics of treatment assignment to experimental units. In base R formulas the nesting needs to be accounted by explicit definition of error terms by means of \code{Error()} within the formula. The numerical example below is just a demonstration of the consequences of using an invalid design. Of the two model formulas used, only one yields a valid data analysis!

\begin{knitrout}\footnotesize
\definecolor{shadecolor}{rgb}{0.969, 0.969, 0.969}\color{fgcolor}\begin{kframe}
\begin{alltt}
\hlstd{y} \hlopt{~} \hlstd{x1} \hlopt{+} \hlstd{x2} \hlopt{+} \hlkwd{Error}\hlstd{(x1} \hlopt{*} \hlstd{x2)}
\end{alltt}
\end{kframe}
\end{knitrout}

Packages \pkgname{nlme} and \pkgname{lme4} use their own extensions to base R's model formula syntax to allow the description of nesting and distinguishing fixed and random effects. Additive models have required other extensions, most of them specific to individual packages. These fall outside the scope of this book.

\begin{warningbox}
  \Rlang will accept any syntactically correct model formula, even when the results of the fit are not interpretable. It is the responsibility of the user to ensure that models are meaningful. The most common, and dangerous, mistake are missing simple interactions in factorial ANOVA.

  Fitting models like those below to three-way ANOVA should be avoided. In both cases simpler terms are missing, while interaction(s) that include the missing term are included in the model. Such models are not interpretable, as the variation from the missing term(s) ends being ``disguised'' within the remaining terms, distorting their apparent significance and parameter estimates.

\begin{knitrout}\footnotesize
\definecolor{shadecolor}{rgb}{0.969, 0.969, 0.969}\color{fgcolor}\begin{kframe}
\begin{alltt}
\hlstd{y} \hlopt{~} \hlstd{A} \hlopt{+} \hlstd{B} \hlopt{+} \hlstd{A} \hlopt{*} \hlstd{C} \hlopt{+} \hlstd{B} \hlopt{*} \hlstd{C} \hlopt{+} \hlstd{A} \hlopt{*} \hlstd{B}
\hlstd{y} \hlopt{~} \hlstd{A} \hlopt{+} \hlstd{B} \hlopt{+} \hlstd{C} \hlopt{+} \hlstd{A} \hlopt{*} \hlstd{B} \hlopt{*} \hlstd{C}
\end{alltt}
\end{kframe}
\end{knitrout}

  Models such as those below are interpretable, even though ``incomplete'' (not including all possible interactions).
\begin{knitrout}\footnotesize
\definecolor{shadecolor}{rgb}{0.969, 0.969, 0.969}\color{fgcolor}\begin{kframe}
\begin{alltt}
\hlstd{y} \hlopt{~} \hlstd{A} \hlopt{+} \hlstd{B} \hlopt{+} \hlstd{C} \hlopt{+} \hlstd{A} \hlopt{*} \hlstd{C} \hlopt{+} \hlstd{B} \hlopt{*} \hlstd{C} \hlopt{+} \hlstd{A} \hlopt{*} \hlstd{B}
\hlstd{y} \hlopt{~} \hlstd{A} \hlopt{+} \hlstd{B} \hlopt{+} \hlstd{C} \hlopt{+} \hlstd{A} \hlopt{*} \hlstd{B}
\end{alltt}
\end{kframe}
\end{knitrout}

  The full three-way factorial models includes all possible combinations of factors.
\begin{knitrout}\footnotesize
\definecolor{shadecolor}{rgb}{0.969, 0.969, 0.969}\color{fgcolor}\begin{kframe}
\begin{alltt}
\hlstd{y} \hlopt{~} \hlstd{A} \hlopt{+} \hlstd{B} \hlopt{+} \hlstd{C} \hlopt{+} \hlstd{A} \hlopt{*} \hlstd{C} \hlopt{+} \hlstd{B} \hlopt{*} \hlstd{C} \hlopt{+} \hlstd{A} \hlopt{*} \hlstd{B} \hlopt{+} \hlstd{A} \hlopt{*} \hlstd{B} \hlopt{*} \hlstd{C}
\end{alltt}
\end{kframe}
\end{knitrout}

\end{warningbox}

\begin{explainbox}
  \textbf{Manipulation of model formulas.} Being this a book about the \Rlang language, it is pertinent to describe how formulas can be manipulated. As seen in chapter \ref{chap:R:data} almost everything in the \Rlang language is an object that can be stored and manipulated. Model formulas are also objects, objects of class \code{formula}. The first consequence is that formulas as any other \Rlang objects can be saved in variables including in lists. Why is this useful? For example if we want to fit several different models to the same data, we can write a for loop that walks through a list of model formulas. Or we can write a new function that accepts one or more formulas as arguments.

\begin{knitrout}\footnotesize
\definecolor{shadecolor}{rgb}{0.969, 0.969, 0.969}\color{fgcolor}\begin{kframe}
\begin{alltt}
\hlkwd{class}\hlstd{(y} \hlopt{~} \hlstd{x)}
\end{alltt}
\begin{verbatim}
## [1] "formula"
\end{verbatim}
\end{kframe}
\end{knitrout}

\begin{knitrout}\footnotesize
\definecolor{shadecolor}{rgb}{0.969, 0.969, 0.969}\color{fgcolor}\begin{kframe}
\begin{alltt}
\hlstd{a} \hlkwb{<-} \hlstd{y} \hlopt{~} \hlstd{x}
\hlkwd{class}\hlstd{(a)}
\end{alltt}
\begin{verbatim}
## [1] "formula"
\end{verbatim}
\begin{alltt}
\hlstd{plyr}\hlopt{::}\hlkwd{is.formula}\hlstd{(a)}
\end{alltt}
\begin{verbatim}
## [1] TRUE
\end{verbatim}
\end{kframe}
\end{knitrout}

Method \Rfunction{is.formula()} is not part of the \Rlang language, but instead defined in package \pkgname{plyr}.

The use of \code{for} \emph{loops} for iteration is described in section \ref{sec:script:flow:control} on page \pageref{sec:script:flow:control}. For now, you need only to know that the statement in the body of the loop is executed once for each member of the \code{formulas} list, with \code{formula} taking successively the value of each member of \code{formulas}.
\begin{knitrout}\footnotesize
\definecolor{shadecolor}{rgb}{0.969, 0.969, 0.969}\color{fgcolor}\begin{kframe}
\begin{alltt}
  \hlstd{my.data} \hlkwb{<-} \hlkwd{data.frame}\hlstd{(}\hlkwc{x} \hlstd{=} \hlnum{1}\hlopt{:}\hlnum{10}\hlstd{,} \hlkwc{y} \hlstd{= (}\hlnum{1}\hlopt{:}\hlnum{10}\hlstd{)} \hlopt{/} \hlnum{2} \hlopt{+} \hlkwd{rnorm}\hlstd{(}\hlnum{10}\hlstd{))}
\hlstd{anovas} \hlkwb{<-} \hlkwd{list}\hlstd{()}
\hlstd{formulas} \hlkwb{<-} \hlkwd{list}\hlstd{(}\hlkwc{a} \hlstd{= y} \hlopt{~} \hlstd{x} \hlopt{-} \hlnum{1}\hlstd{,} \hlkwc{b} \hlstd{= y} \hlopt{~} \hlstd{x,} \hlkwc{c} \hlstd{= y} \hlopt{~} \hlstd{x} \hlopt{+} \hlstd{x}\hlopt{^}\hlnum{2}\hlstd{)}
\hlkwa{for} \hlstd{(formula} \hlkwa{in} \hlstd{formulas) \{}
 \hlstd{anovas} \hlkwb{<-} \hlkwd{c}\hlstd{(anovas,} \hlkwd{list}\hlstd{(}\hlkwd{lm}\hlstd{(formula,} \hlkwc{data} \hlstd{= my.data)))}
 \hlstd{\}}
 \hlkwd{str}\hlstd{(anovas,} \hlkwc{max.level} \hlstd{=} \hlnum{1}\hlstd{)}
\end{alltt}
\begin{verbatim}
## List of 3
##  $ :List of 12
##   ..- attr(*, "class")= chr "lm"
##  $ :List of 12
##   ..- attr(*, "class")= chr "lm"
##  $ :List of 12
##   ..- attr(*, "class")= chr "lm"
\end{verbatim}
\end{kframe}
\end{knitrout}

As could be expected a conversion constructor is available with name \code{as.formula}. It becomes useful when formulas are input interactively by the user or read from text files. We can convert a character string into a formula.

\begin{knitrout}\footnotesize
\definecolor{shadecolor}{rgb}{0.969, 0.969, 0.969}\color{fgcolor}\begin{kframe}
\begin{alltt}
\hlstd{my.string} \hlkwb{<-} \hlstr{"y ~ x"}
\hlkwd{lm}\hlstd{(}\hlkwd{as.formula}\hlstd{(my.string),} \hlkwc{data} \hlstd{= my.data)}
\end{alltt}
\begin{verbatim}
## 
## Call:
## lm(formula = as.formula(my.string), data = my.data)
## 
## Coefficients:
## (Intercept)            x  
##     -0.4382       0.5050
\end{verbatim}
\end{kframe}
\end{knitrout}

As there are many functions available in base \Rlang and through packages for the manipulation of character strings, it is straightforward to build model formulas programmatically as strings. We can use functions like \code{paste()} to assemble a formula as text, and then use \code{as.formula()} to convert it to an object of class \code{formula}, usable for fitting a model.

\begin{knitrout}\footnotesize
\definecolor{shadecolor}{rgb}{0.969, 0.969, 0.969}\color{fgcolor}\begin{kframe}
\begin{alltt}
\hlstd{my.string} \hlkwb{<-} \hlkwd{paste}\hlstd{(}\hlstr{"y"}\hlstd{,} \hlstr{"x"}\hlstd{,} \hlkwc{sep} \hlstd{=} \hlstr{"~"}\hlstd{)}
\hlkwd{lm}\hlstd{(}\hlkwd{as.formula}\hlstd{(my.string),} \hlkwc{data} \hlstd{= my.data)}
\end{alltt}
\begin{verbatim}
## 
## Call:
## lm(formula = as.formula(my.string), data = my.data)
## 
## Coefficients:
## (Intercept)            x  
##     -0.4382       0.5050
\end{verbatim}
\end{kframe}
\end{knitrout}

For the reverse operation of converting a formula into a string, we have available methods \code{as.character()} and \code{format()}. The first of these methods returns a character vector containing the components of the formula as individual strings, while \code{format()} returns a single character string with the formula formatted for printing.

\begin{knitrout}\footnotesize
\definecolor{shadecolor}{rgb}{0.969, 0.969, 0.969}\color{fgcolor}\begin{kframe}
\begin{alltt}
\hlstd{formatted.string} \hlkwb{<-} \hlkwd{format}\hlstd{(y} \hlopt{~} \hlstd{x)}
\hlstd{formatted.string}
\end{alltt}
\begin{verbatim}
## [1] "y ~ x"
\end{verbatim}
\begin{alltt}
\hlkwd{as.formula}\hlstd{(formatted.string)}
\end{alltt}
\begin{verbatim}
## y ~ x
\end{verbatim}
\end{kframe}
\end{knitrout}

It is also possible to \emph{edit} formula objects with method \Rfunction{update()}. In the replacement formula, a dot can replace either the left hand side (lhs) or the right hand side (rhs) of the existing formula in the replacement formula. We can also remove terms as can be seen below. In some cases the dot corresponding to the lhs con be omitted, but including it makes the syntax clearer.

\begin{knitrout}\footnotesize
\definecolor{shadecolor}{rgb}{0.969, 0.969, 0.969}\color{fgcolor}\begin{kframe}
\begin{alltt}
\hlstd{my.formula} \hlkwb{<-} \hlstd{y} \hlopt{~} \hlstd{x1} \hlopt{+} \hlstd{x2}
\hlkwd{update}\hlstd{(my.formula, .} \hlopt{~} \hlstd{.} \hlopt{+} \hlstd{x3)}
\end{alltt}
\begin{verbatim}
## y ~ x1 + x2 + x3
\end{verbatim}
\begin{alltt}
\hlkwd{update}\hlstd{(my.formula, .} \hlopt{~} \hlstd{.} \hlopt{-} \hlstd{x1)}
\end{alltt}
\begin{verbatim}
## y ~ x2
\end{verbatim}
\begin{alltt}
\hlkwd{update}\hlstd{(my.formula, .} \hlopt{~} \hlstd{x3)}
\end{alltt}
\begin{verbatim}
## y ~ x3
\end{verbatim}
\begin{alltt}
\hlkwd{update}\hlstd{(my.formula, z} \hlopt{~} \hlstd{.)}
\end{alltt}
\begin{verbatim}
## z ~ x1 + x2
\end{verbatim}
\begin{alltt}
\hlkwd{update}\hlstd{(my.formula, .} \hlopt{+} \hlstd{z} \hlopt{~} \hlstd{.)}
\end{alltt}
\begin{verbatim}
## y + z ~ x1 + x2
\end{verbatim}
\end{kframe}
\end{knitrout}

R provides high level functions for model selection. Consequently many R users will rarely need to edit model formulas in their scripts. For example, step wise model selection is possible with \Rlang method \code{step()}.

A matrix of dummy coefficients can be derived from a model formula, a type of contrast and the data for the explanatory variables.
\begin{knitrout}\footnotesize
\definecolor{shadecolor}{rgb}{0.969, 0.969, 0.969}\color{fgcolor}\begin{kframe}
\begin{alltt}
\hlstd{treats.df} \hlkwb{<-} \hlkwd{data.frame}\hlstd{(}\hlkwc{A} \hlstd{=} \hlkwd{rep}\hlstd{(}\hlkwd{c}\hlstd{(}\hlstr{"yes"}\hlstd{,} \hlstr{"no"}\hlstd{),} \hlkwd{c}\hlstd{(}\hlnum{4}\hlstd{,} \hlnum{4}\hlstd{)),}
                        \hlkwc{B} \hlstd{=} \hlkwd{rep}\hlstd{(}\hlkwd{c}\hlstd{(}\hlstr{"white"}\hlstd{,} \hlstr{"black"}\hlstd{),} \hlnum{4}\hlstd{))}
\hlstd{treats.df}
\end{alltt}
\begin{verbatim}
##     A     B
## 1 yes white
## 2 yes black
## 3 yes white
## 4 yes black
## 5  no white
## 6  no black
## 7  no white
## 8  no black
\end{verbatim}
\end{kframe}
\end{knitrout}

The default contrasts types currently in use.
\begin{knitrout}\footnotesize
\definecolor{shadecolor}{rgb}{0.969, 0.969, 0.969}\color{fgcolor}\begin{kframe}
\begin{alltt}
\hlkwd{options}\hlstd{(}\hlstr{"contrasts"}\hlstd{)}
\end{alltt}
\begin{verbatim}
## $contrasts
##         unordered           ordered 
## "contr.treatment"      "contr.poly"
\end{verbatim}
\end{kframe}
\end{knitrout}

A model matrix for a model for a two-way factorial design with no interaction term.
\begin{knitrout}\footnotesize
\definecolor{shadecolor}{rgb}{0.969, 0.969, 0.969}\color{fgcolor}\begin{kframe}
\begin{alltt}
\hlkwd{model.matrix}\hlstd{(}\hlopt{~} \hlstd{A} \hlopt{+} \hlstd{B, treats.df)}
\end{alltt}
\begin{verbatim}
##   (Intercept) Ayes Bwhite
## 1           1    1      1
## 2           1    1      0
## 3           1    1      1
## 4           1    1      0
## 5           1    0      1
## 6           1    0      0
## 7           1    0      1
## 8           1    0      0
## attr(,"assign")
## [1] 0 1 2
## attr(,"contrasts")
## attr(,"contrasts")$A
## [1] "contr.treatment"
## 
## attr(,"contrasts")$B
## [1] "contr.treatment"
\end{verbatim}
\end{kframe}
\end{knitrout}

A model matrix for a model for a two-way factorial design with interaction term.
\begin{knitrout}\footnotesize
\definecolor{shadecolor}{rgb}{0.969, 0.969, 0.969}\color{fgcolor}\begin{kframe}
\begin{alltt}
\hlkwd{model.matrix}\hlstd{(}\hlopt{~} \hlstd{A} \hlopt{*} \hlstd{B, treats.df)}
\end{alltt}
\begin{verbatim}
##   (Intercept) Ayes Bwhite Ayes:Bwhite
## 1           1    1      1           1
## 2           1    1      0           0
## 3           1    1      1           1
## 4           1    1      0           0
## 5           1    0      1           0
## 6           1    0      0           0
## 7           1    0      1           0
## 8           1    0      0           0
## attr(,"assign")
## [1] 0 1 2 3
## attr(,"contrasts")
## attr(,"contrasts")$A
## [1] "contr.treatment"
## 
## attr(,"contrasts")$B
## [1] "contr.treatment"
\end{verbatim}
\end{kframe}
\end{knitrout}

\end{explainbox}

\section{Time series}

Longitudinal data, when replicated is usually named repeated measurements, but when not replicated, is named time series.
Base \Rlang provides special support for the analysis of time series data, while repeated measurements can be analysed with nested linear models, mixed-effects models and additive models.

Time series data are data collected in such a way that there is only one observation, possibly of multiple variables, available at each time step. This brief section introduces only the most basic aspects of time-series analysis. In most cases time steps are of uniform duration and occur regularly. Given this properties, it is possible to take advantage of this for their storage. \Rlang not only provides methods for the analysis and manipulation of this type data, but also a specialized class for their storage, \code{ts}. A regular time series does not store the time value at each observation: only a combination of two of start time, step size and end time needs to be stored.

We start by creating a time series from a numeric vector. By now, you surely guessed that you need to use a constructor called \Rfunction{ts()} or a conversion constructor called \Rfunction{as.ts()} and that you can look up the arguments they accept by reading the corresponding help pages.

For example for a time series of monthly values we could use.
\begin{knitrout}\footnotesize
\definecolor{shadecolor}{rgb}{0.969, 0.969, 0.969}\color{fgcolor}\begin{kframe}
\begin{alltt}
\hlstd{my.ts} \hlkwb{<-} \hlkwd{ts}\hlstd{(}\hlnum{1}\hlopt{:}\hlnum{10}\hlstd{,} \hlkwc{start} \hlstd{=} \hlnum{2018}\hlstd{,} \hlkwc{deltat} \hlstd{=} \hlnum{1}\hlopt{/}\hlnum{12}\hlstd{)}
\hlkwd{class}\hlstd{(my.ts)}
\end{alltt}
\begin{verbatim}
## [1] "ts"
\end{verbatim}
\begin{alltt}
\hlkwd{str}\hlstd{(my.ts)}
\end{alltt}
\begin{verbatim}
##  Time-Series [1:10] from 2018 to 2019: 1 2 3 4 5 6 7 8 9 10
\end{verbatim}
\end{kframe}
\end{knitrout}

We next use a data set included in R.
\begin{knitrout}\footnotesize
\definecolor{shadecolor}{rgb}{0.969, 0.969, 0.969}\color{fgcolor}\begin{kframe}
\begin{alltt}
\hlkwd{class}\hlstd{(austres)}
\end{alltt}
\begin{verbatim}
## [1] "ts"
\end{verbatim}
\begin{alltt}
\hlkwd{is.ts}\hlstd{(austres)}
\end{alltt}
\begin{verbatim}
## [1] TRUE
\end{verbatim}
\end{kframe}
\end{knitrout}



This time series of the number of Australian residents is dominated by the increasing trend.
\begin{knitrout}\footnotesize
\definecolor{shadecolor}{rgb}{0.969, 0.969, 0.969}\color{fgcolor}\begin{kframe}
\begin{alltt}
\hlkwd{plot}\hlstd{(austres)}
\end{alltt}
\end{kframe}

{\centering \includegraphics[width=.76\textwidth]{figure/pos-ts-02-1} 

}



\end{knitrout}

A different example, in this case meteorological data, shows an important cyclic component. The annual cycle of mean air temperatures is clearly seen in the plot.
\begin{knitrout}\footnotesize
\definecolor{shadecolor}{rgb}{0.969, 0.969, 0.969}\color{fgcolor}\begin{kframe}
\begin{alltt}
\hlkwd{data}\hlstd{(nottem)}
\hlkwd{is.ts}\hlstd{(nottem)}
\end{alltt}
\begin{verbatim}
## [1] TRUE
\end{verbatim}
\begin{alltt}
\hlkwd{plot}\hlstd{(nottem)}
\end{alltt}
\end{kframe}

{\centering \includegraphics[width=.76\textwidth]{figure/pos-ts-03-1} 

}



\end{knitrout}

In the next two code chunks two different approaches to time series decomposition are used. In the first one we use a moving average to capture the trend, while in the second approach we use loess for the decomposition, a method for which the acronym STL is used. LOESS is a local-regression-based smoothing method.
\begin{knitrout}\footnotesize
\definecolor{shadecolor}{rgb}{0.969, 0.969, 0.969}\color{fgcolor}\begin{kframe}
\begin{alltt}
\hlstd{nottem.celcius} \hlkwb{<-} \hlstd{(nottem} \hlopt{-} \hlnum{32}\hlstd{)} \hlopt{*} \hlnum{5}\hlopt{/}\hlnum{9}
\hlkwd{plot}\hlstd{(}\hlkwd{decompose}\hlstd{(nottem.celcius))}
\end{alltt}
\end{kframe}

{\centering \includegraphics[width=.76\textwidth]{figure/pos-ts-04-1} 

}



\end{knitrout}

\begin{knitrout}\footnotesize
\definecolor{shadecolor}{rgb}{0.969, 0.969, 0.969}\color{fgcolor}\begin{kframe}
\begin{alltt}
\hlkwd{plot}\hlstd{(}\hlkwd{stl}\hlstd{(nottem.celcius,} \hlkwc{s.window} \hlstd{=} \hlnum{6}\hlstd{))}
\end{alltt}
\end{kframe}

{\centering \includegraphics[width=.76\textwidth]{figure/pos-ts-05-1} 

}



\end{knitrout}

\section{Multivariate statistics}

\subsection{Multivariate analysis of variance}

Multivariate methods take into account several response variables simultaneously, as part of a single analysis. In practice it is usual to use contributed packages for multivariate data analysis in \Rlang, except for simple cases. We will look first at \emph{multivariate} ANOVA or MANOVA. In the same way as \Rfunction{aov()} is a wrapper that uses internally \Rfunction{lm()}, \Rfunction{manova()} is a wrapper that uses internally \Rfunction{aov()}.

Multivariate model formulas in some cases use the same syntax as univariate ones, but contain more than one response variable on their left hand side (lhs).
\begin{knitrout}\footnotesize
\definecolor{shadecolor}{rgb}{0.969, 0.969, 0.969}\color{fgcolor}\begin{kframe}
\begin{alltt}
\hlstd{y1} \hlopt{+} \hlstd{y2} \hlopt{+} \hlstd{y3} \hlopt{~} \hlstd{x1} \hlopt{*} \hlstd{x2}
\end{alltt}
\begin{verbatim}
## y1 + y2 + y3 ~ x1 * x2
\end{verbatim}
\end{kframe}
\end{knitrout}

In other cases, a special syntax is used. In the case of function \Rfunction{manova()} we use function \code{cbind()} (column bind) to assemble the lhs of the formula.
\begin{knitrout}\footnotesize
\definecolor{shadecolor}{rgb}{0.969, 0.969, 0.969}\color{fgcolor}\begin{kframe}
\begin{alltt}
\hlkwd{data}\hlstd{(iris)}
\hlstd{m.fit} \hlkwb{<-} \hlkwd{manova}\hlstd{(}\hlkwd{cbind}\hlstd{(Petal.Length, Petal.Width)} \hlopt{~}  \hlstd{Species,} \hlkwc{data} \hlstd{= iris)}
\hlkwd{anova}\hlstd{(m.fit)}
\end{alltt}
\begin{verbatim}
## Analysis of Variance Table
## 
##              Df  Pillai approx F num Df den Df    Pr(>F)    
## (Intercept)   1 0.98786   5939.2      2    146 < 2.2e-16 ***
## Species       2 1.04645     80.7      4    294 < 2.2e-16 ***
## Residuals   147                                             
## ---
## Signif. codes:  0 '***' 0.001 '**' 0.01 '*' 0.05 '.' 0.1 ' ' 1
\end{verbatim}
\begin{alltt}
\hlkwd{summary}\hlstd{(m.fit)}
\end{alltt}
\begin{verbatim}
##            Df Pillai approx F num Df den Df    Pr(>F)    
## Species     2 1.0465   80.661      4    294 < 2.2e-16 ***
## Residuals 147                                            
## ---
## Signif. codes:  0 '***' 0.001 '**' 0.01 '*' 0.05 '.' 0.1 ' ' 1
\end{verbatim}
\end{kframe}
\end{knitrout}

\subsection{Principal components analysis}

Principal components analysis is used to simplify a data set by combing variables with similar behaviour into ``principal components''. At a later stage, we frequently try to interpret these components in relation to known and/or assumed independent variables. Base \Rlang's function \Rfunction{prcomp()} computes the principal components and accepts additional arguments for centering and scaling.

\begin{knitrout}\footnotesize
\definecolor{shadecolor}{rgb}{0.969, 0.969, 0.969}\color{fgcolor}\begin{kframe}
\begin{alltt}
\hlstd{pc} \hlkwb{<-} \hlkwd{prcomp}\hlstd{(iris[}\hlkwd{c}\hlstd{(}\hlstr{"Sepal.Length"}\hlstd{,} \hlstr{"Sepal.Width"}\hlstd{,}
                    \hlstr{"Petal.Length"}\hlstd{,} \hlstr{"Petal.Width"}\hlstd{)],}
             \hlkwc{center} \hlstd{=} \hlnum{TRUE}\hlstd{,}
             \hlkwc{scale.} \hlstd{=} \hlnum{TRUE}\hlstd{)}
\end{alltt}
\end{kframe}
\end{knitrout}

\begin{knitrout}\footnotesize
\definecolor{shadecolor}{rgb}{0.969, 0.969, 0.969}\color{fgcolor}\begin{kframe}
\begin{alltt}
\hlstd{pc}
\end{alltt}
\begin{verbatim}
## Standard deviations (1, .., p=4):
## [1] 1.7083611 0.9560494 0.3830886 0.1439265
## 
## Rotation (n x k) = (4 x 4):
##                     PC1         PC2        PC3        PC4
## Sepal.Length  0.5210659 -0.37741762  0.7195664  0.2612863
## Sepal.Width  -0.2693474 -0.92329566 -0.2443818 -0.1235096
## Petal.Length  0.5804131 -0.02449161 -0.1421264 -0.8014492
## Petal.Width   0.5648565 -0.06694199 -0.6342727  0.5235971
\end{verbatim}
\end{kframe}
\end{knitrout}

The rows ``Proportion of Variance'' and ``Cumulative Proportion'' are most informative of the contribution of each principal component (PC) to explaining the variation among observations.
\begin{knitrout}\footnotesize
\definecolor{shadecolor}{rgb}{0.969, 0.969, 0.969}\color{fgcolor}\begin{kframe}
\begin{alltt}
\hlkwd{summary}\hlstd{(pc)}
\end{alltt}
\begin{verbatim}
## Importance of components:
##                           PC1    PC2     PC3     PC4
## Standard deviation     1.7084 0.9560 0.38309 0.14393
## Proportion of Variance 0.7296 0.2285 0.03669 0.00518
## Cumulative Proportion  0.7296 0.9581 0.99482 1.00000
\end{verbatim}
\end{kframe}
\end{knitrout}

Method \code{plot()}
\begin{knitrout}\footnotesize
\definecolor{shadecolor}{rgb}{0.969, 0.969, 0.969}\color{fgcolor}\begin{kframe}
\begin{alltt}
\hlkwd{plot}\hlstd{(pc)}
\end{alltt}
\end{kframe}

{\centering \includegraphics[width=.76\textwidth]{figure/pos-pca-04-1} 

}



\end{knitrout}



\begin{knitrout}\footnotesize
\definecolor{shadecolor}{rgb}{0.969, 0.969, 0.969}\color{fgcolor}\begin{kframe}
\begin{alltt}
\hlkwd{biplot}\hlstd{(pc)}
\end{alltt}
\end{kframe}

{\centering \includegraphics[width=.76\textwidth]{figure/pos-pca-05-1} 

}



\end{knitrout}

Visually more elaborate plots of can be obtained with packages \pkgname{ggplot} described in chapter \ref{chap:R:plotting} starting on page \pageref{chap:R:performance} and package \pkgname{ggbiplot} described in section \ref{sec:plot:ggbiplot} on page \pageref{sec:plot:ggbiplot}.

\begin{playground}
   For growth and morphological data, a log-transformation can be suitable given that variance is frequently proportional to the magnitude of the values measured. We leave as an exercise to repeat the above analysis using transformed values for the dimensions of petals and sepals. How much does the use of transformations change the outcome of the analysis?
\end{playground}

\subsection{What next?}

Other books are available with detailed descriptions of how to do various types of analyses in \Rlang, included thorough descriptions of the methods briefly presented in this chapter and many other methods not mentioned here. Good examples of books with broad scope that can serve as references for the application of various statistical methods are \citetitle{Everitt2009} \autocite{Everitt2011} and \citetitle{Crawley2012} \autocite{Crawley2012}, and the classic reference \citetitle{Venables2002} \autocite{Venables2002}. Many subject specific books are also available, and some suggestions given at the end of the book on page \pageref{chap:R:readings}.

















\chapter{Further reading about R}\label{chap:R:readings}

\begin{VF}
Before you become too entranced with gorgeous gadgets and mesmerizing video displays, let me remind you that information is not knowledge, knowledge is not wisdom, and wisdom is not foresight. Each grows out of the other, and we need them all.

\VA{Arthur C. Clarke}{}
\end{VF}

%\dictum[Arthur C. Clarke]{Before you become too entranced with gorgeous gadgets and mesmerizing video displays, let me remind you that information is not knowledge, knowledge is not wisdom, and wisdom is not foresight. Each grows out of the other, and we need them all.}\vskip2ex

\begin{warningbox}
  This list will be expanded and more importantly reorganized and short comments added for book or group of books.
\end{warningbox}

\section{Introductory texts}

\cite{Allerhand2011,Dalgaard2008,Zuur2009,Teetor2011,Peng2017,Paradis2005,Peng2016}

\section{Texts on specific aspects}

\cite{Chang2013,Fox2002,Fox2010,Faraway2004,Faraway2006,Everitt2011,Wickham2017}

\section{Advanced texts}

\cite{Xie2013,Chambers2016,Wickham2015,Wickham2014advanced,Wickham2016,Pinheiro2000,Murrell2011,Matloff2011,Ihaka1996,Venables2000}

\section{Texts for S/R wisdom}

\cite{Burns1998,Burns2011,Burns2012,Bentley1986,Bentley1988}

\backmatter

\printbibliography

\printindex

\end{document}

\appendix

\chapter{Build information}

\begin{knitrout}\footnotesize
\definecolor{shadecolor}{rgb}{0.969, 0.969, 0.969}\color{fgcolor}\begin{kframe}
\begin{alltt}
\hlkwd{Sys.info}\hlstd{()}
\end{alltt}
\begin{verbatim}
##        sysname        release        version       nodename        machine 
##      "Windows"       "10 x64"  "build 17134"        "MUSTI"       "x86-64" 
##          login           user effective_user 
##       "aphalo"       "aphalo"       "aphalo"
\end{verbatim}
\end{kframe}
\end{knitrout}



\begin{knitrout}\footnotesize
\definecolor{shadecolor}{rgb}{0.969, 0.969, 0.969}\color{fgcolor}\begin{kframe}
\begin{alltt}
\hlkwd{sessionInfo}\hlstd{()}
\end{alltt}
\begin{verbatim}
## R version 3.5.0 Patched (2018-05-13 r74720)
## Platform: x86_64-w64-mingw32/x64 (64-bit)
## Running under: Windows 10 x64 (build 17134)
## 
## Matrix products: default
## 
## locale:
## [1] LC_COLLATE=English_United Kingdom.1252 
## [2] LC_CTYPE=English_United Kingdom.1252   
## [3] LC_MONETARY=English_United Kingdom.1252
## [4] LC_NUMERIC=C                           
## [5] LC_TIME=English_United Kingdom.1252    
## 
## attached base packages:
## [1] tools     stats     graphics  grDevices utils     datasets  methods  
## [8] base     
## 
## other attached packages:
## [1] svglite_1.2.1 stringr_1.3.1 knitr_1.20   
## 
## loaded via a namespace (and not attached):
## [1] compiler_3.5.0  plyr_1.8.4      magrittr_1.5    Rcpp_0.12.17   
## [5] gdtools_0.1.7   stringi_1.2.2   highr_0.6       evaluate_0.10.1
\end{verbatim}
\end{kframe}
\end{knitrout}

%

\end{document}


