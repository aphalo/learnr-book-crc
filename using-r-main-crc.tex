\documentclass[krantz2]{krantz}\usepackage{knitr}%,ChapterTOCs

%\usepackage[utf8]{inputenc}
\usepackage{color}

\usepackage{polyglossia}
\setdefaultlanguage[variant = british, ordinalmonthday = false]{english}

%\usepackage{gitinfo2} % remember to setup Git hooks

\usepackage{hologo}

\usepackage{csquotes}

\usepackage{graphicx}
\DeclareGraphicsExtensions{.jpg,.pdf,.png}

\usepackage{animate}

%\usepackage{microtype}
\usepackage[style=authoryear-comp,giveninits,sortcites,maxcitenames=2,%
    mincitenames=1,maxbibnames=10,minbibnames=10,backref,uniquename=mininit,%
    uniquelist=minyear,sortgiveninits=true,backend=biber]{biblatex}%,refsection=chapter

\newcommand{\href}[2]{\emph{#2} (\url{#1})}

%\usepackage[unicode,hyperindex,bookmarks,pdfview=FitB,%backref,
%            pdftitle={Learn R ...as you learnt your mother tongue},%
%            pdfkeywords={R, statistics, data analysis, plotting},%
%            pdfsubject={R},%
%            pdfauthor={Pedro J. Aphalo}%
%            ]{hyperref}

%\hypersetup{colorlinks,breaklinks,
%             urlcolor=blue,
%             linkcolor=blue,
%             citecolor=blue,
%             filecolor=blue,
%             menucolor=blue}

\usepackage{framed}

\usepackage{abbrev}
\usepackage{usingr}

\usepackage{imakeidx}

% this is to reduce spacing above and below verbatim, which is used by knitr
% to show returned values
\usepackage{etoolbox}
\makeatletter
\preto{\@verbatim}{\topsep=-5pt \partopsep=-4pt \itemsep=-2pt}
\makeatother

%%% Adjust graphic design

% New float "example" and corresponding "list of examples"
%\DeclareNewTOC[type=example,types=examples,float,counterwithin=chapter]{loe}
%\DeclareNewTOC[name=Box,listname=List of Text Boxes, type=example,types=examples,float,counterwithin=chapter,%
%]{lotxb}

% changing the style of float captions
%\addtokomafont{caption}{\sffamily\small}
%\setkomafont{captionlabel}{\sffamily\bfseries}
%\setcapindent{0em}

% finetuning tocs
%\makeatletter
%\renewcommand*\l@figure{\@dottedtocline{1}{0em}{2.6em}}
%\renewcommand*\l@table{\@dottedtocline{1}{0em}{2.6em}}
%\renewcommand*\l@example{\@dottedtocline{1}{0em}{2.3em}}
%\renewcommand{\@pnumwidth}{2.66em}
%\makeatother
%
%% add pdf bookmarks to tocs
%\makeatletter
%\BeforeTOCHead{%
%  \cleardoublepage
%    \edef\@tempa{%
%      \noexpand\pdfbookmark[0]{\list@fname}{\@currext}%
%    }\@tempa
%}

\setcounter{topnumber}{3}
\setcounter{bottomnumber}{3}
\setcounter{totalnumber}{4}
\renewcommand{\topfraction}{0.90}
\renewcommand{\bottomfraction}{0.90}
\renewcommand{\textfraction}{0.10}
\renewcommand{\floatpagefraction}{0.70}
\renewcommand{\dbltopfraction}{0.90}
\renewcommand{\dblfloatpagefraction}{0.70}

\addbibresource{rbooks.bib}
\addbibresource{references.bib}

\makeindex[title=General index]
\makeindex[name=rindex,title=Alphabetic index of \Rlang names]
\makeindex[name=rcatsidx,title=Index of \Rlang names by category]
\IfFileExists{upquote.sty}{\usepackage{upquote}}{}
\begin{document}

% customize chapter format:
%\KOMAoption{headings}{twolinechapter}
%\renewcommand*\chapterformat{\thechapter\autodot\hspace{1em}}

% customize dictum format:
%\setkomafont{dictumtext}{\itshape\small}
%\setkomafont{dictumauthor}{\normalfont}
%\renewcommand*\dictumwidth{0.7\linewidth}
%\renewcommand*\dictumauthorformat[1]{--- #1}
%\renewcommand*\dictumrule{}

%\extratitle{\vspace*{2\baselineskip}%
%             {\Huge\textsf{\textbf{Learn R}\\ \textsl{\huge\ldots as you learnt your mother tongue}}}}

\title{\Huge{\fontseries{ub}\sffamily Learn R\\{\Large\ldots as you learnt your mother tongue}}}

%\subtitle{Git hash: \gitAbbrevHash; Git date: \gitAuthorIsoDate}

\author{Pedro J. Aphalo}

\date{Helsinki, \today}

%\publishers{Draft, 95\% done\\Available through \href{https://leanpub.com/learnr}{Leanpub}}

%\uppertitleback{\copyright\ 2001--2017 by Pedro J. Aphalo\\
%Licensed under one of the \href{http://creativecommons.org/licenses/}{Creative Commons licenses} as indicated, or when not explicitly indicated, under the \href{http://creativecommons.org/licenses/by-sa/4.0/}{CC BY-SA 4.0 license}.}
%
%\lowertitleback{Typeset with \href{http://www.latex-project.org/}{\hologo{XeLaTeX}}\ in Lucida Bright and \textsf{Lucida Sans} using the KOMA-Script book class.\\
%The manuscript was written using \href{http://www.r-project.org/}{R} with package knitr. The manuscript was edited in \href{http://www.winedt.com/}{WinEdt} and \href{http://www.rstudio.com/}{RStudio}.
%The source files for the whole book are available at \url{https://bitbucket.org/aphalo/using-r}.}

%\frontmatter

% knitr setup

















% \thispagestyle{empty}
% \titleLL
% \clearpage

\frontmatter

\maketitle

%\begin{titlingpage}
%  \maketitle
%\titleLL
%\end{titlingpage}

\setcounter{page}{7} %previous pages will be reserved for frontmatter to be added in later.
\tableofcontents
%\include{frontmatter/foreword}
\chapter*{Preface}

\begin{VF}
``Suppose that you want to teach the `cat' concept to a very young child. Do you explain that a cat is a relatively small, primarily carnivorous mammal with retractible claws, a distinctive sonic output, etc.? I'll bet not. You probably show the kid a lot of different cats, saying `kitty' each time, until it gets the idea. To put it more generally, generalizations are best made by abstraction from experience.''

\VA{R. P. Boas}{Can we make mathematics intelligible?}
\end{VF}

%\dictum[R. P. Boas (1981) Can we make mathematics intelligible?, \emph{American Mathematical Monthly} \textbf{88:} 727-731.]{"Suppose that you want to teach the `cat' concept to a very young child. Do you explain that a cat is a relatively small, primarily carnivorous mammal with retractible claws, a distinctive sonic output, etc.? I'll bet not. You probably show the kid a lot of different cats, saying `kitty' each time, until it gets the idea. To put it more generally, generalizations are best made by abstraction from experience."}


% Such pauses are not a miss use of our time. To learn a natural language we need to interact with other speakers, we need feedback. In the case of R, we can get feedback both from the outcomes from our utterances to the computer, and from other R users.}

\vspace{2ex}This book covers different aspects of the use of \Rpgrm. It is meant to be used as a tutorial complementing a reference book about \R, or the documentation that accompanies R and the many packages used in the examples. Explanations are rather short and terse, so as to encourage the development of a routine of exploration. This is not an arbitrary decision, this is the normal \emph{modus operandi} of most of us who use R regularly for a variety of different problems.

I do not discuss here statistics, just \Rpgrm as a tool and language for data manipulation and display. The idea is for you to learn the \Rpgrm language like children learn a language: they work-out what the rules are, simply by listening to people speak and trying to utter what they want to tell their parents. Instead of listening, you will read and execute on a computer \Rlang code statements, try your hand at telling \Rlang what you want it to compute. I do provide explanations and comments, but the idea of these book is mainly for you to use the numerous examples to find-out by yourself the overall patterns and coding philosophy behind the \Rlang language. Instead of parents being the sound board for your first utterances in \langname{R}, the computer will play this role. You will \emph{play} by modifying the examples, see how the computer responds, does \Rlang understand you or not?

When teaching I tend to lean towards challenging students rather than telling a simplified story. I do the same here, because it is what I prefer as a student, and how I learn best myself. Not everybody learns best with the same approach, for me the most limiting factor is for what I listen to, or read, to be in a way or another challenging or entertaining enough to keep my thoughts focused. This I achieve best when making an effort to understand the contents or to follow the thread or plot of a story. So, be warned, reading this book will be about exploring a new world, this book aims to be a travel guide, neither a traveler's account, nor a cookbook of R recipes.

Keep in mind that it is impossible to remember everything about \Rpgrm! \Rpgrm in a broad sense is vast because its capabilities can be expanded with independently developed packages. Learning to use \Rlang consists in learning the basics plus developing the skill of finding your way in \Rlang and its documentation.  In 2017 the number packages available for free in the Comprehensive R Archive Network (CRAN) broke the 10\,000 barrier. CRAN is the most important, but not only, public repository for R packages. How good a command of the \Rlang language and packages a user needs depends on the type activities to be carried out. This book attempts to train you in the use of the \Rlang language itself and some packages that provide extensions for data manipulation and graphical display which are broadly useful. Given the availability of numerous books on statistical analysis with \Rlang, here we will cover only the bare minimum. The same is true for package development in \Rlang. This book seats in-between, aiming at teaching programming in-the-small: the use \Rlang to automate the drudgery of data manipulation from raw data, through data exploration to the production of publication quality illustrations.

As with all ``rich'' languages there are many different ways of doing things in R, and there is no one-size-fits-all solution to a problem. There is always a compromise involved, usually between time spent by the user and processing time required in the computer. Many of the packages that are most popular nowadays did not exist when I started using R, and many of these packages make new approaches available. One could write many different \Rlang books with a given aim and still use substantially different ways of achieving the same results. In this book, I limit myself to packages that are currently popular and/or that I consider elegantly designed. I have in particular tried to limit myself to packages with similar design philosophies, especially in relation to their interfaces. What is elegant design, and in particular what is a friendly user interface depends strongly on each user's preferences and previous experience. Consequently, the contents of the book are strongly biased by my own preferences. I have tried to write examples in ways that execute fast without compromising readability. I encourage readers to take this book as a travel guide, as a starting point for exploring the very many packages, styles and approaches which I have not described.

I will appreciate suggestions for further examples, notification of errors and unclear sections. Many of the examples here have been collected from diverse sources over many years and because of this not all sources are acknowledged. If you recognize any example as yours or someone else's please let me know so that I can add a proper acknowledgement. I warmly thank the students that over the years have asked the questions and posed the problems that have helped me write this text and correct the mistakes and voids of previous versions. I have also received help on on-line forums and in person from numerous people, learnt from archived e-mail list messages, blog posts, books, articles, tutorials, webinars, and by struggling to solve some new problems on my own. In many ways this text owes much more to people who are not authors than to myself. However, as I am the one who has written this version and decided what to include and exclude, as author, I take full responsibility for any errors and inaccuracies.

I have been using \Rpgrm since around 1998 or 1999, but I am still constantly learning new things about \Rpgrm itself and \Rpgrm packages. With time it has replaced in my work as a researcher and teacher several other pieces of software: \pgrmname{SPSS}, \pgrmname{Systat}, \pgrmname{Origin}, \pgrmname{Excel}, and it has become a central piece of the tool set I use for producing lecture slides, notes, books and even web pages. This is to say that it is the most useful piece of software and programming language I have ever learnt to use. Of course, in time it will be replaced by something better, but at the moment it is the ``hot'' thing to learn for anybody with a need to analyse and display data.

\begin{framed}
\noindent\large%
\textbf{I encourage you to approach R, like a child approaches his or hers mother tongue when learning to speak:} Do not struggle, just play! If going gets difficult and frustrating, take a break! If you get a new insight, take a break to enjoy the victory!
\end{framed}

\newpage

\begin{framed}
\noindent
\textbf{Icons used to mark different content.} Throughout the book text boxes marked with icons present different types of information. First of all, we have \emph{playground} boxes indicated with \playicon\ which contain open-ended exercises---ideas and pieces of R code to play with at the R console. A few of these will require more time to grasp, and are indicated with \advplayicon. Boxes providing general information, usually not directly related to \langname{R} as a language, are indicated with \infoicon. Some boxes highlighted with \ilAttention\ give important bits of information that must be remembered when using \langname{R}---i.e.\ explain some unusual feature of the language. Finally, some boxes indicated by \ilAdvanced\ give in depth explanations, that may require you to spend time thinking, which en general can be skipped on first reading, but to which you should return at a later, and peaceful, time with a cup of coffee or tea.
\end{framed}
\newpage

%\newpage
%\begin{infobox}
%\noindent
%\textbf{Status as of 2016-11-23.} I have updated the manuscript to track package updates since the previous version uploaded six months ago, and added several examples of the new functionality added to packages \ggpmisc, \ggrepel, and \ggplot. I have written new sections on packages \viridis, \pkgname{gganimate}, \pkgname{ggstance}, \pkgname{ggbiplot}, \pkgname{ggforce}, \pkgname{ggtern} and \pkgname{ggalt}. Some of these sections are to be expanded, and additional sections are planned for other recently released packages.
%
%With respect to the chapter \textit{Storing and manipulating data with R} I have put it on hold, except for the introduction, until I can see a soon to be published book covering the same subject. Hadley Wickham has named the set of tools developed by him and his collaborators as \textit{tidyverse} to be described in the book titled \textit{R for Data Science} by Grolemund and Wickham (O'Reilly).
%
%An important update to \ggplot was released last week, and it includes changes to the behavior of some existing functions, specially faceting has become extensible through other packages. Several of the new facilities are described in the updated text and code included in this book and this pdf has been generated with up-to-date version of \ggplot and packages as available today from CRAN, except for \pkgname{ggtern} which was downloaded from Bitbucket minutes ago.
%
%The present update adds about 100 pages to the previous versions. I expect to upload a new update to this manuscript in one or two months time.
%
%\textbf{Status as of 2017-01-17.} Added ``playground'' exercises to the chapter describing \ggplot, and converted some of the examples earlier part of the main text into these playground items. Added icons to help readers quickly distinguish playground sections (\textcolor{blue}{\noticestd{"0055}}), information sections (\textcolor{blue}{\modpicts{"003D}}), warnings about things one needs to be specially aware of (\colorbox{yellow}{\typicons{"E136}}) and boxes with more advanced content that may require longer time/more effort to grasp (\typicons{"E04E}). Added to the sections \code{scales} and examples in the \ggplot chapter details about the use of colors in R and \ggplot2. Removed some redundant examples, and updated the section on \code{plotmath}. Added terms to the alphabetical index. Increased line-spacing to avoid uneven spacing with inline code bits.
%
%\textbf{Status as of 2017-02-09.} Wrote section on ggplot2 themes, and on using system- and Google fonts in ggpplots with the help of package \pkgname{showtext}. Expanded section on \ggplot's \code{annotation}, and revised some sections in the ``R scripts and Programming'' chapter. Started writing the data chapter. Wrote draft on writing and reading text files. Several other smaller edits to text and a few new examples.
%
%\textbf{Status as of 2017-02-14.} Wrote sections on reading and writing MS-Excel files, files from statistical programs such as SPSS, SyStat, etc., and NetCDF files. Also wrote sections on using URLs to directly read data, and on reading HTML and XML files directly, as well on using JSON to retrieve measured/logged data from IoT (internet of things) and similar intelligent physical sensors, micro-controller boards and sensor hubs with network access.
%
%\textbf{Status as of 2017-03-25.} Revised and expanded the chapter on plotting maps, adding a section on the manipulation and plotting of image data. Revised and expanded the chapter on extensions to \pkgname{ggplot2}, so that there are no longer empty sections. Wrote short chapter ``If and when R needs help''. Revised and expanded the ``Introduction'' chapter. Added index entries, and additional citations to literature.
%
%\textbf{Status as of 2017-04-04.} Revised and expanded the chapter on using \Rpgrm as a calculator. Revised and expanded the ``Scripts'' chapter. Minor edits to ``Functions'' chapter. Continued writing chapter on data, writing a section on R's native apply functions and added preliminary text for a pipes and tees section. Write intro to `tidyverse' and grammar of data manipulation. Added index entries, and a few additional citations to the literature. Spell checking.
%
%\textbf{Status as of 2017-04-08.} Completed writing first draft of chapter on data, writing all the previously missing sections on the ``grammar of data manipulation''. Wrote two extended examples in the same chapter. Add table listing several extensions to \pkgname{ggplot2} not described in the book.
%
%\textbf{Status as of 2017-04-13.} Revised all chapters correcting some spelling mistakes, adding some explanatory text and indexing all functions and operators used. Thoroughly revised the Introduction chapter and the Preface. Expanded section on bar plots (now bar and column plots). Revised section on tile plots. Expanded section on factors in chapter 2, adding examples of reordering of factor labels, and making clearer the difference between the labels of the levels and the levels themselves.
%
%\textbf{Status as of 2017-04-29.} Tested with R 3.4.0. Package \pkgname{gganimate} needs to be installed from Github as the updated version is not yet in CRAN. Function \code{gg\_animate()} has been renamed \code{gganimate().}
%
%\textbf{Status as of 2017-05-14.} Submitted package \pkgname{learnrbook} to CRAN. Revised code in the book
%to use this new package. Small fixes after more testing. Added examples of plotting and labeling based on fits with \code{method = "nls"}, including use of the new \code{ggpmisc::stat\_fit\_tidy()}.
%
%\textbf{Status as of 2017-06-11.} Added sections on R-code bench marking and profiling for performance optimization. Added also an example of explicit compilation of a function defined in the R language. Added section on functions \code{assign()}, \code{get()} and \code{mget()}.
%
%\textbf{Status as of 2017-08-12.} Various edits to all chapters. Expanded section on \pkgname{ggpmisc} to include the new functionality added in version 0.2.15.9002: \code{geom\_table} and \code{stat\_fit\_tb}. Added section on package \pkgname{ggbeeswarm}. Added sections on packages \pkgname{magick} and on using \pgrmname{ImageJ} from \Rpgrm. Improved indexing and cross references.
%
%\textbf{Status as of 2017-10-25.} Edited the chapter on using R as a calculator, adding examples on insertion and deletion of members of lists and vectors, and also of use of \code{gl()} and \code{reorder()}. Edited sections on scale limits and added new section on coordinate limits to explain more thoroughly their differences and uses in chapter on plotting with \pkgname{ggplot2}. Added a section on package \pkgname{ggsignif} to the chapter on extensions to \pkgname{ggplot2}. Expanded section on \pkgname{ggpmisc} in the same chapter describing new functionality added in version 0.2.16.
%\pkgname{ggplo2} $>=$ 2.2.1.9000 is required by the current development version of \pkgname{ggpmisc}.
%
%\textbf{Status as of 2017-10-30.}  Add section on using pipes with \code{ggplot()} and layers.
%\end{infobox} 
\listoffigures
\listoftables
%\include{frontmatter/contributor}
%\include{frontmatter/symbollist}

\mainmatter

















% !Rnw root = appendix.main.Rnw



\chapter{Data input and output}\label{chap:R:data:io}\label{sec:data:io}

\begin{VF}
Most programmers have seen them, and most good programmers realize they've written at least one. They are huge, messy, ugly programs that should have been short, clean, beautiful programs.

\VA{John Bentley}{Programming Pearls}
\end{VF}



\section{Aims of this chapter}

Base \Rlang and the recommended packages (installed by default) include several functions for importing and exporting data. Contributed packages provide both replacements for some of these functions and support for several additional file formats. In the present chapter we aim at describing both data input and output covering in detail only the most common ``foreign'' data formats.

Developing software to read and write files with data stored in a format one does not have control on, or possibly a format one even needs to guess, can easily result in long, messy, ugly \Rlang scripts. It is also easy to end with functions that usually work but that fail with specific files, or even worse, occasionally silently corrupt the read data. Our aim is to provide guidance for finding available \Rlang functions and in contributed packages that well tested or validated.

In this chapter you will familiarise with how to exchange data between \Rlang and other systems. The functions \code{save()} and \code{load()} that save and read data in \Rlang's native format are described in section \ref{}.

\section{Introduction}

The first step in any data analysis is to input the data into \Rlang. Available sources of data are many and data can be stored or transmitted using various formats, both based on text or binary encodings. It is crucial that data is not altered (corrupted) when read and that in the eventual case of an error, errors are clearly reported. Most dangerous are silent non-catastrophic errors.

The very welcome increase of awareness of the need for open availability of data, makes the output of data from \Rlang into well defined data-exchange formats another crucial step. Consequently, in many cases an important step in data analysis is to export the data for submission to a repository, in addition to publication of the results of the analysis.

Faster internet access to data sources and cheaper random access memory (RAM) has made it possible to efficiently work with relatively large data sets in \Rlang. That \Rlang keeps all data in memory (RAM), imposes limits to the size of data \Rlang functions can operate on. For data sets large enough not to fit in computer RAM, one can use selective reading of data from flat files, or from databases outside of \Rlang.

Some \Rlang packages have made it faster to import data saved in the same formats already supported by base \Rlang, but in some cases providing weaker guarantees of not corrupting the data than base \Rlang. Other contributed packages make it possible to import and export data stored in file formats not supported by base \Rlang functions. Some of these formats are subject-area specific while others are in widespread use across disciplines.

\section{Packages used in this chapter}

\begin{knitrout}\footnotesize
\definecolor{shadecolor}{rgb}{0.969, 0.969, 0.969}\color{fgcolor}\begin{kframe}
\begin{alltt}
\hlkwd{install.packages}\hlstd{(learnrbook}\hlopt{::}\hlstd{pkgs_ch_data)}
\end{alltt}
\end{kframe}
\end{knitrout}

For executing the examples listed in this chapter you need first to load the following packages from the library:

\begin{knitrout}\footnotesize
\definecolor{shadecolor}{rgb}{0.969, 0.969, 0.969}\color{fgcolor}\begin{kframe}
\begin{alltt}
\hlkwd{library}\hlstd{(learnrbook)}
\hlkwd{library}\hlstd{(tibble)}
\hlkwd{library}\hlstd{(magrittr)}
\hlkwd{library}\hlstd{(wrapr)}
\hlkwd{library}\hlstd{(stringr)}
\hlkwd{library}\hlstd{(dplyr)}
\hlkwd{library}\hlstd{(tidyr)}
\hlkwd{library}\hlstd{(readr)}
\hlkwd{library}\hlstd{(readxl)}
\hlkwd{library}\hlstd{(readODS)}
\hlkwd{library}\hlstd{(xlsx)}
\hlkwd{library}\hlstd{(pdftools)}
\hlkwd{library}\hlstd{(foreign)}
\hlkwd{library}\hlstd{(haven)}
\hlkwd{library}\hlstd{(xml2)}
\hlkwd{library}\hlstd{(RNetCDF)}
\hlkwd{library}\hlstd{(ncdf4)}
\hlkwd{library}\hlstd{(lubridate)}
\hlkwd{library}\hlstd{(jsonlite)}
\end{alltt}
\end{kframe}
\end{knitrout}

\begin{infobox}
Some data sets used in this and other chapters are available in package \pkgname{learnrbook}. In addition to the
R data objects, we provide files saved in \emph{foreign} formats, which we used in examples on how to import data. The files can be either read from the \Rlang library, or from a copy in a local folder. In this chapter we
assume the user has copied the folder \code{"extdata"} from the package to his working folder.

Copy the files using:

\begin{knitrout}\footnotesize
\definecolor{shadecolor}{rgb}{0.969, 0.969, 0.969}\color{fgcolor}\begin{kframe}
\begin{alltt}
\hlstd{pkg.path} \hlkwb{<-} \hlkwd{system.file}\hlstd{(}\hlstr{"extdata"}\hlstd{,} \hlkwc{package} \hlstd{=} \hlstr{"learnrbook"}\hlstd{)}
\hlkwd{file.copy}\hlstd{(pkg.path,} \hlstr{"."}\hlstd{,} \hlkwc{overwrite} \hlstd{=} \hlnum{TRUE}\hlstd{,} \hlkwc{recursive} \hlstd{=} \hlnum{TRUE}\hlstd{)}
\end{alltt}
\begin{verbatim}
## [1] TRUE
\end{verbatim}
\end{kframe}
\end{knitrout}

We also make sure the folder used to save data read from the internet, exists.

\begin{knitrout}\footnotesize
\definecolor{shadecolor}{rgb}{0.969, 0.969, 0.969}\color{fgcolor}\begin{kframe}
\begin{alltt}
\hlstd{save.path} \hlkwb{=} \hlstr{"./data"}
\hlkwa{if} \hlstd{(}\hlopt{!}\hlkwd{dir.exists}\hlstd{(save.path)) \{}
  \hlkwd{dir.create}\hlstd{(save.path)}
\hlstd{\}}
\end{alltt}
\end{kframe}
\end{knitrout}
\end{infobox}

\section{File names}\label{sec:files:filenames}

We start with the naming of files as it affects data sharing irrespective of the format used for its storage. The main difficulty is that different operating systems have different rules governing file names and file paths. In many cases like when publishing data files, we need to ensure that file names are valid in multiple operating system (OS). If the script used to create the files is itself expected to be OS agnostic, we need to be also careful to query the OS for file names and paths without making assumptions on the naming rules. This is specially important when developing \Rlang packages.

\begin{warningbox}
For maximum portability, file names should never contain white-space characters and contain at most one dot. For widest possible portability, underscores should be avoided using dashes instead. As an example instead of \code{my data.2019.csv} use \code{my-data-2019.csv}.
\end{warningbox}

\Rlang provides functions which help with portability, by hiding the idiosyncracies of the different OSs from \Rlang code. In scripts these functions should be preferred over direct call to OS commands (i.e.\ using \Rfunction{shell()} or \Rfunction{system()}) whenever possible. As the algorithm needed to extract a file name from a file path, is OS specific, \Rlang provides functions such as \Rfunction{basename()} whose implementation is OS specific but from the side of \Rlang code behave identically---these functions hide the differences among OS from the user of \Rlang. The chunk below can be expected to work correctly under any OS for which \Rlang is available.

\begin{knitrout}\footnotesize
\definecolor{shadecolor}{rgb}{0.969, 0.969, 0.969}\color{fgcolor}\begin{kframe}
\begin{alltt}
\hlkwd{basename}\hlstd{(}\hlstr{"extdata/my-file.txt"}\hlstd{)}
\end{alltt}
\begin{verbatim}
## [1] "my-file.txt"
\end{verbatim}
\end{kframe}
\end{knitrout}

\begin{warningbox}
While in \pgrmname{Unix} and \pgrmname{Linux} folder nesting in file paths is marked with a forward slash characters (\verb|/|), under \pgrmname{MS-Windows} it is marked with backslash characters (\verb|\|). Backslash (\verb|\|) is a scape characters in \Rlang and interpreted as the start of an embedded special sequence of characters (see section \ref{sec:calc:character} on page \pageref{sec:calc:character}). While in \Rlang forward slash (\verb|/|) can be used for file paths under any OS, and escaped backslash (\verb|\\|) is valid only under MS-Windows. Consequently \verb|/| should be always preferred to \verb|\\| to ensure portability, and is the approach used in this book.

\begin{knitrout}\footnotesize
\definecolor{shadecolor}{rgb}{0.969, 0.969, 0.969}\color{fgcolor}\begin{kframe}
\begin{alltt}
\hlkwd{basename}\hlstd{(}\hlstr{"extdata/my-file.txt"}\hlstd{)}
\end{alltt}
\begin{verbatim}
## [1] "my-file.txt"
\end{verbatim}
\begin{alltt}
\hlkwd{basename}\hlstd{(}\hlstr{"extdata\textbackslash{}\textbackslash{}my-file.txt"}\hlstd{)}
\end{alltt}
\begin{verbatim}
## [1] "my-file.txt"
\end{verbatim}
\end{kframe}
\end{knitrout}
\end{warningbox}

The complementary function to \code{basename()} is \Rfunction{dirname()} and extracts the bare path to the containing folder, from a full file path.

\begin{knitrout}\footnotesize
\definecolor{shadecolor}{rgb}{0.969, 0.969, 0.969}\color{fgcolor}\begin{kframe}
\begin{alltt}
\hlkwd{dirname}\hlstd{(}\hlstr{"extdata/my-file.txt"}\hlstd{)}
\end{alltt}
\begin{verbatim}
## [1] "extdata"
\end{verbatim}
\end{kframe}
\end{knitrout}

Functions \Rfunction{getwd()} and \Rfunction{setwd()} can be used to get the path to the current working directory and to set a directory as current, respectively.

\begin{knitrout}\footnotesize
\definecolor{shadecolor}{rgb}{0.969, 0.969, 0.969}\color{fgcolor}\begin{kframe}
\begin{alltt}
\hlcom{# not run}
\hlkwd{getwd}\hlstd{()}
\end{alltt}
\end{kframe}
\end{knitrout}

Function \Rfunction{setwd()} returns the path to the current working directory, allowing us to portably set the working directory to the previous one. Both relative paths (relative to the current working directory), as in the example, or absolute paths (given in full) are accepted as argument. In mainstream OSs \code{"."} indicates the current directory and \code{".."} the directory above the current one.

\begin{knitrout}\footnotesize
\definecolor{shadecolor}{rgb}{0.969, 0.969, 0.969}\color{fgcolor}\begin{kframe}
\begin{alltt}
\hlcom{# not run}
\hlstd{oldwd} \hlkwb{<-} \hlkwd{setwd}\hlstd{(}\hlstr{".."}\hlstd{)}
\hlkwd{getwd}\hlstd{()}
\end{alltt}
\end{kframe}
\end{knitrout}

The returned value is always an absolute full path, so it remains valid even if the path to the working directory changes more than once before being restored.

\begin{knitrout}\footnotesize
\definecolor{shadecolor}{rgb}{0.969, 0.969, 0.969}\color{fgcolor}\begin{kframe}
\begin{alltt}
\hlcom{# not run}
\hlstd{oldwd}
\hlkwd{setwd}\hlstd{(oldwd)}
\hlkwd{getwd}\hlstd{()}
\end{alltt}
\end{kframe}
\end{knitrout}

We can also obtain lists of files and/or directories (= disk folders) portably across OSs.

\begin{knitrout}\footnotesize
\definecolor{shadecolor}{rgb}{0.969, 0.969, 0.969}\color{fgcolor}\begin{kframe}
\begin{alltt}
\hlkwd{head}\hlstd{(}\hlkwd{list.files}\hlstd{())}
\end{alltt}
\begin{verbatim}
## [1] "abbrev.sty"            "aka-r-life-2019.prj"   "anscombe.svg"         
## [4] "aphalo-learnr-001.pdf" "aphalo-learnr-002.pdf" "aphalo-learnr-003.pdf"
\end{verbatim}
\begin{alltt}
\hlkwd{head}\hlstd{(}\hlkwd{list.dirs}\hlstd{())}
\end{alltt}
\begin{verbatim}
## [1] "."                "./.git"           "./.git/hooks"    
## [4] "./.git/info"      "./.git/logs"      "./.git/logs/refs"
\end{verbatim}
\begin{alltt}
\hlkwd{head}\hlstd{(}\hlkwd{dir}\hlstd{())}
\end{alltt}
\begin{verbatim}
## [1] "abbrev.sty"            "aka-r-life-2019.prj"   "anscombe.svg"         
## [4] "aphalo-learnr-001.pdf" "aphalo-learnr-002.pdf" "aphalo-learnr-003.pdf"
\end{verbatim}
\end{kframe}
\end{knitrout}

\begin{playground}
The default argument for parameter \code{path} is the current working directory, under Windows, Unix and Linux indicated by \code{"."}. Convince yourself that this is indeed the default by calling the functions with an explicit argument. After this, play with the functions trying other existing and non-existent paths in your computer.
\end{playground}

\begin{playground}
Use parameter \code{full.names} with \Rfunction{list.files()} to obtain either a list of files paths or bare file names. Similarly investigate how the returned list of files is affected by the argument passed to \code{all.names}.
\end{playground}

\begin{playground}
Compare the behaviour of functions \Rfunction{dir()} and \Rfunction{list.dirs()}, and try by overriding the default arguments of \Rfunction{list.dirs()}, to get the call to return the same output as \Rfunction{dir()} does by default.
\end{playground}

Base \Rlang provides several functions for portably working with files, they are listed in the help page for \code{files} and in individual help pages. Use \code{help("files")} to access the help for this ``family'' of functions.

\begin{knitrout}\footnotesize
\definecolor{shadecolor}{rgb}{0.969, 0.969, 0.969}\color{fgcolor}\begin{kframe}
\begin{alltt}
\hlkwa{if} \hlstd{(}\hlopt{!}\hlkwd{file.exists}\hlstd{(}\hlstr{"xxx.txt"}\hlstd{)) \{}
  \hlkwd{file.create}\hlstd{(}\hlstr{"xxx.txt"}\hlstd{)}
\hlstd{\}}
\end{alltt}
\begin{verbatim}
## [1] TRUE
\end{verbatim}
\begin{alltt}
\hlkwd{file.size}\hlstd{(}\hlstr{"xxx.txt"}\hlstd{)}
\end{alltt}
\begin{verbatim}
## [1] 0
\end{verbatim}
\begin{alltt}
\hlkwd{file.info}\hlstd{(}\hlstr{"xxx.txt"}\hlstd{)}
\end{alltt}
\begin{verbatim}
##         size isdir mode               mtime               ctime
## xxx.txt    0 FALSE  666 2019-07-28 18:36:50 2019-07-28 18:36:50
##                       atime exe
## xxx.txt 2019-07-28 18:36:50  no
\end{verbatim}
\begin{alltt}
\hlkwd{file.rename}\hlstd{(}\hlstr{"xxx.txt"}\hlstd{,} \hlstr{"zzz.txt"}\hlstd{)}
\end{alltt}
\begin{verbatim}
## [1] TRUE
\end{verbatim}
\begin{alltt}
\hlkwd{file.exists}\hlstd{(}\hlstr{"xxx.txt"}\hlstd{)}
\end{alltt}
\begin{verbatim}
## [1] FALSE
\end{verbatim}
\begin{alltt}
\hlkwd{file.exists}\hlstd{(}\hlstr{"zzz.txt"}\hlstd{)}
\end{alltt}
\begin{verbatim}
## [1] TRUE
\end{verbatim}
\begin{alltt}
\hlkwd{file.remove}\hlstd{(}\hlstr{"zzz.txt"}\hlstd{)}
\end{alltt}
\begin{verbatim}
## [1] TRUE
\end{verbatim}
\end{kframe}
\end{knitrout}

\begin{playground}
Function \Rfunction{file.path()} can be used to construct a file path from its components in a way that is portable across OSs. Look at the help page and play with the function to assemble some paths that exist in the computer you are using.
\end{playground}

\section{Text files}\label{sec:files:txt}

In general text files are the most portable approach to data storage but usually also the least efficient with respect to the size of the file. Text files are composed of encoded characters. This makes them easily readable with text editors and easy to use functions in any programming language. On the other hand, how the data encoded as characters is arranged can be based on two different approaches: positional and using a specific character as separator. The positional approach is more concise but almost unreadable to humans as the values run into each other. Reading of data stored using a positional approach requires access to a format definition and was common in FORTRAN and COBOL at the times when perforated cards were used to store data. In the case of separators, different separators are in common use. Comma separated encodings (CSV) use either comma or semicolon to separate the fields or columns. Tabulator, or tab, separated values (TSV) use the tab character as column separator. Sometimes white space is used as separator, but it is not common to use white space to separate columns that are vertically aligned. White space embedded in values can be a source of problems during decoding.

\begin{explainbox}
\textbf{Not all text files are born equal.} When reading text files, and \emph{foreign} binary files which may contain embedded text strings, there is potential for their misinterpretation during the import operation. One common source of problems, is that column headers are to be read as \Rlang names. As earlier discussed, there are strict rules, such as avoiding spaces or special characters if the names are to be used with the normal syntax. On import, some functions will attempt to sanitize the names, but others not. Most such names are still accessible in \Rlang statements, but a special syntax is needed to protect them from triggering syntax errors through their interpretation as something different than variable or function names---in \Rlang jargon we say that they need to be quoted.

Some of the things we need to be on the watch for are:
1) Mismatches between the character encoding expected by the function used to read the file, and the encoding used for saving the file.
2) Leading or trailing (invisible) spaces are present in the character values or column names, causing error that are disconcerting.
3) When the type of values is guessed at, either by reading a certain number of text lines at the top of the file or after reading the whole text, a typing mistake affecting a single character, including the wrong kind of decimal marker, can prevent a whole column to be recognized as numeric.
5) The default decimal marker used for saving and expected when reading text files depends on the locale (language and sometimes country) settings of the computer or program being used.

If you encounter problems after import, such as failure of indexing of data frame columns by name, use function \code{names()} to get the names printed to the console as a character vector. This is useful because character vectors are always printed with each string delimited by quotation marks.

To demonstrate some of these problems in the first statement below I create a data frame with name sanitation disabled. And in the second statement with sanitation enabled. The first statement is equivalent to the default behaviour of functions in package \pkgname{readr} and the the second equivalent to the behaviour of base \Rlang functions. \pkgname{readr} prioritizes the integrity of the original data while \Rlang prioritizes compatibility with R's naming rules.

\begin{knitrout}\footnotesize
\definecolor{shadecolor}{rgb}{0.969, 0.969, 0.969}\color{fgcolor}\begin{kframe}
\begin{alltt}
\hlkwd{data.frame}\hlstd{(}\hlkwc{a} \hlstd{=} \hlnum{1}\hlstd{,} \hlstr{"a "} \hlstd{=} \hlnum{2}\hlstd{,} \hlstr{" a"} \hlstd{=} \hlnum{3}\hlstd{,} \hlkwc{check.names} \hlstd{=} \hlnum{FALSE}\hlstd{)}
\end{alltt}
\begin{verbatim}
##   a a   a
## 1 1  2  3
\end{verbatim}
\begin{alltt}
\hlkwd{data.frame}\hlstd{(}\hlkwc{a} \hlstd{=} \hlnum{1}\hlstd{,} \hlstr{"a "} \hlstd{=} \hlnum{2}\hlstd{,} \hlstr{" a"} \hlstd{=} \hlnum{3}\hlstd{)}
\end{alltt}
\begin{verbatim}
##   a a. X.a
## 1 1  2   3
\end{verbatim}
\end{kframe}
\end{knitrout}

An even more subtle case is when characters can be easily confused by the user reading the output: not too difficult to distinguish are zero and o (\code{a0} vs.\ \code{aO}) or el and one (\code{al} vs.\ \code{a1}), except for the case of some specific fonts. When using encodings capable of storing many character shapes, such as unicode, in some cases two characters with almost identical visual shape may be encoded as different characters.

\begin{knitrout}\footnotesize
\definecolor{shadecolor}{rgb}{0.969, 0.969, 0.969}\color{fgcolor}\begin{kframe}
\begin{alltt}
\hlkwd{data.frame}\hlstd{(}\hlkwc{al} \hlstd{=} \hlnum{1}\hlstd{,} \hlkwc{a1} \hlstd{=} \hlnum{2}\hlstd{,} \hlkwc{aO} \hlstd{=} \hlnum{3}\hlstd{,} \hlkwc{a0} \hlstd{=} \hlnum{4}\hlstd{)}
\end{alltt}
\begin{verbatim}
##   al a1 aO a0
## 1  1  2  3  4
\end{verbatim}
\end{kframe}
\end{knitrout}

In some cases the import may result in very odd looking values stored in \Rlang variables, for example when a CSV file saved with \pgrmname{MS-Excel} is silently encoded using 16-bit unicode format, but read as an 8-bit encoded file.

The hardest part of all these problems is to diagnose their origin, as function arguments and working environment options can in most cases be used to force the correct decoding of text files with diverse characteristics, origins and vintages once one knows what is required. One function in \Rlang \pkgname{tools} package, which is not-exported, can at the time of writing be used to test files for the presence on non-ASCII characters: \Rfunction{tools:::showNonASCIIfile()}. This function takes as argument the path to a file. We first discuss base \Rlang functions and starting from page \pageref{sec:files:readr} we discuss the functions defined in package \pkgname{readr}.
\end{explainbox}

\subsection[Base R and `utils']{Base \Rlang and \pkgname{utils}}

Text files come in many different sizes and formats, but can be divided into two broad groups. Those with fixed format fields, and those with delimited fields. Fixed format fields were especially common in the early days of \langname{FORTRAN} and \langname{COBOL}, and computers with very limited resources. They are usually capable of encoding information using fewer characters than with delimited fields. The best way of understanding the differences is with examples. Although in this section we exemplify the use of functions by passing a file name as argument, URLs, and open file descriptors are also accepted (see section \ref{sec:???} on page \pageref{sec:???}).

In the first example we will read a file with fields solely delimited by ``,''. This is what is called comma-separated-values format (CSV) which can be read and written with \Rfunction{read.csv()} and \Rfunction{write.csv()}, respectively.

Example file \code{not-aligned-ASCII-UK.csv} contains.

\begin{knitrout}\footnotesize
\definecolor{shadecolor}{rgb}{0.969, 0.969, 0.969}\color{fgcolor}\begin{kframe}
\begin{verbatim}
col1,col2,col3,col4
1.0,24.5,346,ABC
23.4,45.6,78,Z Y
\end{verbatim}
\end{kframe}
\end{knitrout}

\begin{knitrout}\footnotesize
\definecolor{shadecolor}{rgb}{0.969, 0.969, 0.969}\color{fgcolor}\begin{kframe}
\begin{alltt}
\hlstd{from_csv_a.df} \hlkwb{<-} \hlkwd{read.csv}\hlstd{(}\hlstr{"extdata/not-aligned-ASCII-UK.csv"}\hlstd{)}
\end{alltt}
\end{kframe}
\end{knitrout}

\begin{knitrout}\footnotesize
\definecolor{shadecolor}{rgb}{0.969, 0.969, 0.969}\color{fgcolor}\begin{kframe}
\begin{alltt}
\hlkwd{sapply}\hlstd{(from_csv_a.df, class)}
\end{alltt}
\begin{verbatim}
##      col1      col2      col3      col4 
## "numeric" "numeric" "integer"  "factor"
\end{verbatim}
\begin{alltt}
\hlstd{from_csv_a.df[[}\hlstr{"col4"}\hlstd{]]}
\end{alltt}
\begin{verbatim}
## [1] ABC Z Y
## Levels: ABC Z Y
\end{verbatim}
\begin{alltt}
\hlkwd{levels}\hlstd{(from_csv_a.df[[}\hlstr{"col4"}\hlstd{]])}
\end{alltt}
\begin{verbatim}
## [1] "ABC" "Z Y"
\end{verbatim}
\end{kframe}
\end{knitrout}

\begin{playground}
Read the file same file \code{not-aligned-ASCII-UK.csv} with function \Rfunction{read.csv2()} instead of \Rfunction{read.csv()}. Although this may look as a waste of time, the point of the exercise is for you to get familiar with \Rlang behaviour in case of such a mistake. This will help you recognize similar errors when they happen accidentally, which is quite common when files are shared.
\end{playground}

Example file \code{aligned-ASCII-UK.csv} contains comma-separated-values with added white space to align the columns, to make it easier to read by humans. The fields contain leading and railing white spaces. These white spaces are read as part of the fields, but ignored when conversion to numeric is done. However they are retained in the case of factor levels and strings leading to sometimes perplexing errors.

\begin{knitrout}\footnotesize
\definecolor{shadecolor}{rgb}{0.969, 0.969, 0.969}\color{fgcolor}\begin{kframe}
\begin{verbatim}
col1, col2, col3, col4
 1.0, 24.5,  346,  ABC
23.4, 45.6,   78,  Z Y
\end{verbatim}
\end{kframe}
\end{knitrout}

\begin{knitrout}\footnotesize
\definecolor{shadecolor}{rgb}{0.969, 0.969, 0.969}\color{fgcolor}\begin{kframe}
\begin{alltt}
\hlstd{from_csv_b.df} \hlkwb{<-} \hlkwd{read.csv}\hlstd{(}\hlstr{"extdata/aligned-ASCII-UK.csv"}\hlstd{)}
\end{alltt}
\end{kframe}
\end{knitrout}

\begin{knitrout}\footnotesize
\definecolor{shadecolor}{rgb}{0.969, 0.969, 0.969}\color{fgcolor}\begin{kframe}
\begin{alltt}
\hlkwd{sapply}\hlstd{(from_csv_b.df, class)}
\end{alltt}
\begin{verbatim}
##      col1      col2      col3      col4 
## "numeric" "numeric" "integer"  "factor"
\end{verbatim}
\begin{alltt}
\hlstd{from_csv_b.df[[}\hlstr{"col4"}\hlstd{]]}
\end{alltt}
\begin{verbatim}
## [1]   ABC   Z Y
## Levels:   ABC   Z Y
\end{verbatim}
\begin{alltt}
\hlkwd{levels}\hlstd{(from_csv_b.df[[}\hlstr{"col4"}\hlstd{]])}
\end{alltt}
\begin{verbatim}
## [1] "  ABC" "  Z Y"
\end{verbatim}
\end{kframe}
\end{knitrout}

By default, column names have been sanitized but factor levels have't and cosequently contain leading white space. By consulting the help page we discover that by passing an additional argument we can change this default and obtain the data read as desired. Most likely the default has been chosen so that by default the data is not altered.

\begin{knitrout}\footnotesize
\definecolor{shadecolor}{rgb}{0.969, 0.969, 0.969}\color{fgcolor}\begin{kframe}
\begin{alltt}
\hlstd{from_csv_e.df} \hlkwb{<-} \hlkwd{read.csv}\hlstd{(}\hlstr{"extdata/aligned-ASCII-UK.csv"}\hlstd{,} \hlkwc{strip.white} \hlstd{=} \hlnum{TRUE}\hlstd{)}
\hlkwd{sapply}\hlstd{(from_csv_e.df, class)}
\end{alltt}
\begin{verbatim}
##      col1      col2      col3      col4 
## "numeric" "numeric" "integer"  "factor"
\end{verbatim}
\begin{alltt}
\hlstd{from_csv_e.df[[}\hlstr{"col4"}\hlstd{]]}
\end{alltt}
\begin{verbatim}
## [1] ABC Z Y
## Levels: ABC Z Y
\end{verbatim}
\begin{alltt}
\hlkwd{levels}\hlstd{(from_csv_e.df[[}\hlstr{"col4"}\hlstd{]])}
\end{alltt}
\begin{verbatim}
## [1] "ABC" "Z Y"
\end{verbatim}
\end{kframe}
\end{knitrout}

The functions from \Rlang \pkgname{utils} package by default convert columns containg character strings into factors, as seen above. This default can be also changed so that character strings are read as is.

\begin{knitrout}\footnotesize
\definecolor{shadecolor}{rgb}{0.969, 0.969, 0.969}\color{fgcolor}\begin{kframe}
\begin{alltt}
\hlstd{from_csv_c.df} \hlkwb{<-} \hlkwd{read.csv}\hlstd{(}\hlstr{"extdata/not-aligned-ASCII-UK.csv"}\hlstd{,}
                          \hlkwc{stringsAsFactors} \hlstd{=} \hlnum{FALSE}\hlstd{)}
\end{alltt}
\end{kframe}
\end{knitrout}

\begin{knitrout}\footnotesize
\definecolor{shadecolor}{rgb}{0.969, 0.969, 0.969}\color{fgcolor}\begin{kframe}
\begin{alltt}
\hlkwd{sapply}\hlstd{(from_csv_c.df, class)}
\end{alltt}
\begin{verbatim}
##        col1        col2        col3        col4 
##   "numeric"   "numeric"   "integer" "character"
\end{verbatim}
\begin{alltt}
\hlstd{from_csv_c.df[[}\hlstr{"col4"}\hlstd{]]}
\end{alltt}
\begin{verbatim}
## [1] "ABC" "Z Y"
\end{verbatim}
\end{kframe}
\end{knitrout}

Decimal points and exponential notation are allowed for floating point values. In English-speaking locales the decimal mark is a point, while in many other locales it is a comma. If comma is used as decimal marker, we can no longer use it as field separator and is usually substituted by semicolon (\verb|;|). In such a case we can use \Rfunction{read.csv2()} and \Rfunction{write.csv2}. Furthermore parameters \code{dec} and code{sep} allow setting them to arbitrary characters. Function \Rfunction{read.table()} does the actual work and functions like \Rfunction{read.csv()} only differ in the default arguments for the different parameters. By default \Rfunction{read.table()} expects fields to be separated by white space (one or more spaces, tabs, new lines, or carriage return). Strings with embedded spaces need to be quoted in the file as shown below.

\begin{knitrout}\footnotesize
\definecolor{shadecolor}{rgb}{0.969, 0.969, 0.969}\color{fgcolor}\begin{kframe}
\begin{verbatim}
col1 col2 col3 col4
 1.0 24.5  346 ABC
23.4 45.6   78 "Z Y"
\end{verbatim}
\end{kframe}
\end{knitrout}

\begin{knitrout}\footnotesize
\definecolor{shadecolor}{rgb}{0.969, 0.969, 0.969}\color{fgcolor}\begin{kframe}
\begin{alltt}
\hlstd{from_txt_b.df} \hlkwb{<-} \hlkwd{read.table}\hlstd{(}\hlstr{"extdata/aligned-ASCII.txt"}\hlstd{,} \hlkwc{header} \hlstd{=} \hlnum{TRUE}\hlstd{)}
\end{alltt}
\end{kframe}
\end{knitrout}

\begin{knitrout}\footnotesize
\definecolor{shadecolor}{rgb}{0.969, 0.969, 0.969}\color{fgcolor}\begin{kframe}
\begin{alltt}
\hlkwd{sapply}\hlstd{(from_txt_b.df, class)}
\end{alltt}
\begin{verbatim}
##      col1      col2      col3      col4 
## "numeric" "numeric" "integer"  "factor"
\end{verbatim}
\begin{alltt}
\hlstd{from_txt_b.df[[}\hlstr{"col4"}\hlstd{]]}
\end{alltt}
\begin{verbatim}
## [1] ABC Z Y
## Levels: ABC Z Y
\end{verbatim}
\begin{alltt}
\hlkwd{levels}\hlstd{(from_txt_b.df[[}\hlstr{"col4"}\hlstd{]])}
\end{alltt}
\begin{verbatim}
## [1] "ABC" "Z Y"
\end{verbatim}
\end{kframe}
\end{knitrout}

With a fixed-width format no delimiters are needed. Decoding is based solely on the position of the characters in the line or record. A file like this cannot be interpreted without a description of the format used for saving the data. Files containing data stored in \emph{fixed width format} can be read with function \Rfunction{read.fwf()}. Records for a single observation, can be stored in a single- or multiple lines. In either case each line with fields of different but fixed known widths.

Function \Rfunction{read.fortran()} is a wrapper on \Rfunction{read.fwf()} that accepts format definitions similar to those used in \langname{FORTRAN}. One particularity of \langname{FORTRAN} \emph{formated data transfer} is that the decimal marker can be omitted in the saved file and its position specified as part of the format definition. A trick used to make text files (or stacks of punch cards!) smaller. Modern versions of \langname{FORTRAN} support reading from and writing to other formats like those based field delimiters described above.

\begin{knitrout}\footnotesize
\definecolor{shadecolor}{rgb}{0.969, 0.969, 0.969}\color{fgcolor}\begin{kframe}
\begin{verbatim}
 10245346ABC
234456 78Z Y
\end{verbatim}
\end{kframe}
\end{knitrout}

\begin{knitrout}\footnotesize
\definecolor{shadecolor}{rgb}{0.969, 0.969, 0.969}\color{fgcolor}\begin{kframe}
\begin{alltt}
\hlstd{from_fwf_a.df} \hlkwb{<-} \hlkwd{read.fortran}\hlstd{(}\hlstr{"extdata/aligned-ASCII.fwf"}\hlstd{,}
                              \hlkwc{format} \hlstd{=} \hlkwd{c}\hlstd{(}\hlstr{"2F3.1"}\hlstd{,} \hlstr{"F3.0"}\hlstd{,} \hlstr{"A3"}\hlstd{),}
                              \hlkwc{col.names} \hlstd{=} \hlkwd{c}\hlstd{(}\hlstr{"col1"}\hlstd{,} \hlstr{"col2"}\hlstd{,} \hlstr{"col3"}\hlstd{,} \hlstr{"col4"}\hlstd{))}
\end{alltt}
\end{kframe}
\end{knitrout}

\begin{knitrout}\footnotesize
\definecolor{shadecolor}{rgb}{0.969, 0.969, 0.969}\color{fgcolor}\begin{kframe}
\begin{alltt}
\hlkwd{sapply}\hlstd{(from_fwf_a.df, class)}
\end{alltt}
\begin{verbatim}
##        col1        col2        col3        col4 
##   "numeric"   "numeric"   "numeric" "character"
\end{verbatim}
\begin{alltt}
\hlstd{from_fwf_a.df[[}\hlstr{"col4"}\hlstd{]]}
\end{alltt}
\begin{verbatim}
## [1] "ABC" "Z Y"
\end{verbatim}
\end{kframe}
\end{knitrout}

\begin{explainbox}
  The file reading functions described above share with \Rfunction{read.table()} the same parameters. In addition to those described above, other frequently useful parameters are \code{skip} and \code{n}, which can be used to skip lines at the top of a file and limit the number of lines (or records) to read; \code{header} that accepts a logical argument indicating if the fields in first line read should be decoded as column names rather than data; \code{na.strings} to which can be passed a character vector with strings to be interpreted as \code{NA}; \code{colClasses} that provides control of the conversion of the fields to \Rlang classes and possibly skipping some columns altogether. All these parameters are described in the corresponding help pages.
\end{explainbox}

\begin{playground}
In reality  \Rfunction{read.csv()}, \code{read.csv2()} and \Rfunction{read.table()} are the same function with different default arguments to several of their parameters. Study the help page, and make by passing suitable arguments \Rfunction{read.csv()} behave as \Rfunction{read.table()}, then make \Rfunction{read.table()} behave as \Rfunction{read.csv2()}.
\end{playground}

\begin{explainbox}
We can read a text files as character strings, without attempting to decode them. This is occasionally useful, such as when we do the decoding as part of our own script. In this case the function to use is \code{readLines()}.
\end{explainbox}

Next we give one example of the use of a \emph{write} function matching one of the \emph{read} functions described above. The \Rfunction{write.csv()} function takes as argument a data frame or an object that can be coerced into a data frame converts it to character strings and saves it to a file. We first create the data frame that we will write to disk.

\begin{knitrout}\footnotesize
\definecolor{shadecolor}{rgb}{0.969, 0.969, 0.969}\color{fgcolor}\begin{kframe}
\begin{alltt}
\hlstd{my.df} \hlkwb{<-} \hlkwd{data.frame}\hlstd{(}\hlkwc{x} \hlstd{=} \hlnum{1}\hlopt{:}\hlnum{5}\hlstd{,} \hlkwc{y} \hlstd{=} \hlnum{5}\hlopt{:}\hlnum{1} \hlopt{/} \hlnum{10}\hlstd{,} \hlkwc{z} \hlstd{= letters[}\hlnum{1}\hlopt{:}\hlnum{5}\hlstd{])}
\end{alltt}
\end{kframe}
\end{knitrout}

We write \code{my.df} to a CSV file suitable for an English language locale, and then display its contents.
\pageref{sec:calc:attributes} for details.)

\begin{knitrout}\footnotesize
\definecolor{shadecolor}{rgb}{0.969, 0.969, 0.969}\color{fgcolor}\begin{kframe}
\begin{alltt}
\hlkwd{write.csv}\hlstd{(my.df,} \hlkwc{file} \hlstd{=} \hlstr{"my-file1.csv"}\hlstd{,} \hlkwc{row.names} \hlstd{=} \hlnum{FALSE}\hlstd{)}
\hlkwd{file.show}\hlstd{(}\hlstr{"my-file1.csv"}\hlstd{,} \hlkwc{pager} \hlstd{=} \hlstr{"console"}\hlstd{)}
\end{alltt}
\end{kframe}
\end{knitrout}

\begin{knitrout}\footnotesize
\definecolor{shadecolor}{rgb}{0.969, 0.969, 0.969}\color{fgcolor}\begin{kframe}
\begin{verbatim}
"x","y","z"
1,0.5,"a"
2,0.4,"b"
3,0.3,"c"
4,0.2,"d"
5,0.1,"e"
\end{verbatim}
\end{kframe}
\end{knitrout}

\begin{explainbox}
In most cases setting, as above, \code{row.names = FALSE} when writing a CSV file will help when it is read. Of course, if row names do contain important information, such as gene tags, you cannot skip writing the row names to the file unless you first copy these data into a column in the data frame. (Row names are stored separately as an attribute in \code{data.frame} objects, see section \ref{sec:calc:attributes} on page for details.)
\end{explainbox}

\begin{playground}
Write the data frame \code{my.df} into text files with functions \Rfunction{write.csv2()} and \Rfunction{write.table()} instead of \Rfunction{read.csv()} and display the files.
\end{playground}

Function \Rfunction{cat()} takes \Rlang objects and writes them after conversion to character strings to the console or a file, inserting one or more characters as separators, by default a space. This separator can be set through parameter \code{sep}. In our example we set \code{sep} to a new line (entered as the escape sequence \code{"\\n"}.

\begin{knitrout}\footnotesize
\definecolor{shadecolor}{rgb}{0.969, 0.969, 0.969}\color{fgcolor}\begin{kframe}
\begin{alltt}
\hlstd{my.lines} \hlkwb{<-} \hlkwd{c}\hlstd{(}\hlstr{"abcd"}\hlstd{,} \hlstr{"hello world"}\hlstd{,} \hlstr{"123.45"}\hlstd{)}
\hlkwd{cat}\hlstd{(my.lines,} \hlkwc{file} \hlstd{=} \hlstr{"my-file2.txt"}\hlstd{,} \hlkwc{sep} \hlstd{=} \hlstr{"\textbackslash{}n"}\hlstd{)}
\hlkwd{file.show}\hlstd{(}\hlstr{"my-file2.txt"}\hlstd{,} \hlkwc{pager} \hlstd{=} \hlstr{"console"}\hlstd{)}
\end{alltt}
\end{kframe}
\end{knitrout}

\begin{knitrout}\footnotesize
\definecolor{shadecolor}{rgb}{0.969, 0.969, 0.969}\color{fgcolor}\begin{kframe}
\begin{verbatim}
abcd
hello world
123.45
\end{verbatim}
\end{kframe}
\end{knitrout}

\subsection[readr]{\pkgname{readr}}\label{sec:files:readr}

\begin{knitrout}\footnotesize
\definecolor{shadecolor}{rgb}{0.969, 0.969, 0.969}\color{fgcolor}\begin{kframe}
\begin{alltt}
\hlkwd{citation}\hlstd{(}\hlkwc{package} \hlstd{=} \hlstr{"readr"}\hlstd{)}
\end{alltt}
\begin{verbatim}
## 
## To cite package 'readr' in publications use:
## 
##   Hadley Wickham, Jim Hester and Romain Francois (2018). readr:
##   Read Rectangular Text Data. R package version 1.3.1.
##   https://CRAN.R-project.org/package=readr
## 
## A BibTeX entry for LaTeX users is
## 
##   @Manual{,
##     title = {readr: Read Rectangular Text Data},
##     author = {Hadley Wickham and Jim Hester and Romain Francois},
##     year = {2018},
##     note = {R package version 1.3.1},
##     url = {https://CRAN.R-project.org/package=readr},
##   }
\end{verbatim}
\end{kframe}
\end{knitrout}

Package \pkgname{readr} is part of the \pkgname{tidyverse} suite. It defines functions that allow faster input and output, and have different default behaviour. Contrary to base \Rlang functions, they are optimized for speed, but may sometimes wrongly decode their input and sometimes silently do this. Base \Rlang functions do less \emph{guessing}, the delimiters must be supplied as arguments. The \pkgname{readr} functions guess more properties of the format, in most cases they succeed, which is very handy, but occasionally they fail. Defaults can be overridden by passing arguments. Another important advantage is that these functions read character strings formatted as dates or times directly into columns of class \code{POSIXct}. All \code{write} functions defined in \pkgname{readr} have an \code{append} parameter, which can be used to change the default behaviour of overwriting an existing file with the same name, to appending the output at its end.

Although in this section we exemplify the use of these functions by passing a file name as argument, as is the case with \Rlang native functions, URLs, and open file descriptors are also accepted (see section \ref{sec:???} on page \pageref{sec:???}). Furthermore, if the file name ends in a tag recognizable as indicating a compressed file format, the file will be uncompressed on-the-fly.

\begin{warningbox}
The names of functions ``equivalent'' to those described in the previous section have names formed by replacing the dot with an underscore, e.g.\ \Rfunction{read\_csv()} $\approx$ \Rfunction{read.csv()}. The similarity refers to the format of the files read, but not the order, names or roles of their formal parameters. Function \code{read\_table()} has a slightly different behaviour to \Rfunction{read.table()}, although they both read fields separated by white space. Other aspects of the default behaviour are also different, for example \pkgname{readr} functions do not convert columns of character strings into factors and row names are not set in the returned \Rclass{tibble}, which inherits from \Rclass{data.frame} but has somehow different behaviour (see section \ref{sec:data:tibble} on page \pageref{sec:data:tibble}).
\end{warningbox}

As we can see in this first example, these functions also report to the console the specifications of the columns, which is important when these are guessed from the file contents, or part of it.

\begin{knitrout}\footnotesize
\definecolor{shadecolor}{rgb}{0.969, 0.969, 0.969}\color{fgcolor}\begin{kframe}
\begin{alltt}
\hlkwd{read_csv}\hlstd{(}\hlkwc{file} \hlstd{=} \hlstr{"extdata/aligned-ASCII-UK.csv"}\hlstd{)}
\end{alltt}


{\ttfamily\noindent\itshape\color{messagecolor}{\#\# Parsed with column specification:\\\#\# cols(\\\#\#\ \  col1 = col\_double(),\\\#\#\ \  col2 = col\_double(),\\\#\#\ \  col3 = col\_double(),\\\#\#\ \  col4 = col\_character()\\\#\# )}}\begin{verbatim}
## # A tibble: 2 x 4
##    col1  col2  col3 col4 
##   <dbl> <dbl> <dbl> <chr>
## 1   1    24.5   346 ABC  
## 2  23.4  45.6    78 Z Y
\end{verbatim}
\end{kframe}
\end{knitrout}

\begin{knitrout}\footnotesize
\definecolor{shadecolor}{rgb}{0.969, 0.969, 0.969}\color{fgcolor}\begin{kframe}
\begin{alltt}
\hlkwd{read_csv}\hlstd{(}\hlkwc{file} \hlstd{=} \hlstr{"extdata/not-aligned-ASCII-UK.csv"}\hlstd{)}
\end{alltt}


{\ttfamily\noindent\itshape\color{messagecolor}{\#\# Parsed with column specification:\\\#\# cols(\\\#\#\ \  col1 = col\_double(),\\\#\#\ \  col2 = col\_double(),\\\#\#\ \  col3 = col\_double(),\\\#\#\ \  col4 = col\_character()\\\#\# )}}\begin{verbatim}
## # A tibble: 2 x 4
##    col1  col2  col3 col4 
##   <dbl> <dbl> <dbl> <chr>
## 1   1    24.5   346 ABC  
## 2  23.4  45.6    78 Z Y
\end{verbatim}
\end{kframe}
\end{knitrout}

Function \Rfunction{read\_table()}, differently to \Rfunction{read.table()} retains quotes as part of read character strings.

\begin{knitrout}\footnotesize
\definecolor{shadecolor}{rgb}{0.969, 0.969, 0.969}\color{fgcolor}\begin{kframe}
\begin{alltt}
\hlkwd{read_table}\hlstd{(}\hlkwc{file} \hlstd{=} \hlstr{"extdata/aligned-ASCII.txt"}\hlstd{)}
\end{alltt}


{\ttfamily\noindent\itshape\color{messagecolor}{\#\# Parsed with column specification:\\\#\# cols(\\\#\#\ \  col1 = col\_double(),\\\#\#\ \  col2 = col\_double(),\\\#\#\ \  col3 = col\_double(),\\\#\#\ \  col4 = col\_character()\\\#\# )}}\begin{verbatim}
## # A tibble: 2 x 4
##    col1  col2  col3 col4     
##   <dbl> <dbl> <dbl> <chr>    
## 1   1    24.5   346 ABC      
## 2  23.4  45.6    78 "\"Z Y\""
\end{verbatim}
\end{kframe}
\end{knitrout}

Because of the misaligned fields in file \code{"not-aligned-ASCII.txt"}, we need to use \Rfunction{read\_table2()}, which allows missalignment of fields, like \Rfunction{read.table()}, instead of \Rfunction{read\_table()} that expects vertically aligned fields across rows. However, in this case the embedded space character in the quoted string is misinterpreted and part of the string dropped with a warning.

\begin{knitrout}\footnotesize
\definecolor{shadecolor}{rgb}{0.969, 0.969, 0.969}\color{fgcolor}\begin{kframe}
\begin{alltt}
\hlkwd{read_table2}\hlstd{(}\hlkwc{file} \hlstd{=} \hlstr{"extdata/not-aligned-ASCII.txt"}\hlstd{)}
\end{alltt}


{\ttfamily\noindent\itshape\color{messagecolor}{\#\# Parsed with column specification:\\\#\# cols(\\\#\#\ \  col1 = col\_double(),\\\#\#\ \  col2 = col\_double(),\\\#\#\ \  col3 = col\_double(),\\\#\#\ \  col4 = col\_character()\\\#\# )}}

{\ttfamily\noindent\color{warningcolor}{\#\# Warning: 1 parsing failure.\\\#\# row col\ \ expected\ \ \ \ actual\ \ \ \ \ \ \ \ \ \ \ \ \ \ \ \ \ \ \ \ \ \ \ \ \ \ \ \ file\\\#\#\ \  2\ \ -- 4 columns 5 columns 'extdata/not-aligned-ASCII.txt'}}\begin{verbatim}
## # A tibble: 2 x 4
##    col1  col2  col3 col4 
##   <dbl> <dbl> <dbl> <chr>
## 1   1    24.5   346 ABC  
## 2  23.4  45.6    78 "\"Z"
\end{verbatim}
\end{kframe}
\end{knitrout}

Function \Rfunction{read\_delim()} with space as delimiter needs to be used.

\begin{knitrout}\footnotesize
\definecolor{shadecolor}{rgb}{0.969, 0.969, 0.969}\color{fgcolor}\begin{kframe}
\begin{alltt}
\hlkwd{read_delim}\hlstd{(}\hlkwc{file} \hlstd{=} \hlstr{"extdata/not-aligned-ASCII.txt"}\hlstd{,} \hlkwc{delim} \hlstd{=} \hlstr{" "}\hlstd{)}
\end{alltt}


{\ttfamily\noindent\itshape\color{messagecolor}{\#\# Parsed with column specification:\\\#\# cols(\\\#\#\ \  col1 = col\_double(),\\\#\#\ \  col2 = col\_double(),\\\#\#\ \  col3 = col\_double(),\\\#\#\ \  col4 = col\_character()\\\#\# )}}\begin{verbatim}
## # A tibble: 2 x 4
##    col1  col2  col3 col4 
##   <dbl> <dbl> <dbl> <chr>
## 1   1    24.5   346 ABC  
## 2  23.4  45.6    78 Z Y
\end{verbatim}
\end{kframe}
\end{knitrout}

Function \Rfunction{read\_tsv()} reads files with tab character as delimiter, and \Rfunction{read\_fwf())} reads files with fixed width fields. There is, however, no equivalent to \Rfunction{read.fortran()}, supporting implicit decimal points.

\begin{playground}
Use the "wrong" \code{read\_} functions to read the example files used above and/or your own files. As mentioned earlier forcing errors will help you learn how to diagnose when such errors are caused by coding mistakes. In this case, as wrongly read data are not always accompanied by error or warning messages, carefully check the returned tibbles for missread data values.
\end{playground}

\begin{explainbox}
The functions from R's \pkgname{utils} read the whole file as text before attempting to guess the class of the columns or their alignment. This is reliable but slow for very large text files. The functions from \pkgname{readr} read only the top 1000 lines by default for guessing, and then rather blindly read the whole files assuming that the guessed at properties also apply to the remaining of the file. This is more efficient, but somehow risky. In earlier versions of \pkgname{readr} a typical failure to correctly decode fields was when numbers are in increasing order and the field widths continue increasing in the lines below those used for guessing, but this case seems to be at the time of writing correctly handled. It also means that in cases when an individual value after \code{guess\_max} lines cannot be converted to numeric, instead of returning a column of character strings, this value is encoded as a numeric \code{NA} with a warning and the column \code{numeric}. To demonstrate this we will drastically reduce \code{guess\_max} from its default so that we can use an example file only a few lines in length.

\begin{knitrout}\footnotesize
\definecolor{shadecolor}{rgb}{0.969, 0.969, 0.969}\color{fgcolor}\begin{kframe}
\begin{alltt}
\hlkwd{read_table2}\hlstd{(}\hlkwc{file} \hlstd{=} \hlstr{"extdata/miss-aligned-ASCII.txt"}\hlstd{)}
\end{alltt}


{\ttfamily\noindent\itshape\color{messagecolor}{\#\# Parsed with column specification:\\\#\# cols(\\\#\#\ \  col1 = col\_character(),\\\#\#\ \  col2 = col\_double(),\\\#\#\ \  col3 = col\_double(),\\\#\#\ \  col4 = col\_character()\\\#\# )}}\begin{verbatim}
## # A tibble: 4 x 4
##   col1   col2  col3 col4 
##   <chr> <dbl> <dbl> <chr>
## 1 1.0    24.5   346 ABC  
## 2 2.4    45.6    78 XYZ  
## 3 20.4   45.6    78 XYZ  
## 4 a      20    2500 abc
\end{verbatim}
\end{kframe}
\end{knitrout}

\begin{knitrout}\footnotesize
\definecolor{shadecolor}{rgb}{0.969, 0.969, 0.969}\color{fgcolor}\begin{kframe}
\begin{alltt}
\hlkwd{read_table2}\hlstd{(}\hlkwc{file} \hlstd{=} \hlstr{"extdata/miss-aligned-ASCII.txt"}\hlstd{,} \hlkwc{guess_max} \hlstd{=} \hlnum{3L}\hlstd{)}
\end{alltt}


{\ttfamily\noindent\itshape\color{messagecolor}{\#\# Parsed with column specification:\\\#\# cols(\\\#\#\ \  col1 = col\_double(),\\\#\#\ \  col2 = col\_double(),\\\#\#\ \  col3 = col\_double(),\\\#\#\ \  col4 = col\_character()\\\#\# )}}

{\ttfamily\noindent\color{warningcolor}{\#\# Warning: 1 parsing failure.\\\#\# row\ \ col expected actual\ \ \ \ \ \ \ \ \ \ \ \ \ \ \ \ \ \ \ \ \ \ \ \ \ \ \ \  file\\\#\#\ \  4 col1 a double\ \ \ \ \ \ a 'extdata/miss-aligned-ASCII.txt'}}\begin{verbatim}
## # A tibble: 4 x 4
##    col1  col2  col3 col4 
##   <dbl> <dbl> <dbl> <chr>
## 1   1    24.5   346 ABC  
## 2   2.4  45.6    78 XYZ  
## 3  20.4  45.6    78 XYZ  
## 4  NA    20    2500 abc
\end{verbatim}
\end{kframe}
\end{knitrout}
\end{explainbox}

The \code{write\_} functions from \pkgname{readr} are the counterpart to the \code{read\_} funtions above and the \code{write.} functions from \pkgname{utils}. In addition to the expected \Rfunction{write\_csv()}, \Rfunction{write\_csv2()}, \Rfunction{write\_tsv()}, \Rfunction{write\_delim()} and \code{read\_fwf()} functions, \pkgname{readr} provides functions that write \pgrmname{MS-Excel}-friendly CSV files. We demonstrate here the use of \Rfunction{write\_excel\_csv()} to produce a text file with comma-separated fields suitable for import into \pgrmname{MS-Excel}.

\begin{knitrout}\footnotesize
\definecolor{shadecolor}{rgb}{0.969, 0.969, 0.969}\color{fgcolor}\begin{kframe}
\begin{alltt}
\hlkwd{write_excel_csv}\hlstd{(my.df,} \hlkwc{path} \hlstd{=} \hlstr{"my-file6.csv"}\hlstd{)}
\hlkwd{file.show}\hlstd{(}\hlstr{"my-file6.csv"}\hlstd{,} \hlkwc{pager} \hlstd{=} \hlstr{"console"}\hlstd{)}
\end{alltt}
\end{kframe}
\end{knitrout}

That saves a file containing the following text.
\begin{knitrout}\footnotesize
\definecolor{shadecolor}{rgb}{0.969, 0.969, 0.969}\color{fgcolor}\begin{kframe}
\begin{verbatim}
x,y,z
1,0.5,a
2,0.4,b
3,0.3,c
4,0.2,d
5,0.1,e
\end{verbatim}
\end{kframe}
\end{knitrout}

\begin{playground}
Compare the output from \Rfunction{write\_excel\_csv()} and \Rfunction{write\_csv()}. What is the difference? Does it matter when you import the written CSV file into Excel (the version you are using, with the locale settings of your computer).
\end{playground}

The pair of functions \Rfunction{read\_lines()} and \Rfunction{write\_lines()} read and write character vectors without conversion. Functions \Rfunction{read\_file()} and \Rfunction{write\_file()} read and write a single object such as a character vector of length one.

\begin{advplayground}
Use \Rfunction{write\_file()} to write a file that can be read with \Rfunction{read\_csv()}.
\end{advplayground}

\section{XML and HTML files}

XML files contain text with special markup. Several modern data exchange formats are based on the \langname{XML} standard (see \url{https://www.w3.org/TR/xml/}) which uses schemas for flexibility. Schemas define specific formats, allowing reading of formats not specifically target during development of the read functions. Even the modern \langname{XHTML} standard used for web pages is based on such schemas, while \langname{HTML} only differs slightly in its syntax.

\subsection['xml2']{\pkgname{xml2}}

\begin{knitrout}\footnotesize
\definecolor{shadecolor}{rgb}{0.969, 0.969, 0.969}\color{fgcolor}\begin{kframe}
\begin{alltt}
\hlkwd{citation}\hlstd{(}\hlkwc{package} \hlstd{=} \hlstr{"xml2"}\hlstd{)}
\end{alltt}
\begin{verbatim}
## 
## To cite package 'xml2' in publications use:
## 
##   Hadley Wickham, James Hester and Jeroen Ooms (2018). xml2: Parse
##   XML. R package version 1.2.0.
##   https://CRAN.R-project.org/package=xml2
## 
## A BibTeX entry for LaTeX users is
## 
##   @Manual{,
##     title = {xml2: Parse XML},
##     author = {Hadley Wickham and James Hester and Jeroen Ooms},
##     year = {2018},
##     note = {R package version 1.2.0},
##     url = {https://CRAN.R-project.org/package=xml2},
##   }
\end{verbatim}
\end{kframe}
\end{knitrout}

Package \pkgname{xml2} provides functions for reading and parsing \langname{XTML} and \langname{HTML} files. This is a vast subject, of which I will only give a brief example.

We first read a web page with function \Rfunction{read\_html()}, and explore its structure.

\begin{knitrout}\footnotesize
\definecolor{shadecolor}{rgb}{0.969, 0.969, 0.969}\color{fgcolor}\begin{kframe}
\begin{alltt}
\hlstd{web_page} \hlkwb{<-} \hlkwd{read_html}\hlstd{(}\hlstr{"http://r4photobiology.info/R/index.html"}\hlstd{)}
\hlkwd{html_structure}\hlstd{(web_page)}
\end{alltt}
\begin{verbatim}
## <html>
##   <head>
##     <title>
##       {text}
##     <meta [name, content]>
##     <meta [name, content]>
##     <meta [name, content]>
##   <body>
##     {text}
##     <hr>
##     <h1>
##       {text}
##     {text}
##     <hr>
##     <p>
##       {text}
##       <a [href]>
##         {text}
##       {text}
##     {text}
##     <address>
##       {text}
##     {text}
\end{verbatim}
\end{kframe}
\end{knitrout}

Next we extract the text from its \code{title} attribute, using functions \Rfunction{xml\_find\_all()} and \Rfunction{xml\_text()}.

\begin{knitrout}\footnotesize
\definecolor{shadecolor}{rgb}{0.969, 0.969, 0.969}\color{fgcolor}\begin{kframe}
\begin{alltt}
\hlkwd{xml_text}\hlstd{(}\hlkwd{xml_find_all}\hlstd{(web_page,} \hlstr{".//title"}\hlstd{))}
\end{alltt}
\begin{verbatim}
## [1] "r4photobiology repository"
\end{verbatim}
\end{kframe}
\end{knitrout}

The functions defined in this package can be used to ``harvest'' data from web pages, but also to read data from files using formats that are defined through \langname{XML} schemas.

\section{Worksheets}\label{sec:files:worksheets}

Microsoft Office, Open Office and Libre Office are the most frequently used suites containing programs based on the worksheet paradigm. There is available a standardized file format for exchange of worksheet data, but it does not support all the features present in native file formats. We will start by considering \pgrmname{MS-Excel}. The file format used by \pgrmname{MS-Excel} has changed significantly over the years, and old formats tend to be less well supported by available \Rlang packages and may require the file to be updated to a more modern format with \pgrmname{MS-Excel} itself before import into \Rlang. The current format is based on XML and relatively simple to decode, older binary formats are more difficult. Work sheets contain code as equations in addition to the actual data. In all cases only values entered as such or those computed by means of the embedded equations can be imported into \Rlang rather than the equations themselves.

\subsection{CSV files as middlemen}

If we have access to the original software used for creating a worksheet or workbook, then exporting worksheets to text files in CSV format and importing these it into \Rlang using the functions described in sections \ref{sec:files:txt} and \ref{sec:files:readr} starting on pages \pageref{sec:files:txt} and \pageref{sec:files:readr} provides a broadly compatible route for importing data---with the caveat that we should take care that delimiters and decimal marks match the expectations of the functions used. This approach is not ideal from the perspective of having to recreate intermediate files. A better approach is, when feasible, to import the data directly from the workbook or worksheets into \Rlang.

\subsection['readxl']{\pkgname{readxl}}\label{sec:files:excel}

\begin{knitrout}\footnotesize
\definecolor{shadecolor}{rgb}{0.969, 0.969, 0.969}\color{fgcolor}\begin{kframe}
\begin{alltt}
\hlkwd{citation}\hlstd{(}\hlkwc{package} \hlstd{=} \hlstr{"readxl"}\hlstd{)}
\end{alltt}
\begin{verbatim}
## 
## To cite package 'readxl' in publications use:
## 
##   Hadley Wickham and Jennifer Bryan (2019). readxl: Read Excel
##   Files. R package version 1.3.1.
##   https://CRAN.R-project.org/package=readxl
## 
## A BibTeX entry for LaTeX users is
## 
##   @Manual{,
##     title = {readxl: Read Excel Files},
##     author = {Hadley Wickham and Jennifer Bryan},
##     year = {2019},
##     note = {R package version 1.3.1},
##     url = {https://CRAN.R-project.org/package=readxl},
##   }
\end{verbatim}
\end{kframe}
\end{knitrout}

Package \pkgname{readxl} supports reading of \pgrmname{MS-Excel} workbooks, selecting worksheets and regions within worksheets specified in ways similar to those used by \pgrmname{MS-Excel} itself. The interface is simple, and the package easy to instal. We will import a file that in \pgrmname{MS-Excel} looks as in the screen capture below.

\begin{center}
\includegraphics[width=0.75\textwidth]{figures/Book1-xlsx.png}
\end{center}

We first list the sheets contained in the workbook file with \Rfunction{excel\_sheets()}.

\begin{knitrout}\footnotesize
\definecolor{shadecolor}{rgb}{0.969, 0.969, 0.969}\color{fgcolor}\begin{kframe}
\begin{alltt}
\hlstd{sheets} \hlkwb{<-} \hlkwd{excel_sheets}\hlstd{(}\hlstr{"extdata/Book1.xlsx"}\hlstd{)}
\hlstd{sheets}
\end{alltt}
\begin{verbatim}
## [1] "my data"
\end{verbatim}
\end{kframe}
\end{knitrout}

In this case the argument passed to \code{sheet} is redundant, as there is only a single worksheet in the file. It is possible to use either the name of the sheet or a positional index (in this case \code{1} would be equivalent to \code{"my data"}). We use function \Rfunction{read\_excel()} to import the worksheet. Being part of the \pkgname{tidyverse} the returned value is a tibble and character columns returned as such.

\begin{knitrout}\footnotesize
\definecolor{shadecolor}{rgb}{0.969, 0.969, 0.969}\color{fgcolor}\begin{kframe}
\begin{alltt}
\hlstd{Book1.df} \hlkwb{<-} \hlkwd{read_excel}\hlstd{(}\hlstr{"extdata/Book1.xlsx"}\hlstd{,} \hlkwc{sheet} \hlstd{=} \hlstr{"my data"}\hlstd{)}
\hlstd{Book1.df}
\end{alltt}
\begin{verbatim}
## # A tibble: 10 x 3
##    sample group observation
##     <dbl> <chr>       <dbl>
##  1      1 a             1  
##  2      2 a             5  
##  3      3 a             7  
##  4      4 a             2  
##  5      5 a             5  
##  6      6 b             0  
##  7      7 b             2  
##  8      8 b             3  
##  9      9 b             1  
## 10     10 b             1.5
\end{verbatim}
\end{kframe}
\end{knitrout}

We can also read a region instead of the whole worksheet.

\begin{knitrout}\footnotesize
\definecolor{shadecolor}{rgb}{0.969, 0.969, 0.969}\color{fgcolor}\begin{kframe}
\begin{alltt}
\hlstd{Book1_region.df} \hlkwb{<-} \hlkwd{read_excel}\hlstd{(}\hlstr{"extdata/Book1.xlsx"}\hlstd{,} \hlkwc{sheet} \hlstd{=} \hlstr{"my data"}\hlstd{,} \hlkwc{range} \hlstd{=} \hlstr{"A1:B8"}\hlstd{)}
\hlstd{Book1_region.df}
\end{alltt}
\begin{verbatim}
## # A tibble: 7 x 2
##   sample group
##    <dbl> <chr>
## 1      1 a    
## 2      2 a    
## 3      3 a    
## 4      4 a    
## 5      5 a    
## 6      6 b    
## 7      7 b
\end{verbatim}
\end{kframe}
\end{knitrout}

Of the remaining arguments, the most useful ones have the same names and play similar roles as in \pkgname{readr} (see section \ref{sec:files:readr} on page \pageref{sec:files:readr}).

\subsection['xlsx']{\pkgname{xlsx}}

\begin{knitrout}\footnotesize
\definecolor{shadecolor}{rgb}{0.969, 0.969, 0.969}\color{fgcolor}\begin{kframe}
\begin{alltt}
\hlkwd{citation}\hlstd{(}\hlkwc{package} \hlstd{=} \hlstr{"xlsx"}\hlstd{)}
\end{alltt}
\begin{verbatim}
## 
## To cite package 'xlsx' in publications use:
## 
##   Adrian A. Dragulescu and Cole Arendt (2018). xlsx: Read, Write,
##   Format Excel 2007 and Excel 97/2000/XP/2003 Files. R package
##   version 0.6.1. https://CRAN.R-project.org/package=xlsx
## 
## A BibTeX entry for LaTeX users is
## 
##   @Manual{,
##     title = {xlsx: Read, Write, Format Excel 2007 and Excel 97/2000/XP/2003 Files},
##     author = {Adrian A. Dragulescu and Cole Arendt},
##     year = {2018},
##     note = {R package version 0.6.1},
##     url = {https://CRAN.R-project.org/package=xlsx},
##   }
## 
## ATTENTION: This citation information has been auto-generated from
## the package DESCRIPTION file and may need manual editing, see
## 'help("citation")'.
\end{verbatim}
\end{kframe}
\end{knitrout}

Package \pkgname{xlsx} can be more difficult to install as it uses Java functions to do the actual work. However, it is more comprehensive, with functions both for reading and writing \pgrmname{MS-Excel} worksheet and workbooks, in different formats including the older binary ones. Similarly to \pkgname{readr} it allows selecting regions of a worksheet to be imported.

Here we use function \Rfunction{read.xlsx()}, indexing the worksheet by name. The returned value is a data frame, and following the expectations of \Rlang \pkgnameNI{utils} character columns are converted in factors by default.

\begin{knitrout}\footnotesize
\definecolor{shadecolor}{rgb}{0.969, 0.969, 0.969}\color{fgcolor}\begin{kframe}
\begin{alltt}
\hlstd{Book1_xlsx.df} \hlkwb{<-} \hlkwd{read.xlsx}\hlstd{(}\hlstr{"extdata/Book1.xlsx"}\hlstd{,} \hlkwc{sheetName} \hlstd{=} \hlstr{"my data"}\hlstd{)}
\hlstd{Book1_xlsx.df}
\end{alltt}
\begin{verbatim}
##    sample group observation
## 1       1     a         1.0
## 2       2     a         5.0
## 3       3     a         7.0
## 4       4     a         2.0
## 5       5     a         5.0
## 6       6     b         0.0
## 7       7     b         2.0
## 8       8     b         3.0
## 9       9     b         1.0
## 10     10     b         1.5
\end{verbatim}
\end{kframe}
\end{knitrout}

With function \Rfunction{write.xlsx()} we can write data frames out to Excel worksheets and even append new worksheets to an existing workbook.

\begin{knitrout}\footnotesize
\definecolor{shadecolor}{rgb}{0.969, 0.969, 0.969}\color{fgcolor}\begin{kframe}
\begin{alltt}
\hlkwd{set.seed}\hlstd{(}\hlnum{456321}\hlstd{)}
\hlstd{my.data} \hlkwb{<-} \hlkwd{data.frame}\hlstd{(}\hlkwc{x} \hlstd{=} \hlnum{1}\hlopt{:}\hlnum{10}\hlstd{,} \hlkwc{y} \hlstd{= letters[}\hlnum{1}\hlopt{:}\hlnum{10}\hlstd{])}
\hlkwd{write.xlsx}\hlstd{(my.data,} \hlkwc{file} \hlstd{=} \hlstr{"extdata/my-data.xlsx"}\hlstd{,} \hlkwc{sheetName} \hlstd{=} \hlstr{"first copy"}\hlstd{)}
\hlkwd{write.xlsx}\hlstd{(my.data,} \hlkwc{file} \hlstd{=} \hlstr{"extdata/my-data.xlsx"}\hlstd{,} \hlkwc{sheetName} \hlstd{=} \hlstr{"second copy"}\hlstd{,} \hlkwc{append} \hlstd{=} \hlnum{TRUE}\hlstd{)}
\end{alltt}
\end{kframe}
\end{knitrout}

When opened in Excel we get a workbook, containing two worksheets, named using the arguments we passed through \code{sheetName} in the code chunk above.
% screen capture to be replaced!!
\begin{center}
\includegraphics[width=0.75\textwidth]{figures/my-data-xlsx.png}
\end{center}

\begin{playground}
If you have some worksheet files available, import them into \Rlang, to get a feel of how the way data is organized in the worksheets affects how easy or difficult it is to import them into \Rlang.
\end{playground}

\subsection['readODS']{\pkgname{readODS}}

Package \pkgname{readODS} provides functions for reading data saved in files that follow the \emph{Open Documents Standard}. Function \Rfunction{read\_ods()} has a similar but simpler user interface to that of \code{read\_excel()} and reads one worksheet at a time, with support only for skipping top rows. The value returned is a data frame.

\begin{knitrout}\footnotesize
\definecolor{shadecolor}{rgb}{0.969, 0.969, 0.969}\color{fgcolor}\begin{kframe}
\begin{alltt}
\hlkwd{read_ods}\hlstd{(}\hlstr{"extdata/Book1.ods"}\hlstd{,} \hlkwc{sheet} \hlstd{=} \hlnum{1}\hlstd{)}
\end{alltt}


{\ttfamily\noindent\itshape\color{messagecolor}{\#\# Parsed with column specification:\\\#\# cols(\\\#\#\ \  sample = col\_double(),\\\#\#\ \  group = col\_character(),\\\#\#\ \  observation = col\_double()\\\#\# )}}\begin{verbatim}
##    sample group observation
## 1       1     a         1.0
## 2       2     a         5.0
## 3       3     a         7.0
## 4       4     a         2.0
## 5       5     a         5.0
## 6       6     b         0.0
## 7       7     b         2.0
## 8       8     b         3.0
## 9       9     b         1.0
## 10     10     b         1.5
\end{verbatim}
\end{kframe}
\end{knitrout}

Function \Rfunction{write\_ods()} writes a data frame into an ODS file.

\section{Statistical software}\label{sec:files:stat}

There are two different comprehensive packages for importing data saved from other statistical such as SAS, Statistica, SPSS, etc. The long time ``standard'' the \pkgname{foreign} package and the much newer \pkgname{haven}. In the case of files saved with old versions of statistical programs, functions from \pkgname{foreign} tend to be more robust than those from \pkgname{haven}.

\subsection[foreign]{\pkgname{foreign}}

\begin{knitrout}\footnotesize
\definecolor{shadecolor}{rgb}{0.969, 0.969, 0.969}\color{fgcolor}\begin{kframe}
\begin{alltt}
\hlkwd{citation}\hlstd{(}\hlkwc{package} \hlstd{=} \hlstr{"foreign"}\hlstd{)}
\end{alltt}
\begin{verbatim}
## 
## To cite package 'foreign' in publications use:
## 
##   R Core Team (2018). foreign: Read Data Stored by 'Minitab', 'S',
##   'SAS', 'SPSS', 'Stata', 'Systat', 'Weka', 'dBase', .... R
##   package version 0.8-71.
##   https://CRAN.R-project.org/package=foreign
## 
## A BibTeX entry for LaTeX users is
## 
##   @Manual{,
##     title = {foreign: Read Data Stored by 'Minitab', 'S', 'SAS', 'SPSS', 'Stata',
## 'Systat', 'Weka', 'dBase', ...},
##     author = {{R Core Team}},
##     year = {2018},
##     note = {R package version 0.8-71},
##     url = {https://CRAN.R-project.org/package=foreign},
##   }
\end{verbatim}
\end{kframe}
\end{knitrout}

Functions in package \pkgname{foreign} allow to import data from files saved by several foreign statistical analysis programs, including \pgrmname{SAS}, \pgrmname{Stata}, \pgrmname{SPPS}, \pgrmname{Systat}, \pgrmname{Octave} among others, and a function for writing data into files with formats native to \pgrmname{SAS}, \pgrmname{Stata}, and \pgrmname{SPPS}. \Rlang documents them in detail in \emph{R Data Import/Export}. As a simple example we use function \Rfunction{read.spss()} to read a \texttt{.sav} file, saved a few years ago with the then current version of \pgrmname{SPSS}. We display only the first six rows and seven columns of the data frame, including a column with dates, which appears as numeric.

\begin{knitrout}\footnotesize
\definecolor{shadecolor}{rgb}{0.969, 0.969, 0.969}\color{fgcolor}\begin{kframe}
\begin{alltt}
\hlstd{my_spss.df} \hlkwb{<-} \hlkwd{read.spss}\hlstd{(}\hlkwc{file} \hlstd{=} \hlstr{"extdata/my-data.sav"}\hlstd{,} \hlkwc{to.data.frame} \hlstd{=} \hlnum{TRUE}\hlstd{)}
\end{alltt}


{\ttfamily\noindent\itshape\color{messagecolor}{\#\# re-encoding from UTF-8}}\begin{alltt}
\hlstd{my_spss.df[}\hlnum{1}\hlopt{:}\hlnum{6}\hlstd{,} \hlkwd{c}\hlstd{(}\hlnum{1}\hlopt{:}\hlnum{6}\hlstd{,} \hlnum{17}\hlstd{)]}
\end{alltt}
\begin{verbatim}
##   block       treat mycotreat water1 pot harvest harvest_date
## 1     0 Watered, EM         1      1  14       1  13653705600
## 2     0 Watered, EM         1      1  52       1  13653705600
## 3     0 Watered, EM         1      1 111       1  13653705600
## 4     0 Watered, EM         1      1 127       1  13653705600
## 5     0 Watered, EM         1      1 230       1  13653705600
## 6     0 Watered, EM         1      1 258       1  13653705600
\end{verbatim}
\end{kframe}
\end{knitrout}

A second example, this time with a simple \code{.sav} file saved 15 years ago.

\begin{knitrout}\footnotesize
\definecolor{shadecolor}{rgb}{0.969, 0.969, 0.969}\color{fgcolor}\begin{kframe}
\begin{alltt}
\hlstd{thiamin.df} \hlkwb{<-} \hlkwd{read.spss}\hlstd{(}\hlkwc{file} \hlstd{=} \hlstr{"extdata/thiamin.sav"}\hlstd{,} \hlkwc{to.data.frame} \hlstd{=} \hlnum{TRUE}\hlstd{)}
\hlkwd{head}\hlstd{(thiamin.df)}
\end{alltt}
\begin{verbatim}
##   THIAMIN CEREAL
## 1     5.2  wheat
## 2     4.5  wheat
## 3     6.0  wheat
## 4     6.1  wheat
## 5     6.7  wheat
## 6     5.8  wheat
\end{verbatim}
\end{kframe}
\end{knitrout}

Another example, for a \pgrmname{Systat} file saved on an PC more than 20 years ago, and read with \Rfunction{read.systat()}.

\begin{knitrout}\footnotesize
\definecolor{shadecolor}{rgb}{0.969, 0.969, 0.969}\color{fgcolor}\begin{kframe}
\begin{alltt}
\hlstd{my_systat.df} \hlkwb{<-} \hlkwd{read.systat}\hlstd{(}\hlkwc{file} \hlstd{=} \hlstr{"extdata/BIRCH1.SYS"}\hlstd{)}
\hlkwd{head}\hlstd{(my_systat.df)}
\end{alltt}
\begin{verbatim}
##   CONT DENS BLOCK SEEDL VITAL BASE ANGLE HEIGHT DIAM
## 1    1    1     1     2    44    2     0      1   53
## 2    1    1     1     2    41    2     1      2   70
## 3    1    1     1     2    21    2     0      1   65
## 4    1    1     1     2    15    3     0      1   79
## 5    1    1     1     2    37    3     0      1   71
## 6    1    1     1     2    29    2     1      1   43
\end{verbatim}
\end{kframe}
\end{knitrout}

The functions in \pkgname{foreign} can return data frames, but not always this is the default.

\subsection[haven]{\pkgname{haven}}

\begin{knitrout}\footnotesize
\definecolor{shadecolor}{rgb}{0.969, 0.969, 0.969}\color{fgcolor}\begin{kframe}
\begin{alltt}
\hlkwd{citation}\hlstd{(}\hlkwc{package} \hlstd{=} \hlstr{"haven"}\hlstd{)}
\end{alltt}
\begin{verbatim}
## 
## To cite package 'haven' in publications use:
## 
##   Hadley Wickham and Evan Miller (2019). haven: Import and Export
##   'SPSS', 'Stata' and 'SAS' Files. R package version 2.1.1.
##   https://CRAN.R-project.org/package=haven
## 
## A BibTeX entry for LaTeX users is
## 
##   @Manual{,
##     title = {haven: Import and Export 'SPSS', 'Stata' and 'SAS' Files},
##     author = {Hadley Wickham and Evan Miller},
##     year = {2019},
##     note = {R package version 2.1.1},
##     url = {https://CRAN.R-project.org/package=haven},
##   }
\end{verbatim}
\end{kframe}
\end{knitrout}

Package \pkgname{haven} is less ambitious with respect to the number of formats supported, or their vintages, providing read and write functions for only three file formats: \pgrmname{SAS}, \pgrmname{Stata} and \pgrmname{SPSS}. On the other hand \pkgname{haven} provides flexible ways to convert the different labelled values that cannot be directly mapped to normal \Rlang modes. They also decode dates and times according to the idiosyncrasies of each of these file formats. In cases when the imported file contains labelled values the returned \Rclass{tibble} object needs some additional attention from the user before obtaining `normal' data-frame-compatible \Rclass{tibble} objects.

We can use function \Rfunction{read\_sav()} to import a \code{.sav} file saved by a recent version of \pgrmname{SPSS}. As in the previous section, we display only the first six rows and seven columns of the data frame, including a column with dates, which appears as numeric.

\begin{knitrout}\footnotesize
\definecolor{shadecolor}{rgb}{0.969, 0.969, 0.969}\color{fgcolor}\begin{kframe}
\begin{alltt}
\hlstd{my_spss.tb} \hlkwb{<-} \hlkwd{read_sav}\hlstd{(}\hlkwc{file} \hlstd{=} \hlstr{"extdata/my-data.sav"}\hlstd{)}
\hlstd{my_spss.tb[}\hlnum{1}\hlopt{:}\hlnum{6}\hlstd{,} \hlkwd{c}\hlstd{(}\hlnum{1}\hlopt{:}\hlnum{6}\hlstd{,} \hlnum{17}\hlstd{)]}
\end{alltt}
\begin{verbatim}
## # A tibble: 6 x 7
##   block           treat mycotreat water1   pot harvest harvest_date
##   <dbl>       <dbl+lbl>     <dbl>  <dbl> <dbl>   <dbl> <date>      
## 1     0 1 [Watered, EM]         1      1    14       1 2015-06-15  
## 2     0 1 [Watered, EM]         1      1    52       1 2015-06-15  
## 3     0 1 [Watered, EM]         1      1   111       1 2015-06-15  
## 4     0 1 [Watered, EM]         1      1   127       1 2015-06-15  
## 5     0 1 [Watered, EM]         1      1   230       1 2015-06-15  
## 6     0 1 [Watered, EM]         1      1   258       1 2015-06-15
\end{verbatim}
\end{kframe}
\end{knitrout}

In this case the dates are correctly decoded.

And an \pgrmname{SPSS}'s \code{.sav} file saved 15 years ago.

\begin{knitrout}\footnotesize
\definecolor{shadecolor}{rgb}{0.969, 0.969, 0.969}\color{fgcolor}\begin{kframe}
\begin{alltt}
\hlstd{thiamin.tb} \hlkwb{<-} \hlkwd{read_sav}\hlstd{(}\hlkwc{file} \hlstd{=} \hlstr{"extdata/thiamin.sav"}\hlstd{)}
\hlstd{thiamin.tb}
\end{alltt}
\begin{verbatim}
## # A tibble: 24 x 2
##    THIAMIN     CEREAL
##      <dbl>  <dbl+lbl>
##  1     5.2 1 [wheat] 
##  2     4.5 1 [wheat] 
##  3     6   1 [wheat] 
##  4     6.1 1 [wheat] 
##  5     6.7 1 [wheat] 
##  6     5.8 1 [wheat] 
##  7     6.5 2 [barley]
##  8     8   2 [barley]
##  9     6.1 2 [barley]
## 10     7.5 2 [barley]
## # ... with 14 more rows
\end{verbatim}
\begin{alltt}
\hlstd{thiamin.tb} \hlkwb{<-} \hlkwd{as_factor}\hlstd{(thiamin.tb)}
\hlstd{thiamin.tb}
\end{alltt}
\begin{verbatim}
## # A tibble: 24 x 2
##    THIAMIN CEREAL
##      <dbl> <fct> 
##  1     5.2 wheat 
##  2     4.5 wheat 
##  3     6   wheat 
##  4     6.1 wheat 
##  5     6.7 wheat 
##  6     5.8 wheat 
##  7     6.5 barley
##  8     8   barley
##  9     6.1 barley
## 10     7.5 barley
## # ... with 14 more rows
\end{verbatim}
\end{kframe}
\end{knitrout}

\begin{playground}
Compare the values returned by different \code{read} functions when applied to the same file on disk. Use \Rfunction{names()}, \Rfunction{str()} and \Rfunction{class()} as tools in your exploration. If you are brave, also use \Rfunction{attributes()}, \Rfunction{mode()}, \Rfunction{dim()}, \Rfunction{dimnames()}, \Rfunction{nrow()} and \Rfunction{ncol()}.
\end{playground}

\begin{playground}
If you use or have used in the past other statistical software or a general purpose language like \langname{Python}, look up some files, and import them into R.
\end{playground}

\section{NetCDF files}

In some fields including geophysics and meteorology NetCDF is a very common format for the exchange of data. It is also used in other contexts in which data is referenced to an array of locations, like with data read from Affymetrix micro arrays used to study gene expression. The NetCDF format allows the storage of metadata together with the data itself in a well organized and standardized format, which is ideal for exchange of moderately large data sets.

Officially described as
\begin{quote}
NetCDF is a set of software libraries and self-describing, machine-independent data formats that support the creation, access, and sharing of array-oriented scientific data.
\end{quote}

As sometimes NetCDF files are large, it is good that it is possible to selectively read the data from individual variables with functions in packages \pkgname{ncdf4} or \pkgname{RNetCDF}. On the other hand, this implies that contrary to other data file reading operations, reading a NetCDF file is done in two or more steps.

\subsection[ncdf4]{\pkgname{ncdf4}}

\begin{knitrout}\footnotesize
\definecolor{shadecolor}{rgb}{0.969, 0.969, 0.969}\color{fgcolor}\begin{kframe}
\begin{alltt}
\hlkwd{citation}\hlstd{(}\hlkwc{package} \hlstd{=} \hlstr{"ncdf4"}\hlstd{)}
\end{alltt}
\begin{verbatim}
## 
## To cite package 'ncdf4' in publications use:
## 
##   David Pierce (2019). ncdf4: Interface to Unidata netCDF (Version
##   4 or Earlier) Format Data Files. R package version 1.16.1.
##   https://CRAN.R-project.org/package=ncdf4
## 
## A BibTeX entry for LaTeX users is
## 
##   @Manual{,
##     title = {ncdf4: Interface to Unidata netCDF (Version 4 or Earlier) Format Data
## Files},
##     author = {David Pierce},
##     year = {2019},
##     note = {R package version 1.16.1},
##     url = {https://CRAN.R-project.org/package=ncdf4},
##   }
## 
## ATTENTION: This citation information has been auto-generated from
## the package DESCRIPTION file and may need manual editing, see
## 'help("citation")'.
\end{verbatim}
\end{kframe}
\end{knitrout}

We first need to read an index into the file contents, and in additional steps we read a subset of the data. With \Rfunction{print()} we can find out the names and characteristics of the variables and attributes. In this example we use long term averages for potential evapotranspiration (PET).

We first open a connection to the file with function \Rfunction{nc\_open()}.

\begin{knitrout}\footnotesize
\definecolor{shadecolor}{rgb}{0.969, 0.969, 0.969}\color{fgcolor}\begin{kframe}
\begin{alltt}
\hlstd{meteo_data.nc} \hlkwb{<-} \hlkwd{nc_open}\hlstd{(}\hlstr{"extdata/pevpr.sfc.mon.ltm.nc"}\hlstd{)}
\hlcom{# very long output}
\hlcom{# print(meteo_data.nc)}
\end{alltt}
\end{kframe}
\end{knitrout}

\begin{playground}
Uncomment the \Rfunction{print()} statement above and study the metadata available for the data set as a whole, and for each variable.
\end{playground}
The dimensions of the array data are described with metadata, mapping indexes to in our examples a grid of latitudes and longitudes and a time vector as a third dimension. The dates are returned as character strings. We get here the variables one at a time with function \Rfunction{ncvar\_get()}.

\begin{knitrout}\footnotesize
\definecolor{shadecolor}{rgb}{0.969, 0.969, 0.969}\color{fgcolor}\begin{kframe}
\begin{alltt}
\hlstd{time.vec} \hlkwb{<-} \hlkwd{ncvar_get}\hlstd{(meteo_data.nc,} \hlstr{"time"}\hlstd{)}
\hlkwd{head}\hlstd{(time.vec)}
\end{alltt}
\begin{verbatim}
## [1] -657073 -657042 -657014 -656983 -656953 -656922
\end{verbatim}
\begin{alltt}
\hlstd{longitude} \hlkwb{<-}  \hlkwd{ncvar_get}\hlstd{(meteo_data.nc,} \hlstr{"lon"}\hlstd{)}
\hlkwd{head}\hlstd{(longitude)}
\end{alltt}
\begin{verbatim}
## [1] 0.000 1.875 3.750 5.625 7.500 9.375
\end{verbatim}
\begin{alltt}
\hlstd{latitude} \hlkwb{<-} \hlkwd{ncvar_get}\hlstd{(meteo_data.nc,} \hlstr{"lat"}\hlstd{)}
\hlkwd{head}\hlstd{(latitude)}
\end{alltt}
\begin{verbatim}
## [1] 88.5420 86.6531 84.7532 82.8508 80.9473 79.0435
\end{verbatim}
\end{kframe}
\end{knitrout}

The \code{time} vector is rather odd, as it contains only month data as these are long-term averages. From the metadata we can infer that they correspond to the months of the year, and we directly generate these, instead of attempting a conversion.

We construct a \Rclass{tibble} object with PET values for one grid point, we can take advantage of \emph{recycling} or short vectors.

\begin{knitrout}\footnotesize
\definecolor{shadecolor}{rgb}{0.969, 0.969, 0.969}\color{fgcolor}\begin{kframe}
\begin{alltt}
\hlstd{pet.tb} \hlkwb{<-}
    \hlkwd{tibble}\hlstd{(}\hlkwc{moth} \hlstd{= month.abb[}\hlnum{1}\hlopt{:}\hlnum{12}\hlstd{],}
           \hlkwc{lon} \hlstd{= longitude[}\hlnum{6}\hlstd{],}
           \hlkwc{lat} \hlstd{= latitude[}\hlnum{2}\hlstd{],}
           \hlkwc{pet} \hlstd{=} \hlkwd{ncvar_get}\hlstd{(meteo_data.nc,} \hlstr{"pevpr"}\hlstd{)[}\hlnum{6}\hlstd{,} \hlnum{2}\hlstd{, ]}
           \hlstd{)}
\hlstd{pet.tb}
\end{alltt}
\begin{verbatim}
## # A tibble: 12 x 4
##    moth    lon   lat   pet
##    <chr> <dbl> <dbl> <dbl>
##  1 Jan    9.38  86.7  4.28
##  2 Feb    9.38  86.7  5.72
##  3 Mar    9.38  86.7  4.38
##  4 Apr    9.38  86.7  6.76
##  5 May    9.38  86.7 16.6 
##  6 Jun    9.38  86.7 28.9 
##  7 Jul    9.38  86.7 22.8 
##  8 Aug    9.38  86.7 12.7 
##  9 Sep    9.38  86.7  4.09
## 10 Oct    9.38  86.7  3.35
## 11 Nov    9.38  86.7  5.08
## 12 Dec    9.38  86.7  5.17
\end{verbatim}
\end{kframe}
\end{knitrout}

If we want to read in several grid points, we can use several different approaches. In this example we take all latitudes along one longitude. Here we avoid using loops altogether when creating a \emph{tidy} \Rclass{tibble} object. However, because of how the data is stored, we needed to transpose the intermediate array before conversion into a vector.

\begin{knitrout}\footnotesize
\definecolor{shadecolor}{rgb}{0.969, 0.969, 0.969}\color{fgcolor}\begin{kframe}
\begin{alltt}
\hlstd{pet2.tb} \hlkwb{<-}
    \hlkwd{tibble}\hlstd{(}\hlkwc{moth} \hlstd{=} \hlkwd{rep}\hlstd{(month.abb[}\hlnum{1}\hlopt{:}\hlnum{12}\hlstd{],} \hlkwd{length}\hlstd{(latitude)),}
           \hlkwc{lon} \hlstd{= longitude[}\hlnum{6}\hlstd{],}
           \hlkwc{lat} \hlstd{=} \hlkwd{rep}\hlstd{(latitude,} \hlkwc{each} \hlstd{=} \hlnum{12}\hlstd{),}
           \hlkwc{pet} \hlstd{=} \hlkwd{as.vector}\hlstd{(}\hlkwd{t}\hlstd{(}\hlkwd{ncvar_get}\hlstd{(meteo_data.nc,} \hlstr{"pevpr"}\hlstd{)[}\hlnum{6}\hlstd{, , ]))}
           \hlstd{)}
\hlstd{pet2.tb}
\end{alltt}
\begin{verbatim}
## # A tibble: 1,128 x 4
##    moth    lon   lat    pet
##    <chr> <dbl> <dbl>  <dbl>
##  1 Jan    9.38  88.5  1.02 
##  2 Feb    9.38  88.5  1.57 
##  3 Mar    9.38  88.5  0.883
##  4 Apr    9.38  88.5  3.55 
##  5 May    9.38  88.5 12.4  
##  6 Jun    9.38  88.5 27.1  
##  7 Jul    9.38  88.5 21.7  
##  8 Aug    9.38  88.5 11.0  
##  9 Sep    9.38  88.5  0.356
## 10 Oct    9.38  88.5 -1.19 
## # ... with 1,118 more rows
\end{verbatim}
\begin{alltt}
\hlkwd{subset}\hlstd{(pet2.tb, lat} \hlopt{==} \hlstd{latitude[}\hlnum{2}\hlstd{])}
\end{alltt}
\begin{verbatim}
## # A tibble: 12 x 4
##    moth    lon   lat   pet
##    <chr> <dbl> <dbl> <dbl>
##  1 Jan    9.38  86.7  4.28
##  2 Feb    9.38  86.7  5.72
##  3 Mar    9.38  86.7  4.38
##  4 Apr    9.38  86.7  6.76
##  5 May    9.38  86.7 16.6 
##  6 Jun    9.38  86.7 28.9 
##  7 Jul    9.38  86.7 22.8 
##  8 Aug    9.38  86.7 12.7 
##  9 Sep    9.38  86.7  4.09
## 10 Oct    9.38  86.7  3.35
## 11 Nov    9.38  86.7  5.08
## 12 Dec    9.38  86.7  5.17
\end{verbatim}
\end{kframe}
\end{knitrout}

\begin{playground}
Play with \code{as.vector(t(ncvar\_get(meteo\_data.nc, "pevpr")[6, , ]))} until you understand what is the effect of each of the nested function calls, starting from \code{ncvar\_get(meteo\_data.nc, "pevpr")}. You will also want to use \Rfunction{str()} to see the structure of the objects returned at each stage.
\end{playground}

\begin{playground}
Instead of extracting data for one longitude across latitudes, extract data across longitudes for one latitude near the Equator.
\end{playground}

\subsection[RNetCDF]{\pkgname{RNetCDF}}

\begin{warningbox}
Package RNetCDF supports NetCDF3 files, but not those saved using the current NetCDF4 format.
\end{warningbox}

\begin{knitrout}\footnotesize
\definecolor{shadecolor}{rgb}{0.969, 0.969, 0.969}\color{fgcolor}\begin{kframe}
\begin{alltt}
\hlkwd{citation}\hlstd{(}\hlkwc{package} \hlstd{=} \hlstr{"RNetCDF"}\hlstd{)}
\end{alltt}
\begin{verbatim}
## 
## To cite package 'RNetCDF' in publications use:
## 
##   Pavel Michna and with contributions from Milton Woods (2017).
##   RNetCDF: Interface to NetCDF Datasets. R package version 1.9-1.
##   https://CRAN.R-project.org/package=RNetCDF
## 
## A BibTeX entry for LaTeX users is
## 
##   @Manual{,
##     title = {RNetCDF: Interface to NetCDF Datasets},
##     author = {Pavel Michna and with contributions from Milton Woods},
##     year = {2017},
##     note = {R package version 1.9-1},
##     url = {https://CRAN.R-project.org/package=RNetCDF},
##   }
## 
## ATTENTION: This citation information has been auto-generated from
## the package DESCRIPTION file and may need manual editing, see
## 'help("citation")'.
\end{verbatim}
\end{kframe}
\end{knitrout}

We first need to read an index into the file contents, and in additional steps we read a subset of the data. With \Rfunction{print.nc()} we can find out the names and characteristics of the variables and attributes. We open the connection with function \Rfunction{open.nc()}.

\begin{knitrout}\footnotesize
\definecolor{shadecolor}{rgb}{0.969, 0.969, 0.969}\color{fgcolor}\begin{kframe}
\begin{alltt}
\hlstd{meteo_data.nc} \hlkwb{<-} \hlkwd{open.nc}\hlstd{(}\hlstr{"extdata/meteo-data.nc"}\hlstd{)}
\hlkwd{str}\hlstd{(meteo_data.nc)}
\end{alltt}
\begin{verbatim}
##  'NetCDF' num 65536
\end{verbatim}
\begin{alltt}
\hlcom{# very long output}
\hlcom{# print.nc(meteo_data.nc)}
\end{alltt}
\end{kframe}
\end{knitrout}

The dimensions of the array data are described with metadata, mapping indexes to in our examples a grid of latitudes and longitudes and a time vector as a third dimension. The dates are returned as character strings. We get variables, one at a time, with function \Rfunction{var.get.nc()}.

\begin{knitrout}\footnotesize
\definecolor{shadecolor}{rgb}{0.969, 0.969, 0.969}\color{fgcolor}\begin{kframe}
\begin{alltt}
\hlstd{time.vec} \hlkwb{<-} \hlkwd{var.get.nc}\hlstd{(meteo_data.nc,} \hlstr{"time"}\hlstd{)}
\hlkwd{head}\hlstd{(time.vec)}
\end{alltt}
\begin{verbatim}
## [1] 20080902 20080903 20080904 20080905 20080906 20080907
\end{verbatim}
\begin{alltt}
\hlstd{longitude} \hlkwb{<-}  \hlkwd{var.get.nc}\hlstd{(meteo_data.nc,} \hlstr{"lon"}\hlstd{)}
\hlkwd{head}\hlstd{(longitude)}
\end{alltt}
\begin{verbatim}
## [1] 19.5 20.5 21.5 22.5 23.5 24.5
\end{verbatim}
\begin{alltt}
\hlstd{latitude} \hlkwb{<-}  \hlkwd{var.get.nc}\hlstd{(meteo_data.nc,} \hlstr{"lat"}\hlstd{)}
\hlkwd{head}\hlstd{(latitude)}
\end{alltt}
\begin{verbatim}
## [1] 59.5 60.5 61.5 62.5 63.5 64.5
\end{verbatim}
\end{kframe}
\end{knitrout}

We construct a \Rclass{tibble} object with values for midday UV Index for 26 days. For convenience, we convert the strings into \Rlang datetime objects.

\begin{knitrout}\footnotesize
\definecolor{shadecolor}{rgb}{0.969, 0.969, 0.969}\color{fgcolor}\begin{kframe}
\begin{alltt}
\hlstd{uvi.tb} \hlkwb{<-}
    \hlkwd{tibble}\hlstd{(}\hlkwc{date} \hlstd{=} \hlkwd{ymd}\hlstd{(time.vec,} \hlkwc{tz}\hlstd{=}\hlstr{"EET"}\hlstd{),}
           \hlkwc{lon} \hlstd{= longitude[}\hlnum{6}\hlstd{],}
           \hlkwc{lat} \hlstd{= latitude[}\hlnum{2}\hlstd{],}
           \hlkwc{uvi} \hlstd{=} \hlkwd{var.get.nc}\hlstd{(meteo_data.nc,} \hlstr{"UVindex"}\hlstd{)[}\hlnum{6}\hlstd{,}\hlnum{2}\hlstd{,]}
           \hlstd{)}
\hlstd{uvi.tb}
\end{alltt}
\begin{verbatim}
## # A tibble: 26 x 4
##    date                  lon   lat   uvi
##    <dttm>              <dbl> <dbl> <dbl>
##  1 2008-09-02 00:00:00  24.5  60.5 2.36 
##  2 2008-09-03 00:00:00  24.5  60.5 1.19 
##  3 2008-09-04 00:00:00  24.5  60.5 1.29 
##  4 2008-09-05 00:00:00  24.5  60.5 3.24 
##  5 2008-09-06 00:00:00  24.5  60.5 2.36 
##  6 2008-09-07 00:00:00  24.5  60.5 2.69 
##  7 2008-09-08 00:00:00  24.5  60.5 1.46 
##  8 2008-09-09 00:00:00  24.5  60.5 1.87 
##  9 2008-09-10 00:00:00  24.5  60.5 0.900
## 10 2008-09-11 00:00:00  24.5  60.5 2.50 
## # ... with 16 more rows
\end{verbatim}
\end{kframe}
\end{knitrout}

\section{Remotely located data}\label{sec:files:remote}

Many of the functions described above accept am URL address in place of file name. Consequently files can be read remotely, without a separate step. This can be useful, especially when file names are generated within a script. However, one should avoid, especially in the case of servers open to public access, not to generate unnecessary load on server and/or network traffic by repeatedly downloading the same file. Because of this, our first example reads a small file from my own web site. See section \ref{sec:files:txt} on page \pageref{sec:files:txt} for details of the use of these and other functions for reading text files.

\begin{knitrout}\footnotesize
\definecolor{shadecolor}{rgb}{0.969, 0.969, 0.969}\color{fgcolor}\begin{kframe}
\begin{alltt}
\hlstd{logger.df} \hlkwb{<-}
      \hlkwd{read.csv2}\hlstd{(}\hlkwc{file} \hlstd{=} \hlstr{"http://r4photobiology.info/learnr/logger_1.txt"}\hlstd{,}
                \hlkwc{header} \hlstd{=} \hlnum{FALSE}\hlstd{,}
                \hlkwc{col.names} \hlstd{=} \hlkwd{c}\hlstd{(}\hlstr{"time"}\hlstd{,} \hlstr{"temperature"}\hlstd{))}
\hlkwd{sapply}\hlstd{(logger.df, class)}
\hlkwd{sapply}\hlstd{(logger.df, mode)}
\end{alltt}
\end{kframe}
\end{knitrout}

\begin{knitrout}\footnotesize
\definecolor{shadecolor}{rgb}{0.969, 0.969, 0.969}\color{fgcolor}\begin{kframe}
\begin{alltt}
\hlstd{logger.tb} \hlkwb{<-}
    \hlkwd{read_csv2}\hlstd{(}\hlkwc{file} \hlstd{=} \hlstr{"http://r4photobiology.info/learnr/logger_1.txt"}\hlstd{,}
              \hlkwc{col_names} \hlstd{=} \hlkwd{c}\hlstd{(}\hlstr{"time"}\hlstd{,} \hlstr{"temperature"}\hlstd{))}
\hlkwd{sapply}\hlstd{(logger.tb, class)}
\hlkwd{sapply}\hlstd{(logger.tb, mode)}
\end{alltt}
\end{kframe}
\end{knitrout}

While functions in package \pkgname{readr} support the use of URLs, those in packages \pkgname{readxl} and \pkgname{xlsx} do not. Consequently we need to first download the file writing a file locally, that we can read as described in section \ref{sec:files:excel} on page \pageref{sec:files:excel}.

\begin{knitrout}\footnotesize
\definecolor{shadecolor}{rgb}{0.969, 0.969, 0.969}\color{fgcolor}\begin{kframe}
\begin{alltt}
\hlkwd{download.file}\hlstd{(}\hlstr{"http://r4photobiology.info/learnr/my-data.xlsx"}\hlstd{,}
              \hlstr{"data/my-data-dwn.xlsx"}\hlstd{,}
              \hlkwc{mode} \hlstd{=} \hlstr{"wb"}\hlstd{)}
\end{alltt}
\end{kframe}
\end{knitrout}

Functions in package \pkgname{foreign}, as well as those in package \pkgname{haven} support URLs. See section \ref{sec:files:stat} on page \pageref{sec:files:stat} for more information about importing this kind of data into R.

\begin{knitrout}\footnotesize
\definecolor{shadecolor}{rgb}{0.969, 0.969, 0.969}\color{fgcolor}\begin{kframe}
\begin{alltt}
\hlstd{remote_thiamin.df} \hlkwb{<-}
  \hlkwd{read.spss}\hlstd{(}\hlkwc{file} \hlstd{=} \hlstr{"http://r4photobiology.info/learnr/thiamin.sav"}\hlstd{,}
            \hlkwc{to.data.frame} \hlstd{=} \hlnum{TRUE}\hlstd{)}
\hlkwd{head}\hlstd{(remote_thiamin.df)}
\end{alltt}
\end{kframe}
\end{knitrout}

\begin{knitrout}\footnotesize
\definecolor{shadecolor}{rgb}{0.969, 0.969, 0.969}\color{fgcolor}\begin{kframe}
\begin{alltt}
\hlstd{remote_my_spss.tb} \hlkwb{<-}
    \hlkwd{read_sav}\hlstd{(}\hlkwc{file} \hlstd{=} \hlstr{"http://r4photobiology.info/learnr/thiamin.sav"}\hlstd{)}
\hlstd{remote_my_spss.tb}
\end{alltt}
\end{kframe}
\end{knitrout}

Function \Rfunction{download.file()} in \Rlang default \pkgname{utils} package can be used to download files using URLs. It supports differemt modes such as binary or text, and write or append, and different methods such as internal, wget and libcurl.

In this example we use a downloaded NetCDF file of long-term means for potential evapotranspiration from NOOA, the same used above in the \pkgname{ncdf4} example. This is a moderately large file at 444~KB. In this case we cannot directly open the connection to the NetCDF file, we first download it (commented out code, as we have a local copy), and then we open the local file.

\begin{knitrout}\footnotesize
\definecolor{shadecolor}{rgb}{0.969, 0.969, 0.969}\color{fgcolor}\begin{kframe}
\begin{alltt}
\hlstd{my.url} \hlkwb{<-} \hlkwd{paste}\hlstd{(}\hlstr{"ftp://ftp.cdc.noaa.gov/Datasets/ncep.reanalysis.derived/"}\hlstd{,}
                \hlstr{"surface_gauss/pevpr.sfc.mon.ltm.nc"}\hlstd{,}
                \hlkwc{sep} \hlstd{=} \hlstr{""}\hlstd{)}
\hlcom{#download.file(my.url,}
\hlcom{#              mode = "wb",}
\hlcom{#              destfile = "extdata/pevpr.sfc.mon.ltm.nc")}
\hlstd{pet_ltm.nc} \hlkwb{<-} \hlkwd{nc_open}\hlstd{(}\hlstr{"extdata/pevpr.sfc.mon.ltm.nc"}\hlstd{)}
\end{alltt}
\end{kframe}
\end{knitrout}

\begin{warningbox}
For portability NetCDF files should be downloaded in binary mode, setting \code{mode = "wb"}, which is required at least under MS-Windows.
\end{warningbox}

\section{Data acquisition from physical devices}\label{sec:data:acquisition}

Numerous modern data acquisition devices based on microcontrolers, including internet-of-things (IoT) devices, have servers (or daemons) that can be queried over a network connection to retrieve either real-time or looged data. Formats based on XML schemas or in JSON format are commonly used.

\subsection[jsonlite]{\pkgname{jsonlite}}

\begin{knitrout}\footnotesize
\definecolor{shadecolor}{rgb}{0.969, 0.969, 0.969}\color{fgcolor}\begin{kframe}
\begin{alltt}
\hlkwd{citation}\hlstd{(}\hlkwc{package} \hlstd{=} \hlstr{"jsonlite"}\hlstd{)}
\end{alltt}
\begin{verbatim}
## 
## To cite jsonlite in publications use:
## 
##   Jeroen Ooms (2014). The jsonlite Package: A Practical and
##   Consistent Mapping Between JSON Data and R Objects.
##   arXiv:1403.2805 [stat.CO] URL https://arxiv.org/abs/1403.2805.
## 
## A BibTeX entry for LaTeX users is
## 
##   @Article{,
##     title = {The jsonlite Package: A Practical and Consistent Mapping Between JSON Data and R Objects},
##     author = {Jeroen Ooms},
##     journal = {arXiv:1403.2805 [stat.CO]},
##     year = {2014},
##     url = {https://arxiv.org/abs/1403.2805},
##   }
\end{verbatim}
\end{kframe}
\end{knitrout}

We give here a simple example using a module from the \href{http://www.yoctopuce.com/}{YoctoPuce} family using a software hub running locally. We retrieve logged data from a YoctoMeteo module.

\begin{infobox}
This example is not run, and needs setting the configuration of the YoctoPuce module beforehand. Fully reproducible examples, including configuration instructions, will be included in a future revision of the manuscript.
\end{infobox}

Here we use function \Rfunction{fromJSON()} to retrieve logged data from one sensor.

\begin{knitrout}\footnotesize
\definecolor{shadecolor}{rgb}{0.969, 0.969, 0.969}\color{fgcolor}\begin{kframe}
\begin{alltt}
\hlstd{hub.url} \hlkwb{<-} \hlstr{"http://127.0.0.1:4444/"}
\hlstd{Meteo01.lst} \hlkwb{<-}
    \hlkwd{fromJSON}\hlstd{(}\hlkwd{paste}\hlstd{(hub.url,} \hlstr{"byName/Meteo01/dataLogger.json"}\hlstd{,}
                   \hlkwc{sep} \hlstd{=} \hlstr{""}\hlstd{))}
\hlkwd{names}\hlstd{(Meteo01.lst)}
\hlstd{Meteo01.lst}
\end{alltt}
\end{kframe}
\end{knitrout}

The minimum, mean and maximum values for each logging interval, need to be split from a single vector. We do this by indexing with a logical vector (recycled). The data returned is \emph{tidy} with respect to the variables, with quantity names and units also returned by the module, as well as the time.

\begin{knitrout}\footnotesize
\definecolor{shadecolor}{rgb}{0.969, 0.969, 0.969}\color{fgcolor}\begin{kframe}
\begin{alltt}
    \hlstd{val.vector} \hlkwb{<-} \hlkwd{unlist}\hlstd{(Meteo01.lst[[}\hlstr{"val"}\hlstd{]])}
    \hlstd{dplyr}\hlopt{::}\hlkwd{transmute}\hlstd{(Meteo01.lst,}
                     \hlkwc{utc.time} \hlstd{=} \hlkwd{as.POSIXct}\hlstd{(utc,} \hlkwc{origin} \hlstd{=} \hlstr{"1970-01-01"}\hlstd{,} \hlkwc{tz} \hlstd{=} \hlstr{"UTC"}\hlstd{),}
                     \hlkwc{qty} \hlstd{= qty.name,}
                     \hlkwc{unit} \hlstd{= qty.unit,}
                     \hlkwc{minimum} \hlstd{= val.vector[}\hlkwd{c}\hlstd{(}\hlnum{TRUE}\hlstd{,} \hlnum{FALSE}\hlstd{,} \hlnum{FALSE}\hlstd{)],}
                     \hlkwc{mean} \hlstd{= val.vector[}\hlkwd{c}\hlstd{(}\hlnum{FALSE}\hlstd{,} \hlnum{TRUE}\hlstd{,} \hlnum{FALSE}\hlstd{)],}
                     \hlkwc{maximum} \hlstd{= val.vector[}\hlkwd{c}\hlstd{(}\hlnum{FALSE}\hlstd{,} \hlnum{FALSE}\hlstd{,} \hlnum{TRUE}\hlstd{)],}
                     \hlstd{dur,}
                     \hlstd{freq)}
\end{alltt}
\end{kframe}
\end{knitrout}

\section{Databases}\label{sec:data:db}

One of the advantages of using databases is that subsets of cases and variables can be retrieved from databases, even remotely, making it possible to work both locally and remotely with huge data sets. One should remember that \Rlang natively keeps whole objects in RAM, and consequently available machine memory limits the size of data sets with which it is possible to work.

\begin{infobox}
The contents of this section is still missing, but will in any case be basic. I recommend the book \citebooktitle{Wickham2017} \autocite{Wickham2017} for learning how to use the packages in the \pkgname{tidyverse} suite, especially in the case of connecting to databases.
\end{infobox}

%\section[Grammar of data manipulation]{The grammar of data manipulation of the \pkgname{tidyverse}}
%
%Packages in \code{tidyverse}, define more user-friendly \emph{apply} functions, which I describe in the next sections. These packages, do much more than providing replacements for \Rlang \emph{apply} functions. They define a ``grammar of data'' for data manipulations like transformations and summaries, based on the same philosophy as that behind the grammar of graphics on which package \pkgname{ggplot2} is based (see Chapter \ref{chap:R:plotting} starting on page \pageref{chap:R:plotting}).
%
%To make the problem of manipulating data, tractable and consistent, the first step is to settle on a certain way of storing data. In \Rlang data frames, variables are most frequently in columns and cases are in rows. This is a good start and also frequently used in other software. The first major inconsistency across programs, and to some extent among \Rlang packages, is how to store data for sequential or repeated measurements. Do the rows represent measuring events, or measured objects? In R, data from individual measuring events are in most cases stored as rows, and if those that correspond to the same object or individual encoded with an index variable. Furthermore, say in a time sequence, the times or dates are stored in an additional variable. \Rlang approach is much more flexible in that it does not assume that observations on different individuals are synchronized. \citeauthor{Wickham2014a} \cite{Wickham2014a} has coined the name ``tidy data'' organized in this manner.
%
%Hadley Wickham, together with collaborators, has developed a set of \Rlang tools for the manipulation, plotting and analysis of \emph{tidy data}, thoroughly described in the recently published book \citebooktitle{Wickham2017} \autocite{Wickham2017}. The book \citebooktitle{Peng2017} \autocite{Peng2017} covers data manipulaiton in the first chapters before moving on to programming. Here we give an overview of the components of the \pkgname{tidyverse} grammar of data manipulation. The book \citebooktitle{Wickham2017} and the documentation included with the various packages should be consulted for a deeper and more detailed discussion. Aspects of the \pkgname{tidyverse} related to reading and writing data files (\pkgname{readr}, \pkgname{readxl}, and \pkgname{xml2}) have been discussed in earlier sections of this chapter, while the use of (\pkgname{ggplot2}) for plotting is described in later chapters.
%
%\subsection{Better data frames}
%
%\section{Grammar for manipulation of character strings}\label{sec:data:strings}
%
%\begin{warningbox}
%  This section will contain an introduction to character-string manipulation with methods from packages \pkgname{stringr} and \pkgname{stringi}.
%\end{warningbox}
%
%\section{Pipes and tees}\label{sec:data:pipes}
%
%Pipes have been part of Unix shells already starting from the early days of Unix in 1973. By the early 1980's the idea had led to the development of many \emph{tools} to be used in \pgrmname{sh} connected by pipes \autocite{Kernigham1981}. Shells developed more recently like the Korn shell, \pgrmname{ksh}, and \pgrmname{bash} maintained support for this approach \autocite{Rosenblatt1993}. The idea behind the concept of data pipe, is that one can directly use the output from one tool as input for the tool doing the next stage in the processing. These tools are simple programs that do a defined operation, such as \pgrmname{ls} or \pgrmname{cat}---from where the names of equivalent functions in \langname{R} were coined.
%
%Apple's OS X is based on Unix, and allows the use of pipes at the command prompt and in shell scripts. Linux uses the tools from the Gnu project that to a large extent replicate and extend the capabilities  by the and also natively supports \emph{pipes} equivalent to those in Unix. In Windows support for pipes was initially partial at the command prompt. Currently, Window's PowerShell supports the use of pipes, as well as some Linux shells are available in versions that can be used under MS-Windows.
%
%Within \Rlang code, the support for pipes is not native, but instead implemented by some recent packages. Most of the packages in the \code{tidyverse} support this new syntax through the use of package \pkgname{magrittr}. The use of pipes has advantages and disadvantages. They are at their best when connecting small functions with rather simple inputs and outputs. They tend, yet, to be difficult to debug, a problem that counterbalances the advantages of the clear and consice notation achieved.
%
%\subsection{Pipes and tees}
%
%The \emph{pipe} operator \Roperator{\%>\%} is defined in package \pkgname{magrittr}, but imported and re-exported by other packages in the \pkgname{tidyverse}. The idea is that the value returned by a function is passed by the pipe operator as the first argument to the next function in the ``pipeline''.
%
%We can chain some of the examples in the previous section into a ``pipe''.
%
%<<pipes-01>>=
%tibble(numbers = 1:9, letters = rep(letters[1:3], 3)) %>%
%  group_by(letters) %>%
%  summarise(mean_numbers = mean(numbers),
%            var_numbers = var(numbers),
%            n = n())
%@
%
%I we want to save the returned value, to me it feels more natural to use a left to right assignment, although the usual right to left one can also be used.
%
%<<pipes-02>>=
%tibble(numbers = 1:9, letters = rep(letters[1:3], 3)) %>%
%  group_by(letters) %>%
%  summarise(mean_numbers = mean(numbers),
%            var_numbers = var(numbers),
%            n = n()) -> summary.tb
%summary.tb
%@
%
%<<pipes-03>>=
%summary.tb <-
%    tibble(numbers = 1:9, letters = rep(letters[1:3], 3)) %>%
%      group_by(letters) %>%
%      summarise(mean_numbers = mean(numbers),
%                var_numbers = var(numbers),
%                n = n())
%summary.tb
%@
%
%As \Rfunction{print()} returns its input, we can also include it in the middle of a pipe as a simple way of visualizing what takes place at each step.
%
%<<pipes-04>>=
%tibble(numbers = 1:9, letters = rep(letters[1:3], 3)) %>%
%  print() %>%
%  group_by(letters) %>%
%  summarise(mean_numbers = mean(numbers),
%            var_numbers = var(numbers),
%            n = n())  %>%
%            print() -> summary.tb
%@
%
%\begin{explainbox}
%\textbf{Why and how we can insert a call to \Rfunction{print()} in the middle of a pipe?} An extremely simple example, with a twist, follows.
%
%<<pipes-expl-01>>=
%print("a") %>% print()
%@
%
%The example above is equivalent to.
%
%<<pipes-expl-02>>=
%print(print("a"))
%@
%
%The examples above are somehow surprising but instructive. Function \Rfunction{print()} returns a value, its first argument, but \emph{invisibly}---see help for \Rfunction{invisible()}. Otherwise default printing would result in the value being printed twice at the \Rlang prompt. We can demonstrate this by saving the value returned by print.
%
%<<pipes-expl-03>>=
%a <- print("a")
%class(a)
%a
%b <- print(2)
%class(b)
%b
%@
%
%\end{explainbox}
%
%\begin{playground}
%Assemble different pipes, predict what will be the output, and check your prediction by executing the code.
%\end{playground}
%
%Although \Roperator{\%>\%} is the most frequently used pipe operator, there are some additional ones available. We start by creating a tibble.
%
%<<pipes-11>>=
%my.tb <- tibble(numbers = 1:9, letters = rep(letters[1:3], 3))
%@
%
%We first demonstrate that the pipe can have at its head a variable with the same operator as we used above, in this case a tibble.
%
%<<pipes-12>>=
%my.tb %>%
%  group_by(letters) %>%
%  summarise(mean_numbers = mean(numbers),
%            var_numbers = var(numbers),
%            n = n())
%my.tb
%@
%
%We could save the output of the pipe to the same variable at the head of the pipe by explicitly using the same name, but operator \Roperator{\%<>\%} does this directly.
%
%<<pipes-13>>=
%my.tb %<>%
%  group_by(letters) %>%
%  summarise(mean_numbers = mean(numbers),
%            var_numbers = var(numbers),
%            n = n())
%my.tb
%@
%
%A few additional operators defined in \pkgname{magrittr} are not re-exported by packages in the \pkgname{tidyverse}, so their use requires \pkgname{magrittr} to be loaded.
%
%When functions have a side-effect like \Rfunction{print()} displaying its input and passing it unchanged as the returned value, we do not need to split flow of processing through a pipe. In real house plumbing, when a split is needed a ``tee'' shaped pipe joint is used. This is where the name tee as used in programming originates. Operator \Roperator{\%T>\%} passes along not the value returned by a function, but instead the value passed to it as input.
%
%As in the previous chunk we assigned the summaries to \code{my.tb}, we need to re-create it to run the next example.
%
%<<pipes-14>>=
%my.tb <- tibble(numbers = 1:9, letters = rep(letters[1:3], 3))
%@
%
%<<pipes-15>>=
%sump <- function(x) {print("hello"); return(NULL)}
%my.tb %>%
%  group_by(letters) %>%
%  summarise(mean_numbers = mean(numbers),
%            var_numbers = var(numbers),
%            n = n()) %T>%
%  sump() -> summary.tb
%@
%
%We can see that the value saved in \code{summary.tb} is the one returned by \Rfunction{summarize()} rather than the one returned by \Rfunction{sump()}.
%
%\begin{playground}
%Look up the help page for operator \Roperator{\%\$\%} and write an example of its use.
%\end{playground}
%
%\section{Extended examples}\label{sec:dataex}
%
%\subsection{Well-plate data}\label{sec:dataex:well:plate}
%
%Our first example attempts to simulate data arranged in rows and columns based on spatial position, such as in a well plate. We will use pseudo-random numbers for the fake data---i.e.\ the measured response.
%
%<<well-plate-data-01>>=
%well_data.tb <-
%  as.tibble(matrix(rnorm(50),
%                   nrow = 5,
%                   dimnames = list(as.character(1:5), LETTERS[1:10])))
%# drops names of rows
%well_data.tb <-
%  add_column(well_data.tb, row_ids = 1:5, .before = 1)
%@
%
%In addition, we create a matrix of fake treatment ids.
%<<well-plate-data-02>>=
%well_ids.tb <-
%  as.tibble(matrix(sample(letters, size = 50, replace = TRUE),
%                   nrow = 5,
%                   dimnames = list(as.character(1:5), LETTERS[1:10])))
%# drops names of rows
%well_ids.tb <-
%  add_column(well_ids.tb, row_ids = 1:5, .before = 1)
%@
%
%As we will combine them, the coordinates should be encoded consistently in the two objects.
%I will take the approach of first converting each tibble into a tidy tibble. We use function \Rfunction{gather()} from package \pkgname{tidyr}.
%
%<<well-plate-data-03>>=
%well_data.ttb <- gather(well_data.tb,
%                       key = col_ids, value = reading,
%                       -row_ids)
%well_ids.ttb <- gather(well_ids.tb,
%                       key = col_ids, value = group,
%                       -row_ids)
%@
%
%Now we need to join the two tibbles into a single one. In this case, as we know that the row order in the two tibbles is matched, we could simply use \Rfunction{cbind()}. However, \Rfunction{full\_join()}, from package \pkgname{dplyr} provides a more general and less error prone alternative as it can do the matching based on the values of any variables common to both tibbles, by default all the variables in common, as needed here. We use a ``pipe'', through which, after the join, we remove the ids (assuming they are no longer needed), sort the rows by group, and finally save the result to a new ``tidy'' tibble.
%
%<<well-plate-data-04>>=
%full_join(well_ids.ttb, well_data.ttb) %>%
%  select(-row_ids, -col_ids) %>%
%  arrange(group) -> well.tb
%well.tb
%@
%
%We finally calculate \emph{summaries} by group using function \Rfunction{summarise()}, and store the tibble containing the summaries to variable \code{well\_summaries.tb}.
%
%<<well-plate-data-05>>=
%group_by(well.tb, group) %>%
%  summarise(avg_read = mean(reading),
%            var_read = var(reading),
%            count = n()) -> well_summaries.tb
%well_summaries.tb
%@
%
%We now save the tibbles into an \Rlang data file with function \Rfunction{save()}.
%
%<<well-plate-data-06>>=
%save(well.tb, well_summaries.tb, file = "data/well-data.rda")
%@
%
%\subsection{Seedling morphology}\label{sec:dataex:birch}
%
%We use here data from an experiment on the effects of spacing in the nursery between silver birch seedlings on their morphology. We take one variable from a lager study \autocite{Aphalo2006}, the leaf area at different heights above the ground in 10~cm increments. Area was measured separately for leaves on the main stem and leaves on branches.
%
%In this case, as the columns are badly aligned in the original text file, we use \Rfunction{read.table()} from base R, rather than \Rfunction{read\_table()} from \pkgname{readr}. Afterwards we heavily massage the data into shape so as to obtain a tidy tibble with the total leaf area per height segment per plant. The file contains additional data that we discard for this example.
%
%<<birch-area-01>>=
%as.tibble(read.table("extdata/areatable.dat", header = TRUE)) %>%
%  filter(row %in% 4:8) %>%
%  select(code, tray, row, starts_with("a.")) %>%
%  gather(key = sample, value = area, -tray, -row, -code) %>%
%  mutate(segment = str_extract(sample, "[0-9]{1,2}"),
%         part = ifelse(str_extract(sample, "[bm]") == "b",
%                       "branch", "main")) %>%
%  group_by(tray, code, row, segment) %>%
%  summarise(area_tot = sum(area)) -> birch.tb
%birch.tb
%@
%
%\begin{playground}
%The previous chunk uses a long ``pipe'' to manipulate the data. I built this example interactively, starting at the top, and adding one line at a time. Repeat this process, line by line. If in a given line you do not understand why a certain bit of code is included, look at the help pages, and edit the code to experiment.
%\end{playground}
%
%We now will calculate means per true replicate, the trays. Then use these means to calculate overall means, standard deviations and coefficients of variabilities (\%).
%
%<<birch-area-02>>=
%group_by(birch.tb, tray, row, segment) %>%
%  summarise(area = mean(area_tot)) %>%
%  group_by(row, segment) %>%
%  summarise(mean_area = mean(area),
%            sd_area = sd(area),
%            cv_area = sd_area / mean_area * 100) ->
%  birch_summaries.tb
%birch_summaries.tb
%@
%
%We could be also interested in total leaf area per plant. The code is the same as above, but with no grouping for \code{segment}.
%
%<<birch-area-03>>=
%group_by(birch.tb, tray, row) %>%
%  summarise(area = mean(area_tot)) %>%
%  group_by(row) %>%
%  summarise(mean_area = mean(area),
%            sd_area = sd(area),
%            cv_area = sd_area / mean_area * 100) ->
%  birch_plant_summaries.tb
%birch_plant_summaries.tb
%@
%
%We now save the tibbles into an \Rlang data file.
%
%<<birch-area-04>>=
%save(birch.tb, birch_summaries.tb, birch_plant_summaries.tb,
%     file = "data/birch-data.rda")
%@
%
%\begin{playground}
%Repeat the same calculations for all the rows as I originally did. I eliminated the data from the borders of the trays, as those plants apparently did not really experience as crowded a space as that corresponding to the nominal spacing.
%\end{playground}
%
\begin{infobox}
It is always good to clean up, and in the case of the book, the best way to test that the examples
can be run in a ``clean'' system.

\begin{knitrout}\footnotesize
\definecolor{shadecolor}{rgb}{0.969, 0.969, 0.969}\color{fgcolor}\begin{kframe}
\begin{alltt}
\hlkwd{unlink}\hlstd{(}\hlstr{"./data"}\hlstd{,} \hlkwc{recursive} \hlstd{=} \hlnum{TRUE}\hlstd{)}
\hlkwd{unlink}\hlstd{(}\hlstr{"./extdata"}\hlstd{,} \hlkwc{recursive} \hlstd{=} \hlnum{TRUE}\hlstd{)}
\end{alltt}
\end{kframe}
\end{knitrout}
\end{infobox}

\begin{knitrout}\footnotesize
\definecolor{shadecolor}{rgb}{0.969, 0.969, 0.969}\color{fgcolor}\begin{kframe}
\begin{alltt}
\hlkwd{try}\hlstd{(}\hlkwd{detach}\hlstd{(package}\hlopt{:}\hlstd{jsonlite))}
\hlkwd{try}\hlstd{(}\hlkwd{detach}\hlstd{(package}\hlopt{:}\hlstd{lubridate))}
\hlkwd{try}\hlstd{(}\hlkwd{detach}\hlstd{(package}\hlopt{:}\hlstd{ncdf4))}
\hlkwd{try}\hlstd{(}\hlkwd{detach}\hlstd{(package}\hlopt{:}\hlstd{RNetCDF))}
\hlkwd{try}\hlstd{(}\hlkwd{detach}\hlstd{(package}\hlopt{:}\hlstd{xml2))}
\hlkwd{try}\hlstd{(}\hlkwd{detach}\hlstd{(package}\hlopt{:}\hlstd{haven))}
\hlkwd{try}\hlstd{(}\hlkwd{detach}\hlstd{(package}\hlopt{:}\hlstd{foreign))}
\hlkwd{try}\hlstd{(}\hlkwd{detach}\hlstd{(package}\hlopt{:}\hlstd{pdftools))}
\hlkwd{try}\hlstd{(}\hlkwd{detach}\hlstd{(package}\hlopt{:}\hlstd{xlsx))}
\hlkwd{try}\hlstd{(}\hlkwd{detach}\hlstd{(package}\hlopt{:}\hlstd{readxl))}
\hlkwd{try}\hlstd{(}\hlkwd{detach}\hlstd{(package}\hlopt{:}\hlstd{readr))}
\hlkwd{try}\hlstd{(}\hlkwd{detach}\hlstd{(package}\hlopt{:}\hlstd{tidyr))}
\hlkwd{try}\hlstd{(}\hlkwd{detach}\hlstd{(package}\hlopt{:}\hlstd{dplyr))}
\hlkwd{try}\hlstd{(}\hlkwd{detach}\hlstd{(package}\hlopt{:}\hlstd{stringr))}
\hlkwd{try}\hlstd{(}\hlkwd{detach}\hlstd{(package}\hlopt{:}\hlstd{wrapr))}
\hlkwd{try}\hlstd{(}\hlkwd{detach}\hlstd{(package}\hlopt{:}\hlstd{magrittr))}
\hlkwd{try}\hlstd{(}\hlkwd{detach}\hlstd{(package}\hlopt{:}\hlstd{tibble))}
\hlkwd{try}\hlstd{(}\hlkwd{detach}\hlstd{(package}\hlopt{:}\hlstd{learnrbook))}
\end{alltt}
\end{kframe}
\end{knitrout}



\chapter{Further reading about R}\label{chap:R:readings}

\begin{VF}
Before you become too entranced with gorgeous gadgets and mesmerizing video displays, let me remind you that information is not knowledge, knowledge is not wisdom, and wisdom is not foresight. Each grows out of the other, and we need them all.

\VA{Arthur C. Clarke}{Official website at \url{http://arthurcclarke.org}}
\end{VF}

%\dictum[Arthur C. Clarke]{Before you become too entranced with gorgeous gadgets and mesmerizing video displays, let me remind you that information is not knowledge, knowledge is not wisdom, and wisdom is not foresight. Each grows out of the other, and we need them all.}\vskip2ex

\begin{warningbox}
  This list will be expanded and more importantly reorganized and short comments added for book or group of books.
\end{warningbox}

\section{Introductory texts}

\cite{Allerhand2011,Dalgaard2008,Zuur2009,Teetor2011,Peng2017,Paradis2005,Peng2016}

\section{Texts on specific aspects}

\cite{Chang2013,Fox2002,Fox2010,Faraway2004,Faraway2006,Everitt2011,Wickham2017}

\section{Advanced texts}

\cite{Xie2013,Chambers2016,Wickham2015,Wickham2014advanced,Wickham2016,Pinheiro2000,Murrell2011,Matloff2011,Ihaka1996,Venables2000}

\section{Texts for S/R wisdom}

\cite{Burns1998,Burns2011,Burns2012,Bentley1986,Bentley1988}

\backmatter

\printbibliography

\printindex

\printindex[rcatsidx]

\printindex[rindex]

\end{document}

\appendix

\chapter{Build information}

\begin{knitrout}\footnotesize
\definecolor{shadecolor}{rgb}{0.969, 0.969, 0.969}\color{fgcolor}\begin{kframe}
\begin{alltt}
\hlkwd{Sys.info}\hlstd{()}
\end{alltt}
\begin{verbatim}
##        sysname        release        version       nodename        machine 
##      "Windows"       "10 x64"  "build 18362"        "MUSTI"       "x86-64" 
##          login           user effective_user 
##       "aphalo"       "aphalo"       "aphalo"
\end{verbatim}
\end{kframe}
\end{knitrout}



\begin{knitrout}\footnotesize
\definecolor{shadecolor}{rgb}{0.969, 0.969, 0.969}\color{fgcolor}\begin{kframe}
\begin{alltt}
\hlkwd{sessionInfo}\hlstd{()}
\end{alltt}
\begin{verbatim}
## R version 3.6.1 (2019-07-05)
## Platform: x86_64-w64-mingw32/x64 (64-bit)
## Running under: Windows 10 x64 (build 18362)
## 
## Matrix products: default
## 
## locale:
## [1] LC_COLLATE=English_United Kingdom.1252 
## [2] LC_CTYPE=English_United Kingdom.1252   
## [3] LC_MONETARY=English_United Kingdom.1252
## [4] LC_NUMERIC=C                           
## [5] LC_TIME=English_United Kingdom.1252    
## 
## attached base packages:
## [1] tools     stats     graphics  grDevices utils     datasets  methods  
## [8] base     
## 
## other attached packages:
## [1] readODS_1.6.7 svglite_1.2.2 knitr_1.23   
## 
## loaded via a namespace (and not attached):
##  [1] Rcpp_1.0.1            pillar_1.4.2          compiler_3.6.1       
##  [4] cellranger_1.1.0      highr_0.8             forcats_0.4.0        
##  [7] pdftools_2.2          ncdf4_1.16.1          zeallot_0.1.0        
## [10] jsonlite_1.6          lubridate_1.7.4       evaluate_0.14        
## [13] tibble_2.1.3          RNetCDF_1.9-1         pkgconfig_2.0.2      
## [16] rlang_0.4.0           cli_1.1.0             curl_4.0             
## [19] haven_2.1.1           xfun_0.8              rJava_0.9-11         
## [22] dplyr_0.8.3           stringr_1.4.0         xml2_1.2.0           
## [25] gdtools_0.1.9         vctrs_0.2.0           xlsxjars_0.6.1       
## [28] askpass_1.1           hms_0.5.0             tidyselect_0.2.5     
## [31] glue_1.3.1            qpdf_1.1              R6_2.4.0             
## [34] rematch_1.0.1         fansi_0.4.0           readxl_1.3.1         
## [37] foreign_0.8-71        tidyr_0.8.3           wrapr_1.8.9          
## [40] readr_1.3.1           purrr_0.3.2           magrittr_1.5         
## [43] ellipsis_0.2.0.1      backports_1.1.4       assertthat_0.2.1     
## [46] utf8_1.1.4            xlsx_0.6.1            stringi_1.4.3        
## [49] learnrbook_0.0.2.9001 crayon_1.3.4
\end{verbatim}
\end{kframe}
\end{knitrout}

%

\end{document}


